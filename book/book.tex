% Options for packages loaded elsewhere
\PassOptionsToPackage{unicode}{hyperref}
\PassOptionsToPackage{hyphens}{url}
%
\documentclass[
]{article}
\usepackage{amsmath,amssymb}
\usepackage{lmodern}
\usepackage{iftex}
\ifPDFTeX
  \usepackage[T1]{fontenc}
  \usepackage[utf8]{inputenc}
  \usepackage{textcomp} % provide euro and other symbols
\else % if luatex or xetex
  \usepackage{unicode-math}
  \defaultfontfeatures{Scale=MatchLowercase}
  \defaultfontfeatures[\rmfamily]{Ligatures=TeX,Scale=1}
\fi
% Use upquote if available, for straight quotes in verbatim environments
\IfFileExists{upquote.sty}{\usepackage{upquote}}{}
\IfFileExists{microtype.sty}{% use microtype if available
  \usepackage[]{microtype}
  \UseMicrotypeSet[protrusion]{basicmath} % disable protrusion for tt fonts
}{}
\makeatletter
\@ifundefined{KOMAClassName}{% if non-KOMA class
  \IfFileExists{parskip.sty}{%
    \usepackage{parskip}
  }{% else
    \setlength{\parindent}{0pt}
    \setlength{\parskip}{6pt plus 2pt minus 1pt}}
}{% if KOMA class
  \KOMAoptions{parskip=half}}
\makeatother
\usepackage{xcolor}
\usepackage{graphicx}
\makeatletter
\def\maxwidth{\ifdim\Gin@nat@width>\linewidth\linewidth\else\Gin@nat@width\fi}
\def\maxheight{\ifdim\Gin@nat@height>\textheight\textheight\else\Gin@nat@height\fi}
\makeatother
% Scale images if necessary, so that they will not overflow the page
% margins by default, and it is still possible to overwrite the defaults
% using explicit options in \includegraphics[width, height, ...]{}
\setkeys{Gin}{width=\maxwidth,height=\maxheight,keepaspectratio}
% Set default figure placement to htbp
\makeatletter
\def\fps@figure{htbp}
\makeatother
\setlength{\emergencystretch}{3em} % prevent overfull lines
\providecommand{\tightlist}{%
  \setlength{\itemsep}{0pt}\setlength{\parskip}{0pt}}
\setcounter{secnumdepth}{-\maxdimen} % remove section numbering
\newlength{\cslhangindent}
\setlength{\cslhangindent}{1.5em}
\newlength{\csllabelwidth}
\setlength{\csllabelwidth}{3em}
\newlength{\cslentryspacingunit} % times entry-spacing
\setlength{\cslentryspacingunit}{\parskip}
\newenvironment{CSLReferences}[2] % #1 hanging-ident, #2 entry spacing
 {% don't indent paragraphs
  \setlength{\parindent}{0pt}
  % turn on hanging indent if param 1 is 1
  \ifodd #1
  \let\oldpar\par
  \def\par{\hangindent=\cslhangindent\oldpar}
  \fi
  % set entry spacing
  \setlength{\parskip}{#2\cslentryspacingunit}
 }%
 {}
\usepackage{calc}
\newcommand{\CSLBlock}[1]{#1\hfill\break}
\newcommand{\CSLLeftMargin}[1]{\parbox[t]{\csllabelwidth}{#1}}
\newcommand{\CSLRightInline}[1]{\parbox[t]{\linewidth - \csllabelwidth}{#1}\break}
\newcommand{\CSLIndent}[1]{\hspace{\cslhangindent}#1}
\ifLuaTeX
  \usepackage{selnolig}  % disable illegal ligatures
\fi
\IfFileExists{bookmark.sty}{\usepackage{bookmark}}{\usepackage{hyperref}}
\IfFileExists{xurl.sty}{\usepackage{xurl}}{} % add URL line breaks if available
\urlstyle{same} % disable monospaced font for URLs
\hypersetup{
  hidelinks,
  pdfcreator={LaTeX via pandoc}}

\author{}
\date{}

\begin{document}

% Insert the title page at the beginning
% titlepage.tex
\begin{titlepage}
\centering
\Huge \textbf{Κατασκευή Συστήματων Δίαδρασης} \\
\vspace{1cm}
\LARGE ORGANIZATION : asd-xc \\
\vfill
\Large \textbf{Συγγραφέας : Κωνσταντίνος Χωριανόπουλος} \\
\vspace{0.5cm}
\Large Book pdf created by : Νικόλας Νάτσος (YGNT7777)
\end{titlepage}


\hypertarget{ux3baux3b1ux3c4ux3b1ux3c3ux3baux3b5ux3c5ux3ae-ux3c4ux3bfux3c5-ux3b2ux3b9ux3b2ux3bbux3afux3bfux3c5}{%
\section{Κατασκευή του
Βιβλίου}\label{ux3baux3b1ux3c4ux3b1ux3c3ux3baux3b5ux3c5ux3ae-ux3c4ux3bfux3c5-ux3b2ux3b9ux3b2ux3bbux3afux3bfux3c5}}

\begin{quote}
Όπως η τέχνη θεωρήθηκε ως μίμηση της ζωής, έτσι και οι τέχνες των
υπολογιστών μπορούν να θεωρηθούν ως η μίμηση της ίδιας της δημιουργίας
Alan Kay
\end{quote}

\hypertarget{ux3c4ux3b9-ux3b5ux3afux3bdux3b1ux3b9-ux3adux3bdux3b1-ux3b2ux3b9ux3b2ux3bbux3afux3bf}{%
\subsection{Τι είναι ένα
βιβλίο;}\label{ux3c4ux3b9-ux3b5ux3afux3bdux3b1ux3b9-ux3adux3bdux3b1-ux3b2ux3b9ux3b2ux3bbux3afux3bf}}

Για τους περισσότερους αναγνώστες αυτή είναι μάλλον ρητορική, αν όχι
περιττή ερώτηση. Δεν υπάρχει αμφιβολία ότι ένα βιβλίο είναι ένα φυσικό
αντικείμενο που περιέχει δεμένες σελίδες. Πράγματι, αυτός είναι ένας από
τους τέσσερις ορισμούς που βρίσκουμε για τη φύση του σύγχρονου
βιβλίου.\footnote{Borsuk (2018)} Ένα βιβλίο είναι πρώτα από όλα ένα
αντικείμενο, αλλά είναι, επίσης, και το περιεχόμενό του, το οποίο μπορεί
να είναι διαθέσιμο σε άλλες μορφές, όπως είναι η ιστοσελίδα ή ο
ηλεκτρονικός αναγνώστης, τα οποία έχουν πολύ διαφορετική φυσική μορφή
από το βιβλίο, αλλά έχουν το ίδιο ακριβώς περιεχόμενο. Επίσης, ένα
βιβλίο είναι μια ιδέα, με την έννοια ότι το νόημα του θα μπορούσε να
γίνει διαθέσιμο μέσω άλλων σημαντικών μορφών, όπως για παράδειγμα ένα
βίντεο. Τέλος, ένα βιβλίο είναι μια διεπαφή, γιατί, μέσω της οργάνωσης
του περιεχομένου σε σελίδες, ο αναγνώστης μπορεί να πλοηγηθεί όπως θέλει
και όπως το σχεδίασαν ο συγγραφέας και ο εκδότης. Με αυτό το βιβλίο,
επιχειρούμε να προσθέσουμε έναν ακόμη ορισμό για τη φύση του βιβλίου,
\emph{ένα βιβλίο είναι επίσης και η διαδικασία κατασκευής του.}

\hypertarget{ux3c0ux3ceux3c2-ux3baux3b1ux3c4ux3b1ux3c3ux3baux3b5ux3c5ux3acux3b6ux3bfux3c5ux3bcux3b5-ux3adux3bdux3b1-ux3b2ux3b9ux3b2ux3bbux3afux3bf}{%
\subsection{Πώς κατασκευάζουμε ένα
βιβλίο;}\label{ux3c0ux3ceux3c2-ux3baux3b1ux3c4ux3b1ux3c3ux3baux3b5ux3c5ux3acux3b6ux3bfux3c5ux3bcux3b5-ux3adux3bdux3b1-ux3b2ux3b9ux3b2ux3bbux3afux3bf}}

Η διαδικασία κατασκευής των περισσότερων βιβλίων τις τελευταίες
δεκαετίες μετά την διάδοση του επιτραπέζιου υπολογιστή και της
επιφάνειας εργασίας, γίνεται με τις αντίστοιχες εφαρμογές. Ενδεικτικά,
ένας συγγραφέας θα χρησιμοποιήσει μια εφαρμογή όπως το Microsoft Word ή
το Apple Pages για να γράψει ένα βιβλίο και στην συνέχεια η εκδοτική
ομάδα θα το μετατρέψει στο τελικό βιβλίο με ένα πρόγραμμα όπως το Adobe
InDesign. Οι εφαρμογές αυτές μπορεί να είναι από διαφορετικές εταιρείες
και να αλλάζουν σταδιακά, αλλά η βασική τους φιλοσοφία είναι η ίδια
ακριβώς και τοποθετεί έναν αδιαπέραστο τοίχο ανάμεσα στην συγγραφή και
την παραγωγή του βιβλίου.

Η διαδικασία κατασκευής αυτού του βιβλίου γκρεμίζει τον τοίχο που
χωρίζει την συγγραφή από την παραγωγή με μια διαδικασία κατασκευής που
βασίζεται στην τεχνολογία λογισμικού και στα εργαλεία της γραμμής
εντολών. Τα πλεονεκτήματα αυτής της επιλογής είναι πάρα πολλά, ανάμεσα
στα οποία, το πιο απλό και χρήσιμο είναι ότι οι διορθώσεις που γίνονται
από τον συγγραφέα και τους συντελεστές της έκδοσης περνάνε απευθείας στο
τελικό βιβλίο, αφού υπάρχει μόνο ένα πηγαίο κείμενο σε ένα μόνο αρχείο.
Αντίθετα, στην επιτραπέζια σελιδοποίηση βιβλίων, για κάθε αλλαγή που
κάνει ο συγγραφέας στο δικό του αρχείο, θα πρέπει να περαστεί
χειροκίνητα στο διαφορετικό αρχείο παραγωγής. Η συντήρηση δύο διακριτών
αρχείων με το ίδιο περιεχόμενο είναι μια από τις πιο κακές πρακτικές
στην πληροφορική, γιατί είναι θέμα χρόνου τα δύο αυτά αρχεία να χάσουν
τον συγχρονισμό τους.

Η πιο καλή πρακτική για την συντήρηση πολλών διαφορετικών εκδοχών του
ίδιου αρχείου είναι ένα σύστημα ελέγχου εκδόσεων. Για αυτόν τον λόγο, η
συγγραφή αυτού του βιβλίου έχει γίνει στην πλατφόρμα του Github, το
οποίο βασίζεται στο εργαλείο git για τον έλεγχο εκδόσεων αρχείων
κειμένου. Με αυτόν τον τρόπο, για κάθε αλλαγή που γίνεται διατηρείται
ιστορικό. Αν και δεν είναι πιθανό να θέλουμε να γυρίσουμε σε παλιότερες
εκδόσεις του αρχείου κειμένου, είναι πολύ πιθανό να θέλουμε να γίνουν
διορθώσεις από τρίτους, όπως είναι η γλωσσική επιμέλεια. Τα σύγχρονα
συστήματα ελέγχου εκδόσεων διευκολύνουν την συνεργασία πολλών συγγραφέων
πάνω σε αρχεία κειμένου, όπου εκτός από τις αλλαγές κρατάνε και το
ιστορικό των συγγραφέων. Επομένως, η παραγωγή του βιβλίου μπορεί να
γίνει από την ομάδα συγγραφής και υποστήριξης με χρήση απλών εργαλείων
λογισμικού και αρχεία απλού κειμένου, και η συνεισφορά του κάθε μέλους
να τεκμηριώνεται αυτόματα από την δραστηριότητά του.

Καθώς το πηγαίο κείμενο και η διαδικασία παραγωγής του τελικού βιβλίου
βρίσκονται αποθηκευμένα σε δημόσια αποθετήρια με έλεγχο εκδόσεων σε μια
συνεργατική πλατφόρμα ανάπτυξης λογισμικού, αυτόματα προκύπτουν πρόσθετα
πλεονεκτήματα. Για παράδειγμα, ο επιμελής αναγνώστης μπορεί να διορθώσει
σφάλματα και αυτόματα να προστεθεί στους συντελεστές. Επίσης, το
συνολικό έργο μπορεί να διακλαδωθεί σε νέες κατευθύνσεις. Η δυνατότητα
διακλαδώσεων επιτρέπει την προσθήκη νέου περιεχομένου το οποίο
προαιρετικά θα μπορούσε να προστεθεί και στο κεντρικό αποθετήριο του
βιβλίου. Με αυτόν τον τρόπο, όχι μόνο η διαδικασία παραγωγής, αλλά και η
ίδια η συγγραφή του βιβλίου μετασχηματίζεται προς μια συνεργατική
κατεύθυνση. Η δυνατότητα αυτή έχει ήδη χρησιμεύσει ως άσκηση για τους
φοιτητές των αντίστοιχων μαθημάτων. Επιπλέον, η δυνατότητα διαφανούς και
τεκμηριωμένης συνεργασίας μπορεί να χρησιμεύσει σε συλλογικούς τόμους.

Πέρα από τις πρακτικές βελτιώσεις στην παραγωγή που συνοδεύουν αυτήν την
διαδικασία κατασκευής, υπάρχουν επιπλέον κίνητρα που ενθάρρυναν αυτές
τις προδιαγραφές για αυτό το έργο συγγραφής. Η επεξεργασία εγγράφων ήταν
από τις πρώτες δημοφιλείς χρήσεις των προσωπικών υπολογιστών και
συνεχίζει να έχει έναν σημαντικό ρόλο. Οι περισσότεροι χρήστες
υπολογιστών έχουν ήδη μια σχετική εμπειρία από τις αντίστοιχες γραφικές
εφαρμογές, αλλά πολλοί λίγοι γνωρίζουν ότι το ίδιο αποτέλεσμα μπορούν να
το πετύχουν με έναν πολύ διαφορετικό τρόπο. Η θεωρητική κατανοήση των
διαφορετικών μορφών στα συστήματα διάδρασης είναι το κεντρικό θέμα σε
αυτό το βιβλίο, άρα θα ήταν σχεδόν αντιφατικό να χρησιμοποιήσουμε την
κυρίαρχη μορφή διάδρασης, ειδικά αφού έχει και τόσα μειονεκτήματα.

\leavevmode\vadjust pre{\hypertarget{fig:book-making}{}}%
\begin{figure}
\hypertarget{fig:book-making}{%
\centering
\includegraphics{images/book-making.jpg}
\caption{Εικόνα 1: Για την κατασκευή της μορφής του βιβλίου, είτε αυτή
είναι ηλεκτρονική, είτε είναι φυσική, μπορεί να χρησιμοποιηθεί ένας
εξομοιωτής τερματικού. Η ροή της εργασίας είναι παρόμοια με αυτήν της
δεκαετίας του 1970, με την διαφορά ότι σε μια μεγάλη οθόνη μπορεί να
γίνει πολυπλεξία πολλών τερματικών, όπου στο καθένα τρέχουν διαφορετικά
μικρά προγράμματα και εντολές δημιουργώντας έτσι ένα ολοκληρωμένο και
ταυτόχρονα δυναμικό περιβάλλον επεξεργασίας κειμένου και
σελιδοποίησης.}\label{fig:book-making}
}
\end{figure}

Η κατασκευή του βιβλίου μπορεί να θεωρηθεί παρόμοια με την κατασκευή
\{Εικόνα~1\} ενός συστήματος διάδρασης. Παραδοσιακά, η κατασκευή του
βιβλίου γίνεται από δύο διακριτές ομάδες, δηλαδή τη συγγραφική και την
εκδοτική ομάδα. Αντίστοιχα, η κατασκευή ενός συστήματος διάδρασης
συνήθως έχει δύο όψεις, τους προγραμματιστές και τους σχεδιαστές. Στην
κατασκευή συστημάτων διάδρασης είδαμε ότι η βέλτιστη πρακτική είναι να
έχουμε μια σύνθεση αυτών των δύο διαστάσεων που έχει σφαιρική κατανόηση
του αντικειμένου ή τουλάχιστον μια γεφύρωση της απόστασης ανάμεσά τους.
Αυτό ακριβώς το κεντρικό θεώρημα της κατασκευής της διάδρασης
εφαρμόζουμε και στην κατασκευή αυτού του βιβλίου. Ο συγγραφέας του
βιβλίου μπορεί να είναι ταυτόχρονα και εκδότης, αλλά, κυρίως,
αντιλαμβάνεται αυτήν την παραδοσιακά διακριτή διαδικασία ως μια σύνθεση,
όπου η συγγραφή συντελείται μαζί με την παραγωγή. Εκτός από μια πρακτική
εφαρμογή της θεωρίας του βιβλίου, αυτή η οπτική επιτρέπει και στον
αναγνώστη να γίνει συμμέτοχος. Ο αναγνώστης μπορεί να αντιγράψει, να
μελετήσει, να επεξεργαστεί και τελικά να κατανοήσει καλύτερα αυτό το
βιβλίο και κυρίως το πνεύμα του, μέσα από την διαδικασία της ίδιας της
κατασκευής του που είναι διαθέσιμη στο αποθετήριο
\url{https://github.com/pibook}

\hypertarget{ux3b2ux3b9ux3b2ux3bbux3b9ux3bfux3b3ux3c1ux3b1ux3c6ux3afux3b1}{%
\subsection*{Βιβλιογραφία}\label{ux3b2ux3b9ux3b2ux3bbux3b9ux3bfux3b3ux3c1ux3b1ux3c6ux3afux3b1}}
\addcontentsline{toc}{subsection}{Βιβλιογραφία}

\hypertarget{refs}{}
\begin{CSLReferences}{0}{0}
\end{CSLReferences}

Borsuk, Amaranth. 2018. \emph{The Book}. MIT Press.

\hypertarget{ux3c3ux3cdux3bdux3c4ux3bfux3bcux3bf-ux3b2ux3b9ux3bfux3b3ux3c1ux3b1ux3c6ux3b9ux3baux3cc}{%
\section{Σύντομο
βιογραφικό}\label{ux3c3ux3cdux3bdux3c4ux3bfux3bcux3bf-ux3b2ux3b9ux3bfux3b3ux3c1ux3b1ux3c6ux3b9ux3baux3cc}}

\hypertarget{ux3baux3c9ux3bdux3c3ux3c4ux3b1ux3bdux3c4ux3afux3bdux3bfux3c2-ux3c7ux3c9ux3c1ux3b9ux3b1ux3bdux3ccux3c0ux3bfux3c5ux3bbux3bfux3c2}{%
\subsection{Κωνσταντίνος
Χωριανόπουλος}\label{ux3baux3c9ux3bdux3c3ux3c4ux3b1ux3bdux3c4ux3afux3bdux3bfux3c2-ux3c7ux3c9ux3c1ux3b9ux3b1ux3bdux3ccux3c0ux3bfux3c5ux3bbux3bfux3c2}}

Ο συγγραφέας αυτού του βιβλίου έχει εργαστεί για περισσότερα από είκοσι
χρόνια στην έρευνα και ανάπτυξη συστημάτων διάδρασης μεταξύ ανθρώπου και
υπολογιστή. Στις αρχές της πρώτης δεκαετίας αυτού του αιώνα, έκανε
σημαντικές συνεισφορές για τον μετασχηματισμό της παθητικής κατανάλωσης
τηλεοπτικού περιεχομένου προς μια περισσότερο διαδραστική κατεύθυνση.
Μια δεκαετία αργότερα, διαπίστωσε, με έκπληξη, ότι η πλειονότητα των
χρηστών είχε στραφεί σε νέες ψηφιακές υπηρεσίες, οι οποίες θεωρούν τα
σύγχρονα επιτραπέζια και κινητά συστήματα σαν συσκευές κατανάλωσης,
παρόμοια με την παραδοσιακή τηλεόραση. Τα τελευταία χρόνια, εργάζεται
για την διάδοση συστημάτων και πρακτικών που υποστηρίζουν περισσότερο
υγιείς και βιώσιμες ανθρώπινες αξίες και βασίζονται σε ανθρωπιστικές και
διαφανείς τεχνολογίες.

\hypertarget{ux3bfux3c1ux3b9ux3c3ux3bcux3ccux3c2}{%
\section{Ορισμός}\label{ux3bfux3c1ux3b9ux3c3ux3bcux3ccux3c2}}

\begin{quote}
Ο προγραμματισμός είναι ένας τρόπος σκέψης, όχι μια μηχανιστική
δεξιότητα. Το να μάθεις τους βρόγχους for δεν σημαίνει πως μαθαίνεις να
προγραμματίζεις, όπως δεν σημαίνει πως μαθαίνεις να ζωγραφίζεις
μαθαίνοντας για τα μολύβια. Bret Victor
\end{quote}

\hypertarget{ux3c0ux3b5ux3c1ux3afux3bbux3b7ux3c8ux3b7}{%
\subsubsection{Περίληψη}\label{ux3c0ux3b5ux3c1ux3afux3bbux3b7ux3c8ux3b7}}

Η διάδραση του ανθρώπου με υπολογιστές έχει καθιερωθεί στις περισσότερες
ανθρώπινες δραστηριότητες, από την εργασία μέχρι την εκπαίδευση και τη
διασκέδαση. Ο προγραμματισμός της διάδρασης που απαιτείται για την
κατασκευή άρτιων συστημάτων είναι μια σύνθετη έννοια και μια διαδικασία
που προϋποθέτει δεξιότητες τόσο τεχνολογικές, όσο και ανθρωπιστικές. Για
παράδειγμα, δεν αρκεί ένας κατασκευαστής να είναι ικανός
προγραμματιστής, θα πρέπει να έχει και άριστη κατανόηση του ανθρώπινου
παράγοντα αλλά και της διαδικασίας σχεδίασης. Αν και υπάρχουν πολλοί
ικανοί κατασκευαστές συστημάτων, όπως και σχεδιαστές με γνώσεις
ανθρωπιστικών επιστημών, για να λύσουμε ένα πρόβλημα από δύο οπτικές
(μηχανή-άνθρωπος) απαιτείται μια σύνθεση των δύο οπτικών. Η ενότητα αυτή
ορίζει ποιος είναι αυτός ο γόνιμος συνδυασμός και γιατί είναι
απαραίτητος στον προγραμματισμό διαδραστικών συστημάτων.

\hypertarget{ux3b7-ux3b1ux3beux3afux3b1-ux3c4ux3b7ux3c2-ux3baux3b1ux3c4ux3b1ux3c3ux3baux3b5ux3c5ux3aeux3c2-ux3c3ux3c5ux3c3ux3c4ux3b7ux3bcux3acux3c4ux3c9ux3bd-ux3b4ux3b9ux3acux3b4ux3c1ux3b1ux3c3ux3b7ux3c2}{%
\subsection{Η αξία της κατασκευής συστημάτων
διάδρασης}\label{ux3b7-ux3b1ux3beux3afux3b1-ux3c4ux3b7ux3c2-ux3baux3b1ux3c4ux3b1ux3c3ux3baux3b5ux3c5ux3aeux3c2-ux3c3ux3c5ux3c3ux3c4ux3b7ux3bcux3acux3c4ux3c9ux3bd-ux3b4ux3b9ux3acux3b4ux3c1ux3b1ux3c3ux3b7ux3c2}}

Για κάθε σύστημα που δίνει ρόλο στον ανθρώπινο παράγοντα, θα πρέπει
πρώτα να τεκμηριώσουμε πώς ορίζουμε και πώς αντιλαμβανόμαστε τον
άνθρωπο. Η κατανόηση αυτή δεν μπορεί να είναι αντικειμενική, γιατί,
ανάλογα με τη γνωστική περιοχή, υπάρχει μια διαφορετική οπτική γωνία,
ενώ, ακόμη και μέσα στην ίδια γνωστική περιοχή, έχουμε διαφορετικές
σχολές. Για παράδειγμα, στην περιοχή της σύγχρονης δυτικής ιατρικής
δίνεται έμφαση στα επιμέρους ανθρώπινα όργανα, ενώ στις αντίστοιχες
παραδοσιακές ανατολικές, δίνεται έμφαση σε μια ολιστική αντίληψη. Ακόμη,
στην περιοχή της βιολογίας δίνεται έμφαση στις αντικειμενικές
διαδράσεις, οι οποίες έχουν τα κύταρα σε κλίμακα, τα οποία λίγο
διαφέρουν από τα αντίστοιχα συγγενικών βιολογικών οργανισμών. Από την
άλλη, στην ψυχολογία και στην κοινωνιολογία υπάρχουν σχολές που δίνουν
έμφαση στην ατομική και στη συλλογική συμπεριφορά, οι οποίες
διαμορφώνονται κυρίως από το εκπαιδευτικό και πολιτισμικό περιβάλλον των
ανθρώπων. Πράγματι, το αξιακό σύστημα ενός πολιτισμού καθορίζει στον
μεγαλύτερο βαθμό και τον ορισμό που έχουμε για τον άνθρωπο, ενώ
σημαντικό ρόλο παίζει και η οπτική γωνία των επιμέρους γνωστικών
περιοχών. Κατά τον ίδιο τρόπο, στην περιοχή της διάδρασης υπάρχουν
διαφορετικές σχολές, αλλά και διαφορετικές χρονικές περίοδοι με
διαφορετικά αξιακά συστήματα.

\leavevmode\vadjust pre{\hypertarget{fig:augmentation-typewriter}{}}%
\begin{figure}
\hypertarget{fig:augmentation-typewriter}{%
\centering
\includegraphics{images/augmentation-typewriter.jpg}
\caption{Εικόνα 1: Η επαύξηση της ανθρώπινης σκέψης με τη συνδρομή
κάποιου εργαλείου είναι μια πολύ δυνατή ιδέα, αλλά, ταυτόχρονα, είναι
κάπως δύσκολο να την μοιραστούμε γιατί δεν υπάρχουν πολλά πετυχημένα
παραδείγματα. Για αυτόν τον σκοπό, ο Douglas Engelbart συνήθιζε να κάνει
ένα πείραμα ως αναλογία, στο οποίο ζητούσε από τους συνομιλητές του να
γράψουν το ίδιο κείμενο με μια γραφομηχανή, με ένα στυλό, και, τέλος, με
ένα στυλό δεμένο με ένα τούβλο.}\label{fig:augmentation-typewriter}
}
\end{figure}

\leavevmode\vadjust pre{\hypertarget{fig:nls-cscw}{}}%
\begin{figure}
\hypertarget{fig:nls-cscw}{%
\centering
\includegraphics{images/nls-cscw.jpg}
\caption{Εικόνα 2: Ο επεξεργαστής κειμένου στο NLS δεν σχεδιάστηκε για
να γράφει κείμενο στα αρχεία ενός χρήστη, γιατί αυτό θα ήταν απλά μια
κακή προσομοίωση της γραφομηχανής. Ο σκοπός αυτού του συστήματος
διάδρασης είναι η συνεργασία πολλών χρηστών ταυτόχρονα πάνω σε ένα
επαυξημένο έγγραφο, το οποίο επιτρέπει, εκτός από την είσοδο νέων
δεδομένων, την πλοήγηση και την οπτικοποίηση διαφορετικών όψεων. Η
συνεργασία μπορεί να γίνει τόσο με το πληκτρολόγιο, όσο και με τις
χειρονομίες από τον δείκτη του ποντικιού, καθώς και με ταυτόχρονη
προβολή βίντεο από τον χρήστη, με στόχο την επαύξηση της νοημοσύνης,
μέσω της συνεργασίας ανθρώπων και υπολογιστών.}\label{fig:nls-cscw}
}
\end{figure}

Τα συστήματα διάδρασης έχουν ενσωματώσει διαχρονικά πολλές επιρροές από
διαφορετικές γνωστικές περιοχές, ενώ και το πολιτισμικό υπόβαθρο των
σημαντικότερων συντελεστών διαφέρουν σημαντικά σε κάθε περίοδο. Η αρχική
θεμελίωση της περιοχής της κατασκευής των συστημάτων διάδρασης έγινε την
δεκαετία του 1960, \footnote{Licklider (1960), Engelbart (1962)} με το
όραμα ενός συχνού και προχωρημένου χρήστη, ο οποίος συνεργάζεται με
άλλους χρήστες μέσω του συστήματος, \footnote{(\textbf{Εικόνα?})~1
  Επαύξηση της ταχύτητας συγγραφής κειμένου (Doug Engelbart Institute)}
\footnote{(\textbf{Εικόνα?})~2 Συνεργασία με το σύστημα NLS (Doug
  Engelbart Institute)} με στόχο την κατανόηση πολύπλοκων φυσικών ή
κοινωνικών φαινομένων. Εδώ, ο άνθρωπος είναι μια δυναμική οντότητα που
μαθαίνει συνέχεια και μετασχηματίζεται μέσα από τη συνεργασία και τη
διάδραση, ενώ και τα προβλήματα που αντιμετωπίζει δεν είναι σαφώς
καθορισμένα.

Μια δεκαετία αργότερα, η πιο συστηματική θεμελίωση έγινε από γνωστικούς
ψυχολόγους, \footnote{Card, Newell, and Moran (1983)} οι οποίοι
μοντελοποιούν τον άνθρωπο ως έναν επεξεργαστή πληροφορίας, ο οποίος
βρίσκεται μόνος του μπροστά σε ένα τερματικό και εκτελεί
επαναλαμβανόμενες απλές διεργασίες, οι οποίες δεν απαιτούν κάποια
δεξιότητα παρά μόνο μια σύντομη εκπαίδευση. Στην περίπτωση αυτή, ο
άνθρωπος είναι κάτι σταθερό, σαν ένα γρανάζι μιας μηχανής που εκτελεί
συνέχεια την ίδια διεργασία.

Το παράδειγμα της στατικής θεώρησης του ανθρώπου είναι αυτό που
επικράτησε, γιατί σίγουρα είναι το πιο εύκολο να μάθουν οι χρήστες και
ταυτόχρονα είναι το πιο αποδοτικό για όσους οργανώνουν τις διεργασίες
των χρηστών. Αν, δηλαδή, θεωρήσουμε τον άνθρωπο ως το στατικό εξάρτημα
μιας μηχανής που παραμένει πάντα το ίδιο και είναι σχετικά προβλέψιμο,
τότε το συνολικό σύστημα είναι πιο εύκολο να κατασκευαστεί. Ταυτόχρονα,
όμως, η στατική θεώρηση του ανθρώπινου παράγοντα δεν αφήνει πολλά
περιθώρια για την βελτίωση και την επαύξησή του προς μια νέα κατάσταση
με περισσότερες γνώσεις και δεξιότητες, από την οποία νέα κατάσταση θα
αλληλεπιδράσει με τα νέα μηχανήματα που θα κατασκευάσει.\footnote{Engelbart
  (1962)}

Έχοντας δώσει ένα πλαίσιο και ενδεικτικά παραδείγματα για τον ορισμό του
ανθρώπου, θα πρέπει να κάνουμε το ίδιο και για τον υπολογιστή. Για
παράδειγμα, η χρήση της λέξης υπολογιστής για την περιγραφή των
δημοφιλών επιτραπέζιων και κινητών συστημάτων είναι μια σχετικά πρόσφατη
σύμβαση. Μέχρι και την δεκαετία του 1950, η λέξη υπολογιστής αναφέρεται
με απόλυτη σαφήνεια σε μια ανθρώπινη δουλειά γραφείου, \footnote{(\textbf{Εικόνα?})~3
  Οι πρώτοι υπολογιστές ήταν άνθρωποι (NASA)} που αφορούσε τον
υπολογισμό λογαριθμικών πινάκων για χρήση στην πρόβλεψη της τροχιάς
πυραύλων και στη θαλάσσια πλοήγηση. Η εφεύρεση και η διάδοση των πρώτων
κεντρικών υπολογιστών λίγο αργότερα άλλαξε εντελώς το νόημα της λέξης,
αφού την ίδια ακριβώς δουλειά, δηλαδή τον υπολογισμό των λογαριθμικών
πινάκων, την έκαναν πλέον μηχανήματα.

\leavevmode\vadjust pre{\hypertarget{fig:human-computers}{}}%
\begin{figure}
\hypertarget{fig:human-computers}{%
\centering
\includegraphics{images/human-computers.jpg}
\caption{Εικόνα 3: Από τον 17ο αιώνα και μέχρι τη δεκαετία του 1950 ο
όρος υπολογιστής αναφέρεται σε έναν άνθρωπο που κάνει υπολογισμούς για
να ετοιμάσει τριγωνομετρικούς και λογαριθμικούς πίνακες, οι οποίοι
χρησιμοποιούνταν ευρέως στη ναυτική πλοήγηση. Με την εμφάνιση των πρώτων
ηλεκτρονικών και, λίγο αργότερα, ψηφιακών υπολογιστών, η χρήση του όρου
άλλαξε και αναφέρεται πλέον σε συσκευές και όχι σε
ανθρώπους.}\label{fig:human-computers}
}
\end{figure}

\leavevmode\vadjust pre{\hypertarget{fig:transistor-radio}{}}%
\begin{figure}
\hypertarget{fig:transistor-radio}{%
\centering
\includegraphics{images/transistor-radio.png}
\caption{Εικόνα 4: Το τρανζιστοράκι της δεκαετίας του 1950 ήταν συνώνυμο
με την πρώτη πραγματικά φορητή και δημοφιλή εκδοχή του ραδιοφώνου με
τρανζίστορ. Ταυτόχρονα ήταν και η πρώτη δημοφιλής εφαρμογή της εφεύρεσης
του τρανζίστορ, το οποίο στη συνέχεια θα μπει σε πολλές συσκευές που
μέχρι τότε δούλευαν με λυχνίες, όπως ο υπολογιστής. Εκτός από τη χρήση
τρανζίστορ, αυτό που είναι κοινό και στις δύο αυτές περιπτώσεις
(τρανζιστοράκι, υπολογιστής) είναι πως η ονομασία τους γίνεται με βάση
την εσωτερική τους λειτουργία και όχι με βάση τη χρήση τους, η οποία
στην περίπτωση του υπολογιστή είναι ένα ανοιχτό
θέμα.}\label{fig:transistor-radio}
}
\end{figure}

Την ίδια ακριβώς περίοδο, η εφεύρεση του τρανζίστορ επέτρεψε την
κατασκευή μικρών φορητών ραδιοφώνων, τα οποία έμειναν γνωστά για μια
γενιά ανθρώπων ως τρανζιστοράκια. \footnote{(\textbf{Εικόνα?})~4
  Ραδιόφωνο με τρανζίστορ (Bullock Museum)} Με το ίδιο σκεπτικό, θα
μπορούσαμε να ονοματίσουμε τους λεγόμενους κινητούς και φορετούς
υπολογιστές επίσης τρανζιστοράκια, αφού είναι γεμάτοι από αυτά.
Ταυτόχρονα, η ονομασία έξυπνο τηλέφωνο και η συσχέτισή της με τις
δημοφιλείς συσκευές Android και iOS αποτελεί μια ακόμη χειρότερη
παρανόηση για μια κατηγορία προϊόντων, τα οποία οι κατασκευαστές τους
θέλουν να θεωρούνται έξυπνα. Επομένως, το γεγονός ότι ένα κατασκεύασμα
χρησιμοποιεί κάποιο υλικό ή κάποιο λογισμικό για να κάνει μία λειτουργία
δεν μπορεί να καθορίζει την ονομασία του, ούτε φυσικά μπορούμε να
βασιζόμαστε σε χαρακτηρισμούς που δίνει ο κατασκευαστής. \footnote{Lanier
  (2014)}

Έχοντας περιγράψει ένα πλαίσιο επιμέρους ορισμών για τον άνθρωπο και τον
υπολογιστή, μπορούμε να περάσουμε στο κεντρικό μας θέμα που είναι η
διάδρασή τους. Μια από τις αρχικές και πολύ δημοφιλείς θεωρήσεις της
διάδρασης είναι αυτή του αυτοματισμού ή της κατασκευής έξυπνων
συστημάτων. Ο στόχος, δηλαδή, της κατασκευής της διάδρασης είναι να μην
έχουμε καθόλου διάδραση ή, αν η διάδραση είναι απαραίτητη, αυτή να
γίνεται με φυσικούς τρόπους, όπως είναι η ομιλία. Η περιοχή της Τεχνητής
Νοημοσύνης εστιάζει πάνω σε αυτόν τον στόχο και μετά από πενήντα χρόνια
έχει σημειώσει πρόοδο, αλλά με αρκετούς περιορισμούς, καθώς και με
παραδοχές που δεν είναι πάντα εποικοδομητικές για τις δυνατότητες του
ανθρώπου. \footnote{Engelbart (1962), Weizenbaum (1976)} Αντίθετα, η
σχολή της επαυξημένης νοημοσύνης θεωρεί πως η κατασκευή συστημάτων
διάδρασης μπορεί να γίνει με τέτοιο τρόπο ώστε όχι μόνο να βελτιώνουν
τις επιδόσεις μας σε μια διεργασία αλλά και να αυξάνουν τη συλλογική
νοημοσύνη. Για αυτόν τον σκοπό, η διάδραση με έναν επεξεργαστή κειμένου
είναι κάτι περισσότερο. Για παράδειγμα, στο σύστημα NLS, η επεξεργασία
κειμένου γίνεται με ένα εξειδικευμένο πληκτρολόγιο ακόρντων, το οποίο
αυξάνει την ταχύτητα και την εργονομία. Ανάμεσα σε πολλές άλλες πρωτιές,
το σύστημα NLS επιτρέπει τη συνεργασία μεταξύ χρηστών που βρίσκονται σε
διαφορετικά τερματικά σε πραγματικό χρόνο.\footnote{Licklider (1960)}
Επίσης, γίνεται συνεργατικά με μια δομημένη μορφή κειμένου, η οποία
βασίζεται στο υπερκείμενο και στο ιστορικό των εγγράφων, έτσι ώστε να
μην υπάρχουν αμφισημίες και ασάφειες.

Το πιο χαρακτηριστικό παράδειγμα διάδρασης είναι αυτό του επιτραπέζιου
υπολογιστή και της γραφικής επιφάνειας εργασίας που ελέγχεται από τον
χρήστη με συσκευές εισόδου όπως το πληκτρολόγιο και το ποντίκι.
\footnote{Freiberger and Swaine (1984), Hiltzik (1999), Hertzfeld (2004)}
Η επιτραπέζια μορφή υπολογιστή και διάδρασης είναι σημαντική, γιατί ήταν
η πρώτη που ξέφυγε από τις μέχρι τότε πολύ εξειδικευμένες εφαρμογές,
όπως είναι οι βάσεις δεδομένων. Έτσι, μπόρεσε να διευκολύνει τις
εργασίες και την καθημερινότητα πάρα πολλών χρηστών με την επεξεργασία
κειμένου, την ανάκτηση πληροφορίας από τον ιστό και την επικοινωνία μέσω
του ηλεκτρονικού ταχυδρομείου. Αν και το μοντέλο διάδρασης με τον
επιτραπέζιο υπολογιστή δεν είναι ούτε τόσο δημοφιλές ούτε τόσο εύκολο
όσο αυτό του κινητού υπολογισμού με τα έξυπνα κινητά και τις ταμπλέτες,
έχει ιδιαίτερο ενδιαφέρον, γιατί έχει μείνει σχετικά ίδιο από τότε που
δημιουργήθηκε, πράγμα που επιβεβαιώνει, ότι ανεξάρτητα από τη ραγδαία
τεχνολογική εξέλιξη, η ανθρώπινη διάδραση κινείται σε πιο αργούς
ρυθμούς.\footnote{Waldrop (2001)} Σε αυτό το πλαίσιο, η ευχρηστία και η
εμπειρία του χρήστη αποτελούν τους κεντρικούς ορισμούς και έχουν
καθορίσει τον τρόπο με τον οποίο επεξεργαζόμαστε έγγραφα. \footnote{(\textbf{Εικόνα?})~5
  Επεξεργαστής κειμένου Bravo (Nathan Lineback)} \footnote{(\textbf{Εικόνα?})~6
  Επεξεργαστής κειμένου WordStar (Wikimedia)}

\leavevmode\vadjust pre{\hypertarget{fig:xerox-bravo}{}}%
\begin{figure}
\hypertarget{fig:xerox-bravo}{%
\centering
\includegraphics{images/xerox-bravo.png}
\caption{Εικόνα 5: Η εφαρμογή Bravo ήταν ο πρώτος επεξεργαστής κειμένου
με οπτική απεικόνιση στις αρχές της δεκαετίας του 1970 και λειτουργούσε
στον υπολογιστή Xerox Alto, ο οποίος είχε οθόνη σε θέση πορτρέτου. Η
είσοδος από την πλευρά του χρήστη ήταν τροπική, κάτι που βελτιώθηκε στον
διάδοχο του, τον Gypsy.}\label{fig:xerox-bravo}
}
\end{figure}

\leavevmode\vadjust pre{\hypertarget{fig:wordstar-editor}{}}%
\begin{figure}
\hypertarget{fig:wordstar-editor}{%
\centering
\includegraphics{images/wordstar-editor.png}
\caption{Εικόνα 6: Οι πρώτοι επεξεργαστές κειμένου για τους
μικρο-υπολογιστές της δεκαετίας του 1980 εμπνέονται από τον Xerox Bravo
και προσπαθούν να αποδώσουν, σε μια οθόνη απλού κειμένου χωρίς πολλαπλές
γραμματοσειρές, την τελική εμφάνιση και ροή του κειμένου στην τυπωμένη
σελίδα. Το αποτέλεσμα είναι ένας συνδυασμός από τα μειονεκτήματα των δύο
κόσμων, χωρίς κανένα πλεονέκτημα, αλλά αυτό δεν θα εμποδίσει το WordStar
και τους ανταγωνιστές του να κυριαρχίσουν στην
αγορά.}\label{fig:wordstar-editor}
}
\end{figure}

Αν και η ευχρηστία είναι μία από τις βασικές αξίες της διάδρασης, δεν
είναι η μόνη, ενώ η σημασία της μπορεί να είναι πολύ μικρή σε ορισμένες
περιπτώσεις. Αν, για παράδειγμα, το σύστημα διάδρασης έχει εφαρμογή στη
διασκέδαση, τότε θέλουμε η διάδραση να προσφέρει, εκτός από απλή
ευχρηστία, ψυχαγωγία και μάθηση. Στην περίπτωση της βελτίωσης ενός
συστήματος που είτε προϋπάρχει είτε είναι παρόμοιο με υπάρχοντα
συστήματα, ο πιο απλός τρόπος από πλευράς κόστους, αποτελεσματικότητας
και ταχύτητας είναι να βασιστούμε σε επιτυχημένα ιστορικά παραδείγματα.
Πράγματι, η αντιλαμβανόμενη ευχρηστία σχετίζεται περισσότερο με την
οικειότητα του χρήστη με ένα σύστημα,\footnote{Raskin (2000)} παρά με
την αντικειμενική επίδοσή του μετά από κάποια μικρή περίοδο εξάσκησης.
Για παράδειγμα, η εκμάθηση του πληκτρολογίου ακόρντων ή μιας τροπικής
διεπαφής, απαιτεί μερικές ώρες εκπαίδευσης, αλλά έχει διαχρονικό κέρδος
τόσο για την απόδοση όσο και για την εργονομία κατά την επεξεργασία
κειμένου. Παρομοίως, η διακόσμηση της διεπαφής με εικονίδια και
περίτεχνα περιγράμματα που αλληλοκαλύπτονται παρέχει μόνο υποκειμενική
αισθητική οικειότητα, ενώ δεν έχει καμία αντικειμενική απόδοση, ειδικά
για τον συχνό χρήστη.

Το γεγονός αυτό, από μόνο του, αποτελεί τη σημαντικότερη γνώση σε αυτήν
την περιοχή και ταυτόχρονα δείχνει ότι με αυτήν την νοοτροπία η εφικτή
καινοτομία μπορεί να παρουσιαστεί μόνο σταδιακά και ποτέ ως μετάβαση
παραδείγματος. Η κατανόηση της καινοτομίας περισσότερο ως γραμμικής
βελτίωσης παρά ως σημαντικής μετατόπισης παραδείγματος αποτελεί για την
περιοχή της διάδρασης μία από τις σημαντικότερες αντιφάσεις, η οποία
θεμελειώθηκε στα τέλη της δεκαετίας του 1970 κατά τη μετάβαση από το
Xerox PARC στην Apple. Πράγματι, η Apple, με το κίνητρο της διάδοσης των
προσωπικών συστημάτων διάδρασης σε όσο γίνεται μεγαλύτερο πληθυσμό,
δημιουργεί με μεγάλη ακρίβεια συστήματα τα οποία είναι πολύ εύκολα για
τον περιστασιακό χρήστη. Στη συνέχεια μοχλεύει την οικειότητα του χρήστη
με μια διεπαφή ώστε να δημιουργήσει την επόμενη έκδοση, όπου το κίνητρο
παραμένει η οικειότητα και όχι η βέλτιστη διεπαφή, ειδικά αν ο χρήστης
είναι συχνός. Φυσικά, τα κυρίαρχα συστήματα έχουν σημαντικές αρετές,
εκτός από την υποκειμενική ευχρηστία τους για τον άπειρο και
περιστασιακό χρήστη και μας παρέχουν ένα πλαίσιο κατανόησης της
κατασκευής συστημάτων διάδρασης, έτσι ώστε να μπορέσουμε να φτιάξουμε
εναλλακτικά, με τις ίδιες τεχνικές και τις ίδιες τεχνολογίες αλλά με
διαφορετική φιλοσοφία, κίνητρα και κατεύθυνση.

\hypertarget{ux3baux3b1ux3c4ux3b1ux3c3ux3baux3b5ux3c5ux3ae-ux3c5ux3c0ux3bfux3b4ux3b5ux3afux3b3ux3bcux3b1ux3c4ux3bfux3c2-ux3baux3b1ux3b9-ux3b5ux3c0ux3b1ux3bdux3acux3bbux3b7ux3c8ux3b7}{%
\subsection{Κατασκευή υποδείγματος και
επανάληψη}\label{ux3baux3b1ux3c4ux3b1ux3c3ux3baux3b5ux3c5ux3ae-ux3c5ux3c0ux3bfux3b4ux3b5ux3afux3b3ux3bcux3b1ux3c4ux3bfux3c2-ux3baux3b1ux3b9-ux3b5ux3c0ux3b1ux3bdux3acux3bbux3b7ux3c8ux3b7}}

Η διαδικασία κατασκευής ενός λειτουργικού υποδείγματος είναι χρήσιμη ως
μηχανισμός κατανόησης της διάδρασης που θέλουμε να υλοποιήσουμε, ενώ
όταν το υπόδειγμα είναι σε μια πρώτη ικανοποιητική μορφή τότε μπορεί να
χρησιμοποιηθεί για δοκιμές με τους τελικούς χρήστες ή ακόμη και να
δημοσιευτεί ως αρχική έκδοση (π.χ. έκδοση άλφα, βήτα, από την ορολογία
της τεχνολογίας λογισμικού). Αυτή η προσέγγιση είναι γνωστή στη
βιβλιογραφία και ως \emph{το υπόδειγμα ως προδιαγραφές}. Δηλαδή, αντί να
ετοιμάσουμε ένα λεπτομερές συμβόλαιο το οποίο θα περιγράφει με λέξεις
και διαγράμματα το αποτέλεσμα, έχουμε το ίδιο το αποτέλεσμα (εν τη
γενέσει του) να δηλώνει τις ίδιες τις προδιαγραφές του. Αυτό είναι μια
σχετικά απλή ιδέα, η οποία όμως φέρνει σε μεγάλη αντίθεση την περιοχή
της κατασκευής συστημάτων διάδρασης με τις συγγενείς περιοχές της
επιστήμης των μηχανικών, ακόμη και με τη γονική περιοχή της τεχνολογίας
λογισμικού. Επίσης, αποδεικνύει την κεντρική θέση αυτού του βιβλίου, ότι
η κατασκευή συστημάτων διάδρασης είναι μια νέα περιοχή που, ναι μεν έχει
αρκετές ομοιότητες με άλλες, αλλά τελικά έχει τόσες διαφορές απαιτώντας
ουσιαστικά διαφορετική αντιμετώπιση.

\leavevmode\vadjust pre{\hypertarget{fig:sage-lightgun}{}}%
\begin{figure}
\hypertarget{fig:sage-lightgun}{%
\centering
\includegraphics{images/sage-lightgun.jpg}
\caption{Εικόνα 7: Ένας από τους μεγαλύτερους σε μέγεθος και πολύ
ισχυρός υπολογιστής της δεκαετίας του 1950 ήταν ο SAGE (Semi-Automatic
Ground Environment), ο οποίος συγκέντρωνε τα δεδομένα πτήσεων από ένα
δίκτυο ραντάρ και τα πρόβαλε στην οθόνη μαζί με απλά γραφικά χάρτη. Ο
χρήστης μπορούσε να επιλέξει ένα σημείο στο ραντάρ με τη βοήθεια μια
αρχικής μορφής πένας για να ταυτοποιήσει το αντίστοιχο
αντικείμενο.}\label{fig:sage-lightgun}
}
\end{figure}

\leavevmode\vadjust pre{\hypertarget{fig:sketchpad-interaction}{}}%
\begin{figure}
\hypertarget{fig:sketchpad-interaction}{%
\centering
\includegraphics{images/sketchpad-interaction.png}
\caption{Εικόνα 8: Το σύστημα Sketchpad θεωρείται το πρώτο σημαντικό
σύστημα αλληλεπίδρασης ανθρώπου και υπολογιστή με χρήση γραφικών. Οι
κεντρικοί υπολογιστές εκείνης της εποχής συνήθως βασίζονταν σε διάτρητες
κάρτες ή σε τερματικά γραμμής. Με αυτόν τον τρόπο, το Sketchpad έδειξε
ότι είναι εφικτό να έχουμε έναν υπολογιστή, ο οποίος να μην αλληλεπιδρά
με τον χρήστη με τον τρόπο που που έχει προγραμμιστεί ο ίδιος ο
υπολογιστής, αφού μέχρι τότε η διάδραση γινόταν ασύγχρονα με εργασίες
δέσμης σε διάτρητες κάρτες.}\label{fig:sketchpad-interaction}
}
\end{figure}

Αν και υπάρχουν πάρα πολλές τεχνικές και μεθοδολογίες και ακόμη
περισσότερα εργαλεία και δομές για την κατασκευή της διάδρασης, αν
έπρεπε να τα συνοψίσουμε όλα σε μία πρόταση, θα λέγαμε ότι \emph{η
κατασκευή της διάδρασης είναι η επαναληπτική κατασκευή ενός
υποδείγματος.} Αυτή η επανατοποθέτηση του προβλήματος της κατασκευής μας
επιτρέπει να στρέψουμε την προσοχή μας στη φύση και στον ρόλο του
υποδείγματος ως ένα είδος ζωντανών και ευμετάβλητων προδιαγραφών ενός
νέου συστήματος. Η κατασκευή ενός νέου συστήματος διάδρασης, όσο νέα και
αν είναι, συνήθως βασίζεται σε κάποια υλικά που είναι ήδη διαθέσιμα. Για
παράδειγμα, ο μετασχηματισμός της διάδρασης πέρα από τον τηλέτυπο
ξεκίνησε με νέες συσκευές εισόδου, όπως η πένα, καθώς και με τη χρήση
της οθόνης από ραντάρ για την απεικόνηση γραφικών. \footnote{(\textbf{Εικόνα?})~7
  SAGE Lightgun (MIT)} Το σύστημα SAGE δημιουργήθηκε στο MIT και οι
συσκευές διάδρασης ήταν οικείες. Φυσικά απαιτήθηκε πολυετής εργασία και
ένας νέος ψηφιακός υπολογιστής για να μπορέσουν οι ίδιες συσκευές
διάδρασης να παίξουν έναν νέο ρόλο στο σύστημα Sketchpad. \footnote{(\textbf{Εικόνα?})~8
  Το σύστημα Sketchpad ως μετασχηματισμός της αλληλεπίδρασης (History
  Computer)} Αρκεί μια οπτική σύγκριση των δύο συστημάτων για να δούμε
ότι τα αρχέτυπα για τις βασικές συσκευές εισόδου και εξόδου με τον
χρήστη είναι οι ίδιες ακριβώς και στα δύο συστήματα. Φυσικά διαφέρει
πολύ ο κεντρικός υπολογιστής και κυρίως διαφέρει η κατασκευή της
διάδρασης, αφού το Sketchpad δεν είναι απλά μια αναπαράσταση της
πραγματικότητας, αλλά ένας συνεργάτης που διευκολύνει τη σχεδίαση
μηχανολογικών συστημάτων.

Το υπόδειγμα θα πρέπει να είναι διαδραστικό, διαφορετικά είναι σκόπιμο
να το χαρακτηρίσουμε ως ένα αρχικό προσχέδιο. Τα προσχέδια και τα
υποδείγματα κάθε άλλο παρά νέα είναι στην περιοχή των μηχανικών. Οι
αρχιτέκτονες μηχανικοί ξεκινάνε τη σχεδίαση στο χαρτί, γιατί αυτός είναι
παραδοσιακά ο πιο γρήγορος τρόπος αναπαράστασης μιας ιδέας και γιατί
αυτό βοηθάει τη σχεδιαστική σκέψη. Αντίστοιχα, η κατασκευή της διάδρασης
είναι σκόπιμο να ξεκινήσει από ένα σύντομο αφηγηματικό σενάριο, το οποίο
θα συνοδεύεται από μερικές ενδεικτικές οθόνες. Παρά τις ομοιότητες με
τους αρχιτέκτονες μηχανικούς, σημαντικές διαφορές στη μέθοδο προκύπτουν
επειδή το αποτέλεσμα του προγραμματισμού της διάδρασης δεν είναι κάτι
στέρεο και σταθερό, αλλά κάτι πολύ ρευστό, ευμετάβλητο και φευγαλέο, το
οποίο αλλάζει ανάλογα με τη χρήση. Επιπλέον, τόσο τα προσχέδια όσο και
τα πρωτότυπα θα πρέπει να αντικατοπτρίζουν την κίνηση που υπάρχει εκ των
πραγμάτων σε μια διάδραση, κάτι που δύσκολα γίνεται στο χαρτί ή με απλές
εικόνες. Για τον λόγο αυτό, στα προσχέδια του προγραμματισμού της
διάδρασης θέλουμε να δημιουργήσουμε τουλάχιστον ένα λειτουργικό
υπόδειγμα, \footnote{(\textbf{Εικόνα?})~9 Λειτουργικό υπόδειγμα για
  έξυπνο ρολόϊ (Eric Migicovsky)} \footnote{(\textbf{Εικόνα?})~10
  Λειτουργικό πρωτότυπο για το Handspring (Copyright 2002 by the
  Association for Computing Machinery, Inc)} ώστε να μπορεί ο σχεδιαστής
και οι άλλοι συμμετέχοντες της ομάδας όχι μόνο να φανταστούν το τελικό
προϊόν, αλλά και να το χρησιμοποιήσουν.

\leavevmode\vadjust pre{\hypertarget{fig:pebble-hifi}{}}%
\begin{figure}
\hypertarget{fig:pebble-hifi}{%
\centering
\includegraphics{images/pebble-hifi.png}
\caption{Εικόνα 9: Το Arduino στη βασική του μορφή χρησιμοποιείται κατά
το στάδιο της ανάπτυξης λειτουργικών υποδειγμάτων υψηλής πιστότητας για
νέα συστήματα διάδρασης, όπως το έξυπνο ρολόϊ pebble, αφού επιτρέπει τον
γρήγορο έλεγχο, ενώ διευκολύνει και τη μετάβαση στην παραγωγή, καθώς το
κύκλωμά του είναι ελεύθερα διαθέσιμο για χρήση και
μετατροπή.}\label{fig:pebble-hifi}
}
\end{figure}

\leavevmode\vadjust pre{\hypertarget{fig:handspring-buck}{}}%
\begin{figure}
\hypertarget{fig:handspring-buck}{%
\centering
\includegraphics{images/handspring-buck.jpg}
\caption{Εικόνα 10: Στην φάση της μετάβασης από τα αρχικά προσχέδια σε
λειτουργικά πρωτότυπα υπάρχει πολύ μεγάλη ασάφεια αναφορικά με το
λογισμικό και υλικό ειδικά στις περιπτώσεις που έχουμε νέες συσκευές,
όπως ένα έξυπνο κινητό ή άλλες συσκευές διάχυτου υπολογισμού. Το
πρωτότυπο υψηλής πιστότητας τύπου Buck γερυφώνει αυτήν την μετάβαση με
την δημιουργική επαναχρησιμοποίηση υλικού και λογισμικού που ήδη
υπάρχει, ακόμη και αν αυτά δεν θα είναι ίδια στο τελικό προϊόν, αρκεί να
είναι αντιπροσωπευτικά της διάδρασης.}\label{fig:handspring-buck}
}
\end{figure}

Οι διαφορές από τους αρχιτέκτονες μηχανικούς στην κατασκευή του
πρωτοτύπου συνεχίζονται στην περίπτωση της μακέτας. Ενώ η μακέτα είναι
για τους αρχιτέκτονες ένα προχωρημένο πρωτότυπο το οποίο αναπαριστά υπό
κλίμακα σε τρεις διαστάσεις το μελλοντικό προϊόν, στον προγραμματισμό
της διάδρασης, ένα διαδραστικό πρωτότυπο είναι σχεδόν το ίδιο με το
τελικό προϊόν. Η σημαντικότερη όμως διαφορά σε σχέση με τους
αρχιτέκτονες και τις άλλες συγγενείς επιστήμες του μηχανικού είναι ότι
ένα διαδραστικό πρωτότυπο και, φυσικά, το τελικό προϊόν δεν ακολουθούν
καθόλου διακριτά στάδια κατά τις φάσεις της σχεδίασης, της παραγωγής και
της βελτίωσης. Για παράδειγμα, η πρώτη εμπορική έκδοση του δημοφιλούς
Apple iPhone δεν είχε εφαρμογές άλλων κατασκευαστών λογισμικού, παρά
μόνο τις επίσημες εφαρμογές της εταιρείας. Ήταν αυτό το \emph{τελικό
προϊόν} ή μήπως ήταν ένα πολύ \emph{προχωρημένο πρωτότυπο}; Μπορεί από
την πλευρά του υλικού η συσκευή να βελτιώθηκε σταδιακά, όμως από την
πλευρά του λογισμικού, η νέα δυνατότητα του συστήματος να δέχεται
πρόσθετες εφαρμογές δημιούργησε ουσιαστικά ένα καινούργιο προϊόν.
Επομένως, θα μπορούσαμε να χαρακτηρίσουμε το πρώτο εμπορικό iPhone ως
ένα προχωρημένο διαδραστικό πρωτότυπο των σύγχρονων iPhone, τα οποία δεν
έχουν πάψει να εξελίσσονται. Σε συνδυασμό με το λογισμικό που διατίθεται
από την ίδια την εταιρεία, στο οποίο συχνά συνεισφέρουν ανεξάρτητοι
προγραμματιστές, νέο, εξωτερικό υλικό προστίθεται και διευκολύνει
σημαντικές ανθρώπινες δραστηριότητες οι οποίες έχουν να κάνουν με τις
συναλλαγές, την υγεία, τη δημιουργία και τη διασκέδαση, παράγοντας
ουσιαστικά ένα οικοσύστημα διάδρασης.

Οι βασικές τεχνολογίες και ο αντίστοιχος προγραμματισμός της δικτύωσης,
της αποθήκευσης και της επεξεργασίας δεδομένων και, κυρίως, της εισόδου
και εξόδου της διεπαφής με τον άνθρωπο είναι δομικά στοιχεία του
συστήματος. Επομένως, θέλουμε άμεση και εύκολη πρόσβαση σε όλα αυτά
μαζί, χωρίς να πρέπει να ανησυχούμε για τις λεπτομέρειες της υλοποίησης.
Αν και οι λεπτομέρειες της υλοποίησης έχουν μεγάλη σημασία όταν το
σύστημά μας αφορά τις ζωές πολλών ανθρώπων, σε αυτήν τη φάση της
ανάπτυξης (κατά την οποία δεχόμαστε ότι δεν ξέρουμε τι ακριβώς
ετοιμάζουμε, ούτε το πώς αυτό θα επηρεάσει την καθημερινότητα των
ανθρώπων) είναι σκόπιμο να μην ασχοληθούμε με αυτές. Με αυτό το
δεδομένο, η επιλογή των εργαλείων ανάπτυξης (ειδικά της γλώσσας
προγραμματισμού και των βιβλιοθηκών) απλουστεύεται, αλλά σε καμία
περίπτωση δεν μπορεί να χαρακτηριστεί εύκολη, πράγμα που θα δούμε στο
αντίστοιχο κεφάλαιο των εργαλείων της κατασκευής της διάδρασης.

Με δεδομένη την ανάγκη ανάπτυξης δεξιοτήτων που θεμελιώνουν τον ψηφιακό
αλφαβητισμό πέρα από την απλή χρήση, προς τη βαθύτερη κατανόηση
λειτουργίας και ιδανικά τη δημιουργία νέων διαδράσεων, ένα ερώτημα που
προκύπτει αφορά στην επιλογή του εργαλείου, την οργάνωση, αλλά και τη
διαδικασία δημιουργίας της διάδρασης. Για να απαντήσουμε σε αυτό το
ερώτημα θα πρέπει να ανατρέξουμε στη φύση της κατασκευής της διάδρασης.
Το βασικό στοιχείο αυτής της περιοχής είναι ότι οι τελικές προδιαγραφές
του συστήματος είναι άγνωστες κατά το αρχικό στάδιο, ενώ είναι σίγουρο
ότι ακόμη και αν έχουμε τις πρώτες εκδόσεις σε λειτουργικό επίπεδο, οι
προδιαγραφές θα συνεχίζουν να προσαρμόζονται με τη χρήση και τη
διαδικασία της επαναληπτικής αξιολόγησης. Επομένως, τα κατάλληλα
εργαλεία, οι διαδικασίες, και οι δομές θα πρέπει να μπορούν να αλλάζουν
γρήγορα, τόσο τα ίδια όσο και τα δημιουργήματά τους. Επομένως, τα
καταλληλότερα εργαλεία δεν είναι τα έτοιμα, αλλά είναι αυτά που θα
κατασκευάσουμε και θα προσαρμόσουμε έτσι ώστε να ταιριάζουν στο πεδίο
του προβληματος.

\hypertarget{ux3b7-ux3b4ux3b9ux3acux3b4ux3c1ux3b1ux3c3ux3b7-ux3c3ux3b5-ux3bcux3b5ux3b3ux3b1ux3bbux3cdux3c4ux3b5ux3c1ux3b7-ux3baux3bbux3afux3bcux3b1ux3baux3b1}{%
\subsection{Η διάδραση σε μεγαλύτερη
κλίμακα}\label{ux3b7-ux3b4ux3b9ux3acux3b4ux3c1ux3b1ux3c3ux3b7-ux3c3ux3b5-ux3bcux3b5ux3b3ux3b1ux3bbux3cdux3c4ux3b5ux3c1ux3b7-ux3baux3bbux3afux3bcux3b1ux3baux3b1}}

\leavevmode\vadjust pre{\hypertarget{fig:logo-robot}{}}%
\begin{figure}
\hypertarget{fig:logo-robot}{%
\centering
\includegraphics{images/logo-robot.jpg}
\caption{Εικόνα 11: Η πρώτη εφαρμογή των υπολογιστών στη βασική
εκπαίδευση εντοπίζεται σχεδόν παράλληλα με τα πρώτα γραφικά περιβάλλοντα
διάδρασης. Ο Seymour Papert σχεδίασε ένα σύστημα το οποίο επιτρέπει
ακόμη και σε παιδιά προσχολικής ηλικίας να εξερευνήσουν σημαντικές
έννοιες από τα μαθηματικά, τις φυσικές επιστήμες και τη μηχανική, χωρίς
να απαιτεί γνώσεις από τα ανώτερα γνωστικά και συμβολικά επίπεδα αυτών
των περιοχών. Σταδιακά, το ρομπότ συμπληρώθηκε με την οθόνη των
γραφικών, η οποία έδωσε πρόσβαση σε περισσότερα σχολεία και μαθητές,
καθώς το κόστος μειώθηκε.}\label{fig:logo-robot}
}
\end{figure}

\leavevmode\vadjust pre{\hypertarget{fig:children-alto}{}}%
\begin{figure}
\hypertarget{fig:children-alto}{%
\centering
\includegraphics{images/children-alto.png}
\caption{Εικόνα 12: To Xerox Alto ήταν ένα ενδιάμεσο πρωτότυπο για το
Dynabook, το οποίο απευθύνεται σε παιδιά, γι' αυτό και οι χρήστες στις
πρώτες δοκιμές ήταν πολύ συχνά παιδιά από το δημοτικό. Με αυτόν τον
τρόπο, το σύστημα διάδρασης με ποντίκι και με γραφικό περιβάλλον το
οποίο αναπτύχθηκε εκεί απευθύνεται κυρίως σε χρήστες μικρότερης ηλικίας.
Ταυτόχρονα, το σύστημα αυτό δεν περιλαμβάνει λειτουργικό σύστημα ή
αρχεία και προτρέπει τους μικρούς χρήστες να αναπτύξουν μαζί με την
καθοδήγηση του δασκάλου τις εφαρμογές που τους
ενδιαφέρουν.}\label{fig:children-alto}
}
\end{figure}

Η κατασκευή της διάδρασης έχει παραμείνει μια φευγαλέα περιοχή, επειδή
σε κάθε χρονική περίοδο έχουμε διαφορετικές μορφές υπολογιστών (π.χ.
επιτραπέζιος, κινητός, φορετός, διάχυτος) και διεπαφών με τους χρήστες
(π.χ. γραμμή εντολών, γραφικό περιβάλλον, χειρονομίες, φυσική γλώσσα).
Για παράδειγμα, ένας χρήστης υπολογιστών ο οποίος έλαβε βασική,
δευτεροβάθμια και τριτοβάθμια εκπαίδευση τη δεκαετία του 1970 ή το πολύ
μέχρι τα μισά της δεκαετίας του 1980, είναι πολύ πιθανό να έχει μεγάλη
εξοικείωση με τη γραμμή εντολών και τους επιτραπέζιους υπολογιστές, αφού
αυτή ήταν η βασική μορφή στα χρόνια της εκπαίδευσής του. Αντίστοιχα,
ένας χρήστης που έλαβε την εκπαίδευσή του μετά το 2000 και κατά τη
δεκαετία του 2010, είναι πολύ πιθανό να μην έχει καθόλου προσωπικό
επιτραπέζιο υπολογιστή, αφού οι βασικές διεργασίες του χρήστη αυτήν τη
χρονική περίοδο (π.χ. αναζήτηση στον παγκόσμιο ιστό, κοινωνική δικτύωση,
ψηφιακό περιεχόμενο, κτλ.) μπορούν να γίνουν εξίσου καλά, αν όχι
καλύτερα, με έναν κινητό υπολογιστή. Βλέπουμε, λοιπόν, ότι, στην πράξη,
ο ψηφιακός αλφαβητισμός ως βασική δεξιότητα πρόσβασης στην πληροφορία
είναι μια έννοια περισσότερο σχετική με τη δημογραφία και την ημερομηνία
γέννησης, παρά μια διαχρονική αξία.

Ο διαδραστικός τρόπος σκέψης είναι στενά συνδεδεμένος με το εκάστοτε
νόημα που αποδίδουμε στον ψηφιακό αλφαβητισμό. Η αρχική κατασκευή και
αρχική θεώρηση των συστημάτων διάδρασης στόχευε σε έναν προχωρημένο και
συχνό χρήστη, για τον οποίο η ευχρηστία ταυτίζεται κυρίως με τη βελτίωση
της εργασίας αλλά και με την εργονομία, αφού θα πρέπει να εργάζεται για
πολλές ώρες. Λίγο αργότερα, οι ερευνητές προσπάθησαν να διευκολύνουν την
εξέλιξη των δυνατοτήτων του ανθρώπου με διαδραστικά συστήματα μάθησης,
\footnote{(\textbf{Εικόνα?})~11 Χελώνα ρομπότ (Seymour Papert)} τα οποία
δεν εξασκούν απλά τις γνώσεις και τις δεξιότητες, αλλά καλλιεργούν τον
τρόπο σκέψης. \footnote{Papert (1980), Kay (1993)} Από την άλλη πλευρά,
η τεχνολογία της διάδρασης μπορεί να θεωρηθεί ως μια νέα τεχνολογία
γραφής. Πράγματι, οι άνθρωποι μαθαίνουν σχετικά εύκολα να μιλάνε, αλλά η
δεξιότητα της ανάγνωσης απαιτεί πολλά χρόνια εκπαίδευσης. Επομένως, μια
γόνιμη αναλογία είναι να θεωρήσουμε τη διάδραση ως ένα νέο σύστημα
γραφής, το οποίο απαιτεί πολλά χρόνια εκπαίδευσης, \footnote{(\textbf{Εικόνα?})~12
  Παιδιά με το Xerox Alto (the PARC Library)} ειδικά για όσους θέλουν να
βελτιωθούν, ενώ, ταυτόχρονα, δεν φαίνεται να έχει κάποιο τερματικό
σημείο. \footnote{Kay (1993)}

\leavevmode\vadjust pre{\hypertarget{fig:kidsim}{}}%
\begin{figure}
\hypertarget{fig:kidsim}{%
\centering
\includegraphics{images/kidsim.png}
\caption{Εικόνα 13: Η έκθεση στον προγραμματισμό της διάδρασης από
μικρές ηλικίες έχει αναγνωριστεί ως μια σημαντική αξία του ψηφιακού
αλφαβητισμού και έχει γίνει μια διαχρονική προσπάθεια να φτιαχτούν
οπτικές γλώσσες προγραμματισμού ώστε η διάδραση με τους υπολογιστές να
είναι κάτι περισσότερο από μια απλή κατανάλωση έτοιμων εμπειριών. Το
KidSim βασίζεται στην οπτική διάδραση με παραδείγματα, όπου ο κώδικας
παράγεται αυτόματα, χωρίς κείμενο ή τουβλάκια.}\label{fig:kidsim}
}
\end{figure}

\leavevmode\vadjust pre{\hypertarget{fig:mit-scratch}{}}%
\begin{figure}
\hypertarget{fig:mit-scratch}{%
\centering
\includegraphics{images/mit-scratch.png}
\caption{Εικόνα 14: Το περιβάλλον προγραμματισμού MIT Scratch έδωσε τη
δυνατότητα σε πολλές ομάδες χρηστών, ακόμη και μικρών ηλικιών, να
δημιουργήσουν εύκολα και χωρίς τυπική εκπαίδευση το δικό τους λογισμικό,
το οποίο συνήθως έχει τη μορφή μιας διαδραστικής ιστορίας. Η ευχρηστία
του βασίζεται στην οπτικοποίηση απλών εντολών με τη μορφή δομικών
στοιχείων, τα οποία συνδέονται μεταξύ τους, αλλά αυτός ο τρόπος
προγραμματισμού είναι μονοθεματικός και δεν επιτρέπει τη δημιουργία
κλίμακας ούτε τη δημιουργία μεγαλύτερων
συστημάτων.}\label{fig:mit-scratch}
}
\end{figure}

Η θέση του ψηφιακού αλφαβητισμού ως κάτι περισσότερο από εκπαίδευση στην
απλή χρήση των τεχνολογιών πληροφόρησης και επικοινωνίας έχει
διαπιστωθεί από τα μισά της δεκαετίας του 1990, όταν τα γραφικά
περιβάλλοντα διεπαφής με τον χρήστη είχαν κλείσει μία δεκαετία εμπορικής
ζωής. Οι ερευνητές διαπίστωσαν ότι τα παιδιά που μεγάλωσαν με τη γραφική
επιφάνεια εργασίας είχαν μεν μεγαλύτερη εξοικείωση με την παρουσία του
υπολογιστή στη ζωή τους, αλλά είχαν πολύ μικρότερες δεξιότητες στη
δημιουργική χρήση του. Η διαπίστωση αυτή οδήγησε σε μια σειρά από
προσπάθειες τόσο στο λογισμικό όσο και στο υλικό υπολογιστών, έτσι ώστε
να κρατήσουμε την προσβασιμότητα των σύγχρονων υπολογιστών, χωρίς όμως
να χάσουμε τις δεξιότητες που προσφέρει η κατασκευή της διάδρασης. Για
παράδειγμα, οι ερευνητές δημιούργησαν λογισμικό όπως τα
KidSim,\footnote{(\textbf{Εικόνα?})~13 Λογισμικό κατασκευής
  προσομοιώσεων για παιδιά (Allen Cypher)} Etoys, και
Scratch,\footnote{(\textbf{Εικόνα?})~14 Εκπαιδευτικό λογισμικό οπτικού
  προγραμματισμού (Wikimedia)} τα οποία βασίζονται στον οπτικό
προγραμματισμό και στον προγραμματισμό με βάση παραδείγματα χρήσης.
Αντίστοιχα, για την περίπτωση του υλικού υπολογιστή δημιουργήθηκε το
RaspberryPi, το οποίο είναι πολύ οικονομικό και συνδέεται με την
τηλεόραση, έτσι ώστε να έχει όσο γίνεται μεγαλύτερη διάχυση στους νέους
χρήστες υπολογιστών, οι οποίοι διαφορετικά θα μεγάλωναν μόνο με οθόνες
αφής. Τέλος, δημιουργήθηκαν πολλά απτικά προϊόντα προγραμματισμού, τα
οποία δεν έχουν καθόλου οθόνη και τα οποία βασίζονται στην οργάνωση
απτών αντικειμένων, ενώ και το αποτέλεσμά τους μπορεί να είναι η φυσική
κίνηση και η διάδραση με ένα ρομπότ.

Σε πρώτη ανάγνωση, η κατασκευή της διάδρασης φαίνεται το άθροισμα, ή
ίσως η τομή, των επιμέρους περιοχών του προγραμματισμού υπολογιστή και
της διάδρασης ανθρώπου και υπολογιστή. Στην πράξη όμως, η πρόσθεση των
γνώσεων προγραμματισμού σε εκείνες της διάδρασης ανθρώπου και υπολογιστή
δεν αποτελεί μια ικανή συνθήκη για τη δημιουργία νέων επινοήσεων, ικανών
να επαναπροσδιορίσουν τις ανθρώπινες και τις κοινωνικές δραστηριότητες.
Αν και η ύπαρξη βασικών επιμέρους γνώσεων, είτε στον ίδιο τον
κατασκευαστή, είτε στα μέλη μιας ομάδας συνεργασίας, αποτελεί μια
αναγκαία συνθήκη, υπάρχει επιπλέον η ανάγκη για γνώσεις σε ένα υψηλότερο
επίπεδο, στο επίπεδο της κατασκευής της διάδρασης. Σε αυτό το υψηλότερο
επίπεδο αφαίρεσης των επιμέρους λεπτομερειών, εστιάζουμε στα εργαλεία,
στις δομές, και στις διαδικασίες που θα δώσουν δημιουργικές λύσεις σε
υπάρχοντα προβλήματα και θα επαυξήσουν τις δυνατότητές μας.

\leavevmode\vadjust pre{\hypertarget{fig:linux}{}}%
\begin{figure}
\hypertarget{fig:linux}{%
\centering
\includegraphics{images/linux.png}
\caption{Εικόνα 15: Οι δύο βασικές προσεγγίσεις στη διάθεση του
λογισμικού είναι αυτή του ανοικτού (π.χ. Linux) και του κλειστού κώδικα
(π.χ. Microsoft Windows), οι οποίες εμφανίζονται ως αντίπαλες, αλλά σε
κάποιες περιπτώσεις μπορεί να λειτουργούν και συμπληρωματικά, όπως στην
περίπτωση του λογισμικού ανοιχτού κώδικα Red Hat Enterprise Linux, το
οποίο παρέχεται ως εμπορική υπηρεσία. Το πιο ενδιαφέρον, όμως, είναι ότι
μια συλλογική προσπάθεια όπως το Linux, η οποία δεν έχει στόχο το
κέρδος, μπορεί και παράγει ένα αποτέλεσμα εφάμιλλο των εμπορικών
υπηρεσιών.}\label{fig:linux}
}
\end{figure}

\leavevmode\vadjust pre{\hypertarget{fig:napster}{}}%
\begin{figure}
\hypertarget{fig:napster}{%
\centering
\includegraphics{images/napster.jpg}
\caption{Εικόνα 16: Στα τέλη της δεκαετίας του 1990, η συμπίεση των
αρχείων μουσικής σε σχετικά μικρά αρχεία και ο εύκολος συνεργατικός
διαμοιρασμός τους στο δίκτυο με το λογισμικό Napster άλλαξε μέσα σε πολύ
λίγα χρόνια το οικονομικό μοντέλο της διανομής της μουσικής και
συνεχίζει να επηρεάζει τον τρόπο με τον οποίο διανέμονται όλα τα ψηφιακά
αγαθά. Επομένως, η κατασκευή του λογισμικού διάδρασης δεν είναι απλά
ένας ψηφιακός μετασχηματισμός των φυσικών διαδικασιών, αλλά ενδέχεται να
επηρεάσει την ίδια τη φύση των παραδοσιακών
οργανισμών.}\label{fig:napster}
}
\end{figure}

Όπως οι οπτικές γλώσσες προγραμματισμού υψηλού επιπέδου (π.χ. Scratch)
έχουν μικρή μόνο σχέση με τις αντίστοιχες γλώσσες προγραμματισμού που
είναι κοντά στη μηχανή (π.χ. Assembly, C), έτσι και ο προγραμματισμός
της διάδρασης έχει μικρή μόνο σχέση με τις βασικές επιμέρους περιοχές
του, όπως εκείνη του προγραμματισμού ΗΥ. Στην πράξη, ο προγραμματιστής
της διάδρασης είναι χρήσιμο να ξέρει τις βασικές έννοιες του
προγραμματισμού, όπως είναι η μεταβλητή και οι συνθήκες, αλλά από εκεί
και πέρα η δεξιότητά του θα αυξηθεί περισσότερο αν μάθει να χρησιμοποιεί
και να κατασκευάζει νέες βιβλιοθήκες και εργαλεία, παρά αν μάθει όλες
τις αλγοριθμικές λεπτομέρειες που κάνουν ένα πρόγραμμα υπολογιστή
αποδοτικό (π.χ. ταχύτητα, μνήμη). Επομένως, αν και μιλάμε για κατασκευή
της διάδρασης, στην πράξη, η κατασκευή αυτή, όπου υπάρχει, αφορά
περισσότερο τη δημιουργική σύνθεση και τη χρήση έτοιμων και νέων
βιβλιοθηκών και εργαλείων με απλές δομές ελέγχου, με τελικό σκοπό την
επαύξηση των ανθρώπινων και κοινωνικών δραστηριοτήτων. Από αυτήν την
πλευρά, ένα από τα πιο φιλικά και εύχρηστα συστήματα διάδρασης είναι η
γραμμή εντολών των συστημάτων τύπου UNIX, όπου απλές εντολές, οι
ανακατεύθυνση της εισόδου και εξόδου, καθώς και μικρά προγράμματα
μπορούν να συνθέσουν νέες εφαρμογές.

Πέρα από την ανάγκη κατανόησης του σύγχρονου ψηφιακού κόσμου, οι γνώσεις
και οι δεξιότητες της κατασκευής της διάδρασης δημιουργούν νέα προϊόντα
και υπηρεσίες που επηρεάζουν τις προσωπικές αντιλήψεις, τις συνήθειες,
τους θεσμούς και τις μορφές κοινωνικής οργάνωσης. Στην εποχή μας, όπου η
χρήση του υπολογιστή έχει κατηγορηθεί για την αύξηση της ανεργίας μέσω
του αυτοματισμού και της αύξησης της παραγωγικότητας, μπορούμε να δούμε
με αισιοδοξία μια αχαρτογράφητη πτυχή του υπολογιστή ως μέσου και
εργαλείου δημιουργίας ενός νέου επιπέδου ανθρώπινης δραστηριότητας.

Το λειτουργικό σύστημα των επιτραπέζιων υπολογιστών προοριζόταν αρχικά
προς πώληση ως ένα προϊόν, συσκευασμένο σε ένα κουτί. Στη συνέχεια,
έγινε αντιληπτό ότι το περιεχόμενο του κουτιού ποτέ δεν ήταν το τελικό,
αφού λίγες μέρες μετά τη συσκευασία και τη διανομή του, γίνονταν ήδη
βελτιώσεις στον πηγαίο κώδικα. Η ανάπτυξη του διαδικτύου ως καναλιού
διανομής επέτρεψε στο λογισμικό να βρει τον χαρακτήρα που του ταιριάζει
περισσότερο, ως υπηρεσία, αν και υπάρχει μια υβριδική ισορροπία ανάμεσα
στα δύο όταν το λογισμικό συνοδεύει κάποια συσκευή. Διαπιστώνουμε ότι το
λογισμικό είναι αρκετά διαφορετικό από άλλα προϊόντα και υπηρεσίες
αναφορικά με τις παρακάτω ιδιότητες: α) την πνευματική ιδιοκτησία, β)
την εμπορευσιμότητα, ιδιότητες τις οποίες οποίες μελετάμε παρακάτω.
\footnote{(\textbf{Εικόνα?})~15 Εμπορική εκμετάλευση ανοιχτού κώδικα
  (Ubuntu)} \footnote{(\textbf{Εικόνα?})~16 Napster (Wikipedia)}

Υπάρχουν πολλά είδη δικαιωμάτων πνευματικής ιδιοκτησίας, όπως το
εμπορικό σήμα, η πατέντα και η πνευματική ιδιοκτησία. Τα περισσότερα
έργα λογισμικού αντιμετωπίζονται όπως τα λογοτεχνικά βιβλία και έχουν
πνευματική ιδιοκτησία, αν και υπάρχουν περιπτώσεις στο λογισμικό της
διεπαφής ανθρώπου και υπολογιστή, στις οποίες έχει γίνει μία προσπάθεια
για τη δημιουργία πατέντας. Για παράδειγμα, στα τέλη της δεκαετίας του
1980, η Apple προσπάθησε να προστατεύσει το Γραφικό Περιβάλλον
Εργασίας\\
απέναντι στον ανταγωνισμό της Microsoft. Επειδή το λογισμικό δεν είναι
ούτε βιβλίο αλλά ούτε και βιομηχανικό αντικείμενο, τα υπάρχοντα είδη
πνευματικής ιδιοκτησίας ίσως να μην του ταιριάζουν. Άλλωστε, αυξάνονται
οι περιπτώσεις για τις οποίες το λογισμικό δίνεται με άδεια ανοικτού
κώδικα ή παρέχει κάποια Διεπαφή Προγραμματισμού Εφαρμογών και στη
συνέχεια ο δημιουργός αναζητεί αμοιβή μέσα από την πώληση της
τεχνογνωσίας του. Όσο χαρακτηριστική είναι η περίπτωση αυτοδημιούργητων
ιδιοκτητών \emph{κλειστού κώδικα} όπως ο Bill Gates της Microsoft, άλλο
τόσο ενδιαφέρουσα είναι η περίπτωση του Linus Torvalds με τα ανοικτού
κώδικα Linux, ο οποίος επέλεξε να δώσει δωρεάν τον καρπό της προσπάθειάς
του. Και στις δύο περιπτώσεις είχαμε τη δημιουργία μιας πολύ μεγάλης
βιομηχανίας και πολλών θέσεων εργασίας, παρόλο που η προσέγγιση του
καθενός ήταν διαμετρικά αντίθετη.

Η εμπορευσιμότητα ενός αγαθού ή μιας υπηρεσίας εξαρτάται από πολλούς
παράγοντες, αλλά ο σημαντικότερος είναι η δυνατότητα που υπάρχει για
εύκολη γεωγραφική διανομή. Το λογισμικό, το οποίο ξεκίνησε ως μέρος του
υλικού και στη συνέχεια έγινε δίσκος που αγοραζόταν από τα ράφια του
λιανεμπορίου, σταδιακά, έχει μετατραπεί σε μια υπηρεσία διαθέσιμη στο
διαδίκτυο. Στην ίδια συζήτηση έχει ενδιαφέρον να αναφερθούμε και στη
δουλειά του προγραμματιστή λογισμικού, αλλά και στις δυνατότητες καθώς
και στους κινδύνους από την εμπορευσιμότητα αυτής της εργασίας. Για
παράδειγμα, μια υπηρεσία στον ιστό είναι διαθέσιμη παντού, πράγμα που
σημαίνει ότι τελικά θα πρέπει να ανταγωνιστεί αντίστοιχες προσπάθειες
από όπου και αν προέρχονται, είτε από τις τεχνολογικά ανεπτυγμένες
χώρες, είτε από τις χώρες με το εξειδικευμένο εργατικό δυναμικό χαμηλού
κόστους. Πέρα από τις ευκαιρίες για μια διευρυμένη αγορά, μέσα σε αυτό
το παγκοσμιοποιημένο πλαίσιο επαγγελματικής δραστηριότητας, είναι μάλλον
αφελές να κρατάμε κλειστή τη διεπαφή με ένα λογισμικό, αφού μέσα σε ένα
μικρό σχετικά χρονικό διάστημα κάποιος μπορεί να φτιάξει κάτι παρόμοιο ή
κάτι καλύτερο. Μια περισσότερο αποτελεσματική στρατηγική είναι να
κάνουμε διαθέσιμο τον πηγαίο κώδικα ελπίζοντας σε συνεισφορές για τη
βελτίωσή του και ταυτόχρονα να μαθαίνουμε από τις διαδράσεις που κάνουν
οι χρήστες. Έτσι, θα βελτιώνουμε την υπηρεσία και, κυρίως, θα αυξάνουμε
τη γνώση που έχουμε για το τι συνιστά ανά πάσα στιγμή μια χρήσιμη και
επιθυμητή υπηρεσία, το οποίο είναι και το ζητούμενο για ένα σχετικά
βιώσιμο ανταγωνιστικό πλεονέκτημα.

Από την άλλη πλευρά, το λογισμικό λειτουργεί παρόμοια με τη μηχανή
εσωτερικής καύσης και τη βιομηχανική ρομποτική αναφορικά με την
αυτοματοποίηση της ανθρώπινης δραστηριότητας. Η αυτοματοποίηση
θεωρείται, συνήθως, αρετή, αφού επιτρέπει στον άνθρωπο να ασχοληθεί με
κάτι διαφορετικό από τις μηχανικές, επίπονες και επαναλαμβανόμενες
διεργασίες. Στην πράξη, η αυτοματοποίηση επέτρεψε τη μετάβαση από την
αγροτική στη βιομηχανική εποχή και έπειτα στην εποχή των υπηρεσιών. Η
άλλη όψη του νομίσματος, όμως, περιγράφει μια επίπονη περίοδο μετάβασης
από τη μια εποχή στην επόμενη. Όπως οι μηχανές εσωτερικής καύσης
διευκόλυναν την εργασία και αύξησαν την παραγωγικότητα κατά τη μετάβαση
από την αγροτική στη βιομηχανική εποχή και όπως η ρομποτική και ο
αυτοματισμός μείωσαν στη συνέχεια την ανάγκη για ανθρώπινη εργασία στα
εργοστάσια, έτσι και το λογισμικό διάδρασης, με τον ίδιο τρόπο, έρχεται
να αυτοματοποιήσει πάρα πολλές εργασίες οι οποίες γίνονταν με τη
μεσολάβηση ανθρώπων στη βιομηχανία των υπηρεσιών (π.χ. τράπεζες,
ασφάλειες, ταξίδια, κτλ.). Αν η ιστορία είναι ένας σωστός οδηγός, τότε
θα πρέπει να αναζητήσουμε την επόμενη βιομηχανική επανάσταση ανάμεσα
στις δυνατότητες που μας προσφέρει η κατασκευή της διάδρασης για νέες
υπηρεσίες και αγαθά, τα οποία με τη σειρά τους θα ορίσουν μια νέα αγορά.

Στη βιομηχανική εποχή (19ος αιώνας), όταν οι μηχανές αντικατέστησαν το
μεγαλύτερο μέρος της ανθρώπινης χειρωνακτικής εργασίας, σε πρώτη φάση
δημιούργησαν στρατιές ανέργων, σε δεύτερη φάση, όμως, τα προϊόντα
ορισμένων δημιουργικών ανθρώπων που βασίστηκαν στις μηχανές (π.χ.
αεροπλάνο, αυτοκίνητο κ.ά.) δημιούργησαν νέους κλάδους εργασίας και
ανθρώπινης δραστηριότητας, αθροιστικά πολύ μεγαλύτερους από αυτούς που
αρχικά κατέστρεψαν. Για παράδειγμα, τόσο η βιομηχανία του τουρισμού όσο
και η αύξηση της οικονομικής δραστηριότητας με τη συγκέντρωση των
ανθρώπων στις πόλεις, ήταν παράπλευρες ωφέλειες του αεροπλάνου και του
αυτοκινήτου, αντίστοιχα.

Είναι αλήθεια ότι το πρώτο κύμα διάχυσης του ΗΥ, με πρωταγωνιστή τον
επιτραπέζιο ΗΥ, κατάφερε να αυτοματοποιήσει ένα πολύ μεγάλο μέρος της
εργασίας γραφείου, με αποτέλεσμα την απώλεια θέσεων εργασίας στον πυρήνα
της οικονομίας των υπηρεσιών. Σε αναλογία με τη βιομηχανική εποχή, η
ενσωμάτωση και η διάχυση του ΗΥ στην καθημερινότητα με νέα προϊόντα και
υπηρεσίες ενδέχεται να δημιουργήσει αθροιστικά περισσότερες θέσεις
εργασίας από εκείνες που χάθηκαν, αρκεί να βρεθούν οι δημιουργικοί και
καταρτισμένοι κατασκευαστές της διάδρασης, οι οποίοι θα οραματιστούν και
θα υλοποιήσουν αυτούς τους νέους κλάδους της ανθρώπινης δραστηριότητας.

Η κατασκευή της διάδρασης δεν είναι πανάκεια και σίγουρα δεν είναι λύση
σε σημαντικά προβλήματα που έχουν να κάνουν με τη φτώχεια και την υγεία.
Από την άλλη πλευρά, η κατασκευή της διάδρασης είναι σίγουρα μια λύση
συμβατή με την πολύ σημαντική ανάγκη που αφορά τη δυνατότητά μας να
οραματιστούμε και να δημιουργήσουμε ένα διαφορετικό και νέο επίπεδο
ανθρώπινης δραστηριότητας σε σημαντικούς τομείς όπως είναι η εργασία, ο
πολιτισμός, και η εκπαίδευση. Τέλος, είναι σίγουρα μία από τις λίγες
λύσεις που έχουμε για να θέσουμε σε λειτουργία τον εκδημοκρατισμό των
ψηφιακών μέσων σχεδίασης και παραγωγής, τα οποία έχουν τη δυνατότητα να
περάσουν την οικονομία στο επόμενο στάδιο μετά τη βιομηχανία των
υπηρεσιών γραφείου, στην εποχή της δημιουργίας φαντασιακών μηχανών πέρα
από την προσομοίωση του φυσικού και βιολογικού κόσμου.

\hypertarget{ux3b7-ux3c0ux3b5ux3c1ux3afux3c0ux3c4ux3c9ux3c3ux3b7-ux3c4ux3bfux3c5-minecraft}{%
\subsection{Η περίπτωση του
Minecraft}\label{ux3b7-ux3c0ux3b5ux3c1ux3afux3c0ux3c4ux3c9ux3c3ux3b7-ux3c4ux3bfux3c5-minecraft}}

Πριν ο προγραμματισμός της διάδρασης αποκτήσει πρωταγωνιστικό ρόλο στην
έρευνα και στη βιομηχανία, η επιτυχία ενός προϊόντος μπορούσε να
μετρηθεί από τις πωλήσεις που έκανε ή από τις καλές κριτικές που
έπαιρνε. Η άνοδος του προγραμματισμού της διάδρασης προσθέτει νέες
μετρικές, όπως τη συμμετοχή των χρηστών στην επέκταση του αρχικού
προϊόντος. Στον χώρο της ψυχαγωγίας μέσω υπολογιστή, μία από τις πιο
επιτυχημένες περιπτώσεις είναι αυτή του Minecraft, στο οποίο οι
χρήστες-παίκτες είναι εκείνοι που κατασκευάζουν από κοινού το περιβάλλον
του παιχνιδιού.\footnote{(\textbf{Εικόνα?})~17 Minecraft (Microsoft)}

\leavevmode\vadjust pre{\hypertarget{fig:minecraft-end-user}{}}%
\begin{figure}
\hypertarget{fig:minecraft-end-user}{%
\centering
\includegraphics{images/minecraft-end-user.png}
\caption{Εικόνα 17: Η ιδέα να δημιουργούνται πίστες από τους τελικούς
χρήστες δεν είναι καινούργια και έχει δοκιμαστεί με επιτυχία σε αρκετά
παιχνίδια ως πρόσθετη λειτουργία. Το Minecraft είναι από την αρχή
σχεδιασμένο με σκοπό οι τελικοί χρήστες να σχεδιάζουν τον εικονικό
κόσμο.}\label{fig:minecraft-end-user}
}
\end{figure}

\leavevmode\vadjust pre{\hypertarget{fig:learntomod}{}}%
\begin{figure}
\hypertarget{fig:learntomod}{%
\centering
\includegraphics{images/learntomod.jpg}
\caption{Εικόνα 18: Η ευελιξία και η επεκτασιμότητα του Minecraft δεν
σταματούν στη δυνατότητα κατασκευής του σκηνικού της δράσης, αλλά
επεκτείνονται στη δυνατότητα προγραμματισμού της συμπεριφοράς και στη
δημιουργία νέων αντικειμένων.}\label{fig:learntomod}
}
\end{figure}

Η έμφαση στην κατασκευή του εικονικού κόσμου του παιχνιδιού από τους
τελικούς χρήστες βασίζεται σε μια συμμετοχική φιλοσοφία, η οποία είναι
εντελώς διαφορετική από την παροχή μιας προκατασκευασμένης εμπειρίας,
όπως είναι το σύνηθες στα περισσότερα βιντεο-παιχνίδια. Οι δημιουργοί
του Minecraft ανάγνωσαν έγκαιρα την ανάγκη των χρηστών για μεγαλύτερη
δυνατότητα προσωπικής έκφρασης μέσα από τις ψηφιακές δραστηριότητές
τους. Εκτός από τη συμμετοχή των χρηστών στην κατασκευή του εικονικού
κόσμου του παιχνιδιού, οι κατασκευαστές του Minecraft έχουν προχωρήσει
ένα βήμα παρακάτω, στη διευκόλυνση της ανάπτυξης επεκτάσεων, \footnote{(\textbf{Εικόνα?})~18
  Κατασκευή επεκτάσεων στο Minecraft (LearnToMod)} τα οποία αλλάζουν τη
συμπεριφορά του παιχνιδιού ή προσθέτουν λειτουργίες. Μία από τις πιο
ενδιαφέρουσες λειτουργίες προσθέτει τη δυνατότητα της εκμάθησης
προγραμματισμού για τον υπολογιστή. Με δεδομένη την εμβύθιση των χρηστών
στο σύστημα διάδρασης του Minecraft, η ελπίδα είναι ότι η εκμάθηση του
προγραμματισμού υπολογιστών μπορεί να κινητοποιηθεί από το ίδιο το μέσο,
με σκοπό την κατασκευή νέων συμπεριφορών για τον κόσμο του Minecraft.

Η ενεργή συμμετοχή των χρηστών στην κατασκευή παιχνιδιών δεν είναι κάτι
νέο, ούτε ήταν το Minecraft η πρώτη ανάλογη περίπτωση. Στις αρχές της
δεκαετίας του 1990 η δημοφιλής σειρά βιντεο-παιχνιδιών Doom έδινε τη
δυνατότητα στους χρήστες να κατασκευάσουν τις δικές τους πίστες, πράγμα
που διατηρούσε το ενδιαφέρον τους για μεγαλύτερο χρονικό διάστημα. Στη
δυνατότητα mods του Minecraft και ειδικά στην ευελιξία που δίνει για να
εξυπηρετήσει διαφορετικούς σκοπούς, βλέπουμε τη διαφορετική φιλοσοφία
απέναντι στην ιδιοκτησία που έχουν οι εταιρείες του διαδικτύου σε σχέση
με τις παλιότερες εταιρείες κατασκευής παιχνιδιών, οι οποίες ήθελαν να
έχουν όσο γίνεται μεγαλύτερο έλεγχο τόσο στους χαρακτήρες όσο και στο
εικονικό περιβάλλον του παιχνιδιού, ενώ πολλές φορές είχαν κάνει την
πρόσβαση στα βιντεο-παιχνίδια τους δύσκολη, σε μια προσπάθεια να
αποτρέψουν την παράνομη αντιγραφή.

Η ίδια η ιστορία της ανάπτυξης του Minecraft έχει ιδιαίτερο ενδιαφέρον
και πέρα από τη φύση της διάδρασης, η οποία όπως είδαμε παραπάνω
βασίζεται στη συμμετοχή των χρηστών τόσο στο περιεχόμενο του παιχνιδιού
όσο και στην επέκταση της ίδιας της συμπεριφοράς του με τροποποιήσεις. Η
αρχική ανάπτυξη του παιχνιδιού έγινε από έναν μόνο έμπειρο κατασκευαστή,
ο οποίος άφησε τη θέση υπαλλήλου που είχε σε εταιρεία ανάπτυξης
βιντεο-παιχνιδιών για να υλοποιήσει τις δικές του ιδέες. Η έμπνευση ήρθε
από την ενασχόλησή του με βιντεο-παιχνίδια από μικρούς ανεξάρτητους
παραγωγούς, οπότε δημιουργήθηκε η πρώτη έκδοση ενός παιχνιδιού, στο
οποίο ο χρήστης μπορούσε να τοποθετήσει αντικείμενα σε χώρο τριών
διαστάσεων. Στη συνέχεια, πρόσθεσε τη δυνατότητα κατασκευής από πολλούς
χρήστες, καθώς και από εχθρούς τους. Από εκεί και πέρα, η δημοσιότητα
ήρθε από μόνη της, όταν οι χρήστες του παιχνιδιού άρχισαν να δημιουργούν
όμορφες κατασκευές. Βλέπουμε, λοιπόν, ότι η προσωπική έκφραση μέσω του
υπολογιστή στην περίπτωση του Minecraft δεν είναι μόνο ένα προνόμιο του
δημιουργού του αλλά και όλων των τελικών χρηστών, οι οποίοι με αυτόν τον
τρόπο νιώθουν το παιχνίδι δικό τους.

\hypertarget{ux3b7-ux3c0ux3b5ux3c1ux3afux3c0ux3c4ux3c9ux3c3ux3b7-ux3c4ux3bfux3c5-xerox-star}{%
\subsection{Η περίπτωση του Xerox
Star}\label{ux3b7-ux3c0ux3b5ux3c1ux3afux3c0ux3c4ux3c9ux3c3ux3b7-ux3c4ux3bfux3c5-xerox-star}}

\leavevmode\vadjust pre{\hypertarget{fig:xerox-star-pc}{}}%
\begin{figure}
\hypertarget{fig:xerox-star-pc}{%
\centering
\includegraphics{images/xerox-star-pc.jpg}
\caption{Εικόνα 19: Ο επιτραπέζιος υπολογιστής με πληκτρολόγιο, ποντίκι
και γραφική επιφάνεια εργασίας (παράθυρα, εικονίδια, φάκελοι), ο οποίος
δημιουργήθηκε από τη Xerox στα τέλη της δεκαετίας του 1970 λίγο διαφέρει
από τον μοντέρνο επιτραπέζιο υπολογιστή.}\label{fig:xerox-star-pc}
}
\end{figure}

\leavevmode\vadjust pre{\hypertarget{fig:xerox-star-genealogy}{}}%
\begin{figure}
\hypertarget{fig:xerox-star-genealogy}{%
\centering
\includegraphics{images/xerox-star-genealogy.png}
\caption{Εικόνα 20: Το γενεαλογικό δέντρο του Xerox Star περιέχει
σημαντικούς προγόνους (π.χ. Memex, NLS, Sketchpad κ.ά.), καθώς και
εξίσου σημαντικούς απογόνους (π.χ. Macintosh), με τα περισσότερα
στοιχεία της διάδρασης (π.χ. γραφική επιφάνεια εργασίας, απευθείας
χειρισμός κτλ.) να παραμένουν τα ίδια για
δεκαετίες.}\label{fig:xerox-star-genealogy}
}
\end{figure}

Η γραφική επιφάνεια εργασίας όπως είναι διαθέσιμη σε πολλούς εμπορικούς
επιτραπέζιους υπολογιστές λίγο διαφέρει από εκείνη που είχε ο
υπολογιστής Xerox Star\footnote{(\textbf{Εικόνα?})~19 Ο επιτραπέζιος
  υπολογιστής Xerox Star (Xerox PARC)} που δημιουργήθηκε στο ερευνητικό
κέντρο PARC.\footnote{Johnson et al. (1989)} Η γραφική επιφάνεια
εργασίας σε συνδυασμό με το ποντίκι και το πληκτρολόγιο αποτελεί έναν
ιδιαίτερα αποδοτικό τρόπο εργασίας με επεξεργαστές κειμένου και
προγράμματα επεξεργασίας εικόνας και γραφικών. Η βασική διαφορά που έχει
σε σχέση με το εμπορικά επιτυχημένο Apple Macintosh είναι ότι δεν έχει
εφαρμογές, γιατί η βασική μεταφορά διάδρασης είναι ένα έγγραφο, όπου
ανάλογα με το αντικείμενο που επεξεργάζεται ο χρήστης έχει στην διάθεση
του τα αντίστοιχα εργαλεία. Δεν είναι τυχαίο ότι η δημιουργία της
γραφικής επιφάνειας εργασίας έγινε από το ερευνητικό κέντρο PARC της
εταιρείας XEROX κατά τη διάρκεια μελέτης και αυτοματοποίησης της
εργασίας σε εκδοτικούς οργανισμούς. Πράγματι, ο στόχος αυτού του
προϊόντος είναι η προσομοίωση του φυσικού χαρτιού στην οθόνη του
υπολογιστή, έτσι ώστε να διευκολύνει τις εκδοτικές δραστηριότητες.
Αξίζει να δούμε λίγο πιο προσεκτικά τα χαρακτηριστικά του Star γιατί θα
καταλάβουμε καλύτερα τους λόγους που η γραφική επιφάνεια εργασίας πήρε
αυτήν τη μορφή και όχι κάποια άλλη η οποία θα ήταν αποδεκτή, αν το
πλαίσιο ανάπτυξης και οι ανάγκες των χρηστών ήταν διαφορετικές.

Τα βασικά συστατικά της γραφικής επιφάνειας εργασίας υπήρχαν από
προηγούμενες ερευνητικές και εμπορικές προσπάθειες (π.χ. ποντίκι,
ηλεκτρονική επεξεργασία κειμένου σε οθόνη), αλλά ήταν το Xerox Star ο
πρώτος υπολογιστής που ολοκλήρωνε τις κατακερματισμένες προσπάθειες σε
μια χρήσιμη και εύχρηστη συσκευή. Η κινητήριος δύναμη αυτής της
δημιουργικής και ολοκληρωμένης σύνθεσης ήταν η ανθρωποκεντρική σχεδίαση
και ανάπτυξη του συστήματος Star με έμφαση στις ανάγκες ενός πελάτη,
ενός εκδοτικού οίκου που έκανε επεξεργασία κειμένου και σελιδοποίηση
εγγράφων και βιβλίων. Για τον σκοπό αυτό, οι ερευνητές έκαναν παρατήρηση
του τρόπου εργασίας σε ένα γραφείο της εποχής που διαχειριζόταν έγγραφα.
Με αυτόν τον τρόπο διαπίστωσαν ότι υπήρχε ανάγκη για εύκολη επεξεργασία
και αποθήκευση ενός εγγράφου που περιείχε πολυμεσικά στοιχεία, καθώς και
για τον διαμοιρασμό του. Οι παραπάνω προδιαγραφές που προέκυψαν από τις
ανάγκες των χρηστών σε συνδυασμό με το πλαίσιο χρήσης (γραφείο εκδοτικού
οργανισμού) οδήγησαν στη δημιουργία της γραφικής επιφάνειας εργασίας.

Η γραφική επιφάνεια εργασίας είναι μια συμπαγής και συνεπής σύνθεση από
επιμέρους στοιχεία διάδρασης. Συνήθως αναφέρεται και ως μοντέλο Windows
Icons Menus Pointer (WIMP), καθώς τα βασικά της στοιχεία είναι τα
παράθυρα, τα εικονίδια, τα μενού και ο δείκτης. Τα παράθυρα
αντιπροσωπεύουν έγγραφα της ίδιας ή άλλων εφαρμογών, τα εικονίδια
αντιπροσωπεύουν εφαρμογές, φακέλους και αρχεία, ενώ τα μενού επιτρέπουν
ενέργειες πάνω σε αντικείμενα ή αλλαγή της κατάστασης μιας εφαρμογής. Ο
δείκτης επιτρέπει την πλοήγηση ανάμεσα σε παράθυρα, εικονίδια και μενού,
καθώς και την επιλογή αντικειμένων. Ο δείκτης συνήθως ελέγχεται από ένα
ποντίκι, αλλά αυτό δεν είναι η μόνη πιθανή συσκευή εισόδου, αφού ένας
δείκτης μπορεί επίσης να ελέγχεται από διαφορετικές συσκευές εισόδου
όπως είναι η πένα ή ακόμη και απευθείας με την αφή. Σε κάθε περίπτωση,
αυτό που είναι σημαντικό στη γραφική επιφάνεια εργασίας είναι να έχουμε
απευθείας χειρισμό των στοιχείων της από τον δείκτη. Τα παραπάνω
χαρακτηριστικά τα συναντάμε με διαφορετική αισθητική και μικρές
παραλλαγές σε εναλλακτικά λειτουργικά συστήματα με γραφική επιφάνεια
εργασίας.

Συνοπτικά, αυτό που ονομάζουμε γραφική επιφάνεια εργασίας είναι το
αποτέλεσμα της δημιουργικής ολοκλήρωσης ενός συνόλου από προηγούμενες
τεχνολογίες, για την εξυπηρέτηση των αναγκών μιας δεδομένης ομάδας
χρηστών. Έχοντας αναλύσει παραπάνω τη γραφική επιφάνεια εργασίας από την
πλευρά των χρηστών, οι οποίοι ήταν εργαζόμενοι γραφείου (κυρίως
εκδοτικών οίκων ή συναφών οργανισμών), θα αναλύσουμε επίσης την πλευρά
της τεχνολογίας, η οποία είχε βασιστεί σε προηγούμενα έργα. Ανάμεσα στις
πολλές επιρροές του Xerox Star, η σημαντικότερη ήταν η υιοθέτηση της
συσκευής εισόδου ποντίκι και, κυρίως, ο τρόπος με τον οποίο ο δείκτης
του ποντικιού επέτρεπε τη διάδραση με μια αφαιρετική αναπαράσταση της
πληροφορίας σε μια οθόνη. Η σημασία αυτής της τεχνολογικής καινοτομίας
μπορεί να γίνει κατανοητή αν αναλογιστούμε ότι μέχρι τότε η χρήση του
υπολογιστή βασιζόταν στη στενή σύνδεση της εισόδου με την έξοδο για τον
χρήστη, αφού για παράδειγμα είχαμε κείμενο σε οθόνη κειμένου, τα οποία
διαχειριζόμασταν μόνο με το πληκτρολόγιο, χωρίς να υπάρχουν ενδιάμεσα
επίπεδα αφαιρετικότητας του είδους της πληροφορίας. Τελικά, η συνεισφορά
του υπολογιστή Star ήταν πολύ μεγαλύτερη από την αλλαγή του τρόπου που
κάνουμε αυτό που λέμε δουλειά γραφείου, αφού η ιστορία ανάπτυξής του
\footnote{(\textbf{Εικόνα?})~20 Το γενεαλογικό δένδρο του Xerox Star
  (Digibarn)} δείχνει τη μέθοδο, τα εργαλεία και τους κανόνες για να
σχεδιάσουμε και να κατασκευάσουμε νέους τρόπους διάδρασης με τον
υπολογιστή για άλλες ομάδες χρηστών και για διαφορετικό πλαίσιο
δραστηριότητας.

\hypertarget{ux3c3ux3cdux3bdux3c4ux3bfux3bcux3b7-ux3b2ux3b9ux3bfux3b3ux3c1ux3b1ux3c6ux3afux3b1-ux3c4ux3bfux3c5-ivan-sutherland}{%
\subsection{Σύντομη βιογραφία του Ivan
Sutherland}\label{ux3c3ux3cdux3bdux3c4ux3bfux3bcux3b7-ux3b2ux3b9ux3bfux3b3ux3c1ux3b1ux3c6ux3afux3b1-ux3c4ux3bfux3c5-ivan-sutherland}}

Ο Ivan Sutherland μεγάλωσε παίζοντας μια γερμανική εκδοχή των Lego, στην
οποία ένα μικρό σετ από γεωμετρικά σχήματα μπορεί να δώσει μορφή σε
πολύπλοκες κατασκευές όπως είναι μια γέφυρα. Το 1963 κατασκεύασε το
διαδραστικό σύστημα σχεδίασης γραφικών Sketchpad, \footnote{(\textbf{Εικόνα?})~21
  Ivan Sutherland (MIT)} το οποίο μπορούσε να χρησιμοποιηθεί για τη
μοντελοποίηση πολύπλοκων συστημάτων όπως μια γέφυρα. \footnote{(\textbf{Εικόνα?})~22
  Sketchpad (Wikipedia)} Αν και έγινε γνωστός για αυτό το σύστημα, η
συνεισφορά του είναι πολύ ευρύτερη, καθώς δημιούργησε μια μεγάλη
κοινότητα με ανθρώπους και εταιρείες.

\leavevmode\vadjust pre{\hypertarget{fig:sutherland-profile}{}}%
\begin{figure}
\hypertarget{fig:sutherland-profile}{%
\centering
\includegraphics{images/sutherland-profile.jpg}
\caption{Εικόνα 21: Στο σύστημα Sketchpad, εκτός από την υποστήριξη της
δημιουργίας προσχεδίων, υπήρχε η δυνατότητα επίλυσης εξισώσεων, όπως
αυτές ορίζονται από τα γραφικά που αναπαριστούν μια γέφυρα. O Ivan
Sutherland ξεκίνησε από την περιοχή των διαδραστικών γραφικών και
συνέχισε ως μέντορας ερευνητών και στέλεχος πολλών οργανισμών που άφησαν
το αποτύπωμα τους στην ευρύτερη περιοχή των
υπολογιστών.}\label{fig:sutherland-profile}
}
\end{figure}

\leavevmode\vadjust pre{\hypertarget{fig:sketchpad-drafting}{}}%
\begin{figure}
\hypertarget{fig:sketchpad-drafting}{%
\centering
\includegraphics{images/sketchpad-drafting.jpg}
\caption{Εικόνα 22: Η βασική λειτουργία του Sketchpad είναι η υποστήριξη
της διαδικασίας της δημιουργίας προσχεδίων για τον κλάδο της
μηχανολογίας. Το Sketchpad, εκτός απο τη διάδραση με την πένα, παρέχει
μια σειρά εντολών που ορίζουν περιορισμούς πάνω στα αρχικά
σχεδιαγράμματα, έτσι ώστε να είναι εύκολος ο μετασχηματισμός τους με
βάση κάποιους κανόνες, όπως οι παράλληλες γραμμές και οι ορθές
γωνίες.}\label{fig:sketchpad-drafting}
}
\end{figure}

Αμέσως μετά τη δημιουργία του Sketchpad στο MIT, συνέχισε με την
δημιουργία του Sword of Damocles στο Harvard, όπου τα γραφικά ακολουθούν
την κίνηση του κεφαλιού του χρήστη. Το σύστημα αυτό αρχικά δημιουργήθηκε
για τη διευκόλυνση της προσγείωσης ενός ελικοπτέρου με κάμερες στη βάση
του και για την προβολή του βίντεο σε δύο οθόνες που βρίσκονται πάνω σε
ένα κράνος. Για να το πετύχει αυτό έβαλε στη θέση του βίντεο από τις
κάμερες, έναν υπολογιστή που παράγει δυναμικά γραφικά. Το σύστημα αυτό
είναι θεμελιώδες για τα συστήματα εικονικής πραγματικότητας των επόμενων
δεκαετιών.

Εκτός από τη συνεισφορά του στην επιστήμη των διαδραστικών γραφικών, στη
συνέχεια της καριέρας του δημιούργησε μία από τις σημαντικότερες
εταιρείες, η οποία κατασκεύασε προσομοιωτές πτήσης για τα αεροσκάφη της
Boing, έτσι ώστε να γίνει καλύτερη η εκπαίδευση των πιλότων αλλά και η
σχεδίαση της πολύπλοκης διεπαφής πτήσης. Η βελτίωση της εκπαίδευσης και
της διεπαφής στα αεροπλάνα θεωρείται ότι βελτίωσε με την σειρά της την
ασφάλεια των πτήσεων. Στον χώρο της ψηφιακής ψυχαγωγίας, ο μαθητής του
Edwin Catmull εμπνεύστηκε από τις δυνατότητες των γραφικών στον
υπολογιστή και προσπάθησε να εκπληρώσει το παιδικό του όνειρο να γίνει
σχεδιαστής κινούμενων γραφικών με την κατασκευή νέων τεχνολογιών και
εργαζόμενος σε εταιρείες μέχρι τη δημιουργία της Pixar.

Παράλληλα με την εταιρεία, συμβούλευε διδακτορικούς φοιτητές στο
πανεπιστήμιο της Utah, ανάμεσα τους τον Alan Kay, καθώς και τους
δημιουργούς άλλων σημαντικών εταιρειών, όπως η Pixar και η Adobe.
Επίσης, ήταν υπεύθυνος για τη συνέχιση της χρηματοδότησης του
προγράμματος DARPA αμέσως μετά την αποχώρηση του Licklider και με αυτόν
τον τρόπο συνέχισε να στηρίζει τη βασική έρευνα στις περιοχές των
γραφικών και της διάδρασης που γινόταν στα MIT, Stanford, Xerox PARC,
RAND. Συνολικά, ο Ivan Sutherland φαίνεται να έχει επηρεάσει ένα πολύ
μεγάλο μέρος των σύγχρονων τεχνολογιών είτε με τις δικές του καινοτομίες
είτε με αυτές των ανθρώπων που επηρέασε με τη δουλειά του.

\hypertarget{ux3b2ux3b9ux3b2ux3bbux3b9ux3bfux3b3ux3c1ux3b1ux3c6ux3afux3b1}{%
\subsection*{Βιβλιογραφία}\label{ux3b2ux3b9ux3b2ux3bbux3b9ux3bfux3b3ux3c1ux3b1ux3c6ux3afux3b1}}
\addcontentsline{toc}{subsection}{Βιβλιογραφία}

\hypertarget{refs}{}
\begin{CSLReferences}{0}{0}
\end{CSLReferences}

Card, Stuart K, Allen Newell, and Thomas P Moran. 1983. \emph{The
Psychology of Human-Computer Interaction}. L. Erlbaum Associates Inc.

Engelbart, Douglas C. 1962. \emph{Augmenting Human Intellect: A
Conceptual Framework}. SRI, Menlo Park, CA.

Freiberger, Paul, and Michael Swaine. 1984. \emph{Fire in the Valley:
The Making of the Personal Computer}. McGraw-Hill, Inc.

Hertzfeld, Andy. 2004. \emph{Revolution in the Valley: The Insanely
Great Story of How the Mac Was Made}. " O'Reilly Media, Inc.".

Hiltzik, Michael. 1999. {``Dealers of Lightning: Xerox PARC and the
Dawning of the Computer Age.''}

Johnson, Jeff, Teresa L. Roberts, William Verplank, David Canfield
Smith, Charles H. Irby, Marian Beard, and Kevin Mackey. 1989. {``The
Xerox Star: A Retrospective.''} \emph{Computer} 22 (9): 11--26.

Kay, Alan C. 1993. {``The Early History of Smalltalk.''} \emph{ACM
SIGPLAN Notices} 28 (3): 69--95.

Lanier, Jaron. 2014. \emph{Who Owns the Future?} Simon; Schuster.

Licklider, Joseph Carl Robnett. 1960. {``Man-Computer Symbiosis.''}
\emph{IRE Transactions on Human Factors in Electronics}, no. 1: 4--11.

Papert, Seymour. 1980. \emph{Mindstorms: Children, Computers, and
Powerful Ideas}. Basic Books, Inc.

Raskin, Jef. 2000. \emph{The Humane Interface: New Directions for
Designing Interactive Systems}. Addison-Wesley Professional.

Waldrop, M Mitchell. 2001. \emph{The Dream Machine: JCR Licklider and
the Revolution That Made Computing Personal}. Viking Penguin.

Weizenbaum, Joseph. 1976. \emph{Computer Power and Human Reason: From
Judgment to Calculation.} WH Freeman \& Co.

\hypertarget{ux3bcux3adux3b8ux3bfux3b4ux3bfux3c2}{%
\section{Μέθοδος}\label{ux3bcux3adux3b8ux3bfux3b4ux3bfux3c2}}

\begin{quote}
Κάνοντας σωστά τη σωστή σχεδίαση. Bill Buxton
\end{quote}

\hypertarget{ux3c0ux3b5ux3c1ux3afux3bbux3b7ux3c8ux3b7}{%
\subsubsection{Περίληψη}\label{ux3c0ux3b5ux3c1ux3afux3bbux3b7ux3c8ux3b7}}

Η ανθρωποκεντρική σχεδίαση έχει στόχο τη σχεδίαση και τη βελτίωση των
συστημάτων διάδρασης ανθρώπου και υπολογιστή. Οι περισσότερες τεχνικές,
κυρίως στην πρακτική εφαρμογή τους, δίνουν έμφαση στη βελτίωση
συστημάτων που υπάρχουν ή συστημάτων που βρίσκονται στο στάδιο της
σχεδίασης. Η βελτιστοποίηση ενός συστήματος είναι ένα σημαντικό θέμα,
αλλά ακόμη σημαντικότερο είναι το να αποκτήσουμε την αυτοπεποίθηση της
καταλληλότητας των προδιαγραφών του. Για αυτόν τον σκοπό, τόσο αυτό το
κεφάλαιο όσο και τα υπόλοιπα κεφάλαια του βιβλίου εστιάζουν περισσότερο
στην επανάληψη των βημάτων, παρά σε αυτά καθαυτά τα βήματα που συνιστούν
τον κύκλο της ανθρωποκεντρικής σχεδίασης. Στην προηγούμενη ενότητα
είδαμε \emph{τι} είναι η διάδραση με συσκευές χρήστη και ποιες βασικές
μορφές πήρε τις πρώτες δεκαετίες. Εδώ θα μελετήσουμε το \emph{πώς} θα
σχεδιάσουμε τη διάδραση.

\hypertarget{ux3b2ux3b5ux3bbux3c4ux3b9ux3ceux3bdux3bfux3bdux3c4ux3b1ux3c2-ux3c4ux3b9ux3c2-ux3b1ux3bdux3b8ux3c1ux3ceux3c0ux3b9ux3bdux3b5ux3c2-ux3b4ux3c5ux3bdux3b1ux3c4ux3ccux3c4ux3b7ux3c4ux3b5ux3c2}{%
\subsection{Βελτιώνοντας τις ανθρώπινες
δυνατότητες}\label{ux3b2ux3b5ux3bbux3c4ux3b9ux3ceux3bdux3bfux3bdux3c4ux3b1ux3c2-ux3c4ux3b9ux3c2-ux3b1ux3bdux3b8ux3c1ux3ceux3c0ux3b9ux3bdux3b5ux3c2-ux3b4ux3c5ux3bdux3b1ux3c4ux3ccux3c4ux3b7ux3c4ux3b5ux3c2}}

Αν και μας ενδιαφέρει η κατασκευή συστημάτων διάδρασης, εδώ θα
εστιάσουμε περισσότερο στη σχεδίαση της διάδρασης μεταξύ ανθρώπου και
συσκευής, γιατί η διάδραση εξαρτάται εξίσου από την αντίληψη που έχει ο
άνθρωπος για τη συσκευή αλλά και από τη λειτουργία της αντίστοιχης
συσκευής. Υπάρχει η αντίληψη ότι η λεπτομερής σχεδίαση της διάδρασης
πριν από την υλοποίηση των αντίστοιχων λειτουργιών του συστήματος μπορεί
να προσφέρει αποτελεσματικότερη διάδραση και επιπλέον μειωμένο κόστος
και χρόνο ανάπτυξης. Πράγματι, η επαναληπτική σχεδίαση, η κατασκευή και
η αξιολόγηση πρωτοτύπων επιτρέπει την οικονομική και τη γρήγορη απόρριψη
ιδεών που δεν είναι αποτελεσματικές. Μια συμπληρωματική θεώρηση
τοποθετεί την κατασκευή ενός ελάχιστου εφικτού προϊόντος στο κέντρο της
σχεδίασης, το οποίο χρησιμοποιείται από την ομάδα ανάπτυξης για να
φτιαχτούν νέα συστήματα με την τεχνική της αναδρομής. Σε αυτήν την
περίπτωση, το πρωτότυπο χρησιμοποιείται ενεργά και σταδιακά
μετασχηματίζεται μαζί με την ομάδα ανάπτυξης. Η ανάπτυξη αυτής της
δεξιότητας, της κατασκευής της διάδρασης, αν και φαίνεται κοινή λογική
δεν είναι καθόλου εύκολη στην πράξη. Μαθαίνεται μόνο με την εμπειρία,
όπως για παράδειγμα μαθαίνει κάποιος να γράφει ή να ζωγραφίζει καλύτερα.
Μαζί με την υλοποίηση, η κατανόηση των αναγκών του χρήστη και η
τεκμηρίωση ενός αξιακού συστήματος για την κατασκευή και την αξιολόγηση
της διάδρασης είναι οι βασικοί πυλώνες που θα μελετήσουμε στις επόμενες
ενότητες.

\leavevmode\vadjust pre{\hypertarget{fig:apple-lisa}{}}%
\begin{figure}
\hypertarget{fig:apple-lisa}{%
\centering
\includegraphics{images/apple-lisa.jpg}
\caption{Εικόνα 1: Το σύστημα Apple Lisa κατασκευάστηκε με στόχο την
αγορά γραφείου και ήταν το πρώτο σύστημα της εταιρίας με γραφικό
περιβάλλον εργασίας. Αν και απέτυχε εμπορικά, οι τεχνολογίες διάδρασης
ενσωματώθηκαν στον Apple Macintosh.}\label{fig:apple-lisa}
}
\end{figure}

\leavevmode\vadjust pre{\hypertarget{fig:apple-macintosh}{}}%
\begin{figure}
\hypertarget{fig:apple-macintosh}{%
\centering
\includegraphics{images/apple-macintosh-1984.jpg}
\caption{Εικόνα 2: Η πρώτη εμπορική επιτυχία για τη γραφική επιφάνεια
εργασίας ήρθε με τον υπολογιστή Apple Macintosh, ο οποίος θεωρείται
εγγονός του Xerox Star, γιος του Apple Lisa και ο βασικός πρόγονος για
όλες τις σύγχρονες γραφικές επιφάνειες εργασίας, όπως τα Microsoft
Windows 95. O Macintosh κατάφερε να συνθέσει και να βελτιώσει πολλές
πτυχές της διεπαφής με τον χρήστη και να τις προσφέρει σε μια μορφή και
με μια λειτουργία που αποτελούν σημεία αναφοράς για τις επόμενες
δεκαετίες.}\label{fig:apple-macintosh}
}
\end{figure}

Για αρκετές δεκαετίες, η βασική αξία στην κατασκευή της διάδρασης είναι
η ευχρηστία ενός συστήματος, η οποία συνήθως ορίζεται ως η ευκολία
χρήσης του συστήματος από κάποιον χρήση με την ελάχιστη δυνατή
εκπαίδευση. Αν και αυτός ο στόχος έχει οδηγήσει στη δημιουργία πολύ
εύχρηστων συστημάτων διάδρασης, τα οποία είναι προσβάσιμα ακόμη και από
νήπια, έχει και κάποια σημαντικά μειονεκτήματα. Το σημαντικότερο
πρόβλημα είναι πως θεωρεί τον ανθρώπινο παράγοντα ως μια στατική
οντότητα, η οποία δεν αλλάζει και δεν βελτιώνεται, κάτι που μετατρέπεται
σε αυτοεκπληρούμενη προφητεία κατά την διάθεση των αντίστοιχων εύχρηστων
συστημάτων. Πράγματι, οι χρήστες αυτών των συστημάτων δεν χρειάζεται να
γνωρίζουν πολλά και τελικά γνωρίζουν όλο και λιγότερα, ενώ, ταυτόχρονα,
ακονίζουν όλο και λιγότερο τις δεξιότητες τους. Επιπλέον, η προσήλωση
στην ευχρηστία έχει οδηγήσει στην κατασκευή νέων συστημάτων διάδρασης
που μοιάζουν μόνο με τα προηγούμενα τους, γεγονός που έχει καθυστερήσει
την αναζήτηση εναλλακτικών προς άλλες κατευθύνσεις. \footnote{(\textbf{Εικόνα?})~1
  Apple Lisa 1983 (Apple)} \footnote{(\textbf{Εικόνα?})~2 Apple
  Macintosh 1984 (wikimedia)}

Ίσως έχετε συναντήσει ξανά τον όρο της ανθρωποκεντρικής σχεδίασης της
διάδρασης ανθρώπου και υπολογιστή με έμφαση στη μοντελοποίηση του χρήστη
και με σκοπό την αυτοματοποίηση των διεργασιών του. Σε αυτό το βιβλίο η
έμφαση δεν δίνεται στη μοντελοποίηση των δεξιοτήτων και της συμπεριφοράς
του χρήστη, ούτε στην αυτοματοποίηση των δραστηριοτήτων του (έμμεση
διάδραση). Η έμφαση της ανθρωποκεντρικής σχεδίασης δίνεται στη σχεδίαση
και στην υλοποίηση της διάδρασης με συσκευές χρήστη για τις περιπτώσεις
όπου απαιτείται η ενεργή συμβολή του χρήστη (άμεση διάδραση). Για να
μπορέσουμε να κατανοήσουμε τη διάδραση ανθρώπου και υπολογιστή, θα
πρέπει να καταλάβουμε πρώτα τις ιδιότητες του ανθρώπου καθώς και εκείνες
του υπολογιστή. Η κατασκευή ενός διαδραστικού συστήματος υπολογισμού
βασίζεται σε προδιαγραφές που εκφράζουν τις ανάγκες που αυτό θα
εξυπηρετεί. \footnote{Papanek and Fuller (1972), Thackara (2006)} Αυτές
οι ανάγκες, με τη σειρά τους, καταγράφονται σε σχέση με τις δεξιότητες
του ανθρώπου, του υπολογιστή, καθώς και σε σχέση με τις ιδιότητες της
μεταξύ τους διάδρασης. \footnote{Norman (2013)}

Η κατασκευή της διάδρασης δεν είναι μόνο η σχεδίαση της εμφάνισης και
των λειτουργιών μιας συσκευής ή ενός συστήματος συσκευών και υπηρεσιών,
αλλά κάτι συνολικότερο, το οποίο λαμβάνει υπόψη του τον τρόπο που οι
άνθρωποι σκέφτονται και επιτελούν τις εργασίες τους. Επίσης, οι συσκευές
που χρησιμοποιούν οι άνθρωποι είναι κάτι περισσότερο από τα συστήματα
εισόδου και εξόδου, οπότε η σχεδίαση πρέπει να εξετάσει ένα ολόκληρο
οικοσύστημα, το οποίο αποτελείται από τεκμηρίωση, υποστήριξη, εκπαίδευση
και διαδικασίες. \footnote{(\textbf{Εικόνα?})~3 Ελάχιστο εφικτό σύστημα
  NLS (Doug Engelbart Institute)} \footnote{(\textbf{Εικόνα?})~4
  Αξιολόγηση εναλλακτικών συσκευών εισόδου (Doug Engelbart Institute)}

Επομένως, υπάρχουν περιπτώσεις στις οποίες η μελέτη της συνολικής
υπάρχουσας κατάστασης μπορεί να δείξει ότι δεν απαιτείται κάποιο νέο
τεχνολογικό σύστημα, αλλά απλώς μια αναδιάταξη, ή μια βελτίωση των
επιμέρους τμημάτων αυτού του οικοσυστήματος. Για αυτόν τον λόγο,
κρίνεται σκόπιμο να θεωρήσουμε ότι δεν κατασκευάζουμε απλά τη διάδραση
με μια συσκευή ή με ένα σύστημα, αλλά κάτι ευρύτερο. Κατασκευάζουμε μια
παρέμβαση στον τρόπο που ένας ή περισσότεροι άνθρωποι εκτελούν
διαδικασίες είτε αυτές είναι εργασιακές είτε είναι ψυχαγωγικές. Σε αυτό
το πλαίσιο, η ερώτηση που θα μας απασχολήσει στο παρόν κεφάλαιο είναι
\emph{τι είναι η κατασκευή της διάδρασης ως διαδικασία;}

\leavevmode\vadjust pre{\hypertarget{fig:nls-radar-keypad}{}}%
\begin{figure}
\hypertarget{fig:nls-radar-keypad}{%
\centering
\includegraphics{images/nls-radar-keypad.jpg}
\caption{Εικόνα 3: Το σύστημα NLS που παρουσιάστηκε τελικά είχε ποντίκι
με τρία κουμπιά, ακόρντα πέντε πλήκτρων και ορθογώνια οθόνη. Μερικά
χρόνια νωρίτερα η ερευνητική ομάδα ξεκίνησε με το διαθέσιμο υλικό
εκείνης της εποχής, που ήταν η στρογγυλή οθόνη του ραντάρ και ένα
χειροποίητο ποντίκι με ένα πλήκτρο, έτσι ώστε να κατασκευάσει μια πρώτη
έκδοση για το λογισμικό διάδρασης και να διαπιστώσει στην πράξη ποιες
αλλαγές απαιτούνται για την επόμενη έκδοση των συσκευών διάδρασης σε μια
επαναλληπτική διαδικασία σχεδίασης.}\label{fig:nls-radar-keypad}
}
\end{figure}

\leavevmode\vadjust pre{\hypertarget{fig:nls-input}{}}%
\begin{figure}
\hypertarget{fig:nls-input}{%
\centering
\includegraphics{images/nls-input.jpg}
\caption{Εικόνα 4: Η εφεύρεση και η τελική επιλογή της συσκευής εισόδου
ποντίκι για το σύστημα NLS δεν ήταν τυχαία, αλλά βασιζόταν σε μια
συγκριτική αξιολόγηση εναλλακτικών συσκευών εισόδου. Για τον σκοπό αυτό,
δοκίμασαν τις επιδόσεις των χρηστών με πολλές διαφορετικές συσκευές
εισόδου, όπως ήταν η πένα, ο τροχός κύλισης, ο κατευθυντικός μοχλός,
καθώς και διεπαφές για τα πόδια και το κεφάλι, αλλά το ποντίκι ήταν
καλύτερο σε ακρίβεια και ταχύτητα.}\label{fig:nls-input}
}
\end{figure}

Τα τρία βασικά στάδια της ανθρωποκεντρικής σχεδίασης (κατανόηση των
αναγκών του χρήστη, εναλλακτικά σχέδια και κατασκευή πρωτοτύπου,
αξιολόγηση πρωτοτύπων με χρήστες) εκτελούνται κυκλικά και άρα η
επανάληψη βρίσκεται στον πυρήνα της ανθρωποκεντρικής σχεδίασης της
διάδρασης.

Η ανθρωποκεντρική σχεδίαση της διάδρασης ανθρώπου και υπολογιστή δεν
είναι κάτι νέο. Αν μάλιστα θεωρήσουμε και τις δράσεις που έχουν συμβεί
έξω από την επιστημονική κοινότητα, μπορούμε να δούμε ότι είναι τόσο
παλιά όσο η προσπάθεια κάποιων κατασκευαστών να φτιάξουν μηχανές και
εργαλεία που βασίζονται στις δυνατότητες και τις δεξιότητες του
ανθρώπου. Ίσως, το πιο ενδιαφέρον παράδειγμα από το μακρινό παρελθόν να
είναι το σφυρί, ένα ξύλο δεμένο σε μια πέτρα, το οποίο βελτίωσε πάρα
πολύ την ευχρηστία της πέτρας, την οποία μέχρι τότε έπρεπε οι άνθρωποι
να τη χρησιμοποιήσουν κρατώντας την. Αντίστοιχα, μπορούμε να θεωρήσουμε
ότι και η διαδικασία της ανθρωποκεντρικής σχεδίασης δεν είναι κάτι
καινούργιο, αφού η δοκιμή και το σφάλμα είναι μια σχεδόν διαισθητική
δραστηριότητα, η οποία συμβαίνει σε κάθε διαδικασία αλληλεπίδρασης με το
περιβάλλον μας. Η διαφορά είναι ότι η περιοχή της διάδρασης ανθρώπου και
υπολογιστή έχει καταγράψει μια περισσότερο συστηματική μεθοδολογία για
την παραπάνω διαδικασία, η οποία μέχρι τότε συνέβαινε πιο πολύ ως
αυτοσχεδιασμός, και όχι συστηματικά.

Στο πρόσφατο παρελθόν, η αρχή της ανθρωποκεντρικής σχεδίασης εντοπίζεται
στην περιοχή της Εργονομίας, η οποία μελετά τις σωματικές δυνατότητες
του ανθρώπου για κίνηση. Στην περίπτωση της εργονομίας, η
ανθρωποκεντρική σχεδίαση έχει σημαντικό σύμμαχο τη σχετικά καλώς
ορισμένη διακύμανση των μετρικών που περιγράφουν το ανθρώπινο σώμα και
τις κινήσεις του. Στην πορεία, ήρθε να προστεθεί και η περιοχή της
γνωστικής επιστήμης, η οποία δίνει έμφαση στις γνωστικές δυνατότητες του
ανθρώπου για την αντίληψη και την επεξεργασία της πληροφορίας. Στην
περίπτωση της γνωστικής επιστήμης, αν και γίνονται επαναληπτικά
πειράματα επιβεβαίωσης, είναι σίγουρα πιο δύσκολο να θεμελιωθεί μια
θεωρία με βεβαιότητα, αφού αντιλαμβανόμαστε τις λειτουργίες της σκέψης
έμμεσα και όχι άμεσα. Στους παραπάνω βασικούς πυλώνες (γνωστική επιστήμη
και εργονομία), ήρθε να προστεθεί προσφάτως η συναισθηματική και η
αισθητική διάσταση της σχεδίασης για τον άνθρωπο, η οποία έχει τις ρίζες
της στις περιοχές της γραφιστικής και των εφαρμοσμένων τεχνών. Επίσης, η
καλύτερη κατανόηση της διάδρασης του χρήστη με συσκευές επεκτείνεται και
στην ανθρώπινη ψυχολογία, αφού στην πράξη είναι αδύνατο να διαχωρίσουμε
τη λογική από το συναίσθημα. Φαίνεται ότι οι χρήστες θεωρούμε μια όμορφη
διάδραση πιο εύχρηστη, αν και μετρώντας την εν λόγω ευχρηστία με
αντικειμενικά κριτήρια, όπως είναι χρόνος ολοκλήρωσης μιας λειτουργίας,
μπορεί να αποδειχθεί πως δεν είναι.

Ο στόχος της ανθρωποκεντρικής σχεδίασης δεν είναι απλά η βελτιστοποίηση
μιας σχεδίασης, αλλά, πρωτίστως, η εύρεση των ιδιοτήτων της. Αρχικά, οι
περισσότερες μελέτες έδιναν έμφαση στην ακρίβεια της χρήσης ποσοτικών
μεθόδων έρευνας και αξιολόγησης (π.χ. τη χρονομέτρηση της ολοκλήρωσης
μιας λειτουργίας) με στόχο τη βελτιστοποίηση μιας μεμονωμένης
λειτουργίας ή ολόκληρης της σχεδίασης. Τη δεκαετία του 1990, οι δοκιμές
ευχρηστίας και οι ποσοτικές μέθοδοι ήταν πολύ δημοφιλείς και είχαν στόχο
να βελτιστοποιήσουν τον τρόπο που λειτουργούσαν οι προδιαγραφές
σχεδίασης. Σε πολλές περιπτώσεις, οι κατασκευαστές, αν και όντως
βελτίωναν μια σχεδίαση, δεν δούλευαν πάνω σε εκείνη που θα γινόταν
αποδεκτή από τους χρήστες. Τη δεκαετία του 2000, οι κατασκευαστές της
διάδρασης άρχισαν σταδιακά να διερευνούν με ποιoν τρόπο θα σχεδιάσουν
προϊόντα με μεγαλύτερη αποδοχή από το κοινό τους και έτσι άρχισαν να
πειραματίζονται με ποιοτικές μεθόδους έρευνας, κάνοντάς τες πιο
δημοφιλείς και πιο αποδεκτές από την επιχειρηματική κοινότητα. Στην
πράξη, για την κατασκευή της διάδρασης χρησιμοποιούνται διερευνητικές
τεχνικές με πρωτότυπα χαμηλής πιστότητας κατά το πρώτο στάδιο της
κατανόησης των αναγκών και, σταδιακά, με την κατασκευή του πρωτοτύπου
υψηλής πιστότητας εφαρμόζονται περισσότερο ποσοτικές μέθοδοι, κατά τη
φάση της αξιολόγησης με χρήστες.

Ανάμεσα στις πιο δημοφιλείς τεχνικές κατανόησης του χρήστη μπορούμε να
ξεχωρίσουμε την εθνογραφία, η οποία ξεκίνησε από τις μελέτες των
ανθρωπολόγων και προσαρμόστηκε στη σχεδίαση της διάδρασης. Εν συντομία,
όπως οι ανθρωπολόγοι ενσωματώνουν τους εαυτούς τους στην καθημερινότητα
πολύ διαφορετικών πολιτισμών, έτσι και οι σχεδιαστές των νέων διάχυτων
ΗΥ, θα πρέπει, είτε οι ίδιοι, είτε μέσω άλλων ειδικευμένων για αυτόν τον
σκοπό ερευνητών, να μπουν στη ρευστή καθημερινότητα των ανθρώπων για
τους οποίους καλούνται να σχεδιάσουν νέα συστήματα διάδρασης, τα οποία
μπορεί απλά να διευκολύνουν, να επαυξάνουν, ακόμη και να αλλάζουν ριζικά
τον τρόπο με τον οποίο ένας χρήστης ή, ακόμη δυσκολότερα μια ομάδα
ανθρώπων σκέφτονται, αποφασίζουν και δρουν σε έναν κόσμο που γίνεται
αντιληπτός αλλά και επηρεάζεται από διάχυτους υπολογιστές. Αν και η
εθνογραφική μέθοδος είναι μια δημοφιλής επιλογή στη σχεδίαση νέων
συστημάτων, μοιράζεται αρκετές τεχνικές (π.χ. την παρατήρηση) με άλλες
μεθόδους, οπότε η βέλτιστη κατανόηση και η χρήση της προϋποθέτει και τη
γνώση των συμπληρωματικών και, πολλές φορές, επικαλυπτόμενων μεθόδων και
τεχνικών (π.χ. συνεντεύξεις, ομάδες εστίασης, πολιτισμική διερεύνηση,
κτλ.).

Η πολιτισμική διερεύνηση είναι μία από τις πιο απλές και δημοφιλείς
τεχνικές για την καταγραφή της συμπεριφοράς που έχουν οι χρήστες και την
έμμεση αποκάλυψη των αναγκών τους. Η πολιτισμική διερεύνηση βασίζεται
στην αποστολή ενός φακέλου με αντικείμενα καθημερινής χρήσης, τα οποία
έχουν απλές οδηγίες για τους χρήστες. Για παράδειγμα, ένας φάκελος
πολιτισμικής διερεύνησης στα τέλη της δεκαετίας του 1990 συνήθως
περιείχε μια φωτογραφική μηχανή μίας χρήσης, καθώς και την παρότρυνση να
βγάλουν φωτογραφία κάποιο αγαπημένο αντικείμενο ή κάποια αγαπημένη
δραστηριότητα. Εκτός από τη φωτογραφική μηχανή, ένα ακόμη δημοφιλές
αντικείμενο είναι το ημερολόγιο, το οποίο ο χρήστης συμπληρώνει
καταγράφοντας τις δραστηριότητές του, όπως εκπομπές στην τηλεόραση και
συναντήσεις με φίλους. Στο τέλος της χρονικής περιόδου, ο φάκελος της
πολιτισμικής διερεύνησης αποστέλλεται στους ερευνητές, οι οποίοι
χρησιμοποιούν τα περιεχόμενα του φακέλου (φωτογραφίες, αυτοκόλλητα
post-it κτλ.) στον χώρο σχεδίασης, ώστε να μπουν καλύτερα στον κόσμο του
χρήστη. Αν και τα περιεχόμενα του συμπληρωμένου φακέλου πολιτισμικής
διερεύνησης δεν δείχνουν μονοσήμαντα τις προδιαγραφές, το νόημα
βρίσκεται περισσότερο στην καλύτερη εμβύθιση της ομάδας σχεδίασης στον
κόσμο του χρήστη, έτσι ώστε τελικά οι προδιαγραφές που θα καθοριστούν να
είναι συμβατές με το αντίστοιχο πλαίσιο χρήσης του προϊόντος.

Εκτός από την τεχνική της πολιτισμικής διερεύνησης που είδαμε παραπάνω,
άλλη μια τεχνική που είναι απλή, αποτελεσματική και δημοφιλής για την
κατανόηση των ανθρώπινων αναγκών είναι ο καθορισμός αντιπροσωπευτικών
χρηστών. Η τεχνική αυτή βασίζεται στην περιγραφή των ιδιοτήτων ενός
χρήστη, όπως είναι τα δημογραφικά στοιχεία του, οι προτιμήσεις του, και
οι συνήθειές του. Οι αντιπροσωπευτικοί χρήστες οι οποίοι παρουσιάζονται
στις περσόνες μπορεί να είναι υπαρκτά πρόσωπα, αλλά μπορεί να είναι και
φανταστικά πρόσωπα, τα οποία, ομοίως, ανταποκρίνονται σε κάποιες
κατηγορίες χρήστη της εφαρμογής που αναπτύσσουμε. Οι περσόνες αυτές
κατασκευάζονται σε συνεργασία με τους τελικούς χρήστες της εφαρμογής και
με δεδομένα που μαζεύονται από ερωτηματολόγια και συνεντεύξεις. Οι
περσόνες χρησιμοποιούνται από την ομάδα ανάπτυξης σε συνδυασμό με την
τεχνική του αφηγηματικού σεναρίου, το οποίο θα δούμε στην επόμενη
ενότητα της κατασκευής πρωτοτύπου χαμηλής πιστότητας. Για την ακρίβεια,
οι περσόνες είναι, συνήθως, οι πρωταγωνιστές ή κάποιοι σημαντικοί ρόλοι
στα σενάρια που περιγράφουν τη διάδραση ανάμεσα στους χρήστες και στους
υπολογιστές.

Πέρα από τις παραπάνω συστηματικές προσεγγίσεις για την κατανόηση των
αναγκών του χρήστη, υπάρχουν και περισσότερο δημιουργικές απόψεις, οι
οποίες βασίζονται στον αυτοσχεδιασμό και στην έμπνευση. Για παράδειγμα,
για αρκετές από τις συσκευές διάδρασης της Apple δεν έχει γίνει
συστηματική έρευνα των αναγκών του χρήστη, αλλά έχει χρησιμοποιηθεί η
έμπνευση, η διαίσθηση και η δημιουργικότητα της ομάδας σχεδίασης και της
διοίκησης. Τόσο ο σχεδιασμός του iPod όσο και ο σχεδιασμός του iPhone
έχουν στοιχεία διάδρασης που μέχρι τότε δεν είχαν εμφανιστεί σε κάποιο
άλλο εμπορικό προϊόν, αλλά μπήκαν σε αυτά τα προϊόντα, γιατί ο
κατασκευαστής πίστευε ότι αυτό είναι που έχουν ανάγκη οι χρήστες. Είναι
φανερό ότι μια τέτοια προσέγγιση έχει πολύ μεγάλο ρίσκο αποτυχίας, ενώ
απαιτεί και μεγάλα αποθέματα αυτοπεποίθησης, αλλά αν πετύχει, τότε το
αποτέλεσμα είναι ο νικητής να βρίσκεται πολύ μπροστά από τους
ανταγωνιστές, οι οποίοι είναι αναγκασμένοι να επαναπροσδιορίσουν τις
κατηγορίες προϊόντων που προσφέρουν, αφού οι ανάγκες των χρηστών δεν θα
είναι πλέον ίδιες. Είναι χαρακτηριστικό ότι ο αρχικός σχεδιαστής
γραφικών στην εταιρεία Google ανέφερε ότι ένας από τους λόγους της
παραίτησής του ήταν ότι η κυρίαρχη κουλτούρα τεχνοκρατικής αντίληψης της
εταιρείας είχε φτάσει στο σημείο να κάνουν δοκιμές για το αν το πάχος
μιας γραμμής θα έπρεπε να είναι δύο ή τέσσερα εικονοστοιχεία. Επομένως,
εκτός από τη μελέτη των χρηστών, μια ακόμη τεχνική κατανόησης των
αναγκών είναι η διαίσθηση ενός έμπειρου και ταλαντούχου σχεδιαστή, ο
οποίος αντλεί την έμπνευση του από μια ασυνείδητη σύνθεση γνώσεων.

\hypertarget{ux3b5ux3bbux3acux3c7ux3b9ux3c3ux3c4ux3bf-ux3b5ux3c6ux3b9ux3baux3c4ux3cc-ux3c0ux3c1ux3bfux3caux3ccux3bd}{%
\subsection{Ελάχιστο εφικτό
προϊόν}\label{ux3b5ux3bbux3acux3c7ux3b9ux3c3ux3c4ux3bf-ux3b5ux3c6ux3b9ux3baux3c4ux3cc-ux3c0ux3c1ux3bfux3caux3ccux3bd}}

\leavevmode\vadjust pre{\hypertarget{fig:engelbart-mouse}{}}%
\begin{figure}
\hypertarget{fig:engelbart-mouse}{%
\centering
\includegraphics{images/engelbart-mouse.jpg}
\caption{Εικόνα 5: Το αρχικό πρωτότυπο για το ποντίκι είναι ακριβώς το
ίδιο μορφολογικά με τις αντίστοιχες συσκευές που παράγονται με μεγάλη
επιτυχία πενήντα χρόνια μετά, αν και, φυσικά, έχουν βελτιωθεί πολλές
επιμέρους λειτουργικές ιδιότητές του. Οι δύο τροχοί κύλισης δεν
επιτρέπουν τη διαγώνια κίνηση, αλλά αυτό δεν είναι μεγάλο πρόβλημα, αφού
η χρήση του εστιάζεται, κυρίως, στην επεξεργασία κειμένου. Η συσκευή
αυτή είναι μια σημαντική καινοτομία γιατί διαφέρει σημαντικά από
παρόμοιες συσκευές και γιατί επέτρεψε τη διεξαγωγή πειραμάτων, τα οποία
απέδειξαν την υπεροχή του έναντι των εναλλακτικών συσκευών
εισόδου.}\label{fig:engelbart-mouse}
}
\end{figure}

\leavevmode\vadjust pre{\hypertarget{fig:nls-mouse}{}}%
\begin{figure}
\hypertarget{fig:nls-mouse}{%
\centering
\includegraphics{images/nls-mouse.jpg}
\caption{Εικόνα 6: Οι χειριστές του NLS, μετά από μερικές ώρες
εκπαίδευσης, μπορούν να χρησιμοποιήσουν το πληκτρολόγιο ακόρντων σε
συνδυασμό με το ποντίκι τριών πλήκτρων για να κάνουν επεξεργασία
κειμένου και πληκτρολόγηση νέων λέξεων, χωρίς να μετακινήσουν τα χέρια
τους στο κεντρικό πληκτρολόγιο. Με αυτόν τον τρόπο, εκτός από την
ταχύτητα εκτέλεσης της διεργασίας, μειώνεται και η επιβάρυνση των χεριών
που είναι σημαντική για όποιον χρησιμοποιεί τέτοια συστήματα μερικές
ώρες κάθε μέρα.}\label{fig:nls-mouse}
}
\end{figure}

Η ασαφής φύση της διάδρασης, όπως την περιγράψαμε στην προηγούμενη
ενότητα, δεν επιτρέπει σε πολλές περιπτώσεις τη διατύπωση προδιαγραφών,
οι οποίες θα υλοποιηθούν σε ένα επόμενο βήμα της κατασκευής, όπως
συνηθίζεται στις επιστήμες των μηχανικών. Αντίθετα, η κατασκευή της
διάδρασης, συνήθως, βασίζεται σε ένα λειτουργικό υπόδειγμα, το οποίο
παίζει τον ρόλο των ρευστών προδιαγραφών. \footnote{(\textbf{Εικόνα?})~5
  Αρχικό υπόδειγμα για το ποντίκι (Wikipedia)} \footnote{(\textbf{Εικόνα?})~6
  Ποντίκι με τρία πλήκτρα (Doug Engelbart Institute)}

Σε αυτήν την ενότητα περιγράφουμε με περισσότερη λεπτομέρεια τη
διαδικασία, τις τεχνικές και τα εργαλεία για την κατασκευή υποδειγμάτων
διάδρασης με συσκευές χρήστη. Κάθε τεχνική παράγει ένα υπόδειγμα
διαφορετικής πιστότητας και όσο μεγαλύτερη είναι η απαιτούμενη
λειτουργικότητα του υποδείγματος, τόσο περισσότερο χρόνο θέλουμε για να
το φτιάξουμε, ή για να το αλλάξουμε. Επομένως, η επιλογή του αναγκαίου
βαθμού πιστότητας του υποδείγματος και ο καθορισμός της κατάλληλης
τεχνικής κατασκευής του είναι πολύ σημαντικές παράμετροι και
περιγράφονται σε αυτήν την ενότητα.

Η διαδικασία της κατασκευής της διάδρασης είναι ένας κύκλος επανάληψης
στον οποίο δύσκολα θα προσδιορίσουμε πού αρχίζει και πού τελειώνει.
Ειδικά για τα προϊόντα ευρείας χρήσης, είναι η ίδια η χρήση τους που
επαναπροσδιορίζει τη φύση τους σε έναν αέναο κύκλο. Κάποιος θα μπορούσε
να υποστηρίξει ότι όλα αυτά δεν είναι καθόλου νέα, και ότι όλες οι
παραδοσιακές βιομηχανίες (κτήρια, αυτοκίνητα) σταδιακά μεταλλάσσονται
για να εξυπηρετήσουν τους χρήστες τους. Αυτό είναι αλήθεια, αλλά οι
αλλαγές που συνήθως συμβαίνουν σε όλες τις παραπάνω βιομηχανίες είναι
τόσο σταδιακές χρονικά και τόσο προσθετικές δομικά, που και πάλι
αναδεικνύεται αυτή η ιδιαιτερότητα της κατασκευής της διάδρασης, σε
σχέση με τις πολύ συγγενείς του περιοχές.

Συνοπτικά, η κεντρική διαφορά κατασκευής της διάδρασης από άλλες
επιστήμες του μηχανικού είναι ότι τόσο η διαδικασία ανάπτυξης όσο και η
τελική κατασκευή αποτελούν στάδια ενός συνεχώς ανατροφοδοτούμενου
κύκλου. Για παράδειγμα, η δημοφιλής υπηρεσία Google Mail για πολλά
χρόνια είχε την ετικέτα βήτα, ενώ ήταν πλήρως λειτουργική. Βλέπουμε,
λοιπόν, ότι στην περίπτωση του λογισμικού, οι \emph{επίσημες τελικές
εκδόσεις} είναι απλώς προφορικές ή γραπτές δηλώσεις και συμβάσεις του
κατασκευαστή. Σε αντίθεση με ένα σπίτι, το οποίο μετά την παράδοσή του
στον χρήστη δέχεται ελάχιστες μετατροπές, ακόμη κι έπειτα από πολλά
χρόνια.

\leavevmode\vadjust pre{\hypertarget{fig:office-schematic}{}}%
\begin{figure}
\hypertarget{fig:office-schematic}{%
\centering
\includegraphics{images/office-schematic.png}
\caption{Εικόνα 7: Ίσως το πιο διάσημο πρωτότυπο χαμηλής πιστότητας να
είναι το σχεδιάγραμμα της γραφικής επιφάνειας εργασίας που έγινε σε μια
χαρτοπετσέτα από τους ερευνητές του Xerox PARC και χρησιμοποιούσε τη
μεταφορά του γραφείου για την απεικόνιση των
διεπαφών.}\label{fig:office-schematic}
}
\end{figure}

\leavevmode\vadjust pre{\hypertarget{fig:dynabook-spacewar}{}}%
\begin{figure}
\hypertarget{fig:dynabook-spacewar}{%
\centering
\includegraphics{images/dynabook-spacewar.jpg}
\caption{Εικόνα 8: Η βασική εφαρμογή που οδήγησε τον σχεδιασμό του
Dynabook ήταν το βιντεοπαιχνίδι Spacewar, το οποίο εκείνη την εποχή ήταν
σημείο αναφοράς για όλους τους ερευνητές που είχαν πρόσβαση σε έναν
κεντρικό υπολογιστή. Όπως ακριβώς οι ερευνητές υλοποιούσαν, έπαιζαν και
έκαναν μετατροπές στον πηγαίο κώδικα του Spacewar, έτσι και οι χρήστες
του Dynabook θα μπορούσαν να έχουν πρόσβαση σε ένα νέο μέσο επικοινωνίας
και έκφρασης.}\label{fig:dynabook-spacewar}
}
\end{figure}

Τα αρχικά προσχέδια είναι και αυτά πολύ χρήσιμα,\footnote{Buxton (2010)}
για την καλύτερη κατανόηση της διάδρασης, και, κυρίως, για την
επικοινωνία μεταξύ των μελών της ομάδας ανάπτυξης. \footnote{(\textbf{Εικόνα?})~7
  Αρχικό προσχέδιο της επιφάνειας εργασίας (Xerox PARC)} \footnote{(\textbf{Εικόνα?})~8
  Παιδιά παίζουν το Spacewar στο Dynabook (Alan Kay)} Τα αρχικά
προσχέδια, συνήθως, έχουν τη μορφή του αφηγηματικού σεναρίου\footnote{Carroll
  (2000)} και των ενδεικτικών οθονών, αλλά υπάρχουν και άλλες επιλογές,
όπως η ιστοριογραφία, το βίντεο, οι διαδραστικές διαφάνειες και πολλά
άλλα εξειδικευμένα εργαλεία κατασκευής προσχεδίου για το υπόδειγμα.
\footnote{(\textbf{Εικόνα?})~9 Apple Knowledge Navigator (Apple)}
\footnote{(\textbf{Εικόνα?})~10 Sun Starfire (Sun Microsystems)}

Υπάρχουν διάφορες τεχνικές κατασκευής υποδείγματος ανάλογα με το στάδιο
ανάπτυξης και το είδος ενός νέου προϊόντος. Τα τμήματα έρευνας και
ανάπτυξης μιας εταιρείας εντοπίζουν νέες ανάγκες, κατασκευάζουν
υποδείγματα και κάνουν δοκιμές με χρήστες πριν καταλήξουν στο τελικό
προϊόν. Η κατασκευή υποδείγματος είναι μια διαδικασία που κάνουν όλες οι
εταιρείες, η οποία, όμως, διαφέρει ανάλογα με το είδος του προϊόντος και
την οργάνωση της εταιρείας. Για παράδειγμα, μια μεγάλη εταιρεία,
συνήθως, έχει σαφώς ορισμένες διαδικασίες κατασκευής υποδείγματος, οι
οποίες καθορίζουν τον αριθμό των προσπαθειών που γίνονται για κάθε
έκδοση του προϊόντος, καθώς και τις προδιαγραφές του. Αντίθετα, οι
μικρές καινοτόμες εταιρείες χρησιμοποιούν το ίδιο το υπόδειγμα ως
προδιαγραφή. Ακόμη, μπορεί να υπάρχουν διαφορές στα εργαλεία και τα
υλικά που χρησιμοποιούνται για την κατασκευή του υποδείγματος ανάλογα με
το είδος του προϊόντος. Για παράδειγμα, στην αυτοκινητοβιομηχανία
ξεκινούν με σχεδιαγράμματα, συνεχίζουν με μοντέλα τριών διαστάσεων στον
υπολογιστή και καταλήγουν στην κατασκευή απτών υποδειγμάτων. Στην
κατασκευή έξυπνων κινητών τηλεφώνων χρησιμοποιούνται όλες αυτές οι
τεχνικές, ανάλογα με το στάδιο ανάπτυξης του προϊόντος. Ειδικά στην
περίπτωση της κατασκευής λογισμικού διάδρασης, \footnote{Winograd et al.
  (1996), Moggridge (2007)} η διάκριση ανάμεσα στο υπόδειγμα και στο
τελικό προϊόν είναι πολλές φορές δυσδιάκριτη, αφού πολλά από τα
λειτουργικά υποδείγματα διανέμονται ως προϊόντα, ενώ τα προϊόντα με τη
σειρά τους αποτελούν υποδείγματα για την επόμενη έκδοση του προϊόντος.

\leavevmode\vadjust pre{\hypertarget{fig:knowledge-navigator}{}}%
\begin{figure}
\hypertarget{fig:knowledge-navigator}{%
\centering
\includegraphics{images/knowledge-navigator.jpg}
\caption{Εικόνα 9: Στα τέλη της δεκαετίας του 1980 η Apple παρουσίασε
ένα βίντεο με μια μεγάλη επιτραπέζια συσκευή αφής, η οποία είχε τη
δυνατότητα πολυμέσων, υπερμέσων, καθώς και τηλεδιάσκεψης με βίντεο, τα
οποία, όμως, δεν ήταν τεχνολογικά εφικτά τότε.Αυτό δεν εμπόδισε τους
σχεδιαστές της διεπαφής να τα φανταστούν, αφού όλη η παρουσίαση ήταν ένα
ψηφιακά επεξεργασμένο βίντεο και όχι ένα πραγματικό
πρωτότυπο.}\label{fig:knowledge-navigator}
}
\end{figure}

\leavevmode\vadjust pre{\hypertarget{fig:starfire-video}{}}%
\begin{figure}
\hypertarget{fig:starfire-video}{%
\centering
\includegraphics{images/starfire-video.jpg}
\caption{Εικόνα 10: Στις αρχές της δεκαετίας του 1990 η Sun παρουσίασε
ένα βίντεο με μια μεγάλη επιτραπέζια συσκευή χειρονομίας, η οποία
βασίζεται στη συνεργασία μέσω τηλεδιάσκεψης με βίντεο, στην οποία η
πρωταγωνίστρια εργάζεται από το σπίτι της. Το βίντεο υπόδειγμα
χρησιμοποιεί την τεχνική της μπλε οθόνης για να εισάγει τις πρόσθετες
ροές βίντεο πάνω στο αρχικό σκηνοθετημένο
βίντεο.}\label{fig:starfire-video}
}
\end{figure}

Στόχος της σχεδίασης διαδραστικών συστημάτων είναι η μεγιστοποίηση της
ευχρηστίας τους. Υπάρχει μια σειρά κανόνων σχεδίασης που βασίζονται σε
προηγούμενη θεωρία ή/και εμπειρία και οι οποίοι μπορούν να μας βοηθήσουν
στον καθορισμό εύχρηστων διαδραστικών συστημάτων, συμπεριλαμβανομένων
αφηρημένων βασικών αρχών, οδηγιών και άλλων ζητημάτων σχεδίασης. Οι
σχεδιαστικές οδηγίες είναι συλλογές συμβουλών για τους σχεδιαστές
διεπαφών χρήστη, οι οποίες είναι απαραίτητες, προκειμένου να
εξασφαλιστεί ότι το τελικό προϊόν θα είναι φιλικό προς τον χρήστη.
Αρκετά βιβλία και τεχνικές αναφορές περιέχουν μεγάλους καταλόγους από
σχεδιαστικές οδηγίες. Αυτές διαιρούνται σε υποκατηγορίες με πιο
εξειδικευμένες οδηγίες σχεδίασης. Οι περισσότερες έρευνες και προτάσεις
που έχουν γίνει πάνω στις σχεδιαστικές οδηγίες αφορούν τα κλασικά
συστήματα επιτραπέζιων υπολογιστών που χρησιμοποιούνται σε ευρεία
κλίμακα. Όμως, η ραγδαία ανάπτυξη του κινητού υπολογισμού τα τελευταία
χρόνια, προκάλεσε μια έκρηξη στη ζήτηση αντίστοιχων συσκευών.

Η κατασκευή υποδειγμάτων για την κατασκευή της διάδρασης σε συσκευές
πέρα από τον επιτραπέζιο υπολογιστή είναι μια πρόκληση, η οποία είναι
περισσότερο πολύπλοκη από την κατασκευή υποδείγματος για άλλες
περιπτώσεις. Για παράδειγμα, η κατασκευή του υποδείγματος για μια
εφαρμογή που θα εκτελεστεί σε έναν επιτραπέζιο ΗΥ δεν απαιτεί τίποτα
περισσότερο από τον ίδιο τον επιτραπέζιο ΗΥ ανάπτυξης, γιατί και ο
τελικός προορισμός της εφαρμογής θα είναι σε ένα παρόμοιο υλικό και η
διάδραση με τον χρήστη θα γίνεται με τις ίδιες συσκευές εισόδου, δηλαδή,
το πληκτρολόγιο και το ποντίκι. Το ίδιο ισχύει και για την κατασκευή
ενός υποδείγματος για ένα νέο ποντίκι για τον επιτραπέζιο ΗΥ. Με
δεδομένη την εργονομία του χεριού και το πλαίσιο χρήσης του ποντικιού,
το οποίο είναι η μετακίνηση του δείκτη στην οθόνη και η επιλογή με ένα
κουμπί, ο σχεδιαστής έχει αρκετά σημεία αναφοράς στα οποία μπορεί να
βασιστεί.

Αντίθετα, η κατασκευή του υποδείγματος για μια συσκευή διάδρασης χρήστη
με κινητό ή διάχυτο ΗΥ είναι μια πρόκληση, γιατί απαιτεί τη συνεργασία
λογισμικού με την κατασκευή ειδικού υλικού διάδρασης με τον χρήστη.
Καθώς ο διάχυτος υπολογισμός θα φέρνει περισσότερες συσκευές χρήστη σε
περισσότερες πτυχές της ζωής μας, αυξάνεται η ανάγκη για κατασκευή (και
αξιολόγηση από τους χρήστες) υποδειγμάτων υψηλής πιστότητας, τα οποία
συνδυάζουν υλικό με λογισμικό. Για αυτόν τον σκοπό, οι κατασκευαστές της
διάδρασης έχουν αναπτύξει μια τεχνική που συνδυάζει ένα ειδικά φτιαγμένο
υλικό διάδρασης με τον χρήστη, με ένα λογισμικό που εκτελείται σε
επιτραπέζιο ή κινητό ΗΥ, για το οποίο οι προγραμματιστές μπορούν να
χρησιμοποιήσουν τα διαθέσιμα εργαλεία ανάπτυξης. Η τεχνική αυτή δεν
έρχεται να αντικαταστήσει τις τεχνικές κατασκευής υποδείγματος χαμηλής
πιστότητας που μελετήσαμε προηγουμένως, αλλά έρχεται να προστεθεί, ως
ένα ακόμη βήμα, στον επαναληπτικό κύκλο της κατασκευής της διάδρασης.

Ο σκοπός της κατασκευής υποδείγματος υψηλής πιστότητας είναι να το
αξιολογήσουμε με χρήστες σε εργαστηριακό περιβάλλον ή ακόμη και σε
μελέτη στο πεδίο. Τα υποδείγματα χαμηλής πιστότητας (π.χ. σενάριο,
σχεδιάγραμμα, ενδεικτικές οθόνες) είναι κατάλληλα περισσότερο για την
οπτικοποίηση και την επεξεργασία αρχικών ιδεών από τους σχεδιαστές και
τους συνεργάτες τους. Πράγματι, τα υποδείγματα χαμηλής πιστότητας είναι
χρήσιμα για τη γρήγορη και ανεπίσημη επικοινωνία μεταξύ των μελών μιας
ομάδας σχεδίασης και ανάπτυξης, αφού οι λέξεις δεν είναι σχεδόν ποτέ
αρκετές για να περιγράψουν το άρρητο φαινόμενο της διάδρασης. Όταν,
όμως, ο σκοπός είναι να κατανοήσουμε καλύτερα και, κυρίως, να
αξιολογήσουμε ένα υπόδειγμα διάδρασης με χρήστες, τότε θα πρέπει να
έχουμε μεγαλύτερη πιστότητα στη λειτουργία, ώστε να έχουν νόημα και οι
αντιδράσεις των χρηστών που θα καταγραφούν και θα αναλυθούν. Για αυτόν
τον σκοπό, γίνεται η κατασκευή του Buck, \footnote{Pering (2002)} η
οποία αποτελείται από δύο βασικά τμήματα: 1) το λογισμικό που εκτελείται
σε έναν επιτραπέζιο ΗΥ και 2) το υλικό διάδρασης της συσκευής με τον
χρήστη, το οποίο είναι συνδεδεμένο ενσύρματα με τον επιτραπέζιο
υπολογιστή, ώστε να μεταφέρει σε αυτόν για επεξεργασία την είσοδο από
τον χρήστη.

\leavevmode\vadjust pre{\hypertarget{fig:buck}{}}%
\begin{figure}
\hypertarget{fig:buck}{%
\centering
\includegraphics{images/buck.png}
\caption{Εικόνα 11: Στη φάση της μετάβασης από τα αρχικά προσχέδια σε
λειτουργικά πρωτότυπα υπάρχει πολύ μεγάλη ασάφεια αναφορικά με το
λογισμικό και το υλικό, ειδικά στις περιπτώσεις που έχουμε νέες
συσκευές, όπως ένα έξυπνο κινητό ή άλλες συσκευές διάχυτου υπολογισμού.
Το πρωτότυπο υψηλής πιστότητας τύπου Buck γεφυρώνει αυτήν την μετάβαση
με τη δημιουργική επαναχρησιμοποίηση υλικού και λογισμικού που ήδη
υπάρχει, ακόμη και αν αυτά δεν θα είναι ίδια στο τελικό προϊόν, αρκεί να
είναι αντιπροσωπευτικά της διάδρασης.}\label{fig:buck}
}
\end{figure}

\leavevmode\vadjust pre{\hypertarget{fig:kodak-hifi}{}}%
\begin{figure}
\hypertarget{fig:kodak-hifi}{%
\centering
\includegraphics{images/kodak-hifi.png}
\caption{Εικόνα 12: Το πρωτότυπο υψηλής πιστότητας για μια ψηφιακή
φωτογραφική μηχανή της Kodak.}\label{fig:kodak-hifi}
}
\end{figure}

Η κατασκευή υποδείγματος υψηλής πιστότητας τύπου Buck έχει, επίσης,
χρησιμοποιηθεί και από άλλες εταιρείες για προϊόντα που συνδυάζουν το
υλικό με το λογισμικό. Για παράδειγμα, η Kodak το χρησιμοποίησε για να
φτιάξει το πρωτότυπο για ψηφιακές κάμερες. Σε αυτήν την περίπτωση, εκτός
από την κατασκευή του υλικού διάδρασης με τον χρήστη, είχαμε και την
ενσωμάτωση μιας μικρής οθόνης μέσα στο υλικό. Πάντως, το λογισμικό δεν
εκτελείται στην ίδια τη συσκευή, αλλά στον επιτραπέζιο ΗΥ, με τον οποίο
επικοινωνεί μέσω ενός καλωδίου και μιας διεπαφής που φροντίζει για τη
μετατροπή των ενεργειών στη συσκευή του χρήστη σε μορφή κατανοητή από το
λογισμικό του επιτραπέζιου ΗΥ. Και στις δύο περιπτώσεις έχουμε ένα
υπόδειγμα διάδρασης που λίγο μοιάζει με το τελικό προϊόν, όμως
εξυπηρετεί τον σκοπό της αξιολόγησης βασικών λειτουργιών από τους
χρήστες. Τέλος, αξίζει να παρατηρήσουμε ότι, ακόμη και στην κατηγορία
των υποδειγμάτων υψηλής πιστότητας, υπάρχει μια επιμέρους κλίμακα
πιστότητας με κάποια υποδείγματα να είναι περισσότερο κοντά στο τελικό
προϊόν από κάποια άλλα. \footnote{(\textbf{Εικόνα?})~11 Buck prototype
  (Springer Nature)} \footnote{(\textbf{Εικόνα?})~12 Λειτουργικό
  πρωτότυπο για την πρώτη ψηφιακή φωτογραφική μηχανή της Kodak (Kodak)}

Ο βασικός στόχος της κατασκευής της διάδρασης είναι να αναπτύξει
συστήματα και συσκευές που να ανταποκρίνονται στις ανάγκες των χρηστών
κατά τη διαδικασία επίτευξης των στόχων τους σε ένα δεδομένο πλαίσιο
χρήσης. Για αυτόν τον σκοπό, οι σχεδιαστές, συνήθως, εξετάζουν
παράγοντες όπως η ευχρηστία και η μέτρησή της.

\hypertarget{ux3b5ux3c0ux3b1ux3bdux3b1ux3bbux3b7ux3c0ux3c4ux3b9ux3baux3ae-ux3b1ux3beux3b9ux3bfux3bbux3ccux3b3ux3b7ux3c3ux3b7-ux3bcux3b5-ux3c7ux3c1ux3aeux3c3ux3c4ux3b5ux3c2}{%
\subsection{Επαναληπτική αξιολόγηση με
χρήστες}\label{ux3b5ux3c0ux3b1ux3bdux3b1ux3bbux3b7ux3c0ux3c4ux3b9ux3baux3ae-ux3b1ux3beux3b9ux3bfux3bbux3ccux3b3ux3b7ux3c3ux3b7-ux3bcux3b5-ux3c7ux3c1ux3aeux3c3ux3c4ux3b5ux3c2}}

Σε αυτό το μέρος θα εξετάσουμε το κρισιμότερο χαρακτηριστικό ενός
διαδραστικού συστήματος, τη χρήση του από την πλευρά του ανθρώπου. Η
κατασκευή της διάδρασης αφορά τη δημιουργία επεμβάσεων σε συχνά
πολύπλοκες καταστάσεις, όπου εμπλέκονται τόσο άνθρωποι όσο και
ετερόκλητες τεχνολογίες, συμπεριλαμβανομένου του λογισμικού για
επιτραπέζιο ΗΥ, του Web, των κινητών, και των διάχυτων συσκευών. Η
πολυπλοκότητα, συνήθως, σημαίνει ότι κάτι μπορεί να μη γίνει σωστά στην
πρώτη προσπάθεια, αφού η εισαγωγή μιας νέας διάδρασης θα δημιουργήσει
αλλαγές σε ένα ευρύτερο τεχνολογικό και κοινωνικό σύστημα. Συνεπώς,
χρειαζόμαστε επαναληπτικές διαδικασίες και υποδείγματα για δοκιμή και
αξιολόγηση. Η θεωρία (π.χ. οι δυνατότητες του ανθρώπου και οι ιδιότητες
των συσκευών) και τα μοντέλα από τη βιβλιογραφία (π.χ. κανόνες και
μορφότυπα σχεδίασης) μπορούν να βοηθήσουν, παρέχοντας ένα καλό σημείο
εκκίνησης, αλλά η σχεδίαση δεν θα είναι ολοκληρωμένη αν δεν γίνει και
αξιολόγηση, η οποία είναι το αντικείμενο αυτής της ενότητας.

\leavevmode\vadjust pre{\hypertarget{fig:nls-floor}{}}%
\begin{figure}
\hypertarget{fig:nls-floor}{%
\centering
\includegraphics{images/nls-floor.jpg}
\caption{Εικόνα 13: Μετά την εφεύρεση του λογισμικού και του υλικού
διάδρασης, η ομάδα του NLS ασχολήθηκε με την οργάνωση της εργονομίας και
του χώρου εργασίας, έτσι ώστε να διευκολύνεται η επαύξηση της συλλογικής
νοημοσύνης. Για αυτόν τον σκοπό δοκίμασαν διάφορα είδη θέσης εργασίας,
όπως το κάθισμα στο πάτωμα, καθώς και ανοιχτούς χώρους εργασίας, όπου οι
άνθρωποι συνεργάζονται μεταξύ τους και μέσω της επαύξησης των
υπολογιστών.}\label{fig:nls-floor}
}
\end{figure}

\leavevmode\vadjust pre{\hypertarget{fig:nls-desk}{}}%
\begin{figure}
\hypertarget{fig:nls-desk}{%
\centering
\includegraphics{images/nls-desk.jpg}
\caption{Εικόνα 14: Η πολύωρη παραμονή σε έναν σταθμό εργασίας έχει
αρνητικές επιπτώσεις στον άνθρωπο, οι οποίες μπορούν να αμβλυνθούν με
έναν πιο εργονομικό σχεδιασμό των συσκευών αλλά και του λογισμικού. Το
σύστημα NLS βασίζεται σε ένα ειδικά ολοκληρωμένο με τον σταθμό εργασίας
κάθισμα, καθώς και σε μία αποδοτική διάδραση για τον εκπαιδευμένο
χρήστη, οπότε όλες αυτές οι παράμετροι θα πρέπει να αξιολογηθούν
συνδυαστικά μαζί με την κατασκευή του λογισμικού
διάδρασης.}\label{fig:nls-desk}
}
\end{figure}

Το τμήμα αυτό είναι ένα από τα σημαντικότερα από πρακτικής άποψης, καθώς
δίνει συγκεκριμένες κατευθύνσεις για το πώς τελικά αξιολογείται
συστηματικά ένα προϊόν κατασκευής της διάδρασης που απευθύνεται σε
ανθρώπους. Το πρώτο και σημαντικότερο βήμα στην αξιολόγηση με χρήστες,
με την προϋπόθεση ότι έχουμε ήδη ένα λειτουργικό υπόδειγμα υψηλής
πιστότητας, είναι η πιλοτική δοκιμή. Επίσης, οι κανόνες σχεδίασης
μπορούν να χρησιμοποιηθούν από ειδικούς, εκτός από τη δημιουργία
υποδείγματος, στην αξιολόγηση ενός συστήματος διάδρασης και στη βελτίωσή
του. Ακόμη, υπάρχουν οι πειραματικές και οι εργαστηριακές μεθοδολογίες
αξιολόγησης διαδραστικών εφαρμογών. Ιδιαίτερο ενδιαφέρον παρουσιάζουν οι
μέθοδοι αξιολόγησης στο πεδίο, καθώς χρησιμοποιούνται ευρύτατα από
ερευνητές και επαγγελματίες του χώρου. Τέλος, σε κάθε περίπτωση
αξιολόγησης, το πιο σημαντικό είναι να συλλέγουμε δεδομένα διαφορετικού
είδους (π.χ. φυσιομετρικά, συμπεριφοράς, απόψεις), καθώς και να γίνονται
επαναληπτικές πιλοτικές αξιολογήσεις με λίγους χρήστες, πριν
προχωρήσουμε στην τελική αξιολόγηση με περισσότερους χρήστες.
\footnote{(\textbf{Εικόνα?})~13 Οργάνωση του χώρου εργασίας (Doug
  Engelbart Institute)} \footnote{(\textbf{Εικόνα?})~14 Εργονομική θέση
  εργασίας για το σύστημα NLS (Doug Engelbart Institute)}.

Η αξιολόγηση της διάδρασης με μια μικρή ομάδα χρηστών είναι η πιο
δημοφιλής τεχνική αξιολόγησης. Κατά τη φάση της ανάπτυξης, ακόμη και
πέντε χρήστες είναι αρκετοί για να γίνει μια αξιολόγηση της διάδρασης.
Ειδικά στα πρώτα στάδια της ανάπτυξης, όταν η ομάδα κατασκευής προσπαθεί
να κατανοήσει τις ανάγκες των χρηστών και τους τρόπους που μια νέα
διάδραση επηρεάζει τις δραστηριότητές τους, η σημασία της αξιολόγησης
βρίσκεται περισσότερο στις ποιοτικές διαστάσεις της, παρά στις
ποσοτικές. Σε αυτές τις περιπτώσεις, ο μικρός αριθμός χρηστών
συνοδεύεται και από περισσότερα ερωτήματα που έχουμε και θέλουμε να
εξερευνήσουμε με τη συμμετοχή τους. Έτσι, η συλλογή των δεδομένων
βασίζεται περισσότερο στην παρατήρηση και τις ημιδομημένες συνεντεύξεις
με τους χρήστες.

Όταν βρισκόμαστε στα τελικά στάδια της ανάπτυξης ή όταν κάνουμε μόνο
μικρές μετατροπές σε ένα σύστημα διάδρασης που υπάρχει ήδη, τότε είναι
περισσότερο σκόπιμο να χρησιμοποιήσουμε ένα εργαστηριακό πείραμα ή ακόμη
και μια μελέτη στο πεδίο με περισσότερους χρήστες. Σε αυτές τις
περιπτώσεις, εκτός από περισσότερους χρήστες (τουλάχιστον είκοσι), θα
έχουμε και περισσότερο συγκεκριμένα ζητήματα και εναλλακτικές σχεδιάσεις
για τις οποίες θα θέλουμε να εντοπίσουμε με μεγάλη ακρίβεια τις
διαφορές. Αντίστοιχα, για τη συλλογή δεδομένων στην αξιολόγηση με μεγάλο
αριθμό χρηστών ή με μεγάλο αριθμό διαδράσεων, είναι σκόπιμο να έχουμε
περισσότερα είδη δεδομένων. Εκτός από τη βασική παρατήρηση των χρηστών
που εκτελούν διεργασίες με ένα σύστημα διάδρασης μπορούμε να συλλέξουμε
δεδομένα αυτόματα, καταγράφοντας τις λεπτομέρειες των διαδράσεων ή των
βιομετρικών στοιχείων (π.χ. παρακολούθηση της ίριδας του ματιού,
καταγραφή του σφυγμού) σε αρχεία στον υπολογιστή, καθώς και να έχουμε
δομημένα ερωτηματολόγια. \footnote{(\textbf{Εικόνα?})~15 Αξιολόγηση της
  ευχρηστίας με παρατήρηση του χρήστη (Microsoft)} \footnote{(\textbf{Εικόνα?})~16
  Λογισμικό δοκιμών χρήστη για την συσκευή εισόδου ποντίκι (Microsoft)}.

\leavevmode\vadjust pre{\hypertarget{fig:usability-observation}{}}%
\begin{figure}
\hypertarget{fig:usability-observation}{%
\centering
\includegraphics{images/usability-observation.png}
\caption{Εικόνα 15: Η παρατήρηση της δραστηριότητας του χρήστη κατά τη
διάδραση με τον υπολογιστή είναι η πιο δημοφιλής και η πιο απλή τεχνική
αξιολόγησης μιας νέας εφαρμογής ή συσκευής. Για μια νέα συσκευή εισόδου
ο χρήστης εκτελεί τυπικές διεργασίες σε διαθέσιμο λογισμικό ή ακόμη
καλύτερα σε ειδικό λογισμικό, το οποίο καταγράφει με λεπτομέρεια τις
συμπεριφορές.}\label{fig:usability-observation}
}
\end{figure}

\leavevmode\vadjust pre{\hypertarget{fig:mouse-test-software}{}}%
\begin{figure}
\hypertarget{fig:mouse-test-software}{%
\centering
\includegraphics{images/mouse-test-software.png}
\caption{Εικόνα 16: Οι σχεδιαστές νέων συσκευών διάδρασης αναπτύσσουν
ειδικό λογισμικό προσομοίωσης των βασικών διεργασιών που θέλουν να
επιτελεί η νέα συσκευή και το χρησιμοποιούν για να αξιολογήσουν με τη
συμμετοχή των χρηστών εναλλακτικές λύσεις. Για το ποντίκι, οι δοκιμές με
χρήστες αξιολογούν την επιτυχία και τον χρόνο εντοπισμού στόχων και
μετακίνησης αντικειμένων και συγκρίνουν διαφορετικές σχεδιάσεις του
υλικού και διαφορετικές ρυθμίσεις του
λογισμικού.}\label{fig:mouse-test-software}
}
\end{figure}

Υπάρχουν κάποιες περιπτώσεις στην αξιολόγηση της διάδρασης, όπου ο
αριθμός των χρηστών δεν είναι η σημαντικότερη παράμετρος. Για
παράδειγμα, στην αρχική αξιολόγηση της συσκευής εισόδου ποντίκι οι
ερευνητές είχαν μόνο πέντε χρήστες και, παρά τον μικρό (σχετικά) αριθμό
τους, κατέληξαν σε ισχυρά συμπεράσματα αναφορικά με τη συγκριτική
απόδοση των συσκευών εισόδου, τα οποία δεν έχουν αλλάξει πολλές
δεκαετίες μετά. Αντί για τον αριθμό των χρηστών, αυτό που έχει σημασία
είναι ο αριθμός των διαδράσεων που θα αναλύσουμε για να καταλήξουμε σε
συμπεράσματα. Στην περίπτωση της αξιολόγησης της συσκευής εισόδου
ποντίκι, οι ερευνητές έκαναν πολλές προκαταρκτικές δοκιμές με τους
χρήστες μέχρι να διαπιστώσουν ότι η απόδοση τους δεν αλλάζει, και τότε
μόνο έκαναν συλλογή ακόμη περισσότερων διαδράσεων, αρκετών για να
οδηγηθούν σε ασφαλή συμπεράσματα. Το συμπέρασμα είναι ότι, για τον
καθορισμό του αριθμού των χρηστών, θα πρέπει πρώτα να κάνουμε έναν
διαχωρισμό για το αν μας ενδιαφέρει η άποψή τους ή μόνο η απόδοσή τους.

Σε αυτήν την ενότητα μελετήσαμε τις μεθόδους της σχεδίασης της διάδρασης
ανθρώπου και υπολογιστή, οι οποίες από μόνες τους δεν μπορούν να
εγγυηθούν την ποιότητα του αποτελέσματος. Ένα σημαντικό συμπέρασμα που
προκύπτει από τα πολλά ιστορικά παραδείγματα είναι ότι μόνο εκ του
αποτελέσματος γίνεται κατανοητό γιατί κάποιες σχεδιάσεις είναι καλύτερες
από άλλες. Ταυτόχρονα, παραμένει πάντα δύσκολο να προβλέψουμε και να
σχεδιάσουμε με σιγουριά εκείνες τις μελλοντικές διαδράσεις ανθρώπων και
υπολογιστών που είναι περισσότερο αποτελεσματικές. Αν και ο στόχος θα
είναι πάντα φευγαλέος, υπάρχουν θεωρίες, τεχνικές, μοτίβα, τεχνολογίες
και μέθοδοι που αργά ή γρήγορα μας δίνουν σταδιακά καλύτερες λύσεις. Σε
κάθε περίπτωση, είναι σκόπιμο να ξέρουμε τι έχουν δοκιμάσει οι
σχεδιαστές της διάδρασης στο παρελθόν και γιατί (α)πέτυχε.

\hypertarget{ux3b7-ux3c0ux3b5ux3c1ux3afux3c0ux3c4ux3c9ux3c3ux3b7-ux3c4ux3b7ux3c2-ux3c3ux3c5ux3c3ux3baux3b5ux3c5ux3aeux3c2-ux3b5ux3b9ux3c3ux3ccux3b4ux3bfux3c5-ux3c0ux3bfux3bdux3c4ux3afux3baux3b9}{%
\subsection{Η περίπτωση της συσκευής εισόδου
ποντίκι}\label{ux3b7-ux3c0ux3b5ux3c1ux3afux3c0ux3c4ux3c9ux3c3ux3b7-ux3c4ux3b7ux3c2-ux3c3ux3c5ux3c3ux3baux3b5ux3c5ux3aeux3c2-ux3b5ux3b9ux3c3ux3ccux3b4ux3bfux3c5-ux3c0ux3bfux3bdux3c4ux3afux3baux3b9}}

Αν και σήμερα η χρήση της συσκευής εισόδου ποντίκι σε συνδυασμό με τη
γραφική επιφάνεια εργασίας και τις αντίστοιχες εφαρμογές γραφείου και
παραγωγικότητας φαίνεται προφανής επιλογή χωρίς εναλλακτικές, αυτό δεν
ήταν αυτονόητο μερικές δεκαετίες πριν. Στις αρχές της δεκαετίας του
1970, οι ερευνητές είχαν στη διάθεσή τους πολλές διαφορετικές συσκευές
εισόδου για τον ίδιο σκοπό, δηλαδή την εργασία με εφαρμογές επεξεργασίας
κειμένου σε έναν επιτραπέζιο υπολογιστή. Αν και η γραφική επιφάνεια
εργασίας δεν ήταν ακόμη διαθέσιμη με την πλήρη μορφή της, η
λειτουργικότητα των εφαρμογών επέτρεπε πολλές από τις διεργασίες που
υποστηρίζει ένας σύγχρονος επιτραπέζιος υπολογιστής, όπως είναι η
επιλογή μιας λέξης ή πρότασης και η αλλαγή ή η μετακίνησή της. Ο αριθμός
των κουμπιών σε ένα ποντίκι δεν είναι από μόνος του ικανός να καθορίσει
την αποτελεσματικότητά του, αν δεν γνωρίζουμε τις ανάγκες και τις
διεργασίες του χρήστη. Έτσι, το ποντίκι με ένα κουμπί είναι ιδιαιτέρως
κατάλληλο για αρχάριους χρήστες, καθώς δεν επιτρέπει το λάθος, αφού
υπάρχει μόνο μία λειτουργία.\footnote{(\textbf{Εικόνα?})~17 Apple Mouse
  (wikimedia)}

Το πρώτο πράγμα που πρέπει να οριστεί με μεγάλη ακρίβεια σε ένα πείραμα
συγκριτικής αξιολόγησης εναλλακτικών συσκευών εισόδου είναι ο στόχος και
οι αντίστοιχες μετρικές και διεργασίες του χρήστη, οι οποίες θα
μπορούσαν να επιβεβαιώσουν τον βαθμό επιτυχίας του στόχου. Στην
περίπτωση των συσκευών εισόδου, για τη διευκόλυνση της επεξεργασίας
κειμένου, βλέπουμε ότι ανάμεσα στις πολλές λειτουργίες που εκτελεί ένας
χρήστης υπάρχουν κάποιες που ξεχωρίζουν, γιατί είναι πολύ συχνές και
πολύ απλές και αυτές είναι η επιλογή αντικειμένων στην οθόνη, καθώς και
η μετακίνησή τους σε μια άλλη θέση. Με δεδομένο ότι η εφαρμογή
επεξεργασίας κειμένου εκτελείται σε ένα περιβάλλον γραφείου με σκοπό την
αύξηση της παραγωγικότητας, μπορούμε να ορίσουμε ως αντιπροσωπευτικές
μετρικές τον χρόνο που χρειάζεται ο χρήστης για να πραγματοποιήσει τις
παραπάνω βασικές διεργασίες, καθώς και τα λάθη που κάνει. Στη συνέχεια,
οι ερευνητές δοκιμάζουν τις εναλλακτικές λύσεις με τους χρήστες,
συλλέγουν τα δεδομένα μέσω παρατήρησης και, κυρίως, μέσω των αρχείων
διάδρασης του λογισμικού προσομοίωσης, και δημιουργούν γραφήματα για να
τα συγκρίνουν.

\leavevmode\vadjust pre{\hypertarget{fig:apple-mouse}{}}%
\begin{figure}
\hypertarget{fig:apple-mouse}{%
\centering
\includegraphics{images/apple-mouse.jpg}
\caption{Εικόνα 17: Η Apple σκόπιμα έβαλε μόνο ένα κουμπί στο ποντίκι
που συνόδευε τον πρώτο δικό της εμπορικά επιτυχημένο επιτραπέζιο
υπολογιστή με γραφική επιφάνεια εργασίας. Με αυτόν τον τρόπο, αν και
μείωνε τις δυνατότητες της συσκευής εισόδου, την έκανε πιο απλή και
μηδένιζε την πιθανότητα να πατήσει λάθος κουμπί ο αρχάριος σε γραφικά
περιβάλλοντα χρήστης.}\label{fig:apple-mouse}
}
\end{figure}

\leavevmode\vadjust pre{\hypertarget{fig:input-comparison}{}}%
\begin{figure}
\hypertarget{fig:input-comparison}{%
\centering
\includegraphics{images/input-comparison.png}
\caption{Εικόνα 18: Η συσκευή εισόδου ποντίκι αρχικά δεν είχε κάποιο
προφανές πλεονέκτημα σε σχέση με παρόμοιες συσκευές εισόδου, όπως ο
μοχλός και τα πλήκτρα κατεύθυνσης, τα οποία ήταν ήδη διαθέσιμα στην
αγορά και οικεία στους χρήστες. Έπρεπε να γίνουν συγκριτικές δοκιμές
απόδοσης, για να διαπιστωθεί ποια ήταν τελικά περισσότερο κατάλληλη για
την επιλογή κειμένου στην οθόνη, το οποίο αποτελετούσε τότε τη βασική
ανάγκη .}\label{fig:input-comparison}
}
\end{figure}

Από την πλευρά της πειραματικής μεθοδολογίας, αυτό που κάνει τη μελέτη
των συσκευών εισόδου ιδιαίτερα ενδιαφέρουσα είναι η επιλογή των χρηστών
και, ειδικά, ο αριθμός τους. Οι ερευνητές διάλεξαν ως χρήστες τις
γραμματείς που εργάζονταν στο εργαστήριό τους και δικαιολόγησαν την
επιλογή τους με δεδομένο ότι οι διεργασίες που τους έδωσαν, όπως η
επεξεργασία κειμένου, είχαν να κάνουν με δουλειά γραφείου, επομένως,
ήταν εντός του πλαισίου εργασίας τους. Η εύκολη πρόσβαση στους χρήστες
είναι σίγουρα μια σημαντική παράμετρος, ειδικά όταν έχουμε να
κατασκευάσουμε επαναληπτικά μια νέα διάδραση, αλλά αυτό που έχει τη
μεγαλύτερη σημασία είναι ο καθορισμός του πλήθους των χρηστών. Για αυτήν
την παράμετρο, οι ερευνητές της συσκευής εισόδου ποντίκι διάλεξαν μόνο
πέντε χρήστες. Αν και σε πρώτη ανάγνωση ο αριθμός φαίνεται μικρός για
οποιαδήποτε στατιστική ανάλυση, με μια προσεκτικότερη ματιά
διαπιστώνουμε ότι το αντικείμενο ανάλυσης δεν ήταν οι πέντε χρήστες,
αλλά οι διεργασίες που έκαναν με τις εναλλακτικές συσκευές εισόδου. Οι
διεργασίες που έπρεπε να εκτελεστούν από τους πέντε χρήστες του
πειράματος ήταν εκατοντάδες και πάνω σε αυτά τα δεδομένα οι ερευνητές
αιτιολόγησαν τα συμπεράσματά τους.

Τα αποτελέσματα της συγκριτικής μελέτης των εναλλακτικών συσκευών
εισόδου (έμμεσης διάδρασης) για τη μετακίνηση του δείκτη στην οθόνη
\footnote{Card, English, and Burr (1978)} έδειξαν πολλά περισσότερα από
το γεγονός ότι το ποντίκι ήταν η πιο γρήγορη, ακριβής και εργονομική
συσκευή για πολύωρη χρήση. \footnote{(\textbf{Εικόνα?})~18 Συσκευές
  εισόδου για επιλογή κειμένου (Elsevier)} Ανάλογα με τον κάθε χρήστη,
οι ερευνητές διαπίστωσαν ότι από τη στιγμή που μπορεί να θεωρηθεί
έμπειρος, το οποίο πετύχαιναν με τις πολλές επαναλήψεις των τυπικών
διεργασιών, η απόδοση της συσκευής εισόδου ποντίκι σχετιζόταν με τις
δυνατότητες του χρήστη να συντονίζει το χέρι με την όρασή του. Αυτή η
διαπίστωση είναι πολύ σημαντική, καθώς καθορίζει ότι υπάρχει ένα άνω
φράγμα στις επιδόσεις που μπορούμε να πετύχουμε με τη συσκευή εισόδου
ποντίκι, το οποίο δεν εξαρτάται τόσο από τις επιμέρους ιδιότητες της
συσκευής εισόδου, αλλά από τις ιδιότητες του ανθρώπινου καναλιού
επικοινωνίας που συνδέει το χέρι με τον εγκέφαλο και τα μάτια. Η
πληροφορία αυτή είναι πολύ σημαντική, επειδή, όταν γνωρίζουμε τα
ανθρώπινα όρια και τις ιδιότητες μιας νέας συσκευής εισόδου, μπορούμε να
αξιολογήσουμε νέες συσκευές εισόδου με στρατηγικό τρόπο.\footnote{Card,
  Moran, and Newell (2018)} Στην πράξη, βέβαια, οι παραπάνω γνώσεις
έχουν φανεί περισσότερο χρήσιμες όταν έχουμε ήδη έτοιμες κάποιες
συσκευές εισόδου, παρά όταν προσπαθούμε να σχεδιάσουμε κάποια εντελώς
νέα, όπως στην περίπτωση του iPod click wheel.

\hypertarget{ux3b7-ux3c0ux3b5ux3c1ux3afux3c0ux3c4ux3c9ux3c3ux3b7-ux3c4ux3c9ux3bd-microsoft-windows}{%
\subsection{Η περίπτωση των Microsoft
Windows}\label{ux3b7-ux3c0ux3b5ux3c1ux3afux3c0ux3c4ux3c9ux3c3ux3b7-ux3c4ux3c9ux3bd-microsoft-windows}}

Η επιφάνεια εργασίας του επιτραπέζιου υπολογιστή είναι ένα δημοφιλές και
ευέλικτο σύστημα, το οποίο πέρασε από πολλούς κύκλους ανάπτυξης και
προσαρμογής, τόσο κατά τα πρώτα στάδια της δημιουργίας του όσο και κατά
τη διάδοσή του. Η επιφάνεια εργασίας είναι μια πολύ ενδιαφέρουσα μελέτη
περίπτωσης, γιατί η εξέλιξή της ήταν σχετικά αργή, καθώς διάρκησε
περισσότερα από σαράντα χρόνια. Σίγουρα η επιφάνεια εργασίας δεν έγινε
τόσο δημοφιλής ούτε χρονικά ούτε σε κλίμακα όσο η διεπαφή και η διάδραση
με τον κινητό υπολογισμό. Από την άλλη πλευρά, όμως, ο σταδιακός και
διαχρονικός μετασχηματισμός της επιφάνειας εργασίας παρουσιάζει
ενδιαφέρουσες διακυμάνσεις και ομοιότητες ανάμεσα σε ανταγωνιστικά
εμπορικά προϊόντα, τα οποία μπορούν να μας δώσουν πολλά μαθήματα σχετικά
με τον κύκλο της ανθρωποκεντρικής σχεδίασης.

\leavevmode\vadjust pre{\hypertarget{fig:windows95}{}}%
\begin{figure}
\hypertarget{fig:windows95}{%
\centering
\includegraphics{images/windows95.jpg}
\caption{Εικόνα 19: Τα Windows95 ήταν μια πολύ επιτυχημένη προσπάθεια, η
οποία βασίστηκε στην ευρύτατη προβολή, στην οικονομική τιμή, στην
ευελιξία εγκατάστασης σε διαφορετικό υλικό αλλά και στην εύκολη
ενσωμάτωση υλικού από άλλους κατασκευαστές. Ήταν η πρώτη έκδοση με το
κουμπί εκκίνησης και ένα πλήρες παραθυρικό περιβάλλον, το οποίο
περιλάμβανε και την μπάρα ανοικτών εφαρμογών.}\label{fig:windows95}
}
\end{figure}

\leavevmode\vadjust pre{\hypertarget{fig:windows8}{}}%
\begin{figure}
\hypertarget{fig:windows8}{%
\centering
\includegraphics{images/windows8.png}
\caption{Εικόνα 20: Τα Windows8 προσφέρουν μια διεπαφή με τον χρήστη, η
οποία, εμφανισιακά και λειτουργικά, είναι δανεισμένη από τις κινητές
συσκευές, αφού έχουμε μεγάλα ζωντανά εικονίδια και χειρονομίες, δηλαδή
έχουμε μοτίβα διάδρασης που συναντάμε περισσότερο στον κινητό παρά στον
επιτραπέζιο υπολογισμό, καθώς ο στόχος είναι να δημιουργηθεί μια ενιαία
εμπειρία χρήσης ανάμεσα σε πολλές συσκευές διάδρασης. Επιπλέον, για
πρώτη φορά μετά τα Windows 95, δεν υπάρχει το κουμπί
Start.}\label{fig:windows8}
}
\end{figure}

Τα Microsoft Windows είναι σίγουρα το πιο δημοφιλές λειτουργικό σύστημα
με επιφάνεια εργασίας για επιτραπέζιους υπολογιστές και παρουσιάζει
ιδιαίτερο ενδιαφέρον ως μελέτη περίπτωσης, γιατί η εξέλιξή του ήταν
σταδιακή, πράγμα που μας επιτρέπει να βλέπουμε πιο καθαρά τα επιμέρους
στάδια και να τα ερμηνεύουμε, αφού πρώτα τα συνδέσουμε με άλλες σχετικές
εξελίξεις. Επειδή η αποδοχή του βασικού λειτουργικού συστήματος
Microsoft Disk Operating System (MSDOS) ήταν πολύ μεγάλη, η πρώτη έκδοση
του γραφικού περιβάλλοντος ήταν βασισμένη σε αυτό και δεν είχε πολλές
από τις βασικές λειτουργίες της διάδρασης με τη γραφική επιφάνεια
εργασίας, ο οποίες είχαν ήδη εμφανιστεί σε αντίστοιχα προϊόντα
ανταγωνιστικών εταιρειών, όπως ήταν το MacOS. Βλέπουμε ότι ο
κατασκευαστής της διάδρασης, σε ορισμένες περιπτώσεις, μπορεί να
αγνοήσει εφικτές και χρήσιμες δυνατότητες της διάδρασης, προκειμένου να
δώσει βάρος σε παλιές εφαρμογές, απλά και μόνο επειδή οι χρήστες τις
έχουν συνηθίσει ή επειδή οι αντίστοιχοι κατασκευαστές εκείνων των
εφαρμογών δεν είναι έτοιμοι να περάσουν στην επόμενη φάση. Με άλλα
λόγια, βλέπουμε για μια ακόμη φορά ότι ο τεχνολογικός ντετερμινισμός δεν
είναι αρκετός για να σπρώξει μπροστά την ανάπτυξη, αφού υπάρχει και ο
ανθρώπινος και ο κοινωνικός παράγοντας, οι οποίοι είναι εξίσου
σημαντικοί.

Στα μέσα της δεκαετίας του 1990, οι τεχνολογικές συνθήκες έχουν ωριμάσει
τόσο, ώστε μια μεγάλη μερίδα χρηστών λειτουργικών συστημάτων να έχει
αποκτήσει ή να έχει πρόσβαση σε γρηγορότερους επεξεργαστές και σε
ξεχωριστές κάρτες γραφικών. Ταυτόχρονα, η αγορά των οικιακών υπολογιστών
έχει διευρυνθεί αρκετά και καλύπτει πολλές επιμέρους ανθρώπινες
δραστηριότητες. Οπότε, οι κατασκευαστές υλικού έχουν αρχίσει να
διαθέτουν εξειδικευμένο εξοπλισμό, ο οποίος συνδέεται με τον υπολογιστή
για να διευκολύνει τις διεργασίες των χρηστών, όπως μουσική, παιχνίδια,
σχεδίαση κτλ. Το παραπάνω πλαίσιο δημιουργεί τις ιδανικές συνθήκες για
την εισαγωγή των Windows95, \footnote{(\textbf{Εικόνα?})~19 Microsoft
  Windows 95 (Microsoft)} τα οποία έχουν πλέον μια πλήρη γραφική
επιφάνεια εργασίας και υποστηρίζουν την εύκολη προσθήκη νέων
προγραμμάτων και επιπλέον υλικού. Η αποδοχή των Windows95 από την αγορά
μπορεί να συγκριθεί μόνο με αυτή των WindowsXP, σχεδόν δέκα χρόνια μετά,
ενώ η μεγάλη ομοιότητά τους, τόσο με το αρχικό Macintosh OS όσο και με
το Xerox Star, αποτελεί την απόδειξη ότι από μόνη της η ποιότητα της
διάδρασης δεν είναι αρκετή για να καθορίσει την τύχη ενός προϊόντος στην
ευρύτερη αγορά, αλλά απαιτείται και μια καλύτερη κατανόηση των αναγκών
των επιμέρους ομάδων χρηστών. Βλέπουμε ότι η κατασκευή της διάδρασης δεν
βασίζεται μόνο στη σωστή υλοποίηση της διάδρασης αλλά και στη σωστή
κατανόηση των υποκειμενικών αναγκών και των προσδοκιών των χρηστών.

Όπως είδαμε παραπάνω, η κατανόηση και η προσαρμογή στις ανάγκες των
χρηστών είναι μια σημαντική συνθήκη για την επιτυχία της διάδρασης,
αρκεί, βέβαια, η διάδραση να είναι εξίσου καλά προσαρμοσμένη στη συσκευή
του χρήστη. Αν και η Microsoft διατηρεί το προβάδισμα στα λειτουργικά
συστήματα του επιτραπέζιου υπολογιστή, η δεκαετία του 2010 τη βρίσκει να
έχει μείνει πίσω στη ραγδαία αναπτυσσόμενη αγορά του κινητού
υπολογισμού, ο οποίος εκτός από τα έξυπνα κινητά τηλέφωνα, περιλαμβάνει
και τις ταμπλέτες, οι οποίες σταδιακά αντικαθιστούν πολλές από τις
διεργασίες του επιτραπέζιου υπολογιστή. Σε μια προσπάθεια να δώσει
προτεραιότητα στις κινητές συσκευές, η Microsoft εισάγει τα Windows8
\footnote{(\textbf{Εικόνα?})~20 Microsoft Windows 8 χωρίς κουμπί
  εκκίνησης (Microsoft)} με αρχική οθόνη διάδρασης αντίστοιχη με εκείνη
που έχει στις κινητές συσκευές της. Αν και η επιλογή αυτή δημιουργεί μια
πραγματικά ομοιόμορφη και συνεπή εμπειρία για τους χρήστες που κινούνται
ανάμεσα σε πολλούς υπολογιστές (κινητούς και επιτραπέζιους), ταυτόχρονα,
απέχει από το να είναι η βέλτιστη για τον επιτραπέζιο υπολογιστή, με
αποτέλεσμα να μειώνει την ευχρηστία του. Βλέπουμε, λοιπόν, ότι ναι μεν η
εστίαση στον χρήστη έχει προτεραιότητα, αλλά και η κατανόηση της φόρμας
του υπολογιστή δεν πρέπει να αγνοείται. Επομένως, η σωστή κατασκευή της
διάδρασης μπορεί να γίνει κατανοητή ως μια λεπτή ισορροπία ανάμεσα στις
ιδιότητες της συσκευής και στις ανάγκες του χρήστη, ισορροπία που μπορεί
να γίνει εύθραυστη όταν έχουμε πολλούς διαφορετικούς χρήστες, και πολλά
διαφορετικά είδη συσκευής χρήστη.

\hypertarget{ux3c3ux3cdux3bdux3c4ux3bfux3bcux3b7-ux3b2ux3b9ux3bfux3b3ux3c1ux3b1ux3c6ux3afux3b1-ux3c4ux3bfux3c5-douglas-engelbart}{%
\subsection{Σύντομη βιογραφία του Douglas
Engelbart}\label{ux3c3ux3cdux3bdux3c4ux3bfux3bcux3b7-ux3b2ux3b9ux3bfux3b3ux3c1ux3b1ux3c6ux3afux3b1-ux3c4ux3bfux3c5-douglas-engelbart}}

O Douglas Engelbart έγινε διάσημος για τη συσκευή εισόδου ποντίκι,
\footnote{(\textbf{Εικόνα?})~21 Douglas Engelbart (Doug Engelbart
  Institute)} αλλά αυτό θεωρείται μόνο μια μικρή παράπλευρη εφεύρεση σε
σχέση με το σύνολο της συνεισφοράς του στην περιοχή της κατασκευής της
διάδρασης. Το όραμα του περιγράφεται αναλυτικά στην πρόταση
χρηματοδότησης του έργου του, το οποίο οδήγησε σε μια διαδραστική
παρουσίαση, η οποία έμεινε γνωστή ως η μητέρα όλων των παρουσιάσεων.
Εκτός από το ποντίκι, εκείνη η παρουσίαση περιελάμβανε τεχνολογίες
υπερκειμένου και τηλεδιάσκεψης.

\leavevmode\vadjust pre{\hypertarget{fig:engelbart-profile}{}}%
\begin{figure}
\hypertarget{fig:engelbart-profile}{%
\centering
\includegraphics{images/engelbart-profile.jpg}
\caption{Εικόνα 21: Το έργο του Douglas Engelbart θεμελίωσε τη
συνεργατική και διαδραστική χρήση των πληροφοριακών συστήματων, αλλά το
όραμα για την επαύξηση της ανθρώπινης νοημοσύνης με πολύπλοκα συστήματα
διάδρασης αντικαταστάθηκε από τη βολικότητα της καθημερινής
ευχρηστίας.}\label{fig:engelbart-profile}
}
\end{figure}

\leavevmode\vadjust pre{\hypertarget{fig:nls-group}{}}%
\begin{figure}
\hypertarget{fig:nls-group}{%
\centering
\includegraphics{images/nls-group.jpg}
\caption{Εικόνα 22: Το σύστημα NLS περιέχει μια σειρά από διαχρονικές
τεχνολογίες, όπως το υπερκείμενο, τη διάδραση με το ποντίκι και την
τηλεδιάσκεψη, οι οποίες δεν υλοποιήθηκαν απλά ως μια τεχνολογική
παρουσίαση, αλλά γιατί ο συνδυασμός τους εξυπηρετεί τον μεγαλύτερο στόχο
της συνεργασίας των ανθρώπων ώστε να κατανοήσουν πολύπλοκα
προβλήματα.}\label{fig:nls-group}
}
\end{figure}

Η βασική συνεισφορά του, όμως, δεν ήταν τόσο στις σχετικά πρωτοποριακές
τεχνολογίες που παρουσίασε συγκεντρωτικά το 1968 και αμέσως μετά έδωσε
στην ερευνητική κοινότητα, αλλά στο όραμα του για την επαύξηση της
ανθρώπινης νοημοσύνης. Τη δεκαετία του 1960, όπως και σε άλλες δεκαετίες
αργότερα, το κυρίαρχο θέμα δραστηριότητας στην τεχνολογία ήταν αυτό της
τεχνητής νοημοσύνης, όπου ο ανθρώπινος παράγοντας απουσιάζει και οι
εργασίες γίνονται αυτόματα από μια φαινομενικά έξυπνη μηχανή, η οποία
έχει προγραμματιστεί ή εκπαιδευτεί ώστε να κάνει τις ενέργειες των
ανθρώπων.

Αντίθετα με το όραμα της τεχνητής νοημοσύνης, το οποίο βάζει σε δεύτερο
ρόλο τον άνθρωπο, το όραμα της επαυξημένης νοημοσύνης έχει ως
πρωταγωνιστή τον άνθρωπο, ο οποίος, σε συνεργασία με άλλους \footnote{(\textbf{Εικόνα?})~22
  Επαύξηση της συλλογικής νοημοσύνης (Doug Engelbart Institute)} οδηγεί
συνειδητά τη μηχανή σε μια καλύτερη κατανόηση πολύπλοκων φαινομένων,
καθώς και σε καλύτερες συλλογικές αποφάσεις. Δεν είναι τυχαίο ότι αυτό
το έργο δημιουργήθηκε τη δεκαετία του 1960 στις δυτικές ακτές των ΗΠΑ,
όπου υπήρχαν πολιτικές και πολιτισμικές ζυμώσεις για έναν καλύτερο και
διαφορετικό κόσμο.

Για τον σκοπό αυτό, οι χρήστες του συστήματος μέσα από μια εκπαίδευση
γνώσεων και δεξιοτήτων μπορούσαν να έχουν εναλλακτικές απόψεις πάνω στα
δεδομένα τους. Για παράδειγμα, εκτός από το ποντίκι, υπήρχε και μια
συσκευή ακόρντων, η οποία άλλαζε την τροπικότητα της διεπαφής στην
οθόνη. Αν και στην πορεία η ουσία του έργου του χάθηκε για τους πολλούς,
επηρέασε άλλους σημαντικούς ερευνητές όπως ο Alan Kay και ο Ted Nelson,
οι οποίοι συνέχισαν το έργο του στις περιοχές της δεξιοτεχνικής
διάδρασης και του υπερκειμένου, χωρίς όμως ούτε αυτοί να έχουν κάποιο
σημαντικό εμπορικό αποτύπωμα μέχρι σήμερα.

\hypertarget{ux3b2ux3b9ux3b2ux3bbux3b9ux3bfux3b3ux3c1ux3b1ux3c6ux3afux3b1}{%
\subsection*{Βιβλιογραφία}\label{ux3b2ux3b9ux3b2ux3bbux3b9ux3bfux3b3ux3c1ux3b1ux3c6ux3afux3b1}}
\addcontentsline{toc}{subsection}{Βιβλιογραφία}

\hypertarget{refs}{}
\begin{CSLReferences}{0}{0}
\end{CSLReferences}

Buxton, Bill. 2010. \emph{Sketching User Experiences: Getting the Design
Right and the Right Design}. Morgan kaufmann.

Card, Stuart K, William K English, and Betty J Burr. 1978. {``Evaluation
of Mouse, Rate-Controlled Isometric Joystick, Step Keys, and Text Keys
for Text Selection on a CRT.''} \emph{Ergonomics} 21 (8): 601--13.

Card, Stuart K, Thomas P Moran, and Allen Newell. 2018. \emph{The
Psychology of Human-Computer Interaction}. Crc Press.

Carroll, John M. 2000. \emph{Making Use: Scenario-Based Design of
Human-Computer Interactions}. MIT press.

Moggridge, Bill. 2007. \emph{Designing Interactions}. MIT press
Cambridge, MA.

Norman, Don. 2013. \emph{The Design of Everyday Things: Revised and
Expanded Edition}. Basic books.

Papanek, Victor, and R Buckminster Fuller. 1972. \emph{Design for the
Real World}. Thames; Hudson London.

Pering, Celine. 2002. {``Interaction Design Prototyping of Communicator
Devices: Towards Meeting the Hardware-Software Challenge.''}
\emph{Interactions} 9 (6): 36--46.

Thackara, John. 2006. \emph{In the Bubble: Designing in a Complex
World}. MIT press.

Winograd, Terry et al. 1996. \emph{Bringing Design to Software}.
\(\{\)Addison-Wesley Professional\(\}\).

\hypertarget{ux3b1ux3c1ux3c7ux3adux3c4ux3c5ux3c0ux3b1}{%
\section{Αρχέτυπα}\label{ux3b1ux3c1ux3c7ux3adux3c4ux3c5ux3c0ux3b1}}

\begin{quote}
O Leonardo da Vinci δεν μπορούσε να εφεύρει ούτε έναν κινητήρα για
κάποιο από τα οχήματά του. Μπορεί να ήταν ο εξυπνότερος άνθρωπος στην
εποχή του, αλλά γεννήθηκε στη λάθος εποχή. Η εξυπνάδα του δεν μπορούσε
να υπερβεί την εποχή του. Alan Kay
\end{quote}

\hypertarget{ux3c0ux3b5ux3c1ux3afux3bbux3b7ux3c8ux3b7}{%
\subsubsection{Περίληψη}\label{ux3c0ux3b5ux3c1ux3afux3bbux3b7ux3c8ux3b7}}

Σε αυτό το κεφάλαιο μελετάμε τις ιδιότητες του υπολογιστή που επιτρέπουν
τη διάδραση με τον άνθρωπο. Εδώ εστιάζουμε την προσοχή μας στις
ιδιότητες του υπολογιστή και ειδικά στα συστήματα εισόδου και εξόδου. Θα
μελετήσουμε τα παραδοσιακά συστήματα εισόδου και εξόδου, όπως είναι το
πληκτρολόγιο και το ποντίκι, καθώς και τα κινητά και διάχυτα συστήματα
εισόδου και εξόδου, που έχουν πολύ περισσότερες σε αριθμό και είδος
συσκευές διάδρασης με τον χρήστη, όπως τον εντοπισμό γεωγραφικής θέσης,
την αφή, την κάμερα, κτλ.

\hypertarget{ux3b4ux3bfux3bcux3b9ux3baux3ac-ux3c3ux3c4ux3bfux3b9ux3c7ux3b5ux3afux3b1}{%
\subsection{Δομικά
στοιχεία}\label{ux3b4ux3bfux3bcux3b9ux3baux3ac-ux3c3ux3c4ux3bfux3b9ux3c7ux3b5ux3afux3b1}}

Οι περισσότεροι είμαστε πολύ καλοί ή ακόμη και άριστοι χρήστες του
επιτραπέζιου ΗΥ. Η κατεύθυνση του κινητού και του διάχυτου υπολογισμού
αποτελεί μια πρόκληση για όλους τους χρήστες αλλά και για τους
προγραμματιστές επιτραπέζιων ΗΥ, επειδή τα οικεία συστήματα εισόδου και
εξόδου αλλάζουν δραστικά προς την κατεύθυνση της φυσικής διάδρασης (π.χ.
αφή, φυσική γλώσσα, αναγνώριση εικόνας). \footnote{McEwen and Cassimally
  (2013)} Επιπλέον, ο κινητός και διάχυτος υπολογισμός, σε κάποιες
περιπτώσεις, δεν έχει όλους του υπολογιστικούς πόρους (π.χ. επεξεργαστή,
μνήμη, αποθήκευση) στα οποία έχουμε συνηθίσει από τους μοντέρνους ΗΥ
γραφείου.

Ένα μεγάλο μέρος της προόδου στα πρώτα βήματα της διάδρασης έγινε στις
συσκευές εξόδου, όπως είναι η οθόνη και ο τρόπος που απεικονίζεται η
πληροφορία πάνω σε αυτή. Από την πλευρά των συσκευών εισόδου, εκτός από
το κλασικό πληκτρολόγιο, η μεγαλύτερη επιτυχία ήταν το ποντίκι, το οποίο
ανήκει στην ομάδα των συσκευών έμμεσης εισόδου. Με βάση τις δυνατότητες
που έχουν οι αρχικά διαθέσιμες συσκευές εισόδου, εξόδου, και
επεξεργασίας δεδομένων, μια σειρά από μορφές διάδρασης έγιναν διαθέσιμες
και αποδεκτές από τους χρήστες: 1) γραμμή εντολών, 2) μενού και φόρμες,
3) φυσική γλώσσα, 4) απευθείας χειρισμός, 5) επαυξημένη και εικονική
πραγματικότητα.

Τα πρώτα βήματα της διάδρασης έγιναν στον χώρο της εργασίας και ειδικά
στις εκδόσεις έντυπου υλικού και όπως ήταν επόμενο, ένα μεγάλο μέρος από
αυτό που αργότερα έγινε γνωστό ως επιτραπέζιο γραφικό περιβάλλον
εργασίας βασίζεται στις αντίστοιχες ανάγκες. \footnote{Hiltzik (1999)}
Το πληκτρολόγιο ήταν απαραίτητο για την εισαγωγή και την επεξεργασία του
κειμένου, ενώ το καινοτόμο ποντίκι επέτρεψε την εύκολη πλοήγηση ανάμεσα
σε πολλές επιλογές επεξεργασίας του εγγράφου, τις οποίες, σε διαφορετική
περίπτωση, θα έπρεπε να απομνημονεύσει ο χρήστης.

Τα ιστορικά παραδείγματα διάδρασης με συσκευές χρήστη έχουν, σε
ορισμένες περιπτώσεις, κάποιες επικαλύψεις (χρονικές ή στα
χαρακτηριστικά τους), όμως είναι όσο γίνεται περισσότερο ανεξάρτητα. Από
τη μια πλευρά, η χρονολογική επισκόπηση είναι μια ενδιαφέρουσα ιστορική
αναδρομή στην τεχνολογική εξέλιξη, με έμφαση στη διάδραση, αλλά,
ταυτόχρονα, είναι και μια εργαλειοθήκη για τον μελλοντικό σχεδιαστή της
διάδρασης. Εστιάζουμε ειδικότερα στην εξέλιξη των διαδραστικών
συστημάτων και στο πώς έχουν αυξήσει τη χρησιμότητα και την ευχρηστία
των υπολογιστών.

\leavevmode\vadjust pre{\hypertarget{fig:sage-radar}{}}%
\begin{figure}
\hypertarget{fig:sage-radar}{%
\centering
\includegraphics{images/sage-radar.jpg}
\caption{Εικόνα 1: Ο μετασχηματισμός της διάδρασης από τον τηλέτυπο προς
τη γραφική διεπαφή στις αρχές της δεκαετίας του 1960 φαντάζει ως μια
πολύ σημαντική αλλαγή παραδείγματος. Στην πράξη, όμως, μπορούμε να
διαγνώσουμε την έμπνευση των σχεδιαστών των Sketchpad, Spacewar, στο
οικείο για αυτούς σύστημα SAGE, το οποίο χρησιμοποιούσε γραφικά για την
απεικόνιση και την ταυτοποίηση ιπτάμενων αντικειμένων πάνω σε μια οθόνη
ραντάρ.}\label{fig:sage-radar}
}
\end{figure}

\leavevmode\vadjust pre{\hypertarget{fig:spacewar-players}{}}%
\begin{figure}
\hypertarget{fig:spacewar-players}{%
\centering
\includegraphics{images/spacewar-players.jpg}
\caption{Εικόνα 2: Ενα από τα πρώτα δημοφιλή βιντεοπαιχνίδια
δημιουργήθηκε το 1962 και παιζόταν από δύο παίκτες, γιατί ο δυνατός για
την εποχή υπολογιστής δεν είχε αρκετή ισχύ για τον αυτόματο έλεγχο του
αντιπάλου, αφού οι πόροι είχαν χρησιμοποιηθεί για τα γραφικά, τη φυσική
κίνηση και τον έλεγχο της σύγκρουσης. Το παιχνίδι αυτό, αν και
δημιουργήθηκε σε ερευνητικό περιβάλλον για επίδειξη, έγινε σημείο
αναφοράς και αντιγραφής από πολλούς προγραμματιστές και επηρέασε τα
παιχνίδια της επόμενης δεκαετίας, ταοποία δημιούργησαν τη βιομηχανία των
βιντεοπαιχνιδιών.}\label{fig:spacewar-players}
}
\end{figure}

Με το ίδιο σκεπτικό, οι κατασκευαστές των πρώτων συστημάτων διάδρασης
χρησιμοποίησαν συσκευές εισόδου και εξόδου με τον χρήστη, τις οποίες
είχαν ήδη διαθέσιμες από σχετικές τεχνολογίες. \footnote{(\textbf{Εικόνα?})~1
  Γραφικά σε οθόνη ραντάρ (MIT)} \footnote{(\textbf{Εικόνα?})~2 Spacewar
  (MIT)} Πράγματι, ο τηλέτυπος ήταν, επίσης, μια τεχνολογία με σχεδόν
έναν αιώνα λειτουργίας, επομένως, ήταν διαθέσιμος και αξιόπιστος. Στα
πρώτα στάδια, ο τηλέτυπος επέτρεψε στον χρήστη να πληκτρολογήσει το
πρόγραμμα και να βλέπει τι ακριβώς έγραψε στο χαρτί. Καθώς οι
υπολογιστές έγιναν περισσότερο διαδραστικοί με την τεχνολογία του
χρονοδιαμοιρασμού, ο τηλέτυπος επέτρεψε και την απόκριση του υπολογιστή
σε πραγματικό χρόνο πάνω στο ίδιο χαρτί μπροστά στον χρήστη. \footnote{(\textbf{Εικόνα?})~3
  Τηλέτυπος Μοντέλο 33 (Wikipedia)} \footnote{(\textbf{Εικόνα?})~4 JOSS
  (Wikimedia)}

Η καθιέρωση του τηλέτυπου ως βασική συσκευή διάδρασης με τον υπολογιστή
οδήγησε σταδιακά και στον διαχωρισμό της εισόδου από την έξοδο, αφού
στις περισσότερες περιπτώσεις ο προγραμματισμός γινόταν σε εργασίες
δέσμης, οι οποίες είχαν ετεροχρονισμένη έξοδο. Πράγματι, οι περισσότεροι
υπολογιστές εκείνης της εποχής ήταν πολύ ακριβοί και σχετικά αργοί για
διάδραση σε πραγματικό χρόνο. Ακόμη, πολλές δημοφιλείς εφαρμογές, όπως
ήταν οι λογιστικές και η μισθοδοσία, γίνονταν περιοδικά, αλλά με μεγάλη
ανάγκη για ταχύτατη εκτύπωση του αποτελέσματος σε μικρό χρόνο. Με αυτόν
τον τρόπο δημιουργήθηκε η ανάγκη για μια αποκλειστική συσκευή εξόδου, η
οποία οδήγησε στην κατασκευή του εκτυπωτή γραμμής. Βλέπουμε λοιπόν, ότι
αρχικά η διάδραση είχε καθοριστεί από τις υπάρχουσες συσκευές εισόδου
και εξόδου, οι οποίες βασίζονται στο πληκτρολόγιο και στην εκτύπωση
κειμένου, τα οποία άφησαν ένα διαχρονικό αποτύπωμα σε όλα τα σύγχρονα
συστήματα, ενώ παραμένουν θεμελιώδη σε συστήματα που βασίζονται στο
UNIX. Η χρήση του τηλέτυπου οδήγησε στην δημιουργία πολλών μορφών
διάδρασης, όπως είναι η γραμμή εντολών, η φόρμα, τα μενού και η φυσική
γλώσσα.

\leavevmode\vadjust pre{\hypertarget{fig:tty-model33}{}}%
\begin{figure}
\hypertarget{fig:tty-model33}{%
\centering
\includegraphics{images/tty-model33.jpg}
\caption{Εικόνα 3: Το μοντέλο 33 της εταιρείας Teletype αποτελεί μια
ολοκληρωμένη συσκευή εισόδου και εξόδου και ήταν πολύ δημοφιλής με τους
μίνι-υπολογιστές της δεκαετίας του 1960, καθώς και με τους πρώτους
μίκρο-υπολογιστές, αφού για πολλά χρόνια ήταν η πιο οικονομική και
οικεία λύση. Ο χρήστης μπορούσε να αλληλεπιδράσει με τον υπολογιστή με
το πληκτρολόγιο και να δει την έξοδο να τυπώνεται στο χαρτί, ενώ η
διάτρητη χαρτοταινία επέτρεπε τόσο την αποθήκευση δεδομένων, όσο και τη
φόρτωση τους.}\label{fig:tty-model33}
}
\end{figure}

\leavevmode\vadjust pre{\hypertarget{fig:joss}{}}%
\begin{figure}
\hypertarget{fig:joss}{%
\centering
\includegraphics{images/joss.jpg}
\caption{Εικόνα 4: Το περιβάλλον προγραμματισμού JOSS δεν έκανε διάκριση
ανάμεσα στη γλώσσα προγραμματισμού και στο λειτουργικό σύστημα και
επηρέασε την έκδοση της BASIC για μικροϋπολογιστές της δεκαετίας του
1980, γιατί βασιζόταν σε μια σύγχρονη αλληλεπίδραση ανθρώπου και
υπολογιστή, σε αντίθεση με τα συστήματα εκείνης της εποχής, τα οποία
βασίζονταν, κυρίως, σε εργασίες δέσμης. Επίσης, η JOSS δείχνει την
κατανόηση της διάδρασης ως μια διαλογική επικοινωνία ανθρώπου-υπολογιστή
και ταυτόχρονα αποτυπώνει τον ρόλο των συσκευών διάδρασης σε αυτόν τον
διάλογο, που στην περίπτωση της ήταν ο τηλέτυπος.}\label{fig:joss}
}
\end{figure}

Εξίσου καθοριστική ήταν η εφεύρεση της συσκευής εισόδου ποντίκι, η οποία
επιτρέπει την έμμεση διάδραση με μεγάλη ακρίβεια και άνεση, αρχικά με
κείμενο και στη συνέχεια με σημεία της οθόνης.

Η συσκευή εισόδου πένα έχει ιδιαίτερο ενδιαφέρον, καθώς συνδυάζει και
τους δύο τύπους διάδρασης με συσκευές εισόδου, αλλά, κυρίως, επειδή
ιστορικά εμφανίστηκε πολύ νωρίτερα από το ποντίκι, αλλά δεν είχε την
ίδια αποδοχή. Πρώτα από όλα, η πένα είναι μια συσκευή εισόδου που μπορεί
να λειτουργήσει τόσο ως συσκευή άμεσης όσο και έμμεσης διάδρασης. Για
παράδειγμα, μπορούμε να χρησιμοποιήσουμε την πένα απευθείας πάνω σε μια
οθόνη, οπότε έχουμε μια διάδραση που είναι ανάλογη της χρήσης του
μολυβιού πάνω σε χαρτί. Επιπλέον, μπορούμε να χρησιμοποιήσουμε την πένα
ως συσκευή έμμεσης εισόδου, όπως χρησιμοποιούμε το ποντίκι, και τότε η
διαφορά, εκτός από το κράτημα της συσκευής, βρίσκεται στο γεγονός ότι η
πένα μπορεί να έχει ένα-προς-ένα σχέση με τη συσκευή εξόδου, ακόμη και
στην περίπτωση που λειτουργεί ως έμμεση συσκευή εισόδου. Αν και από τα
παραπάνω η πένα φαίνεται να είναι μια ευέλικτη και επιθυμητή συσκευή
εισόδου, στην πράξη έχει αποδειχτεί ότι κουράζει το χέρι, ενώ δεν έχει
την ταχύτητα και την ακρίβεια του ποντικιού. Φυσικά, υπάρχουν κάποιες
επιμέρους χρήσεις, όπως στη σχεδίαση, όπου η πένα μπορεί να έχει πολλά
πλεονεκτήματα έναντι του ποντικιού.

Το ποντίκι στα πρώτα συστήματα διάδρασης ήταν απλώς μια συσκευή επιλογής
κειμένου. Στην πορεία, και καθώς η οθόνη εμπλουτίστηκε με περισσότερα
στοιχεία γραφικών, όπως τα παράθυρα και τα εικονίδια, το ποντίκι κράτησε
τον ρόλο του ως η πιο αποδοτική συσκευή επιλογής στόχου και μετακίνησης
αντικειμένων πάνω στην οθόνη. Από τα πρώτα εμπορικά βήματα, το ποντίκι
είχε διαφορετικό αριθμό πλήκτρων ανάλογα με τις ανάγκες του χρήστη. Για
παράδειγμα, οι υπολογιστές της Apple αρχικά συνοδεύονταν από ποντίκι με
ένα κουμπί, ενώ οι περισσότερες από τις άλλες εμπορικές προτάσεις είχαν
δύο ή τρία κουμπιά. Στο πέρασμα των χρόνων και καθώς το ποντίκι κέρδιζε
τη θέση του σε περισσότερες και πιο πολύπλοκες εφαρμογές, ο σχεδιασμός
του επαυξήθηκε, τόσο με επιπλέον κουμπιά όσο και με νέες λειτουργίες, οι
οποίες γεφύρωσαν το χάσμα με τις πολύ δημοφιλείς οθόνες αφής.

Καθώς ο υπολογιστής απέκτησε μεγαλύτερη ισχύ και δικτύωση και, κυρίως,
καθώς οι ανθρώπινες ανάγκες και χρήσεις του υπολογιστή επεκτάθηκαν και
σε άλλους τομείς πέρα από την εργασία, νέες συσκευές εισόδου και εξόδου,
όπως η κάμερα, το μικρόφωνο, τα ηχεία, απέκτησαν σημασία. Η πρώτη
περίοδος των υπερμέσων και πολυμέσων ήταν περιορισμένη σε στατικά
αποθηκευτικά μέσα, όπως οι οπτικοί δίσκοι, αλλά η εξάπλωση της δικτύωσης
μετέφερε την αποθήκευση, την επεξεργασία και τη διανομή τους μέσω του
δικτύου των υπολογιστών με τη συμμετοχή των χρηστών.

\hypertarget{ux3baux3b1ux3c4ux3b1ux3c3ux3baux3b5ux3c5ux3ae-ux3bdux3adux3c9ux3bd-ux3b4ux3b9ux3b5ux3c0ux3b1ux3c6ux3ceux3bd}{%
\subsection{Κατασκευή νέων
διεπαφών}\label{ux3baux3b1ux3c4ux3b1ux3c3ux3baux3b5ux3c5ux3ae-ux3bdux3adux3c9ux3bd-ux3b4ux3b9ux3b5ux3c0ux3b1ux3c6ux3ceux3bd}}

Η διάκριση ανάμεσα σε συσκευές εισόδου και εξόδου είναι περισσότερο
τεχνητή παρά πραγματική και γίνεται χάριν ανάλυσης, αφού τελικά αυτό που
μας ενδιαφέρει, δηλαδή το μοντέλο της διάδρασης είναι πάντα ένας
συνδυασμός αυτών των δύο. Ο χρήστης μέσω της συσκευής εισόδου θα
μεταδώσει την πρόθεσή του στον υπολογιστή, ο οποίος θα επικοινωνήσει την
κατάστασή του μέσω μιας συσκευής εξόδου. Έχοντας προσεγγίσει τη διάδραση
τόσο από την πλευρά του ανθρώπου όσο και από την πλευρά του υπολογιστή,
θα στρέψουμε την προσοχή μας στον μεταξύ τους διάλογο, όπου θα
εξετάσουμε διάφορα μοντέλα διάδρασης.

Η μεγάλη πρόκληση στη σχεδίαση των μοντέλων διάδρασης βρίσκεται στο
γεφύρωμα των διαφορών που υπάρχουν ανάμεσα στην εικόνα που έχει ο
τελικός χρήστης για το σύστημα και σε εκείνη που έχουν οι κατασκευαστές
του για το πώς λειτουργεί το σύστημα εσωτερικά. Το χάσμα αυτό αποτελεί
πρόκληση, κυρίως γιατί οι ανάγκες των χρηστών είναι ένας κινούμενος
στόχος, αφού οι χρήστες έχουν μεγάλες διαφορές μεταξύ τους και,
επιπλέον, ο ίδιος χρήστης έχει διαφορετικές ανάγκες και προτιμήσεις,
ανάλογα με τη χρονική στιγμή και την περίσταση καθώς και διαχρονικά.
Τέλος, η κάθε τεχνολογική παρέμβαση επηρεάζει ή τουλάχιστον
επαναπροσδιορίζει τις ανάγκες των χρηστών και δημιουργεί μια
αυτοτροφοδοτούμενη ανάγκη για νέα μοντέλα διάδρασης.

\leavevmode\vadjust pre{\hypertarget{fig:chorded-input}{}}%
\begin{figure}
\hypertarget{fig:chorded-input}{%
\centering
\includegraphics{images/chorded-input.jpg}
\caption{Εικόνα 5: Το πληκτρολόγιο ακόρντων επιτρέπει την είσοδο 31
διαφορετικών κωδικών που καλύπουν όλο το λατινικό αλφάβητο ή μπορούν να
αντιστοιχούν σε τροπικές κεντολές. Σε συνδυασμό με μια αποδοτική
συσκευής επιλογής στόχων στην οθόνη, δεν υπάρχει ανάγκη για το
παραδοσιακό πληκτρολόγιο, ειδικά για τη συχνή διεργασία της απλής
επεξεργασίας έτοιμου κειμένου. Αν και είχε ήδη χρησιμοποιηθεί στο
παρελθόν στον τηλέτυπο και στην συντομογραφία, ο συνδυασμός με το
ποντίκι έχει το εργονομικό πλεονέκτημα ότι ο χρήστης δεν χρειάζεται να
μετακινήσει τα χέρια του.}\label{fig:chorded-input}
}
\end{figure}

\leavevmode\vadjust pre{\hypertarget{fig:vpl-data-glove}{}}%
\begin{figure}
\hypertarget{fig:vpl-data-glove}{%
\centering
\includegraphics{images/vpl-data-glove.png}
\caption{Εικόνα 6: Το γάντι δεδομένων περιέχει αισθητήρες που
καταγράφουν την θέση του χεριού και τις κινήσεις των δακτύλων, έτσι ώστε
να υπάρχει λεπτομερής χειρισμός σε περιβάλλον εικονικής πραγματικότητας
ή σε άλλες εφαρμογές, όπως στη ρομποτική.}\label{fig:vpl-data-glove}
}
\end{figure}

Οι σχεδιαστές προσπαθούν να γεφυρώσουν το χάσμα με γενικά μοντέλα
διάδρασης. Τα μοντέλα διάδρασης μας βοηθούν να κατανοήσουμε τι συμβαίνει
κατά την επικοινωνία μεταξύ του χρήση και του συστήματος. Η εργονομία
ασχολείται με τα φυσικά χαρακτηριστικά της διάδρασης και με τον τρόπο
που αυτά επηρεάζουν την αποτελεσματικότητά της. Ο διάλογος μεταξύ του
χρήστη και του συστήματος επηρεάζεται από το στυλ της διεπαφής ανθρώπου
και υπολογιστή, ενώ το στυλ αυτό είναι το βασικό αντικείμενο της
σχεδίασης, όπως τουλάχιστον θα φανεί στον τελικό χρήστη. Η διάδραση
λαμβάνει χώρα μέσα σε ένα κοινωνικό και οργανωτικό πλαίσιο, το οποίο
επηρεάζει τόσο τον χρήστη όσο και το σύστημα. Τα υποδείγματα διάδρασης
παρέχουν μια καλή θεώρηση του ιστορικού των διαδραστικών συστημάτων
υπολογιστών.

Στις προηγούμενες ενότητες είδαμε ξεχωριστά μια σειρά από συσκευές
εισόδου (π.χ. ποντίκι) και εξόδου (π.χ. οθόνη), καθώς και τα στυλ
διάδρασης που επιτρέπουν (π.χ. μενού), αλλά δεν έχουμε δει καθόλου τους
συνδυασμούς τους. Ειδικά η δημοφιλής επιφάνεια εργασίας στον επιτραπέζιο
υπολογιστή είναι παράδειγμα μονοδιάστατης αντίληψης για τη διάδραση
ανθρώπου και υπολογιστή. Αν φανταστούμε πώς βλέπει τον χρήστη ένας
υπολογιστής με διάδραση τύπου επιφάνειας εργασίας, τότε καταλήγουμε ότι
η εικόνα που έχει ο υπολογιστής για εμάς δεν είναι πλήρης, αλλά μοιάζει
με μια παλάμη, ένα δάκτυλο κι ένα μάτι, αφού αυτά αρκούν για αυτόν τον
τύπο διάδρασης. Με δεδομένη τη δυνατότητα του ανθρώπου να εκφράζει τις
προθέσεις του και να προσλαμβάνει ερεθίσματα με ένα πολύ πλουσιότερο
φάσμα κινήσεων και αισθήσεων, καταλήγουμε ότι η διάδραση με την
επιφάνεια εργασίας είναι απλά μια μικρή υποπερίπτωση της διάδρασης
ανθρώπου και υπολογιστή. \footnote{(\textbf{Εικόνα?})~5 Επεξεργασία
  κειμένου και εντολές με ακόρντα (Wikipedia)} \footnote{(\textbf{Εικόνα?})~6
  VPL DataGlove (VPL)}

Υπάρχει μια πολύ μεγάλη ποικιλία από συσκευές εισόδου για τον υπολογιστή
και η επιλογή της κατάλληλης συσκευής εξαρτάται λιγότερο από την
τεχνολογία και περισσότερο από τον χρήστη, το πλαίσιο χρήσης, καθώς και
από τις διεργασίες του χρήστη. Αρχικά, στους πρώτους κεντρικούς
υπολογιστές, η είσοδος βασιζόταν στο χαρτί, αφού το χαρτί ήταν από πολύ
παλιά ένα μέσο οικείο για τον άνθρωπο. Καθώς, όμως, οι υπολογιστές
μετασχηματίζονται, με νέες φόρμες και χρήσεις, σε επιτραπέζιους,
κινητούς και διάχυτους, δημιουργούνται νέοι τρόποι ελέγχου και νέες
συσκευές εισόδου. Μερικές από τις πιο δημοφιλείς συσκευές εισόδου στους
επιτραπέζιους υπολογιστές είναι το πληκτρολόγιο, το ποντίκι, η πένα, το
χειριστήριο (παιχνιδιών), το trackpad, η κάμερα κ.ά. Στον κινητό
υπολογισμό έχουμε επιπλέον συστήματα εισόδου, όπως η γεωγραφική θέση, ο
προσανατολισμός, το επιταγχυσιόμετρο, ενώ στον διάχυτο υπολογισμό (π.χ.
φορετοί υπολογιστές, έξυπνα ρολόγια) έχουμε επιπλέον συστήματα εισόδου,
όπως οι αισθητήρες περιβαλλοντικών και βιολογικών σημάτων. Η επιλογή
συσκευών εισόδου γίνεται περισσότερο περίπλοκη όταν θέλουμε να
συνδυάσουμε διαφορετικές συσκευές εισόδου σε μια πολυτροπική σύνθεση.

\leavevmode\vadjust pre{\hypertarget{fig:trackball}{}}%
\begin{figure}
\hypertarget{fig:trackball}{%
\centering
\includegraphics{images/trackball.jpg}
\caption{Εικόνα 7: Όπως και η πένα εισόδου, έτσι και η μπάλα κύλισης
δημιουργήθηκε αρχικά για να διευκολύνει τον εντοπισμό σημείων πάνω σε
μια οθόνη ραντάρ που οπτικοποιεί πλοία. Η μπάλα κύλισης παράμεινε πάντα
σε χρήση για ορισμένες εφαρμογές και έδωσε την έμπνευση για την πιο
σημαντική βελτίωση στη χρήση του ποντικιού, του οποίου οι τροχοί
αντικαταστάθηκαν από μια μικρή μπάλα.}\label{fig:trackball}
}
\end{figure}

\leavevmode\vadjust pre{\hypertarget{fig:telefunken-ball-mouse}{}}%
\begin{figure}
\hypertarget{fig:telefunken-ball-mouse}{%
\centering
\includegraphics{images/telefunken-ball-mouse.jpg}
\caption{Εικόνα 8: Το αρχικό ποντίκι με τους δύο τροχούς κύλισης, οι
οποίοι ακουμπάνε απευθείας πάνω στο τραπέζι μπορούσε να επιλέγει
κείμενο, αλλά δεν ήταν βέλτιστο για την ελεύθερη επιλογή σημείων στην
οθόνη, γιατί μπορούσε να κινηθεί μόνο σε οριζόντιες και κατακόρυφες
ευθείες γραμμές. Οι μηχανικοί της Telefunken, βασιζόμενοι στη σχεδίαση
της μπάλας κύλισης, την οποία γύρισαν ανάποδα, έφτιαξαν το πρώτο
εύχρηστο ποντίκι για γραφικά
περιβάλλοντα.}\label{fig:telefunken-ball-mouse}
}
\end{figure}

Ένας δημοφιλής και απλός τρόπος για να ταξινομήσουμε τις συσκευές
εισόδου είναι με δύο παραμέτρους, οι οποίες αντιστοιχούν στον αριθμό των
διαστάσεων και στην ιδιότητα που παρακολουθεί η συσκευή εισόδου. Για
παράδειγμα, το ποντίκι μπορεί να έχει μια ροδέλα κύλισης, η οποία
επιτρέπει την εύκολη κατακόρυφη ροή των σελίδων του κειμένου στην οθόνη,
ενώ, ταυτόχρονα, παρακολουθεί τη θέση της συσκευής πάνω στις δύο
διαστάσεις του τραπεζιού για να μετακινήσει αντίστοιχα τον δείκτη στην
οθόνη. Με τη χρήση μιας κάμερας βάθους πεδίου μπορούμε να
παρακολουθήσουμε την κίνηση των δακτύλων ή όλου του σώματος σε τρεις
διαστάσεις. Βλέπουμε, λοιπόν, ότι για την ίδια παράμετρο (αριθμός
διαστάσεων) μπορούμε να έχουμε την ίδια ή διαφορετικές συσκευές εισόδου,
καθώς και ένα μεγάλο εύρος από αισθητήρες, οι οποίοι μπορεί να
αποτελούνται από μηχανικά, ηλεκτρονικά και οπτικά μέρη. Εκτός από την
παράμετρο του αριθμού των διαστάσεων, τις οποίες καταγράφει μια συσκευή
εισόδου, η δεύτερη παράμετρος είναι το είδος της κίνησης που
καταγράφεται. Το είδος της κίνησης μπορεί να είναι η θέση (π.χ. ποντίκι
πάνω σε τραπέζι), η κίνηση (π.χ. ροδέλα κύλισης), και η πίεση (π.χ.
πλήκτρο). Η αποτύπωση των παραμέτρων και η συμπλήρωση των αντίστοιχων
συσκευών εισόδου μας επιτρέπει να δημιουργήσουμε έναν σχεδιαστικό χώρο,
στον οποίο φαίνονται άμεσα οι ευκαιρίες για νέες συσκευές εισόδου.
\footnote{(\textbf{Εικόνα?})~7 Trackball (Courtesy of J. Vardalas)}
\footnote{(\textbf{Εικόνα?})~8 Ποντίκι με μπάλα κύλισης από την
  Telefunken (Computermuseun der Informatik, at University of Stuttgart)}

Η κατασκευή νέων συσκευών εισόδου είναι μια πολύ ενδιαφέρουσα αλλά και
σύνθετη εργασία. Από τη μια πλευρά, η κατασκευή νέων συσκευών εισόδου
δίνει στον άνθρωπο νέες επαυξημένες δυνατότητες για να ενεργήσει στον
κόσμο της πληροφορίας, ο οποίος αντιπροσωπεύει ή ακόμη και ελέγχει τον
πραγματικό κόσμο. Από την άλλη πλευρά, η κατασκευή νέων συσκευών εισόδου
επηρεάζεται από ένα μεγάλο εύρος παραμέτρων, των οποίων η σύνθεση
δύσκολα γίνεται γνωστή σε βάθος από τους σχεδιαστές. Για παράδειγμα, η
σχεδίαση της συσκευής εισόδου ποντίκι υλοποίησε στο ακέραιο το όραμα του
σχεδιαστή της για την επαύξηση της ανθρώπινης σκέψης, αφού έδωσε
πρόσβαση στους υπολογιστές και στην πληροφορία σε ένα μεγάλο εύρος
χρηστών πέρα από τους ειδικούς. Ταυτόχρονα, η κατασκευή της συσκευής
εισόδου `ποντίκι' δεν ήταν άμεσα προφανής, αφού υπάρχουν πάντα πολλές
εναλλακτικές συσκευές εισόδου για τον ίδιο σκοπό. Ακόμη, η συσκευή
εισόδου ποντίκι βρίσκεται σε έναν συνεχή μετασχηματισμό, καθώς
επηρεάζεται από πολλούς παράγοντες, όπως το πλαίσιο χρήσης, οι
προτιμήσεις και η κατανόηση που έχουν οι σχεδιαστές της. Για παράδειγμα,
τα δημοφιλή δικτυακά βιντεοπαιχνίδια δράσης τρίτου προσώπου δημιούργησαν
την ανάγκη για ποντίκια με πολλά πλήκτρα. Αντίστοιχα, η κατασκευή κάθε
νέας συσκευής εισόδου θα πρέπει να ισορροπήσει ανάμεσα σε όλες τις
παραπάνω δυνάμεις.

Οι υπολογιστές έχουν τη δυνατότητα να επεξεργάζονται πληροφορία, την
οποία τελικά αναπαριστούν μέσα από πολλά διαφορετικά κανάλια. Οι πρώτοι
κεντρικοί υπολογιστές είχαν ως έξοδο το χαρτί, ακριβώς όπως είχαν το
χαρτί ως είσοδο, γιατί αυτός ήταν ο πιο αποτελεσματικός τρόπος διάδρασης
με τους ανθρώπους, οι οποίοι, από πολύ παλιά, έχουν μια οικειότητα με το
χαρτί. Καθώς ο υπολογιστής απέκτησε νέες μορφές, όπως είναι ο
επιτραπέζιος, ο κινητός και ο διάχυτος, αναπτύχθηκαν νέοι τρόποι
αναπαράστασης της πληροφορίας. Η οθόνη του υπολογιστή είναι, με μεγάλη
διαφορά, η πιο δημοφιλής συσκευή εξόδου, επειδή μπορεί να έχει πολλά
σχήματα και μεγέθη, ανάλογα με τη φόρμα του υπολογιστή, π.χ.
επιτραπέζιος, φορητός, φορετός, δωματίου. Στην περίπτωση που πρόκειται
για οθόνη εικονοστοιχείων, μπορεί να οπτικοποιήσει την πληροφορία με
πολλούς διαφορετικούς τρόπους. \footnote{(\textbf{Εικόνα?})~9
  Γραφομηχανή τηλεόρασης (Wikimedia)} \footnote{(\textbf{Εικόνα?})~10
  Teletext (wikimedia)} Εκτός από την οθόνη, τα ηχεία επιτρέπουν στον
υπολογιστή να επικοινωνήσει μέσω του ήχου, ή ακόμη και μέσω της φυσικής
γλώσσας με ομιλία. Στις μικρότερες φόρμες των υπολογιστών, π.χ. κινητός,
φορετός, καθώς και στο πλαίσιο χρήσης, στο οποίο η οπτική προσοχή του
ανθρώπου είναι στραμμένη αλλού, π.χ. οδήγηση, άθληση, οι ενδεικτικές
λυχνίες καθώς και η δόνηση αποκτούν σημαντικό ρόλο.

\leavevmode\vadjust pre{\hypertarget{fig:tv-typewriter}{}}%
\begin{figure}
\hypertarget{fig:tv-typewriter}{%
\centering
\includegraphics{images/tv-typewriter.jpg}
\caption{Εικόνα 9: Η μεγάλη διαθεσιμότητα των συσκευών τηλεόρασης τη
δεκαετία του 1970 επέτρεψε σε πολλούς ερασιτέχνες να συνδέσουν ένα
πληκτρολόγιο για να γράψουν κείμενο στην οθόνη της. Αν και οι συσκευές
αυτές δεν είχαν κάποιο επεξεργαστή, στη συνέχεια επεκτάθηκαν για να
συνδεθούν με τους πρώτους μικρο-υπολογιστές.}\label{fig:tv-typewriter}
}
\end{figure}

\leavevmode\vadjust pre{\hypertarget{fig:teletext}{}}%
\begin{figure}
\hypertarget{fig:teletext}{%
\centering
\includegraphics{images/teletext.jpg}
\caption{Εικόνα 10: Το teletext είναι ένα από τα πρώτα λειτουργικά
συστήματα διαδραστικής πληροφόρησης, το οποίο βασίζεται στην αναλογική
τεχνολογία μετάδοσης και απεικόνισης τηλεοπτικού σήματος. Αρχικά
σχεδιάστηκε για τη μετάδοση προαιρετικών υπότιτλων για όσους έχουν
προβλήματα ακοής, αλλά στη συνέχεια έγινε δημοφιλές για τις σελίδες
πληροφόρησης σχετικά με τον καιρό, τις ειδήσεις, αθλητικά και σελίδες
τοπικού ενδιαφέροντος αφού η εμβέλεια του τηλεοπτικού σήματος είναι από
τη φύση της γεωγραφικά περιορισμένη.}\label{fig:teletext}
}
\end{figure}

Η κυριαρχία της οθόνης ως συσκευής εξόδου είναι τόσο μεγάλη που στις
περισσότερες πηγές για την κατασκευή της διάδρασης υπονοείται η χρήση
της, χωρίς να αφήνονται περιθώρια για τη θεώρηση εναλλακτικών συσκευών
εξόδου. Επιπλέον, η ωριμότητα και το προσιτό κόστος των προβολών σε δύο
διαστάσεις έχει αφήσει στο περιθώριο την πιο φυσική για τον άνθρωπο
προβολή σε τρεις διαστάσεις και τα αντίστοιχα εικονικά περιβάλλοντα. Από
μια ανθρωποκεντρική σκοπιά, αν εξετάσουμε τα αισθητήρια όργανα του
ανθρώπου, διαπιστώνουμε ότι η αίσθηση της αφής, ειδικά αυτή στις άκρες
των δακτύλων, είναι από τις πιο πλούσιες σε νευρικές απολήξεις, τόσο σε
εύρος όσο και σε είδος, αφού μπορεί να αντιληφθεί την υφή, το σχήμα και
τη θερμοκρασία των υλικών. Με δεδομένο ότι οι διαθέσιμες τεχνολογίες που
επαυξάνουν την αφή δεν είναι ακόμη τόσο ώριμες όσο οι τεχνολογίες για
τις οθόνες, γίνεται φανερό ότι η κυριαρχία της οθόνης ως συσκευής εξόδου
είναι περισσότερο το αποτέλεσμα μιας τεχνολογικής περίστασης και όχι
τόσο της ανάγκης για ανθρωποκεντρική κατασκευή της διάδρασης. Όπως και
στην περίπτωση των συσκευών εισόδου, έτσι και για τις συσκευές εξόδου τα
πιο ενδιαφέροντα αλλά και δύσκολα στην κατασκευή συστήματα διάδρασης
βασίζονται στη σύνθετη ή στην πολυτροπική διάδραση, η οποία εξετάζεται
στην επόμενη ενότητα.

Με τον όρο φυσική διάδραση \footnote{O'Sullivan and Igoe (2004)}
εννοούμε ένα σύνολο από δεξιότητες που είναι δεδομένες για τους
ανθρώπους, όπως είναι οι χειρονομίες, οι εκφράσεις του προσώπου, η
αναγνώριση της φυσικής γλώσσας. Η φυσική διάδραση ανθρώπου και
υπολογιστή φαίνεται να είναι ο πιο λογικός τρόπος για να γεφυρώσουμε τις
διαφορές που έχουν άνθρωποι και υπολογιστές. Στην πράξη, αν και τα
συστήματα φυσικής διάδρασης έχουν ωριμάσει αρκετά, η αποτελεσματικότητά
τους είναι αποδεκτή μόνο σε πολύ στενά πλαίσια της ανθρώπινης
δραστηριότητας. Οι περιορισμοί που έχουν τα συστήματα φυσικής διάδρασης
έχουν να κάνουν με την αδυναμία να δώσουν ανάδραση για την παρούσα
κατάσταση και, κυρίως, με την περιορισμένη εικόνα που έχει ο χρήστης για
τις πιθανές δράσεις. Από τη μια πλευρά, ο χρήστης δεν χρειάζεται να
προσαρμοστεί στο μοντέλο λειτουργίας του συστήματος (αφού αυτό είναι
τελείως φυσικό και βασίζεται στις ανθρώπινες δεξιότητες), από την άλλη
πλευρά, όμως, η φυσική διάδραση δεν επιτρέπει τη δημιουργία σύνθετων
συστημάτων, αφού είναι δύσκολο να ξέρει ο χρήστης πού βρίσκεται και τι
μπορεί να κάνει. Επιπλέον, η φυσική διάδραση σε πολλές περιπτώσεις είναι
δύσκολο να κατασκευαστεί με τρόπο γενικό, που να καλύπτει όλους τους
ανθρώπους, επειδή υπάρχει μια πολύ μεγάλη διακύμανση σε φυσικές
δεξιότητες, όπως είναι οι χειρονομίες ή η ομιλία.

O σχεδιασμός του υπολογιστή και ειδικά των συσκευών εισόδου και εξόδου
μπορεί να επιφέρει συγκεκριμένη ανάδραση από τους χρήστες. Το αν οι
χρήστες θα αγοράσουν, θα μάθουν, θα χρησιμοποιήσουν ένα προϊόν ή αν θα
συνεργαστούν με άλλους, εξαρτάται σε μεγάλο βαθμό από το πόσο άνετα
νιώθουν όταν βλέπουν και κρατάνε το αντίστοιχο σύστημα, καθώς και από το
πόσο το εμπιστεύονται. Εάν ο υπολογιστής αργεί και είναι ενοχλητικός,
τότε είναι πιθανό οι χρήστες να αποφύγουν τη διάδραση. Εάν όμως το
σύστημα εισόδου και εξόδου είναι ευχάριστο και γρήγορο, τότε η χρήση του
γίνεται περισσότερο επιθυμητή και άνετη, οπότε οι χρήστες είναι πιθανό
να το αγοράσουν και να το χρησιμοποιήσουν. Ταυτόχρονα, κάτι που είναι
πολύ ευχάριστο και εύχρηστο αρχικα συνήθως δεν βελτιώνει τις δεξιότητες
των ανθρώπων και δημιουργεί μια επανάπαυση. Επομένως, οι σχεδιαστές των
διεπαφών θα πρέπει να επιλέξουν τις ιδιότητες της διάδρασης με προσοχή
γιατί οι μακροχρόνιες επιδράσεις είναι σημαντικές, όχι μόνο για την
διεργασία που εκτελείται, αλλά κυρίως για τους ανθρώπους.

\leavevmode\vadjust pre{\hypertarget{fig:makey_makey_front}{}}%
\begin{figure}
\hypertarget{fig:makey_makey_front}{%
\centering
\includegraphics{images/makey-makey.jpg}
\caption{Εικόνα 11: Το Makey Makey είναι ένα Arduino, το οποίο έχει
οργανωθεί και έχει προγραμματιστεί έτσι ώστε να διευκολύνει τον
πειραματισμό με νέες συσκευές εισόδου. Οι χρήστες μπορούν να συνδέσουν
στους ακροδέκτες του κάποιο αγώγιμο υλικό, το οποίο στην συνέχεια θα
παίξει τον ρόλο του δείκτη και της επιλογής στο λογισμικό του
υπολογιστή.}\label{fig:makey_makey_front}
}
\end{figure}

\leavevmode\vadjust pre{\hypertarget{fig:minecraft-pi}{}}%
\begin{figure}
\hypertarget{fig:minecraft-pi}{%
\centering
\includegraphics{images/minecraft-pi.jpg}
\caption{Εικόνα 12: Το RaspberryPi δημιουργήθηκε για να δώσει οικονομική
πρόσβαση στον προγραμματισμό του υπολογιστή σε όσο γίνεται περισσότερους
χρήστες και με ιδιαίτερη έμφαση στα παιδιά. Για αυτόν τον σκοπό, δεν
περιλαμβάνει συσκευές εισόδου και εξόδου, αλλά παρέχει την δυνατότητα
σύνδεσης τόσο με τα παραδοσιακές συσκευές όπως η οθόνη και το
πληκτρολόγιο, αλλά και νέες συσκευές που μπορούν να εφευρεθούν
μελοντικά.}\label{fig:minecraft-pi}
}
\end{figure}

Για την πρακτική κατασκευή συσκευών εισόδου και εξόδου έχουν
δημιουργηθεί συστήματα που διευκολύνουν τις αρχικές δοκιμές και
πειραματισμούς. Τα Lilypad και MakeyMakey είναι μίκρο-επεξεργαστές
(Arduino), οι οποίοι τεκμηριώνουν την ευελιξία της πλατφόρμας και
ανταποκρίνονται στην ανάγκη να προγραμματίσουμε τη διάδραση γρήγορα,
οικονομικά και πέρα από τις συσκευές εισόδου και εξόδου του επιτραπέζιου
υπολογιστή. \footnote{Igoe (2007)} Οι σχεδιαστές του MakeyMakey
παρατήρησαν ότι ένα μεγάλο μέρος της χρήσης του κλασικού Arduino
περιλάμβανε τη δοκιμή νέων συσκευών εισόδου. Οι επίδοξοι σχεδιαστές
έπρεπε να φτιάξουν ένα εξωτερικό κύκλωμα, το οποίο να διευκολύνει την
αγωγιμότητα ανάμεσα στη νέα συσκευή εισόδου και στο Arduino. Ενώ το
Arduino λειτουργεί ως γέφυρα ανάμεσα στην είσοδο του χρήστη και στον
επιτραπέζιο υπολογιστή, ο υπολογιστής επεξεργάζεται την εντολή που
λαμβάνει από το Arduino. Στις περισσότερες περιπτώσεις, μάλιστα, οι
εντολές από το Arduino έχουν αντιστοιχία με κουμπιά από το πληκτρολόγιο,
επειδή αυτή η επιλογή διευκολύνει πολύ την ανάπτυξη του προγράμματος
διάδρασης στον επιτραπέζιο υπολογιστή με οποιοδήποτε λογισμικό, από έναν
φυλλομετρητή μέχρι ένα εξειδικευμένο λογισμικό. Για αυτόν τον λόγο, το
MakeyMakey είναι ένα Arduino που λαμβάνει σήμα εισόδου από αγώγιμα υλικά
και τα μεταφράζει σε πατήματα του πληκτρολογίου. \footnote{(\textbf{Εικόνα?})~11
  Πλατφόρμα πειρασματισμού για νέες συσκευές εισόδου Makey Makey
  (Wikimedia)}

Μαζί με το Arduino, το RaspeberryPi είναι μια ακόμη συσκευή, η οποία
βασίζεται στον ανοικτό κώδικα για να προσφέρει έναν μικρό σε μέγεθος και
οικονομικό υπολογιστή. Οι ομοιότητες ανάμεσα στο Arduino και στο
RaspberryPi δεν σταματούν στον ανοικτό κώδικα και στην οικονομία, αλλά
συνεχίζονται και στα κίνητρα, αφού και στις δύο περιπτώσεις η εκπαίδευση
έπαιξε κυρίαρχο ρόλο, τουλάχιστον στον αρχικό σχεδιασμό. Η διαφορά στην
περίπτωση του RaspberryPi είναι ότι σχεδιάστηκε για την εκπαίδευση στον
προγραμματισμό των υπολογιστών δίνοντας έμφαση στα παιδιά. Το
RaspberryPi εμπνεύστηκε από τους πρώτους οικειακούς υπολογιστές της
δεκαετίας του 1970-80. Ήταν πολύ απλές συσκευές που μπορούσαν να
συνδεθούν στην τηλεόραση και αμέσως μετά κάποιος μπορούσε να αρχίσει να
δημιουργεί (ή έστω να έχει μια διάδραση) σε επίπεδο που να βοηθάει την
κατανόηση της λειτουργίας του υπολογιστή. Σε αντίθεση με την εικόνα που
δίνει ο παγκόσμιος ιστός και τα κοινωνικά δίκτυα (τα κυρίαρχα μέσα με τα
οποία μεγάλωσε η γενιά του 1990-2000), το RaspberryPi βασίζεται στο
λειτουργικό σύστημα Linux και πρεσβεύει μια πιο κοντινή και διαφανή
σχέση με τον υπολογιστή. \footnote{(\textbf{Εικόνα?})~12 MinecraftPi
  (RaspberryPi Foundation)}

Αν και σε πρώτη ανάγνωση τα RaspberryPi και Arduino μπορεί να φαίνονται
ανταγωνιστικά, αφού από την πλευρά της φόρμας και τους κόστους
βρίσκονται στην ίδια κατηγορία, στην πράξη, η λειτουργικότητά τους είναι
συμπληρωματική, κυρίως επειδή καλύπτουν πολύ διαφορετικές ανάγκες. Το
RaspberryPi είναι ένας ισχυρός πολυμεσικός υπολογιστής, κατάλληλος για
πολλές από τις εργασίες που κάνουν οι προσωπικοί επιτραπέζιοι και
φορητοί υπολογιστές. Στο πλαίσιο της κατασκευής συσκευών εισόδου και
εξόδου μπορεί να χρησιμοποιηθεί για την κατασκευή υποδειγμάτων, τα οποία
απαιτούν μεγάλη υπολογιστική ισχύ κατά την επεξεργασία αλλά και την
είσοδο-έξοδο προς τον χρήστη, αφού, εκτός από έναν δυνατό κεντρικό
επεξεργαστή, προσφέρει και ένα μεγάλο εύρος ζώνης στα κανάλια εισόδου
και εξόδου. Αντιθέτως, το Arduino ξεχωρίζει για τη χαμηλή υπολογιστική
ισχύ και το στενό εύρος ζώνης, τα οποία, όμως, έρχονται με μικρότερο
κόστος ενεργειακής λειτουργίας, γεγονός που το κάνει κατάλληλο για την
κατασκευή υποδειγμάτων για συσκευές που πρέπει να έχουν χαμηλή
κατανάλωση ενέργειας, όπως είναι οι φορετοί και οι διάχυτοι υπολογιστές.
Στην πορεία, οι κατασκευαστές των RaspberryPi και Arduino έχουν
δημιουργήσει κάποιες επιμέρους εκδόσεις των βασικών συστημάτων,
διευρύνοντας τις δυνατότητές τους είτε προς την οικονομία είτε προς την
ισχύ, ενώ υπάρχουν και πολλά άλλα παρόμοια συστήματα από ανταγωνιστές.

\hypertarget{ux3c3ux3cdux3bdux3b8ux3b5ux3c4ux3b1-ux3c3ux3c4ux3c5ux3bb-ux3b4ux3b9ux3acux3b4ux3c1ux3b1ux3c3ux3b7ux3c2}{%
\subsection{Σύνθετα στυλ
διάδρασης}\label{ux3c3ux3cdux3bdux3b8ux3b5ux3c4ux3b1-ux3c3ux3c4ux3c5ux3bb-ux3b4ux3b9ux3acux3b4ux3c1ux3b1ux3c3ux3b7ux3c2}}

Στο ερευνητικό σύστημα NLS του Stanford Research Institute, για πρώτη
φορά, τα συστήματα εισόδου και εξόδου του χρήστη είχαν ενδιάμεσα επίπεδα
αφαιρετικότητας, τα οποία επέτρεπαν τον έλεγχο διαφορετικών τύπων
πληροφορίας (π.χ. κειμένου και γραφικών), καθώς και διαφορετικές
συνθέσεις και οργανώσεις της πληροφορίας από μια νέα συσκευή εισόδου
όπως το ποντίκι.

Η δημοφιλής επιφάνεια εργασίας είναι ένα σύνθετο στυλ διάδρασης με τον
χρήστη, το οποίο στηρίζεται στον απευθείας χειρισμό με εικονίδια που
αντιπροσωπεύουν φακέλους και εργασίες, αλλά περιέχει και μενού (π.χ. για
τις επιμέρους λειτουργίες πάνω σε ένα αρχείο ή φάκελο), φόρμες (π.χ. για
τις επιμέρους ρυθμίσεις μιας εφαρμογής ή του λειτουργικού συστήματος),
καθώς και γραμμή εντολών (π.χ. για την αναζήτηση αρχείων ή για το
άνοιγμα εφαρμογών). Επίσης, από την πλευρά των συσκευών εισόδου, η
επιφάνεια εργασίας μπορεί να λειτουργήσει με διαφορετικούς τρόπους
(π.χ., ποντίκι, πληκτρολόγιο, φωνή, γραφή, πένα, κτλ.) και ανάλογα με
τις ανάγκες και τις προτιμήσεις του χρήστη. Η ανάγκη συνδυασμού των
διαφορετικών στυλ διάδρασης δείχνει ότι δεν υπάρχει κάποιο που να
υπερτερεί έναντι των άλλων, αλλά ότι όλα παίζουν έναν διαφορετικό ρόλο,
ανάλογα με τις ανάγκες και τους σκοπούς του χρήστη.

\leavevmode\vadjust pre{\hypertarget{fig:menus-on-windows}{}}%
\begin{figure}
\hypertarget{fig:menus-on-windows}{%
\centering
\includegraphics{images/menus-on-windows.png}
\caption{Εικόνα 13: Σε ένα παραθυρικό περιβάλλον εργασίας τα μενού
μπορεί να βρίσκονται στην κορυφή των παραθύρων της εφαρμογής, το οποίο
προσφέρει μια συνοχή, ειδικά όταν έχουμε πολλά ανοικτά, ορατά και
επικαλυπτόμενα παράθυρα, αλλά, ταυτόχρονα, κάνει κάπως δυσκολότερη την
επιλογή των μενού με τον δείκτη του ποντικιού. Αυτή η σχεδίαση είναι η
πιο δημοφιλής, καθώς χρησιμοποιείται από τα Microsoft Windows και από
πολλές εκδοχές στα παραθυρικά περιβάλλοντα του Linux, ενώ ήταν και η
πρώτη που δοκιμάστηκε εργαστηριακά από τους σχεδιαστές του Apple
Liza.}\label{fig:menus-on-windows}
}
\end{figure}

\leavevmode\vadjust pre{\hypertarget{fig:menus-on-top}{}}%
\begin{figure}
\hypertarget{fig:menus-on-top}{%
\centering
\includegraphics{images/menus-on-top.png}
\caption{Εικόνα 14: Σε ένα παραθυρικό περιβάλλον εργασίας τα μενού
μπορεί να βρίσκονται στην κορυφή της οθόνης, με αυτήν την επιλογή να
είναι αντικειμενικά περισσότερο εργονομική προς τον χρήστη, χωρίς, όμως,
να έχει κερδίσει την αντίστοιχη αποδοχή στην αγορά, αν και ήταν η πρώτη
σχεδίαση που έγινε διαθέσιμη εμπορικά με τον Apple
Macintosh.}\label{fig:menus-on-top}
}
\end{figure}

Το σύνθετο στυλ διάδρασης της επιφάνειας εργασίας, εκτός από τα παράθυρα
και τα εικονίδια, περιέχει και μενού, τα οποία μπορεί να επιλέξει ο
χρήστης και να εξερευνήσει με τη βοήθεια του δείκτη του ποντικιού ή με
κάποια ανάλογη συσκευή εισόδου. Τα μενού αλλάζουν περιεχόμενο ανάλογα με
την εφαρμογή του χρήστη που είναι στο προσκήνιο, αλλά η σημαντικότερη
διαφορά ανάμεσα στα διαθέσιμα λειτουργικά συστήματα αφορά τη θέση τους,
η οποία μπορεί να είναι είτε πάνω στο παράθυρο της εφαρμογής (π.χ.
Microsoft Windows), είτε στην κορυφή της οθόνης (π.χ. Apple MacOS). Τα
γραφικά παραθυρικά συστήματα στο λειτουργικό σύστημα Linux ακολουθούν
έναν από τους δύο τρόπους, με αποτέλεσμα να δημιουργείται μια ασυνέπεια
μεταξύ τους, ακόμη και σε αυτό το βασικό στυλ διάδρασης (μενού). Αν και
τα περισσότερα στοιχεία της διάδρασης με τον χρήστη εμπεριέχουν και μια
διάσταση συνήθειας και προτίμησης, τα μενού στην κορυφή της οθόνης είναι
αντικειμενικά πιο εργονομικά, αφού ο δείκτης δεν μπορεί να κινηθεί πέρα
από αυτήν. \footnote{(\textbf{Εικόνα?})~13 Μενού στην κορυφή των
  παραθύρων (Wikipedia)} \footnote{(\textbf{Εικόνα?})~14 Μενού στην
  κορυφή της οθόνης (Wikipedia)}

Η πολυτροπική διάδραση είναι η πιο ανθρωποκεντρική προσπάθεια για τη
διάδραση, αφού προσπαθεί να χρησιμοποιήσει παράλληλα και συνθετικά όλα
τα διαθέσιμα κανάλια επικοινωνίας ανάμεσα στον άνθρωπο και τον
υπολογιστή. Αρχικά εφαρμόστηκε για να δώσει καθολική πρόσβαση στους
υπολογιστές σε χρήστες που είχαν διαφορετικές ικανότητες. Για
παράδειγμα, ένας χρήστης που δεν βλέπει μπορεί να χρησιμοποιήσει μια
συσκευή εισόδου, όπως ένα ποντίκι, το οποίο παρέχει και ανάδραση με
δόνηση, έτσι ώστε να μπορεί να κάνει επιλογές σε μενού. Στην πορεία,
διαπιστώθηκε ότι οι διαφορετικές ικανότητες των χρηστών δεν
περιορίζονται μόνο σε επιμέρους ομάδες ανθρώπων, αλλά μπορούν να
γενικευτούν σε πολλά πλαίσια χρήσης. Για παράδειγμα, ένας χρήστης που
οδηγεί αυτοκίνητο, μπορεί να θεωρηθεί ότι δεν βλέπει την οθόνη διάδρασης
του αυτοκινήτου. Επιπλέον, η πολυτροπική διάδραση δημιούργησε ένα νέο
επαυξημένο επίπεδο αναφοράς σχετικά με την αντίληψη που έχουμε για τις
ανθρώπινες αισθήσεις, τις οποίες μπορούμε πλέον να χειριστούμε ως απλές
διεπαφές για την πληροφορία, ανεξάρτητα από τη φύση της πληροφορίας. Για
παράδειγμα, η αίσθηση της δόνησης μπορεί να χρησιμοποιηθεί ως διεπαφή
για να μεταφέρει σε έναν χρήστη την πληροφορία της γεωγραφικής
κατεύθυνσης. Με αυτόν τον τρόπο, οι υπάρχουσες αισθήσεις του ανθρώπου
μπορούν να επαυξηθούν με νέες αισθήσεις, των οποίων τη διάδραση με το
περιβάλλον μπορούμε να προγραμματίσουμε και μπορούμε να εκπαιδεύσουμε
τους χρήστες στο να τη χρησιμοποιούν.

Ο συνδυασμός των συστημάτων εισόδου και εξόδου έχει επιτρέψει τη
δημιουργία μιας σειράς από επιτυχημένα στυλ διάδρασης, τα οποία
εμφανίζονται είτε ανεξάρτητα είτε σε σύνθετες μορφές. Το πιο παλιό στυλ
διάδρασης είναι η γραμμή εντολών, σύμφωνα με το οποίο ο χρήστης
πληκτρολογεί τις εντολές. Η φόρμα είναι ένα εξίσου παλιό στυλ διάδρασης,
αφού μπορεί να εμφανιστεί ακόμη και σε τερματικά κειμένου, ενώ το ίδιο
ισχύει και για το μενού~εντολών. Ο απευθείας χειρισμός βασίζεται είτε
στην αφή είτε σε μια συσκευή εισόδου όπως η πένα και το ποντίκι, καθώς
και σε εικονίδια που αναπαριστούν αντικείμενα και δράσεις. Η φυσική
διάδραση βασίζεται στην απευθείας αναγνώριση της ανθρώπινης γλώσσας από
τον υπολογιστή και έχει πολλές μορφές, όπως την αναγνώριση κειμένου από
το πληκτρολόγιο και την αναγνώριση γραφής και ομιλίας. Η επαυξημένη
πραγματικότητα επιτρέπει τη διάδραση με ψηφιακά αντικείμενα, τα οποία
φαίνονται σαν να εμφανίζονται πάνω στον πραγματικό κόσμο, όπως αυτός
φαίνεται μέσα από μια κάμερα. Τέλος, η εικονική πραγματικότητα είναι ένα
στυλ διάδρασης, το οποίο προσομοιώνει τη διάδραση του ανθρώπου με τον
πραγματικό κόσμο και απαιτεί τη χρήση συσκευών εμβύθησης, όπως είναι οι
οθόνες σε μάσκα που αναγνωρίζει το βλέμα και τα γάντια που αναγνωρίζουν
τα δάκτυλα και τις χειρονομίες.

Ένα παράδειγμα σύνθετης διάδρασης, η οποία κάνει ένα βήμα μπροστά και
εμπλουτίζει την επιφάνεια εργασίας με περισσότερες κινήσεις από το
ανθρώπινο ρεπερτόριο, είναι αυτό του υπολογιστή ταμπλέτας με είσοδο από
πένα. Το σύστημα αυτό παρουσιάζει ένα πλεονέκτημα τουλάχιστον στις
διεργασίες επεξεργασίας κειμένου, καθώς ο χρήστης δεν χρειάζεται να
πάρει το χέρι του από το πληκτρολόγιο και να μετακινήσει το ποντίκι για
να επιλέξει μια επιπλέον λειτουργία. Αντί αυτής της εναλλαγής των δύο
διαφορετικών συσκευών εισόδου, η διάδραση με την ταμπλέτα και την πένα
επιτρέπει τη χρήση φυσικής διάδρασης για τη συγγραφή με την πένα, ενώ η
πένα μπορεί να λειτουργήσει και ως δείκτης, ώστε να γίνει η επιλογή
κάποιας επιπλέον λειτουργίας.

Σε έναν επιτραπέζιο υπολογιστή με πληκτρολόγιο, ποντίκι, και οθόνη δύο
διαστάσεων συνήθως έχουμε τα παράθυρα και την επιφάνεια εργασίας ως
βασικό τρόπο της διάδρασης με τον χρήστη. Στην κατηγορία αυτή εμπίπτουν
πολλά συστήματα, τα οποία μπορεί να έχουν επιμέρους διαφορές, τόσο στην
εμφάνιση όσο και στη λειτουργία τους. Για παράδειγμα, η πιο απλή
επιφάνεια εργασίας στο λειτουργικό σύστημα UNIX, το οποίο βασίζεται στο
παραθυρικό σύστημα X-Windows, χρησιμοποιεί τα παράθυρα για να οργανώσει
εφαρμογές, οι οποίες έχουν διεπαφή με κείμενο και όχι με εικονίδια.
Φυσικά, υπάρχουν παραθυρικά συστήματα στο UNIX τα οποία είναι εξίσου
πλούσια με αυτά που συναντάμε στα εμπορικά συστήματα MacOS και Windows.
Ένα μεγάλο μέρος του πλαισίου χρήσης του UNIX αφορά λειτουργίες
διαχείρισης συστήματος στα χαμηλότερα επίπεδα (π.χ., χρήστες, βάση
δεδομένων, δίκτυα, κτλ.), λειτουργίες, δηλαδή, που δεν έχουν μεγάλο
όφελος από την ύπαρξη ενός πλήρους παραθυρικού περιβάλλοντος. Το
συμπέρασμα από αυτό το παράδειγμα είναι ότι τα παράθυρα είναι απλώς ένα
από τα αρχέτυπα διάδρασης που συνιστούν την επιφάνεια εργασίας στους
επιτραπέζιους υπολογιστές και με τον κατάλληλο συνδυασμό των επιμέρους
αρχετύπων διάδρασης, η επιφάνεια εργασίας μπορεί να εξυπηρετήσει
διαφορετικούς χρήστες και τις ανάγκες τους.

Ανάμεσα στους πολλούς τρόπους για να ταξινομήσουμε τις συσκευές εισόδου,
ξεχωρίζουμε τη διάκριση στις κατηγορίες της άμεσης και της έμμεσης
διάδρασης. Το ποντίκι είναι ο βασικός εκπρόσωπος της έμμεσης διάδρασης,
αφού, για να μετακινήσουμε τον δείκτη σε μια συσκευή εξόδου (π.χ. οθόνη)
μετακινούμε μια διαφορετική συσκευή, όπως είναι το ποντίκι. Από την άλλη
πλευρά, η οθόνη αφής ανήκει στις συσκευές εισόδου άμεσης διάδρασης,
αφού, για να επιλέξουμε έναν στόχο ή για να μετακινήσουμε ένα
αντικείμενο το κάνουμε απευθείας με τα χέρια μας πάνω στην οθόνη, χωρίς
να μεσολαβεί κάποια ενδιάμεση μετάφραση. Από την άποψη της διάκρισης σε
άμεσες και έμμεσες συσκευές, ιδιαίτερο ενδιαφέρον έχει και η συσκευή
εισόδου της πένας, η οποία μπορεί να ανήκει και στις δύο κατηγορίες.
\footnote{(\textbf{Εικόνα?})~15 Παρεμπόδιση του στόχου από το χέρι (Fair
  use)} \footnote{(\textbf{Εικόνα?})~16 Έμεση διάδραση σε οθόνη αφής
  (Apple)}

\leavevmode\vadjust pre{\hypertarget{fig:hand-occlusion}{}}%
\begin{figure}
\hypertarget{fig:hand-occlusion}{%
\centering
\includegraphics{images/hand-occlusion.png}
\caption{Εικόνα 15: Η άμεση διάδραση με συσκευές αφής έχει το
μειονέκτημα ότι το χέρι ή τα δάκτυλα μπορεί να καλύπτουν τον στόχο (ή το
αντικείμενο) που θέλουμε να επιλέξουμε ή και να μετακινήσουμε, φαινόμενο
γνωστό ως occlusion (έμφραξη), εκτός αν αυτό είναι αρκετά μεγάλο ή
χρησιμοποιηθούν τεχνικές που αναιρούν το
occlusion.}\label{fig:hand-occlusion}
}
\end{figure}

\leavevmode\vadjust pre{\hypertarget{fig:media-scrub}{}}%
\begin{figure}
\hypertarget{fig:media-scrub}{%
\centering
\includegraphics{images/media-scrub.png}
\caption{Εικόνα 16: Η άμεση διάδραση πάνω σε μια οθόνη αφής δεν
επιτρέπει τον έλεγχο του ρυθμού μετακίνησης, αλλά αυτό μπορεί να γίνει
εφικτό τουλάχιστον για την περίπτωση της γραμμικής μετακίνησης, αν ο
ρυθμός αντιπροσωπεύεται από μια άλλη ορθογωνική κίνηση του δακτύλου,
όπως για παράδειγμα στην περίπτωση της αναζήτησης στον οριζόντιο χρόνο
οπτικοακουστικού περιεχομένου, κατά την οποία ο ρυθμός αλλάζει με την
κατακόρυφη θέση του δακτύλου.}\label{fig:media-scrub}
}
\end{figure}

Αν και σε πρώτη ανάγνωση οι συσκευές άμεσης διάδρασης φαίνεται να έχουν
πολλά πλεονεκτήματα, τουλάχιστον αναφορικά με την ευκολία εκμάθησής
τους, υπάρχουν πολλές περιπτώσεις στις οποίες μια συσκευή έμμεσης
διάδρασης υπερτερεί. Για παράδειγμα, η επιλογή μικρών στόχων ή ακόμη
δυσκολότερα ο χειρισμός τους με άμεση διάδραση (π.χ. πάνω σε μια οθόνη
αφής) δεν είναι εύκολος, εκτός αν υπάρχει ειδική υποστήριξη από το
αντίστοιχο λογισμικό. Για το τελευταίο, χαρακτηριστική περίπτωση είναι η
μικρομετρική αναζήτηση πάνω στον χρόνο για ένα βίντεο ή για ένα μουσικό
τραγούδι. Εκτός από το ποντίκι, οι ερευνητές δοκίμασαν και άλλες
συσκευές εισόδου, όπως την πένα, η οποία επιτρέπει τόσο άμεση όσο και
έμμεση διάδραση.

Η επιφάνεια εργασίας σε συνδυασμό με τις συσκευές εισόδου ποντίκι και
πληκτρολόγιο αντιπροσωπεύει ένα στυλ διάδρασης με τον χρήστη το οποίο
είναι δημοφιλές στους επιτραπέζιους υπολογιστές γραφείου, καθώς είναι
μια κοντινή μεταφορά του πλαισίου εργασίας του χρήστη. Όταν οι
υπολογιστές βρίσκουν εφαρμογή εκτός του πλαισίου του γραφείου είναι
επόμενο να χρειαζόμαστε διαφορετικά μοντέλα διάδρασης, τα οποία ναι μεν
θα ανταποκρίνονται στις βασικές ιδιότητες του ανθρώπου (οι οποίες δεν
είναι πολύ διαφορετικές ανεξάρτητα από το πλαίσιο χρήσης), αλλά θα
ταιριάζουν και στο αντίστοιχο πλαίσιο και στις ανάγκες που αυτό
δημιουργεί, οι οποίες μπορεί να είναι πολύ διαφορετικές σε σχέση με
εκείνες της χρήσης εντός του γραφείου.

Αν και η τεχνολογία των υπολογιστών έχει κάνει πολύ μεγάλη ποσοτική
πρόοδο αναφορικά με την ταχύτητα και το μέγεθος των δεδομένων που
μπορούν να επεξεργαστούν, την ίδια στιγμή, η πρόοδος αυτή δεν έχει
μεγάλο αντίκρισμα στην ποιότητα της διάδρασης ανθρώπου και υπολογιστή. Η
ποιότητα της διάδρασης εξαρτάται τόσο από τον υπολογιστή όσο και από τον
άνθρωπο. Μπορούμε να σκεφτούμε και να δημιουργήσουμε πολλές διαφορετικές
συσκευές εισόδου και εξόδου για την επικοινωνία με τον άνθρωπο, όμως, αν
αυτές δεν είναι συμβατές με τις ανάγκες του, ή αν δεν γίνουν αποδεκτές,
τότε δεν έχουμε πετύχει κάποια πρόοδο. Επομένως, η πρόκληση που
παραμένει ανοικτή είναι να κατασκευάσουμε εκείνες τις συσκευές εισόδου
και εξόδου που είναι κατάλληλες για την φύση και τις ανάγκες του
ανθρώπου.

Το συμπέρασμα από τη σύγκριση της καθιερωμένης διάδρασης με την
επιφάνεια εργασίας και των δυνατοτήτων του ανθρώπου μας δίνει νέους
ορίζοντες για το πεδίο ορισμού του φαινομένου της διάδρασης. Το
διευρυμένο πεδίο ορισμού της διάδρασης μπορεί να προσδιοριστεί με βάση
τις δυνατότητες του ανθρώπου που δεν έχουν αποκτήσει ακόμη ρόλο σε
συνδυασμό με ένα νέο πλαίσιο χρήσης, πέρα από το γραφείο και τον
επιτραπέζιο υπολογιστή.

\hypertarget{ux3b7-ux3c0ux3b5ux3c1ux3afux3c0ux3c4ux3c9ux3c3ux3b7-ux3c4ux3bfux3c5-apple-ipod}{%
\subsection{Η περίπτωση του Apple
iPod}\label{ux3b7-ux3c0ux3b5ux3c1ux3afux3c0ux3c4ux3c9ux3c3ux3b7-ux3c4ux3bfux3c5-apple-ipod}}

Η περίπτωση της συσκευής αναπαραγωγής μουσικής iPod σήμανε την αρχή της
γρήγορης μετατόπισης της διάδρασης από τον επιτραπέζιο προς τον κινητό
και διάχυτο υπολογισμό. Η μετάβαση του υπολογισμού από τον επιτραπέζιο
στον κινητό συνοδεύτηκε από μια ριζική αναθεώρηση τόσο των συστημάτων
εισόδου όσο και των μοντέλων διάδρασης.

\leavevmode\vadjust pre{\hypertarget{fig:ipod-click-wheel}{}}%
\begin{figure}
\hypertarget{fig:ipod-click-wheel}{%
\centering
\includegraphics{images/ipod-click-wheel.jpg}
\caption{Εικόνα 17: Η πρώτη γενιά iPod εισάγει έναν νέο τρόπο πλοήγησης
σε μεγάλες λίστες (μενού) αρχείων μουσικής, τον περιστρεφόμενο τροχό.
Ταυτόχρονα, λειτούργησε ως δούρειος ίππος για τη σταδιακή εισαγωγή και
την εξοικείωση των χρηστών με κινητές συσκευές διάδρασης, οι οποίες
συνδέονται με τον προσωπικό υπολογιστή. Στις επόμενες γενιές iPod οι
σχεδιαστές κρατήσαν τον τρόχο και τον βελτίωσαν μεταφέροντας πάνω του τα
προηγουμένως διακριτά κουμπιά μενού, παύσης, και επόμενο/προηγούμενο.
Επίσης, τόσο το τροχός όσο και τα κουμπιά δουλεύουν πλέον με αισθητήρα
αφής, με το συνολικό αποτέλεσμα να είναι οπτικά πιο απλό για τον χρήστη,
αλλά και να απαιτεί λιγότερες κινήσεις μεγαλύτερης ακρίβειας από τα
δάκτυλα του.}\label{fig:ipod-click-wheel}
}
\end{figure}

\leavevmode\vadjust pre{\hypertarget{fig:itunes3}{}}%
\begin{figure}
\hypertarget{fig:itunes3}{%
\centering
\includegraphics{images/itunes3.png}
\caption{Εικόνα 18: Το iTunes, αν και ξεκίνησε ως μια ταπεινή εφαρμογή
εκτέλεσης μουσικών αρχείων και συγχρονισμού τους με το iPod, μετατράπηκε
σε δούρειο ίππο για τη μεταφόρτωση εφαρμογών και τον συγχρονισμό με το
οικοσύστημα των κινητών συσκευών χρήστη.}\label{fig:itunes3}
}
\end{figure}

Η πρώτη έκδοση του iPod περιέχει έναν μικρό σκληρό δίσκο και λειτουργεί
ως συσκευή αποθήκευσης αρχείων, όπως πολλές αντίστοιχες εμπορικές
συσκευές εκείνης της εποχής, με τη διαφορά ότι ο περιστρεφόμενος τροχός
επιτρέπει τη γρήγορη πλοήγηση σε μεγάλες λίστες μουσικών
αρχείων.\footnote{(\textbf{Εικόνα?})~17 Τροχός πλοήγησης iPod
  (Wikimedia)} Οι άλλες συσκευές της κατηγορίας του, αντί για τον τροχό,
έχουν πλήκτρα βήματος, τα οποία είναι πιο δύσχρηστα, ειδικά στην
περίπτωση που ο σκληρός δίσκος είναι γεμάτος με μουσικά αρχεία. Οι
περισσότερες ανταγωνιστικές συσκευές εισόδου βασίζονται σε μια απευθείας
απεικόνιση της λίστας πολυμεσικών αρχείων με τα κουμπιά εισόδου, τα
οποία είναι οργανωμένα κατακόρυφα. Αν και ο τροχός κύλισης του iPod
παραβαίνει τον κανόνα της απευθείας απεικόνισης, αποδικνύεται
περισσότερο εύχρηστος, γιατί πολύ γρήγορα οι χρήστες καταλαβαίνουν ότι
με την περιστροφή του τροχού μπορούν να κινηθούν πάνω-κάτω και, ακόμη
γρηγορότερα, μπορούν να επιλέξουν ένα αρχείο από την λίστα, γιατί ο
τροχός προσφέρει εκτός από τον έλεγχο της κατεύθυνσης και τον έλεγχο της
επιτάγχυνσης.

Βλέπουμε, λοιπόν, ότι η επιτυχία (κατά ένα μέρος) οφείλεται στη
δημιουργία μιας νέας συσκευής εισόδου, η οποία δίνει έμφαση στην πιο
συχνή διεργασία του χρήστη. Όπως ακριβώς το ποντίκι ήρθε να διευκολύνει
την επιλογή κειμένου στην οθόνη και το έκανε αποδεδειγμένα πολύ καλύτερα
από όλες τις εναλλακτικές (όπως είδαμε στην ερευνητική μελέτη
περίπτωσης), έτσι και ο τροχός του πρώτου iPod είναι ο πιο
αποτελεσματικός τρόπος για την επιλογή μέσα σε λίστες με πολλές
καταχωρήσεις. Κάποιος θα μπορούσε να προσθέσει ότι πέρα από
λειτουργικός, ο τροχός του iPod είναι και διασκεδαστικός, αφού θυμίζει
την κίνηση που κάνουν οι DJs, όταν διαλέγουν μουσική.

Ένα ακόμη στοιχείο που συνέβαλε στην επιτυχία του iPod ήταν η ολοκλήρωσή
του με ένα σύστημα διανομής μουσικής, το οποίο περιλάμβανε μια εφαρμογή
επιτραπέζιου υπολογιστή, καθώς και ένα ηλεκτρονικό κατάστημα. Στα τέλη
της δεκαετίας του 1990, η επικράτηση των αρχείων μουσικής τύπου MP3 και
η εύκολη διανομή τους μέσω του δικτύου, επέτρεψε στους χρήστες να
συγκεντρώσουν πολύ μεγάλες συλλογές μουσικών αρχείων, που ήταν δύσκολο
να οργανώσουν και να ακούσουν. Η εφαρμογή iTunes και το αντίστοιχο
ηλεκτρονικό κατάστημα επεδίωξε να λύσει αυτό το πρόβλημα και το έκανε με
μεγάλη επιτυχία, ενώ ταυτόχρονα το iTunes έγινε η πύλη για την εισαγωγή
νέων πολυμεσικών αρχείων πέρα από τη μουσική, όπως βίντεο και
φωτογραφίες, και αργότερα εφαρμογές για την κατηγορία των έξυπνων
κινητών (iPod Touch, iPhone).

Η εφαρμογή iTunes,\footnote{(\textbf{Εικόνα?})~18 Apple iTunes (Apple)}
εκτός από το να οργανώνει και να διανέμει τα πολυμεσικά πλέον αρχεία του
χρήστη, μετατρέπεται σταδιακά στον Δούρειο Ίππο που θα φέρει τις κινητές
εφαρμογές στη συσκευή του χρήστη. Οι πρώτες εκδόσεις του iTunes είναι
επιτραπέζιες εφαρμογές εκτέλεσης μουσικών αρχείων και συγχρονισμού με το
iPod, ενώ οι τελευταίες εκδόσεις συνδέονται και ενημερώνουν το λογισμικό
για όλες τις κινητές συσκευές του χρήστη (π.χ. smart phone, tablet,
smart watch, κτλ.). Ταυτόχρονα, συνδέονται και με τα οικιακά συστήματα
ψυχαγωγίας, όπως ηχεία και τηλεόραση, ώστε να επιτρέψουν την εκτέλεση
των πολυμεσικών αρχείων σε συσκευές εξόδου υψηλότερης πιστότητας σε
σχέση με εκείνες του επιτραπέζιου υπολογιστή.

Ένα σημαντικό μάθημα που μας δίνει η ιστορία του iPod είναι πως τα
σύγχρονα διαδραστικά συστήματα δεν στέκονται από μόνα τους, αλλά
λειτουργούν σε ένα οικοσύστημα συσκευών και εφαρμογών. Ένα ακόμη μάθημα
είναι πως η εισαγωγή της καινοτομίας πρέπει να γίνει τόσο σταδιακά που η
μετάβαση να είναι διαφανής για τον τελικό χρήστη. Όταν οι πρώτοι χρήστες
του iPod έβαζαν το iTunes, ήθελαν απλώς να οργανώσουν καλύτερα τη
μουσική τους, ένα πρόβλημα που ήδη αντιμετώπιζαν στον επιτραπέζιο
υπολογιστή. Καθώς απέκτησαν οικειότητα με το iTunes, το είδαν να
μεταμορφώνεται σε ένα πολυεργαλείο για όλες τις κινητές συσκευές τους.

\hypertarget{ux3b7-ux3c0ux3b5ux3c1ux3afux3c0ux3c4ux3c9ux3c3ux3b7-ux3c4ux3c9ux3bd-ux3baux3b1ux3b8ux3b7ux3bcux3b5ux3c1ux3b9ux3bdux3ceux3bd-ux3c0ux3c1ux3b1ux3b3ux3bcux3acux3c4ux3c9ux3bd}{%
\subsection{Η περίπτωση των καθημερινών
πραγμάτων}\label{ux3b7-ux3c0ux3b5ux3c1ux3afux3c0ux3c4ux3c9ux3c3ux3b7-ux3c4ux3c9ux3bd-ux3baux3b1ux3b8ux3b7ux3bcux3b5ux3c1ux3b9ux3bdux3ceux3bd-ux3c0ux3c1ux3b1ux3b3ux3bcux3acux3c4ux3c9ux3bd}}

Η κατανόηση και η χρήση μιας συσκευής διέπεται από μερικές βασικές και
διαχρονικές αξίες, οι οποίες παραμένουν ίδιες, ανεξάρτητα από το είδος
και την πολυπλοκότητα που μπορεί να έχει η διάδραση ανθρώπου υπολογιστή.
Στο κλασικό βιβλίο του \emph{Η Σχεδίαση των Καθημερινών Πραγμάτων} ο
Ντον Νόρμαν παραθέτει ένα μικρό σύνολο από βασικές αξίες και δίνει
παραδείγματα καλής και κακής εφαρμογής σε καθημερινά απλά αντικείμενα,
όπως πόρτες και υδραυλικά. Οι βασικές αξίες που πρέπει να έχει μια
συσκευή, ώστε να είναι κατανοητή και εύχρηστη κατά τη διάδραση με τον
άνθρωπο, είναι: affordance, constraint, mapping, feedback. Το affordance
αναφέρεται στις περισσότερο ή λιγότερο προφανείς χρήσεις που επιτρέπει η
ίδια η εμφάνιση και η λειτουργία ενός αντικειμένου. Το constraint
αναφέρεται στους περιορισμούς που σκόπιμα εισάγει ο σχεδιασμός, ώστε να
εμποδίσει κάποιες χρήσεις ή να αποτρέψει το λάθος κατά την σωστή χρήση.
Το mapping αναφέρεται στη φυσική σύνδεση ανάμεσα στις καταστάσεις
λειτουργίας και στον έλεγχο από την πλευρά του χρήστη.\footnote{(\textbf{Εικόνα?})~19
  Συνεπής απεικόνιση ανάμεσα στην είσοδο και στην έξοδο (Free Art
  License)} Τέλος, το feedback αναφέρεται στην συνεχή ανάδραση του
συστήματος, ώστε να είναι πάντα γνωστή η κατάστασή του στον χρήστη.

Η διαπίστωση που επιβεβαιώνεται διαχρονικά στη σχεδίαση των καθημερινών
πραγμάτων\footnote{D. Norman (2013)} είναι ότι οι κατασκευαστές
επαναλαμβάνουν τα ίδια λάθη με την παράλειψη των βασικών αξιών και ότι
οι αξίες αυτές έχουν μείνει αναλλοίωτες. Οι κατασκευαστές κάνουν τα ίδια
λάθη, γιατί κάθε φορά που έχουμε μια νέα τεχνολογική επανάσταση, η
κατασκευή της διάδρασης γίνεται συνήθως από τους κατασκευαστές που έχουν
οικειότητα με τη νέα τεχνολογία και οι οποίοι συνήθως είναι είτε νέοι
είτε επικεντρωμένοι μόνο στην τεχνολογία. Οι βασικές αξίες έχουν μείνει
οι ίδιες, γιατί ο άνθρωπος αλλάζει πολύ πιο αργά από όσο η τεχνολογία.
Τελικά, ο στόχος της ανθρωποκεντρικής κατασκευής συστημάτων είναι να
βρούμε μια ισορροπία ανάμεσα σε όλες τις δυνάμεις που επηρεάζουν τη
σχεδίαση, την κατασκευή, τη διανομή, και τη χρήση των συσκευών
διάδρασης. Για παράδειγμα, ο σχεδιασμός του τυπικού πληκτρολογίου QWERTY
δεν είναι βέλτιστος για την πληκτρολόγηση όσο το σύστημα DVORAK, ωστόσο
επικράτησε, γιατί στα πρώτα στάδια διάδοσης των υπολογιστών με
πληκτρολόγιο, αυτά ήταν διαθέσιμα και οικεία.\footnote{(\textbf{Εικόνα?})~20
  Πληκτρόλογιο Dvorak (Public domain)}

Καθώς η χρήση του υπολογιστή ξέφυγε από το στενό πλαίσιο της εργασίας
και από την αντίληψη του υπολογιστή ως απλού εργαλείο, όπου η απαίτηση
για χρησιμότητα και ευχρηστία είναι κυρίαρχη, δημιουργήθηκε η ανάγκη για
ένα νέο αξιακό σύστημα, το οποίο να βασίζεται περισσότερο στα
συναισθήματα του ανθρώπου και να εμπλέκει πιο πολλές ανθρώπινες
αισθήσεις. Τόσο οι ερευνητικές μελέτες όσο και τα εμπορικά προϊόντα προς
το τέλος της δεκαετίας του 2000 άρχισαν να δίνουν έμφαση όχι μόνο στη
γνωστική επεξεργασία της πληροφορίας αλλά και στα συναισθήματα του
ανθρώπου\footnote{D. A. Norman (2004)} και, αντίστοιχα, η περιοχή της
σχεδίασης της διάδρασης ανθρώπου και υπολογιστή αρχίζει να περιγράφεται
και ως σχεδίαση της εμπειρίας του χρήστη.\footnote{Garrett (2010)} Με
αυτόν τον τρόπο γίνεται ένα ακόμη βήμα μακρύτερα από την αρχική θεώρηση
της περιοχής της διάδρασης, που ήταν γνωστή ως σχεδίαση της διεπαφής
ανθρώπου και υπολογιστή, όπου η διάδραση γινόταν, για παράδειγμα,
αντιληπτή ως η σχεδίαση των παραθύρων και των εικονιδίων της γραφικής
επιφάνειας εργασίας.

\leavevmode\vadjust pre{\hypertarget{fig:mapping-principle}{}}%
\begin{figure}
\hypertarget{fig:mapping-principle}{%
\centering
\includegraphics{images/mapping-principle.jpg}
\caption{Εικόνα 19: Σε μια συσκευή που έχει την έξοδο σε διαφορετικό
σημείο από την είσοδο του χρήστη θα πρέπει να υπάρχει μια συνεπής
απεικόνιση ανάμεσα στην είσοδο και στην έξοδο, όπως στην περίπτωση των
εστιών μιας κουζίνας μαγειρέματος.}\label{fig:mapping-principle}
}
\end{figure}

\leavevmode\vadjust pre{\hypertarget{fig:dvorak-keyboard}{}}%
\begin{figure}
\hypertarget{fig:dvorak-keyboard}{%
\centering
\includegraphics{images/dvorak-keyboard.png}
\caption{Εικόνα 20: Η επίτευξη της ευχρηστίας δεν είναι η μόνη απόλυτη
αξία, καθώς μια συσκευή διάδρασης επηρεάζεται από πολλούς ακόμη
παράγοντες, όπως από την αισθητική, το κόστος αλλά και από τη συνήθεια,
η οποία είναι η κύρια αιτία για το ότι το πιο συνηθισμένο πληκτρολόγιο
για τον επιτραπέζιο υπολογιστή δεν είναι το πιο εύχρηστο. Η οργάνωση των
πλήκτρων κάτω από τα δάκτυλα στο πληκτρολόγιο Dvorak έχει γίνει έτσι,
ώστε τα πιο συχνά γράμματα να βρίσκονται πιο κοντά στα
δάκτυλα.}\label{fig:dvorak-keyboard}
}
\end{figure}

\hypertarget{ux3c3ux3cdux3bdux3c4ux3bfux3bcux3b7-ux3b2ux3b9ux3bfux3b3ux3c1ux3b1ux3c6ux3afux3b1-ux3c4ux3bfux3c5-larry-tesler}{%
\subsection{Σύντομη βιογραφία του Larry
Tesler}\label{ux3c3ux3cdux3bdux3c4ux3bfux3bcux3b7-ux3b2ux3b9ux3bfux3b3ux3c1ux3b1ux3c6ux3afux3b1-ux3c4ux3bfux3c5-larry-tesler}}

Ο Larry Tesler στην πρώτη προγραμματιστική δουλειά του προσπάθησε να
βελτιώσει την ευχρηστία μιας διεπαφής σχεδίασης γραφικών για μεγάλες
οθόνες σε στάδια ποδοσφαίρου, έτσι ώστε να είναι προσβάσιμη από τους
γραφίστες και όχι μόνο από τους προγραμματιστές. Αν και δεν είχε κάποια
εκπαίδευση σε θέματα ευχρηστίας, στο μέλλον θα έκανε πολλές συνεισφορές
σε αυτήν την περιοχή αρχικά εργαζόμενος στο Xerox PARC και στην Apple
και μετά ως σύμβουλος ευχρηστίας πολλών οργανισμών. Η βασική πεποίθηση
του Larry Tesler είναι ότι μια διεπαφή θα πρέπει να είναι εύκολη για τον
αρχάριο και ευκαιριακό χρήστη, μια φιλοσοφία η οποία τελικά επικράτησε
στον κλάδο της σχεδίασης της διάδρασης.

\leavevmode\vadjust pre{\hypertarget{fig:tesler-profile}{}}%
\begin{figure}
\hypertarget{fig:tesler-profile}{%
\centering
\includegraphics{images/tesler-profile.jpg}
\caption{Εικόνα 21: Ο Larry Tesler κατασκεύασε λογισμικό με έμφαση στην
προσβασιμότητα από απλούς χρήστες. Η πιο σημαντική συνεισφορά του είναι
η μη-τροπική αλληλεπίδραση, την οποία δοκίμασε αρχικά στο Xerox Alto με
τον επεξεργαστή κειμένου Gypsy. Η έννοια της ευχρηστίας για τον Larry
Tesler συμπίπτει με την ευκολία για τον περιστασιακό και αρχάριο χρήστη
και τελικά επικράτησε στην περιοχή της
διάδρασης.}\label{fig:tesler-profile}
}
\end{figure}

\leavevmode\vadjust pre{\hypertarget{fig:nomodes}{}}%
\begin{figure}
\hypertarget{fig:nomodes}{%
\centering
\includegraphics{images/nomodes.png}
\caption{Εικόνα 22: Η διάδραση με έναν επεξεργαστή κειμένου γινόταν
αρχικά με την επιλογή κατάστασης λειτουργίας όπως είναι η εισαγωγή
κειμένου και οι εντολές αναζήτησης και αποθήκευσης. Ο τρόπος αυτός, αν
και παραμένει αποδεκτός από τους προχωρημένους χρήστες, αποτέλεσε
εμπόδιο για τους αρχάριους χρήστες, το οποίο οδήγησε τον Larry Tesler να
προσθέση σχετική αναφορά στην πινακίδα του αυτοκινήτου του, καθώς και
στην ονομασία της προσωπικής ιστοσελίδας του.}\label{fig:nomodes}
}
\end{figure}

Η σημαντικότερη συνεισφορά του έγινε πραγματικότητα με την επιμονή του
στη μη-τροπικότητα της διεπαφής. Σε μια εποχή που όλες οι διεπαφές, όπως
η γραμμή εντολών και τα πρώτα γραφικά περιβάλλοντα, βασίζονται στην
τροπικότητα, σχεδίασε, υλοποίησε και δοκίμασε με απλούς χρήστες μια μη
τροπική διεπαφή. Πράγματι, οι αρχάριοι χρήστες φάνηκε να προτιμούν πρώτα
την επιλογή ενός αντικειμένου στη διεπαφή, π.χ. λέξη, παράγραφο, εικόνα
και μετά την επιλογή της ενέργειας, παρά το αντίθετο, που μέχρι τότε
ήταν το κυρίαρχο τροπικό στυλ διάδρασης.

Το μεγαλύτερο μέρος της αρχικής προγραμματιστικής δουλειάς του έγινε για
περιβάλλον γραφείου με έμφαση στον επεξεργαστή κειμένου. \footnote{(\textbf{Εικόνα?})~21
  Larry Tesler (Xerox PARC)} Με αυτόν τον τρόπο μετέτρεψε τον αρχικό
οπτικό επεξεργαστή κειμένου Xerox Bravo, σε μια μη-τροπική έκδοση,
\footnote{(\textbf{Εικόνα?})~22 Μη τροπική διάδραση (Larry Tesler)} που
λεγόταν Gypsy και η οποία ελάχιστα διαφέρει από τους σύγχρονους οπτικούς
επεξεργαστές κειμένου, όπως το Microsoft Word. Επιπλέον, ανέπτυξε το
σύστημα PUΒ, το οποίο αποτέλεσε την έμπνευση για την μελλοντική ανάπτυξη
του συστήματος LaTex.

Η πιο δημοφιλής συνεισφορά του Larry Tesler είναι η διάδραση αντιγραφής
και επικόλλησης μέσω του προχείρου, το οποίο, πλέον, βρίσκεται σε όλες
τις γραφικές διεπαφές. Λιγότερο γνωστή αλλά εξίσου σημαντική είναι η
κατασκευή του πρώτου περιηγητή κώδικα στο περιβάλλον Smalltalk. Μετά τη
δεκαετία του 1980, από την θέση του συμβούλου θα συνεχίσει να επηρεάζει
την κατασκευή της διάδρασης για περισσότερο ή λιγότερο γνωστά συστήματα,
όπως το Apple Newton και το Amazon Books.

\hypertarget{ux3b2ux3b9ux3b2ux3bbux3b9ux3bfux3b3ux3c1ux3b1ux3c6ux3afux3b1}{%
\subsection*{Βιβλιογραφία}\label{ux3b2ux3b9ux3b2ux3bbux3b9ux3bfux3b3ux3c1ux3b1ux3c6ux3afux3b1}}
\addcontentsline{toc}{subsection}{Βιβλιογραφία}

\hypertarget{refs}{}
\begin{CSLReferences}{0}{0}
\end{CSLReferences}

Garrett, Jesse James. 2010. \emph{Elements of User Experience, the:
User-Centered Design for the Web and Beyond}. Pearson Education.

Hiltzik, Michael. 1999. {``Dealers of Lightning: Xerox PARC and the
Dawning of the Computer Age.''}

Igoe, Tom. 2007. \emph{Making Things Talk: Practical Methods for
Connecting Physical Objects}. " O'Reilly Media, Inc.".

McEwen, Adrian, and Hakim Cassimally. 2013. \emph{Designing the Internet
of Things}. John Wiley \& Sons.

Norman, Don. 2013. \emph{The Design of Everyday Things: Revised and
Expanded Edition}. Basic books.

Norman, Donald A. 2004. \emph{Emotional Design: Why We Love (or Hate)
Everyday Things}. Basic Civitas Books.

O'Sullivan, Dan, and Tom Igoe. 2004. \emph{Physical Computing: Sensing
and Controlling the Physical World with Computers}. Course Technology
Press.

\hypertarget{ux3c4ux3b5ux3c7ux3bdux3b9ux3baux3adux3c2}{%
\section{Τεχνικές}\label{ux3c4ux3b5ux3c7ux3bdux3b9ux3baux3adux3c2}}

\begin{quote}
Μπορώ να κατανοήσω μόνο αυτό που μπορώ να φτιάξω. Ρίτσαρντ Φάινμαν
\end{quote}

\hypertarget{ux3c0ux3b5ux3c1ux3afux3bbux3b7ux3c8ux3b7}{%
\subsubsection{Περίληψη}\label{ux3c0ux3b5ux3c1ux3afux3bbux3b7ux3c8ux3b7}}

Αυτό το κεφάλαιο περιγράφει τα βασικά δομικά στοιχεία και τις τεχνικές
της κατασκευής της διάδρασης και απευθύνεται σε όσους έχουν λίγες
γνώσεις στις επιμέρους περιοχές ή στον συνδυασμό τους. Παρέχει συνοπτικά
τις βασικές γνώσεις τόσο για τη γενική πλευρά του προγραμματισμού όσο
και για την ειδική περίπτωση της διάδρασης. Ο αναγνώστης θα μάθει τα
θεμελιώδη στοιχεία τα οποία απαιτούνται για την κατασκευή της διάδρασης.
Επίσης, θα διαβάσει για τις τεχνικές και τις διαδικασίες που
χρησιμοποιούνται για την κατασκευή μιας διεπαφής με τον χρήστη.
Συνοπτικά, το κεφάλαιο αυτό προσφέρει τις βασικές γνώσεις από τις
περιοχές του προγραμματισμού συστημάτων λογισμικού, δίνοντας έμφαση στον
ανθρωποκεντρικό σχεδιασμό.

\hypertarget{ux3c4ux3bf-ux3c0ux3b5ux3c1ux3b9ux3b2ux3acux3bbux3bbux3bfux3bd-ux3b1ux3bdux3acux3c0ux3c4ux3c5ux3beux3b7ux3c2-ux3bbux3bfux3b3ux3b9ux3c3ux3bcux3b9ux3baux3bfux3cd}{%
\subsection{Το περιβάλλον ανάπτυξης
λογισμικού}\label{ux3c4ux3bf-ux3c0ux3b5ux3c1ux3b9ux3b2ux3acux3bbux3bbux3bfux3bd-ux3b1ux3bdux3acux3c0ux3c4ux3c5ux3beux3b7ux3c2-ux3bbux3bfux3b3ux3b9ux3c3ux3bcux3b9ux3baux3bfux3cd}}

Η διαδικασία υλοποίησης ενός συστήματος διάδρασης ανθρώπου και
υπολογιστή μπορεί να διευκολυνθεί, αν ο προγραμματιστής έχει στη διάθεσή
του επιμέρους εργαλεία και τεχνικές, τα οποία βοηθούν στην κατασκευή των
υποδειγμάτων και, κυρίως, στην κατασκευή του τελικού συστήματος
διάδρασης. Τα συστήματα υποστήριξης της κατασκευής διάδρασης είναι
σχετικά απλά στην περίπτωση του παραδοσιακού επιτραπέζιου συστήματος,
γιατί το λεξιλόγιο της διάδρασης (π.χ. παράθυρο, μενού, φόρμα, παλέτα,
έγγραφο, κτλ.) είναι σχετικά περιορισμένο. Η σχεδίαση για τον διάχυτο ΗΥ
έχει αυξήσει και ουσιαστικά έχει αλλάξει τις παραμέτρους της διάδρασης
τόσο, ώστε τα περισσότερα εργαλεία να είναι ακατάλληλα, αφού δεν μπορούν
να δώσουν μια πλήρη εικόνα της εκτέλεσης στην τελική συσκευή του χρήστη.
Από την άλλη πλευρά, οι γενικές τεχνικές των προδιαγραφών διατηρούν την
αξιοπιστία τους, όπως το μοντέλο ελεγκτής-όψη, οι δηλωτικές γλώσσες
προδιαγραφών, και τα διαγράμματα ροής και κατάστασης. \footnote{(\textbf{Εικόνα?})~1
  GRAIL οπτικός προγραμματισμός (RAND)} \footnote{(\textbf{Εικόνα?})~2
  MAX programming language (wikimedia)}

\leavevmode\vadjust pre{\hypertarget{fig:grail-flow}{}}%
\begin{figure}
\hypertarget{fig:grail-flow}{%
\centering
\includegraphics{images/grail-flow.jpg}
\caption{Εικόνα 1: Το πρώτο περιβάλλον οπτικού προγραμματισμού GRAIL
(GRaphical Input Language) απευθύνεται σε επαγγελματίες, οι οποίοι δεν
γνωρίζουν να γράφουν κώδικα, αλλά γνωρίζουν άριστα την ροή εργασίας της
δουλειάς τους, οπότε μπορούν να την περιγράψουν σε ένα ευέλικτο
διάγραμμα ροής στον υπολογιστή.}\label{fig:grail-flow}
}
\end{figure}

\leavevmode\vadjust pre{\hypertarget{fig:max-language}{}}%
\begin{figure}
\hypertarget{fig:max-language}{%
\centering
\includegraphics{images/max-language.jpg}
\caption{Εικόνα 2: Η οπτική γλώσσα προγραμματισμού MAX δημιουργήθηκε για
να διευκολύνει την καλλιτεχνική δημιουργία και βασίζεται στη λογική της
ροής δεδομένων μέσα από τον οπτικά σχεδιασμένο γράφο επεξεργασίας τους,
αντί για κείμενο και εντολές ελέγχου σε γραμμική σειρά, τα οποία
συναντάμε στις παραδοσιακές γλώσσες
προγραμματισμού.}\label{fig:max-language}
}
\end{figure}

Τα περισσότερα βιβλία προγραμματισμού διαλέγουν από την αρχή κάποια από
τα τρέχοντα διαθέσιμα εργαλεία, δίνοντας έμφαση, συνήθως, στη γλώσσα
προγραμματισμού (π.χ. Java), το λειτουργικό σύστημα (π.χ. Windows), και
το περιβάλλον ανάπτυξης (π.χ. Eclipse) και, από εκεί και πέρα,
περιγράφουν τα επιμέρους ζητήματα. Αντίθετα, σε αυτό το βιβλίο,
αντιμετωπίζουμε όλα τα εργαλεία του προγραμματισμού ως ζητούμενα, τα
οποία έχουν διαφορετικές τιμές, ανάλογα με τις απαιτήσεις του έργου. Για
τον σκοπό αυτό, δίνουμε μια επισκόπηση των διαθέσιμων εργαλείων και των
τεχνικών ανάπτυξης με έμφαση στα κριτήρια επιλογής τους, ανάλογα με τις
περιπτώσεις του προγραμματισμού της διάδρασης. Επίσης, σε αντίθεση με τα
περισσότερα βιβλία που επιλέγουν περισσότερο ή λιγότερο ρητά μια
δημοφιλή τεχνική και διαδικασία ανάπτυξης, εδώ περιγράφουμε τις
ιδιότητές τους και αξιολογούμε την καταλληλότητά τους ανάλογα με το
ζητούμενο. Για παράδειγμα, κάποια βιβλία θεωρούν δεδομένο ότι θα πρέπει
να ξεκινήσουμε τον προγραμματισμό μόνο αφού έχουμε καθορίσει με ακρίβεια
τις προδιαγραφές, αλλά υπάρχουν πολλές περιπτώσεις χρήσης, στις οποίες ο
ίδιος ο προγραμματισμός της διάδρασης μπορεί να μας βοηθήσει να
κατανοήσουμε καλύτερα ποιες είναι οι προδιαγραφές. Είναι τόσα πολλά τα
πιθανά επιμέρους εργαλεία που έχει ανάγκη ένας προγραμματιστής, ώστε
δημιουργήθηκε μια νέα κατηγορία υπερ-εργαλείου, το ολοκληρωμένο
περιβάλλον ανάπτυξης, το οποίο περιλαμβάνει όλα τα παραπάνω μέσα στην
ίδια εφαρμογή.

Το ολοκληρωμένο περιβάλλον ανάπτυξης αποτελείται από μια οργάνωση
εργαλείων με εύκολη πρόσβαση σε δομές και σε τεχνικές που βοηθούν στην
παραγωγή του τελικού προϊόντος. Ανάλογα με την εμπειρία και τις
προτιμήσεις του κατασκευαστή, το περιβάλλον ανάπτυξης μπορεί να έχει
πάρα πολλές μορφές και επίπεδα λειτουργίας. Για παράδειγμα, οι αρχάριοι
χρήστες, συνήθως, διευκολύνονται από οπτικά περιβάλλοντα ανάπτυξης
λογισμικού, τα οποία δεν δίνουν άμεση πρόσβαση στον τελικό πηγαίο
κώδικα, αλλά δίνουν πολύ εύκολη πρόσβαση σε βασικά μοτίβα χρήσης. Από
την άλλη πλευρά, οι έμπειροι κατασκευαστές, οι οποίοι θέλουν να φτιάξουν
κάτι εντελώς καινούργιο, όχι μόνο χρησιμοποιούν πολύ απλά και ευέλικτα
εργαλεία (π.χ. έναν απλό επεξεργαστή κειμένου), αλλά ξοδεύουν και αρκετό
χρόνο φτιάχνοντας δικά τους εργαλεία και τεχνικές. Ανάμεσα σε αυτές τις
δύο ακραίες περιπτώσεις, υπάρχουν πάρα πολλά εργαλεία και τεχνικές τις
οποίες μπορεί να χρησιμοποιήσει κάποιος, ανάλογα με τις ικανότητες και
τον σκοπό του. \footnote{(\textbf{Εικόνα?})~3 Ολοκληρωμένο περιβάλλον
  ανάπτυξης Eclipse (Eclipse Public License)} \footnote{(\textbf{Εικόνα?})~4
  Ολοκληρωμένο περιβάλλον ανάπτυξης για νέους προγραμματιστές
  (Processing Foundation)}

\leavevmode\vadjust pre{\hypertarget{fig:eclipse-ide}{}}%
\begin{figure}
\hypertarget{fig:eclipse-ide}{%
\centering
\includegraphics{images/eclipse-ide.png}
\caption{Εικόνα 3: Ένα δημοφιλές περιβάλλον ανάπτυξης ανοικτού κώδικα
είναι το Eclipse, το οποίο χρησιμοποιεί την επιτραπέζια γραφική διεπαφή
για να ολοκληρώσει σε μια εφαρμογή χρήσιμα εργαλεία, όπως είναι ο
επεξεργαστής κειμένου, η αυτόματη μορφοποίηση και η συμπλήρωση κώδικα, ο
αποσφαλματωτής και η υποστήριξη για διαφορετικές γλώσσες
προγραμματισμού. Η αρχική διευκόλυνση που παρέχουν τα ολοκληρωμένα
περιβάλλοντα ανάπτυξης περιορίζει την ευελιξία, ενώ παρόμοια
λειτουργικότητα είναι εφικτή με χειροκίνητο τρόπο σε απλά εργαλεία όπως
τα Vim, GNU Emacs.}\label{fig:eclipse-ide}
}
\end{figure}

\leavevmode\vadjust pre{\hypertarget{fig:processing-ide}{}}%
\begin{figure}
\hypertarget{fig:processing-ide}{%
\centering
\includegraphics{images/processing-ide.png}
\caption{Εικόνα 4: Το περιβάλλον ανάπτυξης Processing μοιάζει σκόπιμα με
μια εφαρμογή εκτέλεσης αρχείων πολυμέσων. Τα προγράμματα ονομάζονται
σχεδιαγράμματα, έτσι ώστε να ενθαρρύνουν την συνεχή αλλαγή
τους.}\label{fig:processing-ide}
}
\end{figure}

Ιδιαίτερη αναφορά αξίζει να γίνει στο ολοκληρωμένο περιβάλλον του
Processing, το οποίο έχει φτιαχτεί σκόπιμα έτσι ώστε να μοιάζει
περισσότερο με εφαρμογή εκτέλεσης πολυμεσικών αρχείων παρά με ένα
προγραμματιστικό περιβάλλον. Σε αντίθεση με τα δημοφιλή ολοκληρωμένα
περιβάλλοντα, η εμφάνιση του Processing είναι πολύ λιτή, πράγμα που έχει
γίνει για να διευκολύνει τον νέο προγραμματιστή. Ίσως περισσότερο
αντισυμβατική και από την εμφάνιση είναι η ορολογία σε αυτό το
περιβάλλον, αφού σκόπιμα αναφέρεται στον πηγαίο κώδικα ως σχέδιο, και
όχι ως αρχείο ή ως κώδικα. Η πρόθεση των σχεδιαστών είναι να παροτρύνουν
τον προγραμματιστή σε αυτό το περιβάλλον να πειραματιστεί και να
βελτιώσει την ιδέα του, κάνοντας δοκιμή και λάθη. Σε αντίθεση λοιπόν με
την παραδοσιακή συμβουλή της αρχικής αναλυτικής σχεδίασης ενός
προγράμματος πριν από την υλοποίησή του, το περιβάλλον Processing
προτρέπει στον αυτοσχεδιασμό και στην εξερεύνηση.

Τα εργαλεία που διευκολύνουν την ανάπτυξη υλικού και λογισμικού
διάδρασης έχουν αποδειχθεί ιδιαίτερα αποτελεσματικά σε πολλές
περιπτώσεις, αρχίζοντας από το γραφικό περιβάλλον του επιτραπέζιου
υπολογιστή. Κάθε κατασκευαστής ενός λειτουργικού συστήματος παρέχει, σε
διαφορετικό βαθμό, ένα σύνολο από μοτίβα, τα οποία επιτρέπουν μια
ομοιόμορφη εμφάνιση και, κυρίως, μια συνεπή συμπεριφορά ανάμεσα στις
πολλές διαφορετικές εφαρμογές χρήστη. Ήταν η Apple αυτή που πρώτη έδωσε
στους κατασκευαστές εφαρμογών ένα σετ από οδηγίες και μοτίβα συνεπούς
σχεδιασμού της διάδρασης με τον υπολογιστή Macintosh, ενώ, ταυτόχρονα,
τα αντίστοιχα εργαλεία ανάπτυξης του λογισμικού σέβονταν αυτές τις
οδηγίες. Για παράδειγμα, η θέση, η εμφάνιση και η λειτουργία των
κουμπιών που καθορίζουν το μέγεθος του παραθύρου έχουν προκαθορισμένες
ιδιότητες, ώστε να μοιάζουν ανάμεσα στις διαφορετικές εφαρμογές του
χρήστη, δημιουργώντας μια αίσθηση οικειότητας. Ο κατασκευαστής εφαρμογών
χρήστη μπορεί πάντα να αγνοήσει τις οδηγίες και τα έτοιμα μοτίβα, αν
επιθυμεί να φτιάξει μια εφαρμογή που έχει λόγους να διαφέρει, όπως για
παράδειγμα σε εφαρμογές μουσικής και επεξεργασίας εικόνας.

Πέρα από την Apple, και οι άλλοι κατασκευαστές λειτουργικών συστημάτων
με γραφικό περιβάλλον, σε μικρότερο (π.χ. Linux) ή μεγαλύτερο βαθμό
(π.χ. Windows), παρέχουν πλέον τα αντίστοιχα σετ οδηγιών, καθώς και
έτοιμα δομικά στοιχεία κατασκευής της διάδρασης. \footnote{(\textbf{Εικόνα?})~5
  Οπτικός προγραμματισμός φόρμας δεδομένων (Microsoft)} Υπάρχουν δύο
κύριοι λόγοι για τους οποίους ένας κατασκευαστής της διάδρασης θα ήθελε
να κινηθεί έξω από την ασφάλεια που του προσφέρει το στενό και
προκαθορισμένο σύνολο οδηγιών που του παρέχει ο κατασκευαστής της
αρχικής πλατφόρμας. Ο πρώτος λόγος είναι να θέλει να φτιάξει μια
εφαρμογή, η οποία πρέπει να δείχνει και να συμπεριφέρεται διαφορετικά,
επειδή αυτό εξυπηρετεί τις ανάγκες του. Χαρακτηριστικά παραδείγματα
είναι τα skinable mp3 players, όπως το WinAmp, καθώς και οι εφαρμογές με
φίλτρα ψηφιακής φωτογραφίας, όπως το Kai Tools. Ο δεύτερος και
σημαντικότερος λόγος που κάνει έναν κατασκευαστή να κινηθεί έξω από τους
κανόνες είναι η ανάπτυξη ενός συστήματος το οποίο δεν μοιάζει καθόλου με
το σύστημα του υπολογιστή εργασίας του, όπως για παράδειγμα η ανάπτυξη
κινητών εφαρμογών σε επιτραπέζιο υπολογιστή.

\leavevmode\vadjust pre{\hypertarget{fig:visual-basic-form-designer}{}}%
\begin{figure}
\hypertarget{fig:visual-basic-form-designer}{%
\centering
\includegraphics{images/visual-basic-form-designer.png}
\caption{Εικόνα 5: Το οπτικό περιβάλλον της Visual Basic έδωσε τη
δυνατότητα σε πολλούς χρήστες, οι οποίοι δεν ήταν ειδικοί της
πληροφορικής, να φτιάξουν προγράμματα για ειδικούς σκοπούς, όπως η
αναζήτηση και η ανάκτηση πληροφορίας από μία βάση δεδομένων, χωρίς να
πρέπει να μάθουν όλες τις λεπτομέρειες της ανάπτυξης του λογισμικού.
Ταυτόχρονα, όμως, ο σχεδιασμός περιορίζεται από τα διαθέσιμα αρχέτυπα,
με αποτέλεσμα τη δημιουργία κινητών εφαρμογών, οι οποίες μοιάζουν και
συμπεριφέρονται όπως οι
επιτραπέζιες.}\label{fig:visual-basic-form-designer}
}
\end{figure}

\leavevmode\vadjust pre{\hypertarget{fig:windows-mobile}{}}%
\begin{figure}
\hypertarget{fig:windows-mobile}{%
\centering
\includegraphics{images/windows-mobile.png}
\caption{Εικόνα 6: Ένα από τα πρώτα λειτουργικά συστήματα με γραφικό
περιβάλλον για κινητά τηλέφωνα, τα windows mobile, μεταφέρουν τις
έννοιες από το λειτουργικό σύστημα του επιτραπέζιου υπολογιστή σε εκείνο
του κινητού. Οι έννοιες αυτές, αν και επιτυχημένες, είναι ακατάλληλες
για το πλαίσιο χρήσης του κινητού
υπολογισμού.}\label{fig:windows-mobile}
}
\end{figure}

Κάποιες από τις πρώτες προσπάθειες κατασκευής κινητών εφαρμογών έμοιαζαν
πολύ με τις αντίστοιχες επιτραπέζιες (π.χ. οι πρώτες εκδόσεις των
Windows Mobile). \footnote{(\textbf{Εικόνα?})~6 Διάδραση με τα πρώτα
  Windows Mobile (Microsoft)} Φυσικά, αυτό δε βοήθησε στην αποδοχή αυτών
των κινητών εφαρμογών στα πρώτα στάδια, μέχρι που η Apple με το iPhone
έδωσε έναν νέο ορισμό του πλαισίου μέσα στο οποίο θα πρέπει να κινούνται
οι εφαρμογές χρήστη στις κινητές συσκευές, για να είναι χρήσιμες,
εύχρηστες και αποδεκτές. Αντίστοιχα, κάθε νέα τεχνολογία που μετατοπίζει
τη διάδραση πέρα από τον επιτραπέζιο υπολογιστή, αντιμετωπίζει τις ίδιες
προκλήσεις. Στα πρώτα στάδια, οι κατασκευαστές εφαρμογών χρήστη θα
δανειστούν (λανθασμένα) πάρα πολλά στοιχεία από συσκευές που φαίνονται
παρόμοιες, αλλά στην πορεία και μετά από μερικούς κύκλους δοκιμής και
λάθους, θα καταλήξουν σε ένα ενημερωμένο σύνολο από οδηγίες και
εργαλεία, που θα τους βοηθήσουν στην παραγωγή κατάλληλων εφαρμογών
χρήστη. Συνοπτικά, όταν κατασκευάζουμε εφαρμογές οι οποίες θα
εκτελεστούν σε υπολογιστή που διαφέρει από τον επιτραπέζιο, θα πρέπει να
προσέχουμε πρώτα από όλα τις συσκευές εισόδου και εξόδου (είναι
πληκτρολόγιο και ποντίκι ή μήπως κάτι άλλο;) και το πλαίσιο χρήσης
(είναι περιβάλλον γραφείου και εργασία με εκδόσεις ή κάτι άλλο;).

Το τελικό αποτέλεσμα και, κυρίως, το πεδίο ορισμού στο οποίο μπορεί να
κινηθεί ένα νέο πρόγραμμα διάδρασης, εξαρτάται από τα βασικά μοτίβα
σχεδίασης που είδαμε παραπάνω. Επίσης, εξαρτάται και από τα εργαλεία,
την οργάνωση και τη διαδικασία κατασκευής. Όπως ακριβώς τα βασικά
σχεδιαστικά και τεχνολογικά μοτίβα, τα οποία έχει στη διάθεσή του ένας
κατασκευαστής μπορούν να δώσουν συγκεκριμένες μορφές και λειτουργίες στη
διάδραση, έτσι και η μέθοδος κατασκευής μπορεί να επιτρέψει ή να
αποτρέψει κάποιες μορφές και λειτουργίες της διάδρασης. Τα πρώτα
συστήματα προγραμματισμού της διάδρασης δεν είχαν καμία διαφορά από
εκείνα για τον προγραμματισμό του συστήματος, οπότε πολλοί δυνητικοί
κατασκευαστές της διάδρασης δεν είχαν καταφέρει να δώσουν τη συνεισφορά
τους. Μετά τη δεκαετία του 1970, οι αντικειμενοστραφείς γλώσσες
προγραμματισμού (π.χ. SmallTalk, C++, Java, JavaScript) και τα οπτικά
περιβάλλοντα ανάπτυξης (π.χ. KidSim, MIT Scratch, Processing) επέτρεψαν
σε γνώστες της περιοχής του προγραμματισμού της διάδρασης να
συμμετάσχουν. Ταυτόχρονα, η διευκόλυνση κάποιων πτυχών του
προγραμματισμού της διάδρασης, ακόμη και από τον τελικό χρήστη,
ολοκληρώνει τη διαχρονική τάση που ενθαρρύνει τη συμμετοχικότητα του
τελικού χρήστη, όχι μόνο στην απλή χρήση, αλλά και στη δημιουργία.

\hypertarget{ux3b5ux3c1ux3b3ux3b1ux3bbux3b5ux3afux3b1-ux3b1ux3bdux3acux3c0ux3c4ux3c5ux3beux3b7ux3c2}{%
\subsection{Εργαλεία
ανάπτυξης}\label{ux3b5ux3c1ux3b3ux3b1ux3bbux3b5ux3afux3b1-ux3b1ux3bdux3acux3c0ux3c4ux3c5ux3beux3b7ux3c2}}

Για πολλά χρόνια η ανάπτυξη και η εκτέλεση εφαρμογών στον επιτραπέζιο
υπολογιστή ήταν μονόδρομος, αφού οι άλλες μορφές υπολογιστή δεν ήταν
ιδιαίτερα διαδεδομένες. \footnote{Olsen (2009)} Πάντα υπήρχαν
υπερ-υπολογιστές, καθώς και παιχνιδομηχανές, αλλά η ανάπτυξη για αυτές
τις πλατφόρμες γινόταν από ειδικευμένο προσωπικό, που λάμβανε την
αντίστοιχη εκπαίδευση. \footnote{Grudin (1990)} Η ανάπτυξη και η
εκτέλεση εφαρμογών διάδρασης στον επιτραπέζιο υπολογιστή έχει πολλές
παραμέτρους, τις οποίες πρέπει να αξιολογήσει ο κατασκευαστής και δεν
είναι καθόλου τετριμμένη περίπτωση. Όμως, έχει ένα βασικό πλεονέκτημα σε
σχέση με την ανάπτυξη για τον κινητό και διάχυτο υπολογισμό. Η βασική
διαφορά στην ανάπτυξη λογισμικού διάδρασης ανάμεσα στον επιτραπέζιο και
τον κινητό ή διάχυτο υπολογισμό είναι το γεγονός ότι το πρόγραμμα
εκτελείται στην πρώτη περίπτωση πάνω στον ίδιο τον υπολογιστή ανάπτυξης,
ενώ στη δεύτερη περίπτωση το πρόγραμμα εκτελείται πάνω σε διαφορετικό
υλικό.

Όταν το πρόγραμμα που κατασκευάζουμε εκτελείται τελικά πάνω σε
διαφορετικό υλικό από εκείνο του υπολογιστή ανάπτυξης, τότε η δυνατότητα
που έχουμε για την εφαρμογή του ανθρωποκεντρικού κύκλου σχεδίασης
μειώνεται ανάλογα με τον βαθμό και το είδος της διάδρασης. Αν, για
παράδειγμα, κατασκευάζουμε ένα πρόγραμμα για ένα έξυπνο κινητό που έχει
πληκτρολόγιο και δεν έχει οθόνη αφής, τότε μπορούμε, σχετικά εύκολα, να
κάνουμε τις επαναληπτικές δοκιμές της διάδρασης πάνω στον επιτραπέζιο
υπολογιστή, ο οποίος έχει πληκτρολόγιο που, ναι μεν διαφέρει από το
μικρό πληκτρολόγιο του κινητού, όμως δεν είναι δραματικά διαφορετικό.
Στην περίπτωση, όμως, που το έξυπνο κινητό έχει μόνο πολυαπτική οθόνη
αφής, τότε η δοκιμή της διάδρασης στον επιτραπέζιο υπολογιστή γίνεται
πιο δύσκολη, αφού, συνήθως, δεν συνοδεύεται από παρόμοια συσκευή
εισόδου. Η δοκιμή της διάδρασης γίνεται ακόμη δυσκολότερη, όταν η
διάδραση βασίζεται σε αισθητήρες εισόδου, όπως ο εντοπισμός θέσης ή το
γυροσκόπιο, αφού αυτά δεν υπάρχουν στον επιτραπέζιο υπολογιστή και
απαιτείται πλέον η σύνδεσή του με την τελική συσκευή για την
πραγματοποίηση των επαναληπτικών δοκιμών κατά το στάδιο της ανάπτυξης.

Όπως είδαμε παραπάνω, το βασικό μειονέκτημα της κατασκευής στην
περίπτωση του κινητού και διάχυτου υπολογισμού είναι ότι τα περισσότερα
εργαλεία ανάπτυξης είναι διαθέσιμα κυρίως για τον επιτραπέζιο
υπολογιστή, ο οποίος μπορεί να διαφέρει, από λίγο έως πάρα πολύ, από την
τελική πλατφόρμα, αναφορικά με τις συσκευές εισόδου και εξόδου. Για
παράδειγμα, ένας επιτραπέζιος υπολογιστής έχει είσοδο, κυρίως, από το
πληκτρολόγιο και το ποντίκι, ενώ ένα έξυπνο κινητό έχει είσοδο, κυρίως,
από μια πολυαπτική οθόνη. Το αποτέλεσμα είναι ότι, εκτός από κάποιες
απλές επιλογές αντικειμένων πάνω στην οθόνη, πολλές από τις πιθανές
διαδράσεις, οι οποίες είναι χρήσιμες στο έξυπνο κινητό, δεν είναι
διαθέσιμες για δοκιμή στην πλατφόρμα ανάπτυξης, όταν αυτή είναι ο
επιτραπέζιος υπολογιστής. Από αυτήν την άποψη θα μπορούσαμε να
υποθέσουμε ότι τα μελλοντικά εργαλεία ανάπτυξης για κινητό υπολογισμό θα
εκτελούνται απευθείας πάνω στο κινητό. Αυτό φυσικά υπαγορεύει ένα πολύ
διαφορετικό μοντέλο ανάπτυξης αναφορικά με τα εργαλεία και τις
διαδικασίες κατασκευής του προγράμματος διάδρασης.

\leavevmode\vadjust pre{\hypertarget{fig:smalltalk}{}}%
\begin{figure}
\hypertarget{fig:smalltalk}{%
\centering
\includegraphics{images/smalltalk.jpg}
\caption{Εικόνα 7: Η αντικειμενοστραφής γλώσσα προγραμματισμού
Smalltalk, η οποία εμπνέεται από τον τρόπο που επικοινωνούν σε πολύ
μεγάλη κλίμακα τα βιολογικά κύτταρα, έδινε έμφαση σε οντότητες υψηλού
επιπέδου και στη διάδραση με τον χρήστη, με αποτέλεσμα να διευκολύνει
την κατασκευή και τις δοκιμές του λογισμικού, γεγονός που τελικά οδήγησε
στους πρώτους επιτυχημένους εμπορικά επιτραπέζιους
υπολογιστές.}\label{fig:smalltalk}
}
\end{figure}

\leavevmode\vadjust pre{\hypertarget{fig:lilith-modula}{}}%
\begin{figure}
\hypertarget{fig:lilith-modula}{%
\centering
\includegraphics{images/lilith-modula.jpg}
\caption{Εικόνα 8: Η εμπειρία της διάδρασης με το Alto εντυπωσίασε τόσο
πολύ τον Niklaus Wirth που αποφάσισε να δημιουργήσει ένα παρόμοιο
μηχάνημα με την ονομασία Lilith. Ακολούθησε μια λιγότερο δυναμική
κατεύθυνση από αυτήν της Smalltalk, με τη δημιουργία της γλώσσας
προγραμματισμού Modula, καθώς και με παράθυρα που δεν
αλληλοεπικαλύπτονται.}\label{fig:lilith-modula}
}
\end{figure}

Η κατασκευή προγραμμάτων διάδρασης διευκολύνεται από εργαλεία και
τεχνικές, \footnote{Graham (2004), McConnell (2004), Thimbleby (2007)}
τα οποία είναι τόσο διαφορετικά όσο και το εύρος των συσκευών εισόδου,
εξόδου και υπολογισμού. \footnote{Noble (2009)} Επιπλέον, τα εργαλεία
και οι διαδικασίες κατασκευής εξαρτώνται από τις προτιμήσεις του
κατασκευαστή, οι οποίες μπορεί να γίνουν αρκετά πολύπλοκες στην
περίπτωση μεγάλων οργανισμών και ομάδων ανάπτυξης, επομένως, τότε
αναφερόμαστε στην κουλτούρα ανάπτυξης του κάθε κατασκευαστή. \footnote{(\textbf{Εικόνα?})~7
  Smalltalk (Xerox PARC)} \footnote{(\textbf{Εικόνα?})~8 Lilith Modula
  (Niklaus Wirth)}

Μετά τον καθορισμό του στόχου και των αναγκών του χρήστη, το επόμενο
βήμα είναι η επιλογή των εργαλείων ανάπτυξης, καθώς και ο καθορισμός του
πλάνου ανάπτυξης που θα διευκολύνει τη σωστή παράδοση του προγράμματος
της διάδρασης. Το πλάνο ανάπτυξης περιλαμβάνει ένα σύνολο από παραδοτέα
της μορφής αναφορά ή υπόδειγμα, ενώ η σωστή οργάνωση της ομάδας
ανάπτυξης περιλαμβάνει ρόλους, όπως προγραμματιστής, δοκιμαστής,
αναλυτής-σχεδιαστής. Σε ένα πραγματικό έργο ανάπτυξης λογισμικού, αν ο
οργανισμός χρησιμοποιήσει περισσότερους ανθρώπινους πόρους από όσους
χρειάζεται, θα πέσει έξω οικονομικά, αφού το να δουλεύει το έργο δεν
είναι ο μοναδικός στόχος ενός οργανισμού. Θα πρέπει το έργο να έχει
παραχθεί και με βιώσιμο κόστος, ώστε να είναι ανταγωνιστικό. Υπό αυτήν
τη σκοπιά, θα πρέπει να γίνει μια συζήτηση για τη σκοπιμότητα της
επιλογής των εργαλείων ανάπτυξης, η οποία να βασίζεται στις δεξιότητες
των προγραμματιστών, αλλά και στους στόχους του έργου. \footnote{Andrew
  and David (2000)}

Το πιο σημαντικό, διαχρονικά, εργαλείο στην ανάπτυξη νέων συστημάτων
είναι ο επεξεργαστής κειμένου. Η σημασία του κειμένου οφείλεται στο
γεγονός ότι οι περισσότερες γλώσσες προγραμματισμού είναι γραπτές. Αν
και η επεξεργασία κειμένου είναι μια σχετικά απλή δραστηριότητα,
υπάρχουν πάρα πολλά είδη επεξεργαστή κειμένου, γιατί οι προτιμήσεις των
προγραμματιστών και οι απαιτήσεις των έργων ανάπτυξης έχουν μεγάλη
ποικιλία. Για παράδειγμα, μπορούμε να χρησιμοποιήσουμε από έναν
επεξεργαστή κειμένου γενικής χρήσης, που συνήθως είναι ελεύθερα
διαθέσιμος με το λειτουργικό σύστημα του υπολογιστή, μέχρι έναν
εξειδικευμένο επεξεργαστή κειμένου που είναι μέρος ενός εξειδικευμένου
συνόλου εργαλείων ανάπτυξης για μια συγκεκριμένη πλατφόρμα υπολογιστή.
Ανάμεσα σε αυτά τα δύο άκρα υπάρχει ένα πολύ μεγάλο φάσμα από είδη
επεξεργαστών κειμένου, τα οποία διευκολύνουν τη συγγραφή, την ανάγνωση
και τις αλλαγές στον κώδικα, καθώς και τις συνήθειες του προγραμματιστή.
\footnote{(\textbf{Εικόνα?})~9 Ολοκληρωμένο περιβάλλον ανάπτυξης σε
  Emacs (Wikipedia)} \footnote{(\textbf{Εικόνα?})~10 Ολοκληρωμένο
  περιβάλλον ανάπτυξης στο τερματικό (Chris Nicola)} Η σχετική σημασία
του επεξεργαστή κειμένου μειώνεται στις περιπτώσεις που έχουμε μια
μετατόπιση προς οπτικές γλώσσες προγραμματισμού και προς ολοκληρωμένα
περιβάλλοντα ανάπτυξης.

\leavevmode\vadjust pre{\hypertarget{fig:emacs-ide}{}}%
\begin{figure}
\hypertarget{fig:emacs-ide}{%
\centering
\includegraphics{images/emacs-ide.png}
\caption{Εικόνα 9: Ο επεξεργαστής κειμένου GNU Emacs επιτρέπει την
επέκτασή του με τη γλώσσα προγραμματισμού LISP και οργανώνει τα δεδομένα
σε παράλληλους αποθηκευτικούς χώρους, οι οποίοι μπορούν να εμφανίζονται
επιλεκτικά στην οθόνη του χρήστη. Με αυτόν τον τρόπο, έχουν αναπτυχθεί
επεκτάσεις για πολλές εφαρμογές, όπως η ηλεκτρονική αλληλογραφία, η
πλοήγηση στην πληροφορία και η κατασκευή λογισμικού σε ολοκληρωμένο
περιβάλλον ανάπτυξης.}\label{fig:emacs-ide}
}
\end{figure}

\leavevmode\vadjust pre{\hypertarget{fig:vim-ide}{}}%
\begin{figure}
\hypertarget{fig:vim-ide}{%
\centering
\includegraphics{images/vim-ide.jpg}
\caption{Εικόνα 10: Ο συνδυασμός ενός ευέλικτου και επεκτάσιμου
επεξεργαστή κειμένου, όπως ο vim, με ένα απλό παραθυρικό περιβάλλον ή
ακόμη και με έναν πολυπλέκτη τερματικών όπως το tmux, επιτρέπει στον
έμπειρο προγραμματιστή να έχει ένα γρήγορο και πλούσιο σε πληροφορία
περιβάλλον που μπορεί να τον ακολουθεί ανεξάρτητα από τις δυνατότητες
του τερματικού υπολογιστή.}\label{fig:vim-ide}
}
\end{figure}

Μετά τη συγγραφή του πηγαίου κώδικα, το επόμενο βασικό εργαλείο που
απαιτείται για τον προγραμματισμό είναι η δυνατότητα της μετάφρασης ή
της μεταγλώττισης σε εκτελέσιμο κώδικα της τελικής πλατφόρμας. Στο
πλαίσιο του προγραμματισμού και ειδικά των λειτουργικών συστημάτων, αυτό
είναι μια μεγάλη ενότητα, αλλά στο πλαίσιο τηςκατασκευής της διάδρασης η
προτεραιότητα είναι στη γρήγορη δημιουργία εναλλακτικών προγραμμάτων
και, κυρίως, στις επαναληπτικές αλλαγές. Για τον σκοπό αυτό, αν υπάρχει
μια παράμετρος της κατασκευής κατά τον προγραμματισμό της διάδρασης που
έχει μεγάλη σημασία, αυτή είναι η ταχύτητα με την οποία μπορεί ο
κατασκευαστής να εναλλάσσει την ανάπτυξη με τη δοκιμή. \footnote{Reas
  and Fry (2007), Victor (2012)} Όσο πιο γρήγορα μπορεί ο κατασκευαστής
να περνάει από το στάδιο της σχεδίασης της διάδρασης στο στάδιο της
δοκιμής της διάδρασης, στο πλαίσιο δοκιμών είτε με ειδικούς είτε με
τελικούς χρήστες, τόσο πιο γρήγορα το πρόγραμμα της διάδρασης θα
αποκτήσει την επιθυμητή ποιότητα.

Καθώς τα προγράμματα διάδρασης γίνονται περισσότερο σύνθετα και
πολύπλοκα, η σημασία των παραπάνω βασικών εργαλείων και διαδικασιών,
π.χ. επεξεργαστής κειμένου, μετατροπή σε εκτελέσιμο, γίνονται λιγότερα
σημαντικά από την πλευρά του προγραμματιστή της διάδρασης, αφού
προτεραιότητα έχει η επιλογή του κατάλληλου πλαισίου προγραμματισμού
ανάλογα με τις ανάγκες. Για παράδειγμα, ο προγραμματισμός της διάδρασης
για μια εφαρμογή που θα εκτελείται στο διαδίκτυο επιβάλλει τη χρήση των
τεχνολογιών του ιστού και ειδικά εκείνων που διευκολύνουν τη δημιουργία
της διάδρασης στο τερματικό του χρήστη. Στην περίπτωση που είναι
αναγκαίο η δικτυακή εφαρμογή να εκτελείται σε τερματικές συσκευές
διαφορετικού μεγέθους, επιβάλλεται η χρήση των αντίστοιχων τεχνολογικών
αρχετύπων που διευκολύνουν την κλιμάκωση της εφαρμογής σε συσκευές
χρήστη με διαφορετικές δυνατότητες, π.χ. επιτραπέζιος, φορητός, κινητός,
τάμπλετ κτλ. Ταυτοχρόνως, αν η δικτυακή φύση της εφαρμογής απαιτεί και
τη διατήρηση της κατάστασης, τότε επιβάλλεται και η χρήση των
τεχνολογιών του εξυπηρετητή σε απομακρυσμένο υπολογιστή.

\leavevmode\vadjust pre{\hypertarget{fig:android-emulator}{}}%
\begin{figure}
\hypertarget{fig:android-emulator}{%
\centering
\includegraphics{images/android-emulator.png}
\caption{Εικόνα 11: Ο εξομοιωτής για τις κινητές συσκευές με λειτουργικό
σύστημα Android, εκτός από την εξομοίωση της επεξεργασίας των δεδομένων,
περιλαμβάνει και μια προσομοίωση κάποιων κουμπιών και διαδράσεων, που,
συνήθως, έχουν τα έξυπνα κινητά.}\label{fig:android-emulator}
}
\end{figure}

\leavevmode\vadjust pre{\hypertarget{fig:geolocation-simulation}{}}%
\begin{figure}
\hypertarget{fig:geolocation-simulation}{%
\centering
\includegraphics{images/geolocation-simulation.png}
\caption{Εικόνα 12: Το iOS SDK περιλαμβάνει τη δυνατότητα προσομοίωσης
της γεωγραφικής θέσης του χρήστη, γιατί η θέση του πάνω στον χάρτη είναι
σημαντική είσοδος για πολλές κινητές εφαρμογές, οι οποίες μπορούν να
αλλάξουν την πληροφορία στην οθόνη και να στείλουν
ειδοποιήσεις.}\label{fig:geolocation-simulation}
}
\end{figure}

Εξίσου πολύπλοκο τεχνολογικό πλαίσιο μπορεί να έχουμε και για την
ανάπτυξη μιας εφαρμογής επιτραπέζιου υπολογιστή, όταν υπάρχει η απαίτηση
η είσοδος να γίνεται από συσκευή χειρονομίας και η έξοδος να γίνεται σε
περιβάλλον εικονικής πραγματικότητας. Γίνεται, λοιπόν, κατανοητό ότι σε
ένα τόσο διευρυμένο τεχνολογικό πλαίσιο, αναφορικά με τις συσκευές
εισόδου και εξόδου με τον χρήστη, ο προγραμματισμός της διάδρασης έχει
περισσότερο να κάνει με τη δοκιμή και την επιλογή των κατάλληλων για την
περίσταση εργαλείων π.χ. βιβλιοθήκη προγραμματισμού, παρά με τις
λεπτομέρειες της υλοποίησης, οι οποίες μπορεί να είναι τόσο διαφορετικές
όσο και οι διαφορετικές πλατφόρμες ανάπτυξης, π.χ. επιτραπέζιος ΗΥ,
απομακρυσμένος εξυπηρετητής και εκτέλεσης, π.χ. έξυπνο κινητό, έξυπνο
ρολόι, μάσκα εμβύθυσης. Φυσικά, υπάρχουν κάποιες σταθερές αξίες που,
ισχύουν ανεξάρτητα από το τεχνολογικό πλαίσιο και τις λεπτομέρειες της
κάθε βιβλιοθήκης προγραμματισμού, όπως είναι ο συνεχής έλεγχος που
είδαμε στην προηγούμενη ενότητα, καθώς και ο έλεγχος σε ένα περιβάλλον
που θα μοιάζει με αυτό της τελικής πλατφόρμας εκτέλεσης, τον οποίο θα
δούμε στην επόμενη ενότητα.

Όταν η τελική εφαρμογή έχει ως πλατφόρμα εκτέλεσης την ίδια την
πλατφόρμα ανάπτυξης, π.χ. ανάπτυξη εφαρμογής για την επιφάνεια εργασίας
σε επιτραπέζιο υπολογιστή, τότε η μετατροπή του πηγαίου κώδικα σε
εκτελέσιμο κώδικα μπορεί να δοκιμαστεί άμεσα από τον προγραμματιστή πάνω
στον ίδιο υπολογιστή. Στην περίπτωση, όμως, που η τελική πλατφόρμα
εκτέλεσης είναι διαφορετική από την πλατφόρμα ανάπτυξης μιας εφαρμογής,
τότε η δουλειά του προγραμματιστή διευκολύνεται από έναν προσομοιωτή.
Στην περίπτωση που η εφαρμογή δεν έχει διεπαφή με τον χρήστη, τότε ο
προσομοιωτής είναι αναγκαίος για τη δοκιμή και την αποσφαλμάτωση του
πηγαίου κώδικα. Όμως, στην πιο ενδιαφέρουσα περίπτωση στην οποία η
τελική εφαρμογή περιλαμβάνει και την ανάγκη για διάδραση με τον χρήστη,
τότε έχουμε την απαίτηση ο προσομοιωτής να είναι κάτι παραπάνω από ένα
μαύρο κουτί. Αν και στην απλή εκτέλεση κώδικα υψηλού επιπέδου για μια
διαφορετική τελική συσκευή είναι δόκιμο να χρησιμοποιήσουμε την έννοια
του εξομοιωτή (emulator), αυτό σίγουρα δεν είναι σκόπιμο για την
περίπτωση του προγραμματισμού της διάδρασης, όπου η χρήση του
προσομοιωτή (simulator) είναι περισσότερο εύστοχη. \footnote{(\textbf{Εικόνα?})~11
  Εξομοιωτής συσκευής Android (Android Open Source Project)} \footnote{(\textbf{Εικόνα?})~12
  Προσομοίωση γεωγραφικής θέσης (Apple)}

Ένας προσομοιωτής για τη δοκιμή εφαρμογών με διάδραση, χρήστη, οι οποίες
εκτελούνται σε διαφορετική πλατφόρμα ανάπτυξης, θα πρέπει να
περιλαμβάνει και την αντίστοιχη διεπαφή ή τουλάχιστον κάποια προσομοίωση
αυτής. Για παράδειγμα, ο προσομοιωτής για τα έξυπνα κινητά τηλέφωνα
περιλαμβάνει οθόνη, τα αντίστοιχα εικονικά κουμπιά και την προσομοίωση
για κάποιες χειρονομίες. Η οθόνη του προσομοιωτή δεν είναι ίδια με αυτήν
της τελικής συσκευής, αφού η οθόνη του υπολογιστή ανάπτυξης τις
περισσότερες φορές έχει διαφορετικές προδιαγραφές, αλλά σίγουρα είναι
πολύ κοντά. Από την άλλη πλευρά, οι συσκευές εισόδου σε ένα έξυπνο
κινητό, π.χ. κουμπιά πάνω στη συσκευή, αισθητήρες κίνησης και θέσης,
πολυαπτική οθόνη, είναι πολύ διαφορετικές από το πληκτρολόγιο και το
ποντίκι του επιτραπέζιου υπολογιστή ανάπτυξης, με αποτέλεσμα ο βαθμός
προσομοίωσης της διάδρασης να είναι μικρός.

Συμπερασματικά, ο προσομοιωτής είναι ένα αναγκαίο κακό που διευκολύνει
μεν τις δοκιμές κατά τα πρώτα στάδια της ανάπτυξης, αλλά δεν μπορεί να
αντικαταστήσει τις δοκιμές στην τελική συσκευή, επειδή η διάδραση με τον
χρήστη δεν μπορεί να προσομοιωθεί εν γένει. Μάλιστα, όσο πιο πολύ
διαφέρει η διάδραση με τον προσομοιωτή από εκείνη με την τελική συσκευή
τόσο πιο αναγκαίο είναι ένα μεγάλο μέρος της ανάπτυξης να γίνει στην
τελική συσκευή. Από την άλλη πλευρά, ο εξομοιωτής είναι αναγκαίος όταν
θέλουμε να κατασκευάσουμε ένα νέο περιβάλλον προγραμματισμού της
διάδρασης. \footnote{Ingalls (2020)}

\leavevmode\vadjust pre{\hypertarget{fig:pygmalion}{}}%
\begin{figure}
\hypertarget{fig:pygmalion}{%
\centering
\includegraphics{images/pygmalion.jpg}
\caption{Εικόνα 13: Το σύστημα Pygmalion, στα μέσα της δεκαετίας του
1970, ήταν εκείνο που έδωσε τον ορισμό για τα εικονίδια, αλλά, κυρίως,
ήταν το πρώτο που επέτρεψε τη δημιουργία λογισμικού με βάση την τελική
συμπεριφορά και το αποτέλεσμα που πρέπει να έχει ένα πρόγραμμα
υπολογιστή (αντί της κυρίαρχης πρακτικής που ήταν να δίνουμε οδηγίες στο
πρόγραμμα για το τι ακριβώς να κάνει).}\label{fig:pygmalion}
}
\end{figure}

\leavevmode\vadjust pre{\hypertarget{fig:programming-example}{}}%
\begin{figure}
\hypertarget{fig:programming-example}{%
\centering
\includegraphics{images/programming-example.jpg}
\caption{Εικόνα 14: Το σύστημα SmallStar επιτρέπει τη δημιουργία νέου
λογισμικού χωρίς τη χρήση γλώσσας προγραμματισμού με κείμενο για τη ροή
εκτέλεσης. Για τον σκοπό αυτό, ο χρήστης κάνει εγγραφή της
αλληλεπίδρασης με τη διεπαφή και συμπληρώνει φόρμες στα σημεία όπου
υπάρχει ασάφεια.}\label{fig:programming-example}
}
\end{figure}

Ο οπτικός προγραμματισμός έχει γνωρίσει μεγάλη αποδοχή στις περιπτώσεις
της εκμάθησης προγραμματισμού, στον αντικειμενοστραφή προγραμματισμό
και, ειδικά, στον σχεδιασμό της διεπαφής με τον χρήστη. Αρχικά, ο
οπτικός προγραμματισμός επιτρέπει την οπτική οργάνωση και την επισκόπηση
στην περίπτωση που έχουμε πηγαίο κώδικα μεγάλης κλίμακας. Σε αυτήν την
περίπτωση, ο οπτικός προγραμματισμός λειτουργεί ως ένα επίπεδο αφαίρεσης
των λεπτομερειών της υλοποίησης, έτσι ώστε ο κατασκευαστής να μπορεί να
εστιάσει αρχικά στον συνδυασμό των επιμέρους αντικειμένων και στη
συνολική αρχιτεκτονική της διάδρασης. Με αυτόν τον τρόπο, ο
αντικειμενοστραφής προγραμματισμός μπορεί να διευκολυνθεί από ένα οπτικό
περιβάλλον προγραμματισμού.

Στην περίπτωση του σχεδιασμού της διεπαφής με τον υπολογιστή, ο οπτικός
προγραμματισμός επιτρέπει στον κατασκευαστή να χρησιμοποιήσει έτοιμα
μοτίβα ή να φτιάξει δικά του. Για παράδειγμα, η Visual Basic ήταν μια
πολύ διαδεδομένη γλώσσα προγραμματισμού για το λειτουργικό σύστημα
Microsoft Windows, γιατί παρείχε ένα οπτικό περιβάλλον σχεδιασμού της
διεπαφής. Ο προγραμματιστής μπορούσε να διαλέξει οπτικά τα εικονίδια, τα
μενού και τις φόρμες που ήθελε να συμπεριλάβει στην εφαρμογή του και
έπειτα να τα συνδυάσει με τις ενέργειες και τις λειτουργίες του
προγράμματος. Με αυτόν τον τρόπο, δημιουργείται ένας διαχωρισμός ανάμεσα
στη διεπαφή και την υλοποίηση των λειτουργιών, ο οποίος διευκολύνει και
τον καταμερισμό της εργασίας ανάμεσα στους προγραμματιστές των
λειτουργιών και σε εκείνους της διεπαφής. \footnote{(\textbf{Εικόνα?})~13
  Pygmalion (David Canfield Smith)} \footnote{(\textbf{Εικόνα?})~14
  Κατασκευή λογισμικού διάδρασης με παραδείγματα (Xerox PARC)}

Οι γλώσσες προγραμματισμού με την ευρεία έννοιά τους είναι οι δομικές
τεχνολογίες για τα συστήματα διάδρασης, αφού με αυτές υλοποιούνται και
με αυτές καθορίζεται η ποιότητα των διαδραστικών συστημάτων. Οι γλώσσες
προγραμματισμού που βασίζονται σε γραπτό κείμενο είναι ο βασικός τρόπος
διάδρασης για όσους θέλουν να κατασκευάσουν ένα διαδραστικό σύστημα,
ακόμη και αν αυτό χρησιμοποιεί μια άλλη συμβολική γλώσσα πέρα από το
κείμενο για την αλληλεπίδραση με τον χρήστη. Για παράδειγμα, το UNIX
κατασκευάστηκε με τη γλώσσα προγραμματισμού C, αλλά οι χρήστες θα
αλληλεπιδράσουν με ένα κέλυφος, το οποίο παρέχει ένα σετ από βασικές
εντολές που μπορούν να συνδυαστούν σε απλά προγράμματα. Αντίθετα, το
σύστημα του Alto βασίζεται μόνο στην Smalltalk, η οποία, εκτός από
γλώσσα προγραμματισμού, καθορίζει και τη συμπεριφορά του συστήματος
διάδρασης συνολικά τόσο στη βάση του όσο και στις τελικές εφαρμογές, ενώ
στο UNIX αν θέλουμε μια νέα διαφορετική εντολή ή κάποιες σημαντικές
αλλαγές στο σύστημα, αυτό μπορεί να γίνει με τη γλώσσα C. Το χάσμα
ανάμεσα στη γλώσσα προγραμματισμού και στη διάδραση γίνεται ακόμη
μεγαλύτερο κατά τη μετάβαση στις γραφικές διεπαφές επιφάνειας εργασίας,
όπου η διάδραση βασίζεται στον απευθείας χειρισμό αντικειμένων στην
οθόνη. Για να γεφυρωθεί αυτό το χάσμα προγραμματισμού, έχει αναπτυχθεί η
τεχνική του προγραμματισμού με παραδείγματα, όπου ο προγραμματιστής,
αντί να περιγράφει στον υπολογιστή πώς θα εκτελέσει μια λειτουργία,
χρησιμοποιεί τη διάδραση με απευθείας χειρισμό, για να δείξει το
αποτέλεσμα που θέλει. Η τεχνική του προγραμματισμού με παραδείγματα
μπορεί θεωρητικά να δώσει τις δυνατότητες του προγραμματισμού σε χρήστες
με μικρότερες δεξιότητες, αλλά, στην πράξη, δεν έχει αποδειχτεί χρήσιμη
ούτε εκπαιδευτικά ούτε στην αποτελεσματικότητα των νέων δημιουργιών,
αφήνοντας έτσι όλο το βάρος στην ανάπτυξη δεξιοτήτων με τις γλώσσες
προγραμματισμού ή, ακόμη καλύτερα, στην ανάπτυξη νέων γλωσσών
προγραμματισμού.

Τελευταία σε αυτήν την ενότητα αφήσαμε τη γλώσσα προγραμματισμού,
επειδή, τουλάχιστον στην περίπτωση του προγραμματισμού της διάδρασης,
έχει τη λιγότερη σημασία σε σχέση με τις παραμέτρους που εξετάσαμε
παραπάνω. Τα περισσότερα βιβλία για τον προγραμματισμό ασχολούνται
αποκλειστικά με μία γλώσσα προγραμματισμού. Αυτό είναι σωστό μόνο στην
περίπτωση που κάποιος θέλει να μάθει την συγκεκριμένη γλώσσα
προγραμματισμού και λάθος όταν κάποιος θέλει να μάθει τη λογική πίσω από
τον προγραμματισμό υπολογιστών, πράγμα πιο σημαντικό από τις συντακτικές
λεπτομέρειες της κάθε γλώσσας. Η γλώσσα προγραμματισμού είναι σίγουρα
μια σπουδαία παράμετρος, τόσο για την εκμάθηση προγραμματισμού όσο και
την κατασκευή της διάδρασης, αλλά δεν είναι η μόνη παράμετρος, ούτε η
σημαντικότερη. Ειδικά για τη συγγραφή κώδικα κατά τον προγραμματισμό της
διάδρασης ισχύει ότι η κατάλληλη γλώσσα είναι εκείνη που διευκολύνει τη
γρήγορη δημιουργία και την επαναληπτική αλλαγή για πολλά εναλλακτικά
υποδείγματα υψηλής πιστότητας.

Για την κατασκευή της διάδρασης, αυτό που έχει μεγαλύτερη σημασία από τη
γλώσσα προγραμματισμού είναι ο μεταφραστής αυτής της γλώσσας στο τελικό
εκτελέσιμο πρόγραμμα για κάποιον υπολογιστή. Ο προγραμματιστής που έχει
την κατανόηση και, κυρίως, τον έλεγχο του μεταφραστή είναι ελεύθερος
τόσο από την ίδια την γλώσσα προγραμματισμού όσο και από το υλικό
εκτέλεσης. Ιδανικά, ο προγραμματιστής που έχει και την κατανόηση του
πεδίου εφαρμογής, μπορεί να δημιουργήσει μια γλώσσα για αυτό ακριβώς το
πεδίο, έτσι ώστε το πρόγραμμα ως κείμενο να είναι πολύ κοντά στις
ιδιότητες αυτού του πεδίου. Για την υλοποίηση του μεταφραστή μιας νέας
γλώσσας μπορεί να χρησιμοποιηθεί μια ήδη υπάρχουσα γλώσσα και η ανάπτυξη
μπορεί να γίνει σε διαφορετικό υπολογιστή από αυτόν της τελικής
εκτέλεσης. Εναλλακτικά, ένας μεταφραστής μπορεί να γραφτεί στην ίδια τη
γλώσσα που μεταφράζει. Σε αυτήν την περίπτωση, η υλοποίηση μπορεί να
ξεκινήσει από ένα υποσύνολο της γλώσσας, η οποία υλοποιείται απευθείας
στο χαρτί σε συμβολική γλώσσα μηχανής. Για την καλύτερη φορητότητα της
γλώσσας, ο προγραμματιστής μπορεί να ορίσει μια ενδιάμεση εικονική
μηχανή, οπότε η γλώσσα μπορεί να τρέξει και σε διαφορετικό υλικό. Αυτού
του είδους τα αυτόνομα συστήματα είναι δυσκολότερο να σχεδιαστούν
αρχικά, αλλά είναι περισσότερο προσαρμόσιμα σε διαφορετικές ανάγκες,
μέσω των επεκτάσεων.

\hypertarget{ux3c3ux3c5ux3bdux3b5ux3c1ux3b3ux3b1ux3c4ux3b9ux3baux3ae-ux3baux3b1ux3c4ux3b1ux3c3ux3baux3b5ux3c5ux3ae-ux3baux3b1ux3b9-ux3b9ux3b4ux3b9ux3bfux3baux3c4ux3b7ux3c3ux3afux3b1}{%
\subsection{Συνεργατική κατασκευή και
ιδιοκτησία}\label{ux3c3ux3c5ux3bdux3b5ux3c1ux3b3ux3b1ux3c4ux3b9ux3baux3ae-ux3baux3b1ux3c4ux3b1ux3c3ux3baux3b5ux3c5ux3ae-ux3baux3b1ux3b9-ux3b9ux3b4ux3b9ux3bfux3baux3c4ux3b7ux3c3ux3afux3b1}}

Σε αυτήν την ενότητα περιγράφουμε τις ιδιότητες που παρουσιάζει μια
γενιά οργανισμών, οι οποίοι προσπαθούν να μεγιστοποιήσουν τους δεσμούς
τους με άλλους συγγενείς οργανισμούς. Το όραμά τους στηρίζεται σε μια
διαφορετική φιλοσοφία για την ιδιοκτησία και την αξία, ενώ ο
ακρογωνιαίος λίθος για αυτό το οικοδόμημα είναι η συνεργασία των
προγραμματιστών μεταξύ τους, καθώς και ο προγραμματισμός της διεπαφής
του προγραμματιστή, ο οποίος αποτελεί ειδική περίπτωση του
προγραμματισμού της διάδρασης με αποδέκτη έναν χρήστη, τον
προγραμματιστή.

Σε πολλά από τα δημοφιλή βιβλία προγραμματισμού και σχεδίασης λογισμικού
υπάρχει η εξειδανικευμένη εικόνα ότι η σχεδίαση ξεκινάει από μια λευκή
σελίδα. Στην πράξη αυτό είναι πολύ σπάνιο, μάλιστα, στις περισσότερες
περιπτώσεις στις οποίες η σχεδίαση ενός προγράμματος ξεκινάει από μια
λευκή σελίδα, συνήθως οδηγείται προς μια σχετικά ελλιπή εκδοχή ενός
προγράμματος που ήδη υπάρχει κάπου αλλού σε πολύ πιο βελτιωμένη μορφή.
Φυσικά, υπάρχουν και οι εξαιρέσεις, για τις οποίες θα πρέπει να
δημιουργηθεί μια πραγματικά πρωτότυπη εφαρμογή στον υπολογιστή, αλλά
στις περισσότερες περιπτώσεις αυτό που βλέπουμε είναι παραλλαγές ή,
ακόμη καλύτερα, δημιουργικές συνθέσεις πάνω σε βασικά αρχέτυπα που ήδη
υπάρχουν και τα οποία αναφέρονται σε κάποιες ανθρώπινες ανάγκες και
συνήθειες.

Στο πλαίσιο του διαμοιρασμού εργαλείων και τεχνικών, έχει δημιουργηθεί
μια ευρεία κίνηση από χομπίστες και ερευνητές, οι οποίοι συνεργάζονται,
είτε σε ειδικές συναντήσεις είτε με τη βοήθεια του δικτύου για τη
σχεδίαση και την κατασκευή νέων εργαλείων που διευκολύνουν τη δουλειά
τους ή απλώς για διασκέδαση, χωρίς να έχουν εξωτερικά κίνητρα. Η κίνηση
του \emph{φτιάξτο μόνος σου} σίγουρα δεν είναι νέα και δεν αφορά μόνο το
υλικό και το λογισμικό, αλλά πλέον αναπτύσσεται και σε αυτόν τον τομέα
πολύ γρήγορα και προσφέρει ιδέες που δεν θα βρούμε στο εμπόριο.

\leavevmode\vadjust pre{\hypertarget{fig:xerox-colab}{}}%
\begin{figure}
\hypertarget{fig:xerox-colab}{%
\centering
\includegraphics{images/xerox-colab.jpg}
\caption{Εικόνα 15: Δύο δεκαετίες μετά τη δημιουργία του επιτραπέζιου
υπολογιστή γραφείου, οι ερευνητές του Xerox PARC δημιουργούν λογισμικό,
το οποίο διευκολύνει τη σύγχρονη συνεργασία διά ζώσης, στο οποίο εκτός
από τους προσωπικούς υπολογιστές, προσθέτουν και τον ηλεκτρονικό πίνακα
τοίχου.}\label{fig:xerox-colab}
}
\end{figure}

\leavevmode\vadjust pre{\hypertarget{fig:makerspace}{}}%
\begin{figure}
\hypertarget{fig:makerspace}{%
\centering
\includegraphics{images/makerspace.jpg}
\caption{Εικόνα 16: Στα εργαστήρια τύπου hackerspace, η τεχνολογική
καινοτομία και η μάθηση αποτελούν βασικό στόχο, τον οποίο πετυχαίνουν
μέσω της συνεργασίας και της πειραματικής εξερεύνησης. Με αυτόν τον
τρόπο, εργαλεία και τεχνικές που στο παρελθόν κατείχαν μόνο μεγάλες
εταιρείες, πλέον βρίσκονται στη διάθεση των τελικών χρηστών, οι οποίοι
έχουν, επίσης, τους ρόλους του σχεδιαστή και του
κατασκευαστή.}\label{fig:makerspace}
}
\end{figure}

Η διαδικασία της συνεργατικής σχεδίασης έρχεται να ενισχύσει τον ήδη
σημαντικό ρόλο του χρήστη στην κατασκευή της διάδρασης. Ο ρόλος του
τελικού χρήστη ενός υπολογιστικού συστήματος είναι κεντρικός στη
διαδικασία της ανθρωποκεντρικής σχεδίασης, αφού με βάση τον χρήστη
καθορίζονται οι προδιαγραφές του συστήματος και γίνονται οι ενδιάμεσες
αξιολογήσεις της καταλληλότητάς του. Στη συνεργατική σχεδίαση ο χρήστης
κρατάει αυτόν τον κεντρικό ρόλο και επιπλέον αναλαμβάνει έναν ρόλο δίπλα
στους σχεδιαστές του συστήματος, το οποίο, πιθανώς, να χρησιμοποιήσει
στο μέλλον είτε ιδιωτικά είτε στην εργασία του. To πρόγραμμα Apple
Hypercard ήταν από τα πρώτα εμπορικά λογισμικά ευρείας χρήσης που έδωσαν
στους τελικούς χρήστες πολλές δυνατότητες αλλαγής της συμπεριφοράς του,
μια πρακτική που συνεχίστηκε με τη δυνατότητα αυτοματοποίησης συχνών
ενεργειών σε επίπεδο λειτουργικού συστήματος (π.χ. Apple Scripts) ή
εφαρμογών γραφείου (π.χ. Microsoft Office Macros). Σταδιακά, η ενίσχυση
του ρόλου του χρήστη ως ισότιμου σχεδιαστή και κατασκευαστή του
προγραμματισμού της διάδρασης καλύπτει όχι μόνο το λογισμικό αλλά και το
υλικό του υπολογιστή, αφού με τη διάχυση των οικονομικών και ευέλικτων
μίκρο-υπολογιστών (π.χ. Arduino, RaspberryPi κτλ.) οι χρήστες μπορούν να
κατασκευάσουν αυτό που θέλουν. Με αυτόν τον τρόπο, στα αρχικά κινήματα
ανεξάρτητων κατασκευαστών βιντεοπαιχνιδιών, έρχονται να προστεθούν τα
ομότιμα εργαστήρια κατασκευής νέων τεχνολογιών διάδρασης, τα οποία
έγιναν γνωστά με ονόματα όπως makerlab και hackerspace. \footnote{(\textbf{Εικόνα?})~15
  Δωμάτιο ηλεκτρονικών συναντήσεων (Mark Stefik)} \footnote{(\textbf{Εικόνα?})~16
  Συνεργατικό εργαστήριο ερασιτεχνικής κατασκευής συστήματων Makerspace
  (Fair use)}

Η ιδιοκτησία ενός συστήματος διάδρασης είναι ένα πολύπλοκο φαινόμενο,
γιατί ένα σύστημα διάδρασης είναι, συνήθως, μια σύνθεση από υλικό και
λογισμικό η οποία απευθύνεται σε έναν άνθρωπο, στον χρήστη του. Από τη
μια πλευρά, το υλικό και το λογισμικό καλύπτονται από διαφορετική
νομοθεσία για την ιδιοκτησία, με το υλικό να καλύπτεται από δίπλωμα
ευρεσιτεχνίας (πατέντα), ενώ το λογισμικό να θεωρείται κείμενο και να
καλύπτεται από την πνευματική ιδιοκτησία. Από την άλλη πλευρά, η
ανθρωποκεντρική διαδικασία ανάπτυξης ενός συστήματος διάδρασης άπτεται
της νομοθεσίας για την εργονομία, η οποία αφορά κυρίως τις πατέντες. Οι
παραδοσιακές επιχειρήσεις στον χώρο του λογισμικού είναι
υπερπροστατευτικές με την πνευματική ιδιοκτησία τους και περιχαρακώνουν
την περιοχή που τους ανήκει. Αντιθέτως, οι επιχειρήσεις που βασίζονται
στις τεχνολογίες του υπολογισμού και του δικτύου προσπαθούν να είναι όσο
γίνεται περισσότερο ανοικτού κώδικα και ταυτόχρονα να δημιουργούν
συνέργειες με άλλες επιχειρήσεις. \footnote{(\textbf{Εικόνα?})~17
  Συνεισφορές σε κοινό έργο στο Github (Github)} \footnote{(\textbf{Εικόνα?})~18
  Προφίλ χρήστη στο Github (Github)}

\leavevmode\vadjust pre{\hypertarget{fig:github-contributions}{}}%
\begin{figure}
\hypertarget{fig:github-contributions}{%
\centering
\includegraphics{images/github-contributions.png}
\caption{Εικόνα 17: Η αναγνώριση της σχετικής συνεισφοράς σε ένα
συνεργατικό έργο έχει μεγάλη σημασία, γιατί τα περισσότερα έργα και οι
περισσότερες επαγγελματικές δραστηριότητες της σύγχρονης οικονομίας
είναι συνεργατικά. Αντίστοιχα, οι ατομικές επιδόσεις σε επιμέρους
μαθήματα δεν δίνουν μια αντιπροσωπευτική εικόνα των δεξιοτήτων ενός
εκπαιδευόμενου.}\label{fig:github-contributions}
}
\end{figure}

\leavevmode\vadjust pre{\hypertarget{fig:github-profile}{}}%
\begin{figure}
\hypertarget{fig:github-profile}{%
\centering
\includegraphics{images/github-profile.png}
\caption{Εικόνα 18: Το προφίλ ενός χρήστη στο αποθετήριο πηγαίου κώδικα
δίνει μια εποπτική εικόνα της ποιότητας και της ποσότητας της
συνεισφοράς του σε συνεργατικά έργα και μπορεί να αποτελέσει μοντέλο για
τη βελτίωση της πιστοποίησης που δίνουν τα μαθήματα και οι
σχολές.}\label{fig:github-profile}
}
\end{figure}

Στην πράξη, το νομικό πλαίσιο είναι τόσο ασαφές και πολύπλοκο εξαιτίας
της φύσης των συστημάτων διάδρασης, που οι εταιρείες οχυρώνονται με όσες
περισσότερες πατέντες μπορούν να αγοράσουν ή να κατοχυρώσουν, κάνουν
εκατέρωθεν μηνύσεις και, τελικά, συμβιβάζονται εξωδικαστικά. Για
παράδειγμα, εταιρείες όπως η Microsoft ή η Apple, οι οποίες αναπτύχθηκαν
πριν από την εξάπλωση της δικτυακής κουλτούρας, βασίζουν τη
δραστηριότητά τους σε σχετικά κλειστά συστήματα, τα οποία προστατεύουν
με πολλούς τρόπους. Ένας τρόπος με τον οποίο προσπάθησαν οι εταιρείες
του χώρου να προστατέψουν το λογισμικό τους και ειδικά το τμήμα της
διεπαφής είναι οι πατέντες. Η Apple, στα τέλη της δεκαετίας του 1980,
είχε κάνει μήνυση στη Microsoft για την ομοιότητα που παρουσίαζε η
διεπαφή των πρώτων εκδόσεων των Windows με το αντίστοιχο λειτουργικό
σύστημα του Macintosh. Μία δεκαετία αργότερα, η Amazon προσπάθησε να
κερδίσει μία πατέντα για τη δυνατότητα που έδινε στους αγοραστές να
ψωνίζουν με ένα μόνο κλικ του ποντικιού ένα προϊόν από το δικτυακό
μαγαζί της.

Πολλοί επικριτές τους έχουν παρομοιάσει τις παραπάνω πατέντες με την
προσπάθεια να κερδίσει μια εταιρεία την πατέντα για ένα εργαλείο όπως το
σφυρί: δεν υπάρχουν πολλοί τρόποι με τους οποίους να μπορεί ο άνθρωπος
να κρατήσει και να χρησιμοποιήσει ένα σφυρί και αν κάποιος κατοχυρώσει
αυτήν την πατέντα, αποκτά ένα ανταγωνιστικό πλεονέκτημα, το οποίο τελικά
δεν θα βοηθήσει την κοινωνία συνολικά. Ενώ, λοιπόν, είναι αποδεκτό ότι η
αποτελεσματική προστασία της πνευματικής ιδιοκτησίας είναι ένα σημαντικό
κίνητρο για τους δημιουργούς, ταυτόχρονα, έχει γίνει κατανοητό ότι
υπάρχει μια πολύ λεπτή διαχωριστική γραμμή ανάμεσα στην καινοτομία που
πρέπει να προστατευτεί και στο προφανές που πρέπει να είναι διαθέσιμο σε
όλους. Δυστυχώς, αυτή η λεπτή διαχωριστική γραμμή δεν είναι ευδιάκριτη,
ενώ, με τη συνεχή εξέλιξη της τεχνολογίας και των ανθρώπινων αναγκών,
είναι μετακινούμενη.

Οι οργανισμοί και οι εταιρείες της οικονομίας του δικτύου εντοπίζουν και
ορίζουν την ταυτότητα και τον σκοπό τους όχι με βάση μια αγορά, αλλά με
βάση τους συνδέσμους συνεργασίας που έχουν με όλους τους παίκτες σε μια
αγορά. Για παράδειγμα, πάρα πολλά από τα δεδομένα της Google και του
Twitter είναι ελεύθερα διαθέσιμα, επειδή, η αύξηση της χρήσης τους κάνει
τις ίδιες τις εταιρείες πιο σημαντικές, αν και έτσι δίνουν πρόσβαση σε
αυτά και στους ανταγωνιστές τους. Με άλλα λόγια, αυξάνει έμμεσα την
αγορά τους. Συνοπτικά, η πρώτη προσέγγιση βλέπει την αγορά σαν μια πίτα
σταθερού μεγέθους, από την οποία προσπαθεί να πάρει το καλύτερο ή το
μεγαλύτερο κομμάτι. Η δεύτερη προσέγγιση φαντάζεται μια πίτα που
μεγαλώνει συνέχεια. Την ενδιαφέρει να κρατήσει το κομμάτι που έχει, ενώ
δεν την πειράζει και να χάσει κάτι από αυτό, αρκεί η συνολική πίτα-αγορά
να μεγαλώνει και το δικό της κομμάτι να βρίσκεται σε ανάπτυξη. Αυτή η
μικρή φαινομενικά διαφορά αντιμετώπισης της αγοράς λογισμικού έχει πολύ
μεγάλες συνέπειες στην επιχειρηματική πρακτική και το ακριβές μείγμα της
μπορεί να υλοποιηθεί με βάση τον τρόπο με τον οποίο ορίζει μια εταιρεία
τη διεπαφή του προγραμματιστή.

\hypertarget{ux3b7-ux3c0ux3b5ux3c1ux3afux3c0ux3c4ux3c9ux3c3ux3b7-ux3c4ux3bfux3c5-arduino}{%
\subsection{Η περίπτωση του
Arduino}\label{ux3b7-ux3c0ux3b5ux3c1ux3afux3c0ux3c4ux3c9ux3c3ux3b7-ux3c4ux3bfux3c5-arduino}}

Το Arduino είναι ένας πολύ δημοφιλής μικροελεγκτής, ο οποίος φτιάχτηκε
με αρχικό σκοπό τον προγραμματισμό και την εκπαίδευση των φοιτητών της
διάδρασης ανθρώπου και υπολογιστή με συστήματα εισόδου-εξόδου, πέρα από
τα κλασικά πληκτρολόγιο, ποντίκι και οθόνη που έχουμε στους
επιτραπέζιους ΗΥ. Πριν το Arduino, οι φοιτητές και οι ερευνητές που
ήθελαν να δημιουργήσουν και να πειραματιστούν με νέα συστήματα εισόδου
και εξόδου έπρεπε πρώτα να συνδέσουν στον επιτραπέζιο υπολογιστή
κάποιους αισθητήρες και ελεγκτές, μέσω ενός εξωτερικού μικροεπεξεργαστή,
ο οποίος μεταφράζει τα αναλογικά σήματα σε ψηφιακά και αντίστροφα. Ένα
μεγάλο μέρος αυτής της διαδικασίας είναι όμοιο, ανεξάρτητα από το είδος
του αισθητήρα που συνδέουμε, επομένως, ένα σημαντικό μέρος της
προ-εργασίας που γινόταν αποτελούσε εμπόδιο και καθυστέρηση για τον
βασικό στόχο, ενώ, ταυτόχρονα, απαιτούσε και ειδικές δεξιότητες στην
ηλεκτρονική και στους μικροεπεξεργαστές, τις οποίες πολλοί δημιουργικοί
κατασκευαστές της διάδρασης δεν είχαν. Αυτήν την ανάγκη ήρθε να καλύψει
το Arduino, το οποίο δημιουργήθηκε από τους καθηγητές της μεταπτυχιακής
σχολής διάδρασης ανθρώπου και υπολογιστή στο Ινστιτούτο Ιβρέα της
Ιταλίας.

\leavevmode\vadjust pre{\hypertarget{fig:arduino-uno}{}}%
\begin{figure}
\hypertarget{fig:arduino-uno}{%
\centering
\includegraphics{images/arduino-uno.jpg}
\caption{Εικόνα 19: Υπάρχουν πάρα πολλά είδη Arduino, τα οποία
εξυπηρετούν διαφορετικές ανάγκες, ανάλογα με την υπολογιστική ισχύ και
το μέγεθος, άλλα όλη η σειρά αυτών των μικροελεγκτών έχει σχεδιαστεί με
βασικό στόχο τη γρήγορη κατασκευή πρωτοτύπων διάδρασης, τα οποία
βασίζονται σε νέες καινοτόμες συσκευές εισόδου και
εξόδου.}\label{fig:arduino-uno}
}
\end{figure}

\leavevmode\vadjust pre{\hypertarget{fig:arduino-ide}{}}%
\begin{figure}
\hypertarget{fig:arduino-ide}{%
\centering
\includegraphics{images/arduino-ide.png}
\caption{Εικόνα 20: Ο προγραμματισμός νέων διεπαφών με τον χρήστη ήταν
μια εξειδικευμένη εργασία, η οποία απαιτούσε γνώσεις τόσο υλικού όσο και
λογισμικού υπολογιστών. Το ολοκληρωμένο περιβάλλον ανάπτυξης του Arduino
περιλαμβάνει βιβλιοθήκες και γλώσσα προγραμματισμού που μοιάζουν με την
δημοφιλή C, τα οποία κρύβουν τις λεπτομέρειες του υλικού και επιτρέπουν
στον κατασκευαστή να εστιάσει στις νέες λειτουργίες που
σχεδιάζει.}\label{fig:arduino-ide}
}
\end{figure}

Το βασικό μοντέλο Arduino Uno\footnote{(\textbf{Εικόνα?})~19 Arduino Uno
  (Arduino)} έρχεται με μια θύρα USB, η οποία αποτελεί το κύριο κανάλι
δικτυακής επικοινωνίας με έναν επιτραπέζιο ΗΥ. Η θύρα USB είναι πολύ
χρήσιμη για να φορτώσουμε μια νέα έκδοση της εφαρμογής μας, καθώς και
για να δοκιμάσουμε μια εφαρμογή, όταν αυτή θα πρέπει να έχει πρόσβαση σε
δεδομένα του ευρύτερου δικτύου του επιτραπέζιου ΗΥ. Αν και αυτές οι
δυνατότητες δικτυακής επικοινωνίας είναι, συνήθως, αρκετές για τα
περισσότερα εκπαιδευτικά και οικιακά έργα που γίνονται με Arduino, είναι
πολύ περιορισμένες για κάτι εμπορικό ή για κάτι που είναι ανεξάρτητο από
τον παραδοσιακό επιτραπέζιο ΗΥ. Για αυτόν τον σκοπό, οι σχεδιαστές του
Arduino έχουν προβλέψει την τοποθέτηση επεκτάσεων με έναν τυποποιημένο
τρόπο που λέγεται shield.

Το Arduino έχει πολλές εισόδους, τόσο ψηφιακές όσο και αναλογικές, που
μπορούν να συνδεθούν με μια μεγάλη ποικιλία απλών αισθητήρων αλλά και με
πιο πολύπλοκες κατασκευές. Ο ευκολότερος τρόπος για να δώσουμε είσοδο
στο Arduino είναι η απευθείας σύνδεση ενός αισθητήρα με τις
ψηφιακές/αναλογικές εισόδους του. Σε πιο πολύπλοκα συστήματα εισόδου, ο
σχεδιαστής μπορεί να φτιάξει ένα ηλεκτρικό κύκλωμα στο συνοδευτικό
breadboard. Εκτός από τη δυνατότητα για είσοδο από εναλλακτικά
συστήματα, πέρα από το πληκτρολόγιο/ποντίκι, το Arduino σχεδιάστηκε για
να δίνει και έξοδο σε εναλλακτικά συστήματα, πέρα από την παραδοσιακή
οθόνη και τον εκτυπωτή. Φυσικά, όπως και στην περίπτωση των εισόδων, οι
χρήστες του Arduino έχουν βρει πολλές ακόμη εφαρμογές, οι περισσότερες
από τις οποίες εμπνέονται από τα συστήματα ελέγχου (π.χ. βιομηχανία,
ασφάλεια κτλ.).

Το Arduino δεν ήταν η πρώτη προσπάθεια κατασκευής ενός μικροελεγκτή που
διασυνδέεται εύκολα με επιπλέον αισθητήρες, αφού στο παρελθόν είχαν
γίνει αντίστοιχες προσπάθειες τόσο από μεγάλα ερευνητικά έργα και
πανεπιστήμια όσο και από εταιρείες, αλλά κανένα δεν είχε την αποδοχή του
Arduino σε τόσο μικρό χρονικό διάστημα. Αν και δεν είναι εύκολο να
εντοπίσουμε όλες τις παραμέτρους που συνέβαλαν στην επιτυχία του,
σίγουρα μια από αυτές ήταν το γεγονός ότι το έργο βασιζόταν σε
τεχνολογία με ευέλικτη άδεια χρήσης, η οποία επέτρεψε σε άλλους
κατασκευαστές να φτιάξουν τις δικές τους εκδοχές. Επιπλέον, η φύση του
ανοικτού κώδικα έδωσε την αυτοπεποίθηση σε πολλούς σχεδιαστές να το
επιλέξουν, αφού έτσι θα είχαν μεγαλύτερη ασφάλεια από πιθανές αλλαγές
που θα αποφάσιζε μονομερώς μια εταιρεία. Στα λίγα χρόνια της κυκλοφορίας
του, η αποδοχή και η ευελιξία του Arduino αποδείχτηκαν τόσο μεγάλες, που
δημιουργήθηκε μια αντίστοιχα μεγάλη και ενεργή κοινότητα χρηστών, οι
οποίοι ασχολούνται με εφαρμογές πολύ πέρα από τους αρχικούς στόχους του
σχεδιασμού του.

Συνολικά, το Arduino δίνει την ελευθερία στον σχεδιαστή της διάδρασης
ανθρώπου και υπολογιστή να σκεφτεί και να κατασκευάσει σχετικά εύκολα
και οικονομικά εναλλακτικούς τρόπους διάδρασης, πέρα από τον επιτραπέζιο
ΗΥ.\footnote{Banzi and Shiloh (2014)} Όπως ακριβώς το Linux και το
Processing, το Arduino βασίζεται περισσότερο σε μια κοινότητα χρηστών
παρά σε μια ιεραρχικά οργανωμένη εταιρεία για την παροχή μιας σειράς
υπηρεσιών, όπως η πώληση, η τεκμηρίωση και η υποστήριξη. Οι ομοιότητες
μεταξύ του Processing και του Arduino δεν σταματούν στα κίνητρα και στην
ανοικτή κοινότητα ανάπτυξης, αλλά συνεχίζονται και στην υιοθέτηση του
περιβάλλοντος ανάπτυξης του Processing από το Arduino.\footnote{(\textbf{Εικόνα?})~20
  Περιβάλλον προγραμματισμού για μικροελεκτή Arduino (Arduino)}
Βλέπουμε, λοιπόν, ότι ένα απλό περιβάλλον ανάπτυξης, το οποίο βασίζεται
στις συνεχείς αλλαγές του κώδικα και στον γρήγορο έλεγχο του
αποτελέσματος, είναι βασική προϋπόθεση για τον προγραμματισμό της
διάδρασης, ανεξάρτητα από το τεχνολογικό πλαίσιο (π.χ., πολυμέσα,
επιτραπέζιος, διάχυτος υπολογισμός).

\hypertarget{ux3b7-ux3c0ux3b5ux3c1ux3afux3c0ux3c4ux3c9ux3c3ux3b7-ux3c4ux3bfux3c5-reactable}{%
\subsection{Η περίπτωση του
Reactable}\label{ux3b7-ux3c0ux3b5ux3c1ux3afux3c0ux3c4ux3c9ux3c3ux3b7-ux3c4ux3bfux3c5-reactable}}

Το Reactable είναι ένα ψηφιακό μουσικό σύστημα, το οποίο βασίζεται στη
διάδραση με την αφή και με απτά αντικείμενα.\footnote{(\textbf{Εικόνα?})~21
  Reactable (Reactable)} Εκτός από την εκθεσιακή του εγκατάσταση και την
πειραματική του χρήση, έχει χρησιμοποιηθεί σε συναυλίες από γνωστά
συγκροτήματα και μουσικούς, όπως οι Coldplay και η Bjork. Το εμπορικό
αυτό σύστημα ξεκίνησε από ένα ερευνητικό έργο σε ένα πανεπιστήμιο και οι
δημιουργοί του έκαναν διαθέσιμο ένα μεγάλο μέρος του λογισμικού και τις
οδηγίες για την κατασκευή του υλικού, έτσι ώστε να μπορούν περισσότεροι
χρήστες να το φτιάξουν και να το τροποποιήσουν.

Το υλικό του Reactable βασίζεται σε έναν βίντεο-προβολέα και μία κάμερα,
που λειτουργούν ως συσκευές άμεσης εισόδου και εξόδου πάνω σε μια
οριζόντια επιφάνεια, η οποία έχει τη μορφή ενός στρογγυλού τραπεζιού. Η
επιφάνεια διάδρασης μπορεί και αναγνωρίζει την αφή σε πολλαπλά σημεία
και την τοποθέτηση αντικειμένων τα οποία έχουν ένα είδος γραμμοκώδικα
(fiducials). Το μεγάλο μέγεθος της οθόνης σε συνδυασμό με τη δυνατότητα
αναγνώρισης πολλαπλών σημείων αφής και πολλών αντικειμένων επιτρέπει την
ταυτόχρονη και συνεργατική διάδραση πολλών χρηστών, καθώς και την
κατασκευή σύνθετων αναπαραστάσεων, τα οποία στην περίπτωση του
Reactable, είναι φίλτρα ηλεκτρονικής μουσικής. Φυσικά, τίποτα δεν
εμποδίζει έναν κατασκευαστή να χρησιμοποιήσει τα αρχέτυπα διάδρασης του
Reactable (πολυαπτικό, απτά αντικείμενα) για να αναπτύξει εφαρμογές σε
άλλα πεδία, όπως π.χ. στην εκπαίδευση και τις καλιτεχνικές
βιντεοεγκαταστάσεις.\footnote{(\textbf{Εικόνα?})~22 Διαδραστικό
  συντριβάνι με το Reactable (Sergi Jorda)}

Από τη σκοπιά των εργαλείων και της διαδικασίας ανάπτυξης, το πιο
ενδιαφέρον τμήμα του λογισμικού ReacTIVision που χρησιμοποιείται στο
σύστημα Reactable είναι το υποσύστημα της προσομοίωσης. Η δοκιμή για
νέες χειρονομίες για συνεργατικές εφαρμογές σε πολυαπτική οθόνη μπορεί
να ξεκινήσει από τον προσομοιωτή που εκτελείται πάνω στον επιτραπέζιο
υπολογιστή και επιτρέπει στον κατασκευαστή να εξερευνήσει σχετικά άμεσα,
μέσα μια από γρήγορη δοκιμή και επανάληψη, πιθανές εναλλακτικές. Στη
συνέχεια, βέβαια, θα πρέπει να έχει στη διάθεσή του και την αντίστοιχη
πραγματική πολυαπτική επιφάνεια, αφού είναι διαφορετικό για τους χρήστες
να χειρίζονται απτά αντικείμενα και τα δάκτυλά τους, από το να
προσομοιώνουν όλες αυτές τις κινήσεις μέσω του ποντικιού.

Συνοπτικά, βλέπουμε ότι η περίπτωση του προγραμματισμού της διάδρασης
για το Reactable, το οποίο αντιπροσωπεύει ένα νέο σύστημα διάδρασης,
απαιτεί εκτός από τη γλώσσα προγραμματισμού πολλές δεξιότητες. Αρχικά,
οι κατασκευαστές του Reactable είχαν ως στόχο να ικανοποιήσουν τις
ανάγκες μιας πολύ συγκεκριμένης ομάδας χρηστών, των μουσικών που παίζουν
ζωντανά ηλεκτρονική μουσική. Με αφετηρία τις ειδικές ανάγκες αυτής της
ομάδας σχεδίασαν και κατασκεύασαν τόσο το υλικό όσο και το λογισμικό για
τη νέα συσκευή διάδρασης. Σε αυτήν την προσπάθεια δεν χρειάστηκε να
επανεφεύρουν τον τροχό. Αντιθέτως, ενσωμάτωσαν όσα περισσότερα έτοιμα
στοιχεία μπορούσαν από σχετικά έργα (π.χ. πολυαπτική οθόνη προβολής,
αναγνώριση εικόνας, πρωτόκολλο μετάδοσης δεδομένων). Σε αναλογία με την
περίπτωση της γραφικής επιφάνειας εργασίας του επιτραπέζιου υπολογιστή,
η οποία στόχευε να διευκολύνει την εργασία στο γραφείο, οι κατασκευαστές
του Reactable οδηγήθηκαν σε μια νέα συσκευή διάδρασης που εξυπηρετεί τις
ανάγκες της ζωντανής ηλεκτρονικής μουσικής με έναν νέο τρόπο, ο οποίος,
όμως, βασίζεται σε στοιχεία από προηγούμενη έρευνα και ταυτόχρονα έχει
την ευελιξία να εξυπηρετεί και σχετικές ομάδες χρηστών.

\leavevmode\vadjust pre{\hypertarget{fig:reactable-music}{}}%
\begin{figure}
\hypertarget{fig:reactable-music}{%
\centering
\includegraphics{images/reactable-music.jpg}
\caption{Εικόνα 21: Το ηλεκτρονικό μουσικό όργανο Reactable βασίζεται σε
μια δημιουργική σύνθεση υλικού και λογισμικού, η οποία είναι εύκολα
διαθέσιμη και η οποία παρότι ξεκίνησε ως ερευνητικό έργο σε ένα
πανεπιστήμιο δεν άργησε να βρει αποδοχή από τη μουσική
αγορά.}\label{fig:reactable-music}
}
\end{figure}

\leavevmode\vadjust pre{\hypertarget{fig:reactable-fountain}{}}%
\begin{figure}
\hypertarget{fig:reactable-fountain}{%
\centering
\includegraphics{images/reactable-fountain.jpg}
\caption{Εικόνα 22: Το διαφορετικό πλαίσιο χρήσης και οι διαφορετικές
ανάγκες των χρηστών του Reactable οδήγησαν τους σχεδιαστές στην
κατασκευή ενός προσομοιωτή για μια νέα κατηγορία εφαρμογών, η οποία
εκτός από τη μουσική χρησιμοποιείται και αλλού, όπως στον έλεγχο της
ροής του νερού και των φωτιστικών εφέ για ένα συντριβάνι στην
Βαρκελώνη.}\label{fig:reactable-fountain}
}
\end{figure}

\hypertarget{ux3c3ux3cdux3bdux3c4ux3bfux3bcux3b7-ux3b2ux3b9ux3bfux3b3ux3c1ux3b1ux3c6ux3afux3b1-ux3c4ux3bfux3c5-bill-atkinson}{%
\subsection{Σύντομη βιογραφία του Bill
Atkinson}\label{ux3c3ux3cdux3bdux3c4ux3bfux3bcux3b7-ux3b2ux3b9ux3bfux3b3ux3c1ux3b1ux3c6ux3afux3b1-ux3c4ux3bfux3c5-bill-atkinson}}

\leavevmode\vadjust pre{\hypertarget{fig:atkinson-profile}{}}%
\begin{figure}
\hypertarget{fig:atkinson-profile}{%
\centering
\includegraphics{images/atkinson-profile.jpg}
\caption{Εικόνα 23: Ο Bill Atkinson ήταν βασικός συντελεστής σε
σημαντικά προϊόντα όπως τα Apple Lisa και Macintosh, καθώς και το Apple
Paint, για το οποίο κατασκεύασε τη διάδραση με τα μενού και τις παλέτες
εργαλείων. Η συνεισφορά του αποτυπώνεται καλύτερα στο Hypercard, το
οποίο έδωσε για πρώτη φορά τη δυνατότητα σε απλούς χρήστες να φτιάξουν
τις δικές τους εφαρμογές πολυμέσων και
υπερκειμένου.}\label{fig:atkinson-profile}
}
\end{figure}

\leavevmode\vadjust pre{\hypertarget{fig:macpaint-prototype}{}}%
\begin{figure}
\hypertarget{fig:macpaint-prototype}{%
\centering
\includegraphics{images/macpaint-prototype.jpg}
\caption{Εικόνα 24: Πριν ακόμη κατασκευαστεί το γραφικό περιβάλλον με τα
παράθυρα, ο Bill Atkinson ανάπτυξε το λογισμικό ψηφιακής επεξεργασίας
εικόνας Sketchpad, το οποίο τρέχει στον αρχικό εξομοιωτή και βοήθησε
στον διαχωρισμό του λειτουργικού συστήματος από τις εφαρμογές αλλά και
στη σχεδίαση στοιχείων του μελλοντικού γραφικού
περιβάλλοντος.}\label{fig:macpaint-prototype}
}
\end{figure}

Το πιο δημοφιλές λογισμικό που δημιούργησε ο Bill Atkinson ήταν η
γραφική διεπαφή χρήστη για την επιφάνεια εργασίας του πρώτου Apple
Macintosh, \footnote{(\textbf{Εικόνα?})~23 Bill Atkinson (Bill Atkinson)}
η οποία καθόρισε τη λειτουργία των μενού που τραβιούνται προς τα κάτω
και το διπλό κλικ. Με αφετηρία έναν εξομοιωτή για τον υπολογιστή Liza
που έτρεχε στον Apple II, κατασκεύασε σταδιακά τη διάδραση για το
παραθυρικό περιβάλλον, καθώς και την εφαρμογή MacPaint, \footnote{(\textbf{Εικόνα?})~24
  Αρχικό MacPaint χωρίς παράθυρα (Bill Atkinson)} ως παράδειγμα για το
νέο είδος φιλικών και δημιουργικών εφαρμογών που θα αναπτύσσονταν
μελλοντικά πάνω σε αυτήν την πλατφόρμα.

Εκτός από το ευρέως γνωστό λογισμικό διάδρασης της πλατφόρμας του
Macintosh, δημιούργησε την εφαρμογή Hypercard, η οποία έδωσε σημαντικές
δυνατότητες ανάπτυξης νέων εφαρμογών σε χρήστες που δεν είχαν γνώσεις
προγραμματισμού. Το Hypercard έδωσε πίσω στους χρήστες της γραφικής
επιφάνειας εργασίας έναν μικρό τουλάχιστον έλεγχο των διαθέσιμων
εφαρμογών, ενώ, παράλληλα, αποτέλεσε το βασικό συγγραφικό εργαλείο για
μια νέα κατηγορία πολυμεσικών εφαρμογών. Με την εφαρμογή Hypercard, οι
χρήστες μπορούσαν να πλοηγούνται σε υπερμεσικό περιεχόμενο και την ίδια
στιγμή να επεξεργάζονται την πληροφορία, δυνατότητες που θα χαθούν σε
μεταγενέστερα συστήματα.

Η έμφαση στην εικόνα, στη δημιουργία και στην ενδυνάμωσή τους μέσω της
τεχνολογίας δεν είναι τυχαία. Ο Bill Atkinson μεγάλωσε με χόμπυ την
φωτογραφία, την οποία συνέχισε να υπηρετεί σε όλη του τη ζωή και η οποία
επηρέασε, λιγότερο ή περισσότερο άμεσα, το λογισμικό που δημιούργησε.
Εκτός από τα QuickDraw, MacPaint, Hypercard, ως απάντηση στο ανερχόμενο
Instagram κατασκεύασε την κινητή εφαρμογή PhotoCard, η οποία
χρησιμοποιεί τις δυνατότητες του έξυπνου κινητού ώστε να δώσει νέα ζωή
στις παραδοσιακές κάρτ ποστάλ. Για την κατασκευή του λογισμικού
διάδρασης χρησιμοποιεί πάντα μεταφορές από τον φυσικό κόσμο, ενώ
προτιμάει να εργάζεται σε απόλυτη συγκέντρωση στο σπίτι και επισκέπτεται
την εταιρεία για να κάνει δοκιμές με τους χρήστες.

Η πολυμεσική επικοινωνία ήταν επίσης το αντικείμενο της εταιρίας General
Magic, η οποία με το λογισμικό Magic Cap ήταν η πρώτη που προσπάθησε να
μεταφέρει τις δυνατότητες του επιτραπέζιου προσωπικού υπολογιστή σε μια
μικρή φορετή συσκευή χεριού. Αν και η εταιρεία αυτή δεν πέτυχε, η
τεχνογνωσία που ανέπτυξαν οι συντελεστές της μεταφέρθηκε στους νεότερους
υπαλλήλους της και οδήγησε τελικά στη δημιουργία των iPhone και Android,
τα οποία κυριάρχησαν δύο δεκαετίες αργότερα. Αυτό που διατρέχει τη
δουλειά του διαχρονικά είναι η έμφαση στη σημασία του προγραμματισμού
που βρίσκεται στην υπηρεσία της ανθρώπινης έκφρασης και
δημιουργικότητας.

\hypertarget{ux3b2ux3b9ux3b2ux3bbux3b9ux3bfux3b3ux3c1ux3b1ux3c6ux3afux3b1}{%
\subsection*{Βιβλιογραφία}\label{ux3b2ux3b9ux3b2ux3bbux3b9ux3bfux3b3ux3c1ux3b1ux3c6ux3afux3b1}}
\addcontentsline{toc}{subsection}{Βιβλιογραφία}

\hypertarget{refs}{}
\begin{CSLReferences}{0}{0}
\end{CSLReferences}

Andrew, Hunt, and Thomas David. 2000. {``The Pragmatic Programmer: From
Journeyman to Master.''} Addison Wesley Longman, Redwood City.

Banzi, Massimo, and Michael Shiloh. 2014. \emph{Getting Started with
Arduino: The Open Source Electronics Prototyping Platform}. Maker Media,
Inc.

Graham, Paul. 2004. \emph{Hackers \& Painters: Big Ideas from the
Computer Age}. " O'Reilly Media, Inc.".

Grudin, Jonathan. 1990. {``The Computer Reaches Out: The Historical
Continuity of Interface Design.''} In \emph{Proceedings of the SIGCHI
Conference on Human Factors in Computing Systems}, 261--68. ACM.

Ingalls, Daniel. 2020. {``The Evolution of Smalltalk: From Smalltalk-72
Through Squeak.''} \emph{Proceedings of the ACM on Programming
Languages} 4 (HOPL): 1--101.

McConnell, Steve. 2004. \emph{Code Complete}. Pearson Education.

Noble, Joshua. 2009. \emph{Programming Interactivity: A Designer's Guide
to Processing, Arduino, and OpenFrameworks}. " O'Reilly Media, Inc.".

Olsen, Dan. 2009. \emph{Building Interactive Systems: Principles for
Human-Computer Interaction}. Cengage Learning.

Reas, Casey, and Ben Fry. 2007. \emph{Processing: A Programming Handbook
for Visual Designers and Artists}. 6812. Mit Press.

Thimbleby, H. 2007. \emph{Press on: Principles of Interaction
Programming}. MIT Press, Cambridge.

Victor, Bret. 2012. {``Learnable Programming: Designing a Programming
System for Understanding Programs.''} 2012.
\url{http://worrydream.com/LearnableProgramming}.

\hypertarget{ux3bcux3bfux3bdux3c4ux3adux3bbux3b1}{%
\section{Μοντέλα}\label{ux3bcux3bfux3bdux3c4ux3adux3bbux3b1}}

\begin{quote}
Το σπουδαίο έργο του Einstein είχε προέλθει από φυσική διαίσθηση και
όταν ο Einstein σταμάτησε να δημιουργεί, ήταν επειδή έπαψε να σκέφτεται
με συγκεκριμένες φυσικές εικόνες και έγινε χειριστής εξισώσεων. Freeman
Dyson
\end{quote}

\hypertarget{ux3c0ux3b5ux3c1ux3afux3bbux3b7ux3c8ux3b7}{%
\subsubsection{Περίληψη}\label{ux3c0ux3b5ux3c1ux3afux3bbux3b7ux3c8ux3b7}}

Η αρχική αντίληψη που είχαν οι περισσότεροι για τη διάδραση του
υπολογιστή με τον άνθρωπο είναι ότι ο υπολογιστής είναι κυρίως ένα
εργαλείο. Ένα εργαλείο που δημιουργήθηκε και χρησιμοποιείται από τον
άνθρωπο για να βελτιώσει τις δραστηριότητές του σε διάφορους τομείς της
ζωής και κυρίως στην εργασία. Για παράδειγμα, στη διάδραση ανθρώπου και
υπολογιστή μια από τις πιο δημοφιλείς εφαρμογές είναι η ψηφιακή
επεξεργασία κειμένου. Στην επεξεργασία κειμένου ο ρόλος του υπολογιστή
ως εργαλείου είναι η βελτίωση της εργασίας που μέχρι τότε γινόταν με
εργαλείο τη γραφομηχανή και, ακόμη παλιότερα, με την πένα. Στην πορεία,
η ευελιξία που έχει ο υπολογιστής στην εκτέλεση διαφορετικών
προγραμμάτων χρήστη και η εφεύρεση νέων στυλ διάδρασης, πέρα από την
εισαγωγή κειμένου πάνω στο πληκτρολόγιο, επέτρεψαν στη διάδραση να έχει
περισσότερους ρόλους πέρα από αυτόν του εργαλείου. Επιπλέον, τα νέα
μοντέλα διάδρασης δίνουν μεγαλύτερη σημασία σε αξίες πέρα από τη
γνωστική επεξεργασία της πληροφορίας, όπως τα συναισθήματα, η κοινωνική
διάσταση, καθώς και η συνολική παρουσία του ανθρώπινου σώματος στον
χώρο.

Καθώς προχωράμε σε νέα μοντέλα διάδρασης δεν σημαίνει ότι τα προηγούμενα
βασικά μοντέλα της διάδρασης χάνονται. Αντίθετα, τα βασικά μοντέλα της
διάδρασης συνεχίζουν να έχουν σημαντικό ρόλο ως~συστατικά στοιχεία σε
πιο σύνθετα συστήματα. Για παράδειγμα, ένα σύστημα συζήτησης εξ
αποστάσεως έχει ως βασικό στοιχείο του το κοινωνικό μοντέλο διάδρασης,
όπου ο υπολογιστής μεσολαβεί στην επικοινωνία και τη συνεργασία δύο ή
περισσότερων ανθρώπων, αλλά μπορεί να περιέχει και το εργαλείο της
ανάκτησης πληροφορίας από παλιότερες συζητήσεις. Επιπλέον, το ίδιο
σύστημα μπορεί να επεκταθεί με το μοντέλο του πράκτορα διάδρασης, ο
οποίος παρακολουθεί εκ μέρους του χρήστη τις συζητήσεις που γίνονται και
τον ενημερώνει όταν υπάρχει κάτι που τον ενδιαφέρει ή κάνει παρεμβάσεις
εκ μέρους του. Επίσης, το παραπάνω σύστημα συνεργασίας μπορεί να
επεκταθεί με τη χρήση διάχυτων συσκευών διάδρασης, οι οποίες μπορούν να
μεταφέρουν και τη μη λεκτική επικοινωνία. Στα επόμενα, περιγράφουμε την
ιστορική εξέλιξη των μοντέλων διάδρασης, καθώς και τη θεωρία για καθένα
ξεχωριστά, ενώ στο επόμενο κεφάλαιο θα δούμε τη σύνθεσή τους.

\hypertarget{ux3bf-ux3c5ux3c0ux3bfux3bbux3bfux3b3ux3b9ux3c3ux3c4ux3aeux3c2-ux3c9ux3c2-ux3b5ux3c4ux3b1ux3afux3c1ux3bfux3c2}{%
\subsection{Ο υπολογιστής ως
εταίρος}\label{ux3bf-ux3c5ux3c0ux3bfux3bbux3bfux3b3ux3b9ux3c3ux3c4ux3aeux3c2-ux3c9ux3c2-ux3b5ux3c4ux3b1ux3afux3c1ux3bfux3c2}}

\leavevmode\vadjust pre{\hypertarget{fig:eliza-chat-bot}{}}%
\begin{figure}
\hypertarget{fig:eliza-chat-bot}{%
\centering
\includegraphics{images/eliza-chat-bot.jpg}
\caption{Εικόνα 1: Το διαδραστικο σύστημα ELIZA βασίζεται στον διάλογο
με τον χρήστη σε φυσική γλώσσα και έγινε πολύ δημοφιλές με ένα σενάριο,
στο οποίο ο υπολογιστής παριστάνει τον ψυχολόγο. Αν και δημιουργήθηκε
για να δείξει τους περιορισμούς στην κατανόηση της φυσικής γλώσσας από
τους υπολογιστές, πολλοί χρήστες προτίμησαν να πιστέψουν ότι μιλάνε με
κάποιον που τους καταλαβαίνει.}\label{fig:eliza-chat-bot}
}
\end{figure}

\leavevmode\vadjust pre{\hypertarget{fig:microsoft-bob}{}}%
\begin{figure}
\hypertarget{fig:microsoft-bob}{%
\centering
\includegraphics{images/microsoft-bob.jpg}
\caption{Εικόνα 2: Παράλληλα με την ανάπτυξη της επιφάνειας εργασίας,
γίνονται προσπάθειες για την ανάπτυξη μιας διαλογικής διάδρασης. Η
περίπτωση του Microsoft Bob δείνει ότι το αποτέλεσμα της
ανθρωποκεντρικής σχεδίασης και των κανόνων σχεδίασης εξαρτάται και από
το πλαίσιο χρήσης.}\label{fig:microsoft-bob}
}
\end{figure}

Το όραμα για διεπαφές φυσικής γλώσσας έχει διατυπωθεί από πολύ νωρίς με
μια υπόθεση στην τελευταία δημοσίευση του Alan Turing, σύμφωνα με την
οποία η ανταλλαγή μηνυμάτων με έναν υπολογιστή μπορεί να καθορίσει το
επίπεδο νοημοσύνης του υπολογιστή. Εξίσου πιθανό, βέβαια, είναι ο
χρήστης να συμπεριφέρεται έτσι ώστε η δική του νοημοσύνη να φαίνεται
μικρότερη από αυτήν του υπολογιστή. Για παράδειγμα, το δημοφιλές ρομπότ
ψυχανάλυσης ELIZA λειτουργεί σε ένα μικρό πεδίο συζήτησης και μπορεί να
πείσει τους χρήστες να μιλήσουν για τα προσωπικά τους σε μια μηχανή και
τελικά να νιώσουν καλύτερα. Αυτό δεν αποδεικνύει τόσο τη δυνατότητα να
έχουμε έξυπνους υπολογιστές όσο τη δυνατότητα ή την ευκολία να
πιστέψουμε αυτήν την ιδέα. Παράλληλα με την ανάπτυξη των τεχνολογιών
διάδρασης, αναπτύσσεται με πολύ μεγαλύτερους ρυθμούς, ο κλάδος της
Τεχνητής Νοημοσύνης, στον οποίο η διάδραση δεν έχει μεγάλη σημασία, αφού
τα συστήματα αυτά σχεδιάζονται με συμβολικούς τρόπους ή με τη μηχανική
μάθηση, προκειμένου να λειτουργούν αυτόνομα. Οι τεχνικές που έχει
αναπτύξει ο κλάδος της Τεχνητής Νοημοσύνης είναι μερικές φορές χρήσιμες
ως υποσυστήματα διάδρασης, αλλά γίνονται λιγότερο χρήσιμες μέχρι και
άγονες, όταν παρουσιάζονται με την τεχνική του ανθρωπομορφισμού.
\footnote{(\textbf{Εικόνα?})~1 ELIZA chat bot (wikimedia)} \footnote{(\textbf{Εικόνα?})~2
  Έξυπνη διεπαφή Microsoft Bob (Microsoft)}

Η αντίληψη του υπολογιστή ως μιας έξυπνης και αυτόνομης οντότητας
εμφανίστηκε για πρώτη φορά στη λογοτεχνία και στις ταινίες επιστημονικής
φαντασίας όπου οι δημιουργοί φαντάζονταν έναν υπολογιστή με ανθρώπινα
στοιχεία διάδρασης. Αυτός ο ανθρωπομορφισμός της διάδρασης ανθρώπου και
υπολογιστή μπορεί να πάρει διάφορες μορφές ή χαρακτηριστικά, τα οποία
συνδέουμε με την ανθρώπινη παρουσία, όπως πρόσωπο, ομιλία, σώμα,
συναίσθημα, κοινή λογική. Η περιοχή της Τεχνητής Νοημοσύνης προσπάθησε
να δώσει πρακτικές λύσεις στα παραπάνω, αλλά τα αποτελέσματά της έχουν
μόνο περιορισμένη χρησιμότητα και σε πολύ καλά ορισμένες εφαρμογές. Για
παράδειγμα, η αναγνώριση γραφής και ομιλίας είναι πλέον εμπορικά
διαθέσιμες ύστερα από περισσότερα από σαράντα χρόνια έρευνας και
αποτυχημένων εμπορικών προϊόντων. Στην πράξη, όμως, δεν είναι όσο
χρήσιμη τη φαντάστηκαν οι δημιουργοί της επιστημονικής φαντασίας, ούτε
τόσο πρακτική όσο τη σχεδίασαν οι επιστήμονες της Τεχνητής Νοημοσύνης,
αφού στις περισσότερες περιπτώσεις η διάδραση γίνεται με συσκευές
εισόδου και επιπρόσθετες διεπαφές που πρέπει να μάθει να χρησιμοποιεί ο
χρήστης.

Ο αυτοματισμός είναι μια εκδοχή της διάδρασης ανθρώπου και υπολογιστή,
στην οποία οι ενέργειες του χρήστη, η διεπαφή του υπολογιστή ή ακόμη και
οι ανάγκες του χρήστη εξάγονται δυναμικά από τον υπολογιστή, ο οποίος
μπορεί επιπλέον να λαμβάνει αποφάσεις με βάση ορισμένους κανόνες. Ο
αυτοματισμός που προσφέρει ένας υπολογιστής μπορεί να διευκολύνει και τη
διάδραση με τον άνθρωπο. Για παράδειγμα, μια λίστα με εφαρμογές μπορεί
να ταξινομηθεί ανάλογα με το ποιες χρησιμοποιούμε συχνότερα. Φυσικά,
αυτό σημαίνει ότι το αντίστοιχο μενού θα είναι πιθανόν διαφορετικό κάθε
φορά που το ζητάει ο χρήστης, πράγμα που ίσως δημιουργεί, εκτός από
διευκόλυνση, και μια ασυνέπεια απέναντι στη συνήθεια του να βρίσκουμε τα
πράγματα εκεί που τα αφήσαμε. \footnote{(\textbf{Εικόνα?})~3 Δυναμικά
  εξατομικευμένα μενού επιλογών (Microsoft)} Αν και ο αυτοματισμός έχει
αποδειχτεί πολύ χρήσιμος σε πολλές εφαρμογές των υπολογιστών, στην
περίπτωση της διάδρασης, η χρήση του αυτοματισμού θέλει προσοχή, κυρίως,
επειδή θα πρέπει να αξιολογηθεί απέναντι στον χρήστη, και όχι μόνο
απέναντι σε κάποιες λειτουργικές και φαινομενικά αντικειμενικές
προδιαγραφές.

Παράλληλα με τους ερευνητές της Τεχνητής Νοημοσύνης που προσπαθούν να
κωδικοποιήσουν την ανθρώπινη λογική και συμπεριφορά σε έναν υπολογιστή,
οι ερευνητές της Ανθρώπινης Επικοινωνίας διαπιστώνουν ότι στην πράξη οι
ανθρώπινες αντιδράσεις κατά τη διάδραση με υπολογιστές και μέσα
επικοινωνίας δεν έχουν διαφορά από τις αντιδράσεις κατά τη διάδραση με
άλλους ανθρώπους και φυσικά αντικείμενα. Η διαπίστωση αυτή δημιούργησε
μια σειρά από ερευνητικές και εμπορικές απόπειρες στην κατεύθυνση του
ανθρωπομορφισμού και του σκευομορφισμού. Το αρχικό κίνητρο σε όλες αυτές
τις προσπάθειες είναι η δημιουργία μιας περισσότερο οικείας και
\emph{φιλικής} διάδρασης με τον άνθρωπο. Στην πορεία, η φιλική διεπαφή
της δεκαετίας του 1990 έδωσε τη θέση της στην πειστική διεπαφή της
δεκαετίας του 2000, όταν οι σχεδιαστές προσπάθησαν να δημιουργήσουν
διεπαφές που θα βοηθούσαν τους χρήστες να αλλάξουν συμπεριφορά.
\footnote{Fogg (2003)} Για παράδειγμα, μια εφαρμογή που καταγράφει τα
βήματα και την απόσταση που διανύουμε καθημερινά με τα πόδια μπορεί να
περιέχει τη στατιστική αποτύπωση των επιδόσεών μας και επιπλέον μια
ανθρωπόμορφη διεπαφή που να μας παροτρύνει ή ακόμη και να μας επικρίνει
(ανάλογα με το ψυχολογικό προφίλ του χρήστη) για τις επιδόσεις μας, έτσι
ακριβώς όπως θα έκανε ένας φίλος ή ένας προσωπικός γυμναστής.
\footnote{(\textbf{Εικόνα?})~4 Ανθρωπομορφισμός και έξυπνες διεπαφές
  (MIT)}

\leavevmode\vadjust pre{\hypertarget{fig:adaptive-menus}{}}%
\begin{figure}
\hypertarget{fig:adaptive-menus}{%
\centering
\includegraphics{images/adaptive-menus.png}
\caption{Εικόνα 3: Οι χρήστες προτιμούν τα στατικά μενού ή τα μενού που
μπορούν να αλλάξουν μόνοι τους, ενώ δεν προτιμούν ούτε έχουν καλή
επίδοση με ένα μενού που αλλάζει αυτόματα ανάλογα με τη συχνότητα
χρήσης.}\label{fig:adaptive-menus}
}
\end{figure}

\leavevmode\vadjust pre{\hypertarget{fig:relational-agent}{}}%
\begin{figure}
\hypertarget{fig:relational-agent}{%
\centering
\includegraphics{images/relational-agent.jpg}
\caption{Εικόνα 4: Ο ανθρωπομορφισμός αρχικά χρησιμοποιήθηκε για να
βελτιώσει τη φιλικότητα της διάδρασης με τον υπολογιστή. Στην πορεία
δοκιμάστηκε και ως ένας τρόπος να αυξηθεί η πειστικότητα των συμβουλών
του υπολογιστή προς έναν χρήστη που θέλει να βελτιώσει τον τρόπο ζωής
του.}\label{fig:relational-agent}
}
\end{figure}

Ίσως μια από τις πιο αποτυχημένες εισαγωγές νέου προϊόντος διάδρασης
ανθρώπου και υπολογιστή να ήταν η προσπάθεια της Microsoft να δώσει μια
πιο φιλική διάδραση στις εφαρμογές της με το Microsoft Bob και το
σχετικό Microsoft Office Clip. Αν και η ιδέα του ανθρωπομορφισμού
βασιζόταν ήδη σε πολυετή επιστημονική έρευνα, \footnote{Reeves and Nass
  (1996)} και παρά το ότι είχαν γίνει πολλές δοκιμές με χρήστες στα
αρχικά στάδια του προϊόντος, τελικά η αποδοχή από το ευρύ κοινό ήταν από
μικρή έως αρνητική.

Εκ των υστέρων, μια πιθανή εξήγηση αυτής της έλλειψης αποδοχής είναι ότι
τουλάχιστον στη δεκαετία του 1990 οι περισσότεροι επιτραπέζιοι
υπολογιστές χρησιμοποιούνταν ακόμη κυρίως στην εργασία και γι' αυτό η
έννοια της φιλικότητας μέσω του ανθρωπομορφισμού δεν είχε μεγάλη σημασία
για τους χρήστες τους. Μπορεί να τους φαινόταν ακόμη και ενοχλητική,
καθώς προσπαθούσαν να συγκεντρωθούν για να εργαστούν στον επεξεργαστή
κειμένου και ξαφνικά, στην οθόνη του τερματικού τους, τους έπιανε
κουβέντα ένας σκύλος! Σε δεύτερη ανάγνωση, η περίπτωση του Microsoft Bob
είναι και μια προειδοποίηση ότι οι τεχνικές της ανθρωποκεντρικής
σχεδίασης δεν είναι ασφαλείς, όχι επειδή είναι λάθος τεχνικές, αλλά,
κυρίως, επειδή έχουν ως στόχο να ικανοποιήσουν ανθρώπινες ανάγκες που
είναι από ασαφείς μέχρι ρευστές.

Οι έξυπνοι πράκτορες είναι μια αυτοματοποιημένη εκδοχή της διάδρασης με
τον υπολογιστή. Ένας τρόπος να οριστούν οι έξυπνοι πράκτορες είναι να
οριστούν ως το συμπλήρωμα του απευθείας χειρισμού. Όπως δηλαδή έχουμε
τον απευθείας χειρισμό για να εκτελέσουμε μια διεργασία στον υπολογιστή,
έτσι έχουμε και τους έξυπνους πράκτορες που μπορούν να εκτελέσουν αυτήν
τη διεργασία για λογαριασμό μας. Από τη μια πλευρά, ο απευθείας
χειρισμός δίνει στον χρήστη τη δυνατότητα να πάρει αποφάσεις, αλλά από
την άλλη πλευρά υπάρχει ένα όριο στην πληροφορία που μπορεί να
επεξεργαστεί ο χρήστης. Επιπλέον, σίγουρα υπάρχουν πολλές αποφάσεις που
ως παρόμοιες ή τουλάχιστον έχοντας κάποιες ιδιότητες που είναι
παρόμοιες, θα μπορούσαν να λαμβάνονται από έναν έξυπνο πράκτορα που
εκπροσωπεί τον χρήστη. Αν και σε πρώτη ανάγνωση φαίνεται ότι οι έξυπνοι
πράκτορες είναι το αντίθετο του απευθείας χειρισμού, εντούτοις είναι
περισσότερο γόνιμο να αντιμετωπίζουμε αυτές τις δύο βασικές φιλοσοφίες
διάδρασης ως συμπληρωματικές. Για παράδειγμα, ο χρήστης με απευθείας
χειρισμό δείχνει στον υπολογιστή ότι κάποια από τα ηλεκτρονικά μηνύματα
του είναι άχρηστα, ενώ άλλα είναι χρήσιμα για την εργασία του, κατά
συνέπεια, ο έξυπνος πράκτορας, την επόμενη φορά που θα δει ένα μήνυμα,
θα μπορέσει να το μεταφέρει στον σωστό φάκελο χωρίς επιπλέον ενέργειες
από την πλευρά του χρήστη. Στην πράξη, αυτά τα συστήματα έχουν βελτιωθεί
πάρα πολύ και στις περισσότερες περιπτώσεις κάνουν λιγότερα λάθη από τον
άνθρωπο, όμως σε κάθε περίπτωση, για όποια λάθη κάνουν δεν φέρουν
ευθύνη, κι αυτό μπορεί να είναι σημαντικό σε κάποιες εφαρμογές.

Όταν η μεταφορά γίνεται ιδιαίτερα αναπαραστατική σε ένα γραφικό
περιβάλλον, τότε μιλάμε για σκευομορφισμό, όπως για παράδειγμα πολλές
εφαρμογές στις πρώτες εκδόσεις του iPhone, οι οποίες μοιάζουν με
αντικείμενα (π.χ. μικρόφωνο για την εγγραφή ηχητικών σημειώσεων) ή
προσομοιώνουν οπτικά την υφή υλικών, όπως π.χ. μέταλλο, ύφασμα, κτλ. Αν
και το αρχικό iPhone ήταν από τις πιο δημοφιλείς συσκευές με τεχνικές
σκευομορφισμού στην έξοδο προς τον χρήστη, σίγουρα δεν ήταν η πρώτη. Στη
δεκαετία του 1990, η σταδιακή μετατροπή και διανομή της μουσικής με
συμπιεσμένα αρχεία τύπου MP3 δημιούργησε την ανάγκη για ευέλικτες
εφαρμογές εκτέλεσης των αρχείων MP3 στον υπολογιστή και ανάμεσα σε αυτές
οι πιο δημοφιλείς ήταν εκείνες που επέτρεπαν στον χρήστη να αλλάξει την
εμφάνιση της εφαρμογής (skinning), με πιο αντιπροσωπευτικό παράδειγμα
την εφαρμογή Winamp. Η χρήση της μεταφοράς και του σκευομορφισμού δεν
έχει από μόνη της πάντα θετικά αποτελέσματα. Για παράδειγμα, στις αρχές
της δεκαετίας του 1990, το λειτουργικό σύστημα Magic Cap για τον κινητό
υπολογισμό εμφάνιζε στην οθόνη ένα γραφείο με οικεία αντικείμενα, όπως
το τηλέφωνο-φαξ, το ημερολόγιο, το σημειωματάριο, τη λίστα επαφών, τα
εισερχόμενα και εξερχόμενα μηνύματα, το αρχείο, κτλ. Η χρήση της
μεταφοράς στην περίπτωση του Magic Cap δεν σταματούσε στο γραφείο, αλλά
συνεχιζόταν με τη μετακίνηση του χρήστη σε δωμάτια, από τα οποία είχε
πρόσβαση σε διαφορετικές λειτουργίες. Αν και είναι δύσκολο έως αδύνατο
να διακρίνουμε τα προβλήματα, καμία από τις παραπάνω μορφές της διεπαφής
με τον χρήστη δεν είχε την αποδοχή των χρηστών, παρόλο που εφάρμοζαν
κάποιους από τους κανόνες της φιλικής για τον άνθρωπο σχεδίασης, όπως
είναι η μεταφορά και ο σκευομορφισμός.

Επιπλέον, με την ανάπτυξη του κινητού και διάχυτου υπολογισμού, στις
αρχές της δεκαετίας του 2000, η διάδραση έκανε το μεγάλο βήμα πέρα από
το πλαίσιο της εργασίας και του γραφείου. Ειδικά για την περίπτωση του
διάχυτου υπολογισμού, όπου έχουμε πολλούς υπολογιστές διαφόρων μορφών
φορετών στον χρήστη ή διάχυτων στο περιβάλλον, η νέα θεώρηση της
διάδρασης βασίστηκε στις φιλοσοφικές θεωρίες για την ενσώματη διάδραση,
οι οποίες περιγράφουν την ανθρώπινη σκέψη, την αντίληψη και τη δράση ως
έννοιες στενά συνδεδεμένες με την ύπαρξη και τις ιδιότητες του
ανθρώπινου σώματος. Οι υπολογιστές, πλέον, εκτός από την κινητή μορφή
τους, μπορούν να φορεθούν και καταγράψουν βιομετρικά στοιχεία του χρήστη
τους, όπως κινήσεις και σφυγμό. Παράλληλα με τη σταδιακή έμφαση στη
συνολική φύση του ανθρώπου, μια μεγάλη μερίδα του επιστημονικού και
εμπορικού κόσμου συνεχίζει να αναζητεί τη χρησιμότητα του υπολογιστή
στον αυτοματισμό, όπου η διάδραση γίνεται αντιληπτή ως επικοινωνία με
έναν έξυπνο βοηθό.

\hypertarget{ux3b5ux3c1ux3b3ux3b1ux3bbux3b5ux3afux3bf}{%
\subsection{Εργαλείο}\label{ux3b5ux3c1ux3b3ux3b1ux3bbux3b5ux3afux3bf}}

\leavevmode\vadjust pre{\hypertarget{fig:superpaint-toolbox}{}}%
\begin{figure}
\hypertarget{fig:superpaint-toolbox}{%
\centering
\includegraphics{images/superpaint-toolbox.jpg}
\caption{Εικόνα 5: Ένα από τα πρώτα προγράμματα επεξεργασίας εικόνας
βασίζεται στην είσοδο και στην έξοδο βίντεο για την αποθήκευση και την
προβολή των εικόνων, οι οποίες ψηφιοποιούνται και μπορούν να
επεξεργαστούν με την παλέτα εργαλείων, η οποία εμφανίζεται σε μια
δεύτερη οθόνη. Η εφαρμογή των εργαλείων από την δεύτερη οθόνη στην
εικόνα γίνεται με έμμεση διάδραση από μια πένα σε μια ταμπλέτα. Η
τεχνολογία αυτή αναπτύχθηκε παράλληλα με τα γραφικά περιβάλλοντα και
οδήγησε σε εφαρμογές ψηφιακής τέχνης και ψηφιακής
κινηματογραφίας.}\label{fig:superpaint-toolbox}
}
\end{figure}

\leavevmode\vadjust pre{\hypertarget{fig:macpaint}{}}%
\begin{figure}
\hypertarget{fig:macpaint}{%
\centering
\includegraphics{images/macpaint.jpg}
\caption{Εικόνα 6: Η εφαρμογή ψηφιακής επεξεργασίας εικόνας MacPaint
χρησιμοποιήθηκε για να διαφημίσει τη φιλικότητα του Apple Macintosh,
καθώς και για να τονίσει τη νέα κατεύθυνση των υπολογιστών εκτός από τα
εργαλεία και ως μέσα δημιουργικής έκφρασης. Αν και δεν έγινε ευρέως
διαθέσιμη στους δημοφιλείς υπολογιστές εκείνης της εποχής, έλαβε άριστες
κριτικές και οδήγησε στη δημιουργία εφαρμογών όπως το
Photoshop.}\label{fig:macpaint}
}
\end{figure}

Αρχικά, τόσο οι πρώτοι κεντρικοί-υπολογιστές και οι μίκρο-υπολογιστές
όσο και ο επιτραπέζιος υπολογιστής θεωρήθηκαν εργαλεία για τη
διευκόλυνση των ανθρώπινων εργασιών. Για παράδειγμα, ο υπολογιστής
μπορούσε να βοηθήσει τον χρήστη στον υπολογισμό της τροχιάς ενός
διαστημοπλοίου, στη σύνταξη μιας γραπτής αναφοράς, στη σχεδίαση μιας
κατασκευής, \footnote{McCullough (1998)} στην επεξεργασία μιας εικόνας.
\footnote{(\textbf{Εικόνα?})~5 Superpaint (Web Archive)} \footnote{(\textbf{Εικόνα?})~6
  MacPaint (Computer History Museum)} Με τη διάδοση του δικτύου Internet
και την αύξηση της ισχύος (ταχύτητα, μνήμη, γραφικά) σε προσιτούς
οικονομικά υπολογιστές αλλά και σε νέες μορφές, π.χ. φορητός,
παιχνιδομηχανή, έξυπνο κινητό τηλέφωνο, κτλ. μια νέα γενιά εφαρμογών
διασκέδασης και επικοινωνίας ήρθε στο προσκήνιο, η οποία απαιτούσε μια
διαφορετική θεώρηση της διάδρασης πέρα από τη χρησιμότητα και την
ευχρηστία. Για παράδειγμα, ψυχαγωγικές εφαρμογές όπως τα βιντεοπαιχνίδια
έχουν σκοπό να διασκεδάσουν τον χρήστη και σε πολλές περιπτώσεις ο
στόχος είναι να δυσκολέψουν τον χρήστη παρά να τον διευκολύνουν, αφού
αυτή η προσέγγιση, στις σωστές δόσεις, θα ενισχύσει την εμβύθιση στην
ψυχαγωγική δραστηριότητα.

Η αρχική εφαρμογή της τεχνολογίας των υπολογιστών ήταν στην
αυτοματοποίηση της ανθρώπινης εργασίας υπολογισμού πινάκων για την
τροχιά πυραύλων, κάτι που άλλαξε με τον χρονοδιαμοιρασμό. Αυτή η
θεμελίωση της περιοχής δημιούργησε την αντίστοιχη κατεύθυνση και θεώρηση
των υπολογιστών ως μηχανές δεδομένων που βοηθάνε στη διαδικασία
παραγωγής λογαριθμικών πινάκων. Σύμφωνα με την αρχική θεώρηση, οι
υπολογιστές είναι μηχανές, σαν αυτές που υπάρχουν σε άλλες παραγωγικές
διαδικασίες. Οι μηχανές αυτές, αφού κατασκευαστούν πολύ προσεκτικά, στη
συνέχεια απαιτούν περιστασιακή μόνο παρέμβαση από τον άνθρωπο κυρίως σε
περίπτωση δυσλειτουργίας. Ο ρόλος μιας υπολογιστικής μηχανής είναι να
λαμβάνει μια είσοδο δεδομένων και να δίνει μια έξοδο δεδομένων, μέσα από
μια αυστηρά προγραμματισμένη διεργασία, σε αναλογία με τις βιομηχανικές
διαδικασίες που μετασχηματίζουν υλικά. Οι κεντρικοί πρωταγωνιστές αυτής
της θεώρησης είναι η IBM, καθώς και μια ομάδα ανθρώπων στο MIT μέχρι τη
δεκαετία του 1960. Η IBM θεωρεί τον υπολογισμό ως μια βιομηχανική
υπηρεσία που τιμολογείται με τον χρόνο εκτέλεσης ενός προγράμματος, το
οποίο λαμβάνει μια είσοδο δεδομένων και, μετά την εκτέλεση μιας δέσμης
προγραμμάτων, δίνει μια έξοδο δεδομένων χωρίς την ενδιάμεση παρέμβαση
κάποιου χειριστή. Σε σχέση με αυτό το μοντέλο λειτουργίας, η ιδέα ότι
ένας χρήστης ή ακόμη και ένα σύνολο χρηστών, αλληλεπιδρούν χωρίς να
υπάρχει ένα δεδομένο πλάνο, σε πραγματικό χρόνο, με ένα τόσο ακριβό
μηχάνημα, είναι πολύ μεγάλη σπατάλη. Πράγματι, ακόμη και οι ίδιοι οι
υποστηρικτές του χρονοδιαμοιρασμού βασίζονται στην παραδοχή ότι κατά
μέσο όρο κάθε χρονική στιγμή ελάχιστοι χρήστες θα κάνουν ενεργή χρήση
των υπολογιστικών πόρων.

Οι υποστηρικτές του χρονοδιαμοιρασμού, βασιζόμενοι στον νόμο του Moore,
ο οποίος προβλέπει τη μείωση του κόστους και, κυρίως, τον μετασχηματισμό
από τον υπολογισμό δεδομένων προς την κατεύθυνση της διάδρασης με την
πληροφορία για τις δουλειές του γραφείου, θεώρησαν τον υπολογιστή ως
εργαλείο στη διάθεση κάθε χρήστη. Η εμφάνιση του Sketchpad από τον Ivan
Sutherland και των πρώτων λειτουργικών συστημάτων χρονοδιαμοιρασμού,
όπως το CTSS κατά τη δεκαετία του 1960, έδωσε ώθηση στη θεώρηση των
υπολογιστών ως εργαλεία που επεκτείνουν τη δυνατότητα των ανθρώπων να
εργάζονται με την πληροφορία. Πράγματι, ένας υπολογιστής
χρονοδιαμοιρασμού επιτρέπει την διάδραση σε πραγματικό χρόνο με πολλούς
χρήστες, οι οποίοι ενδέχεται να εκτελούν διαφορετικά προγράμματα. Ο
J.C.R. Licklider ήταν ο βασικός ενορχηστρωτής αυτής της προσπάθειας
μετασχηματισμού των υπολογιστών από μηχανές παραγωγής σε διαδραστικά
εργαλεία. Για αυτόν τον σκοπό, χρηματοδότησε μια σειρά από έρευνες με
κεντρικό ρόλο τον χρονοδιαμοιρασμό και τη δικτύωση. Αυτά που μετά το
2000 θεωρούνται θεμελιώδη για τις περισσότερες εφαρμογές των
υπολογιστών, εκείνη την εποχή συνάντησαν όχι μόνο δυσκολίες αλλά και την
άρνηση ομάδων και οργανισμών. Ειδικά η IBM, χρειάστηκε να πιεστεί από
τους πελάτες και τους ανταγωνιστές της, όπως η DEC με τη δημοφιλή σειρά
PDP, για να προσφέρει και αυτή ένα σύστημα χρονοδιαμοιρασμού, αφού στο
επιχειρηματικό της μοντέλο δεν υπήρχε η έννοια του χρήστη που
αλληλεπιδρά σε πραγματικό χρόνο. Ακόμη και μετά από είκοσι χρόνια, στις
αρχές της δεκεατίας του 1980, όταν η ΙΒΜ αναγκάζεται, υπό την πίεση των
δημοφιλών μικρο-υπολογιστών και των νέων εφαρμογών γραφείου, όπως το
λογιστικό φύλλο εργασίας VisiCalc, να δημιουργήσει τον προσωπικό της
υπολογιστή, θα αναθέσει σε εξωτερικούς προμηθευτές την ανάπτυξη του
λογισμικού, αφού η διάδραση σε πραγματικό χρόνο είναι κάτι ξένο.

\leavevmode\vadjust pre{\hypertarget{fig:vi-editor}{}}%
\begin{figure}
\hypertarget{fig:vi-editor}{%
\centering
\includegraphics{images/vi-editor.png}
\caption{Εικόνα 7: Ο επεξεργαστής κειμένου vi δημιουργήθηκε για το
λειτουργικό σύστημα UNIX για τερματικά χρήστη που βασίζονται στην οθόνη
και το πληκτρολόγιο. Θεωρείται ο πρώτος δημοφιλής επεξεργαστής κειμένου
που έφυγε από τη διάδραση γραμμής και πέρασε στη διάδραση με την οθόνη
και, επειδή είναι σχετικά απλός, παρά την τροπική λειτουργία του,
χρησιμοποιείται ευρέως, κυρίως, από διαχειριστές
συστήματος.}\label{fig:vi-editor}
}
\end{figure}

\leavevmode\vadjust pre{\hypertarget{fig:unreal-blueprints}{}}%
\begin{figure}
\hypertarget{fig:unreal-blueprints}{%
\centering
\includegraphics{images/unreal-blueprints.png}
\caption{Εικόνα 8: Η κατασκευή της διάδρασης σε ένα βιντεοπαιχνίδι
απαιτεί πολλές διαφορετικές δεξιότητες, όπως κώδικα, γραφικά, ήχο,
κίνηση. Για τη διευκόλυνση της δημιουργικής διαδικασίας, οι σχεδιαστές
βασίζονται σε έτοιμα μοτίβα, τα οποία μπορούν να παραμετροποιήσουν και
να διασυνδέσουν σε διαγράμματα ροής.}\label{fig:unreal-blueprints}
}
\end{figure}

Οι άνθρωποι έμαθαν από τα πρώτα στάδια του επιτραπέζιου υπολογιστή με
γραφική διεπαφή τη δεκαετία του 1980 να τον βλέπουν ως εργαλείο, και
αυτή η αντίληψη παραμένει ισχυρή και σήμερα κατά τη χρήση του σε
εργασιακά περιβάλλοντα. Ως εξελιγμένο εργαλείο, ο ΗΥ με επεξεργαστή
κειμένου επιτρέπει στον χρήστη να προετοιμάσει ένα κείμενο σε ψηφιακή
μορφή, η οποία είναι πιο ευέλικτη στις αλλαγές, στην αποθήκευση και στη
μεταφορά σε σχέση με μια φυσική σελίδα που τυπώνεται στην παραδοσιακή
γραφομηχανή ή που γράφεται με ένα μολύβι. Φυσικά υπάρχουν και σύνθετες
περιπτώσεις, στις οποίες πολλά εργαλεία συμπληρώνουν το ένα το άλλο για
να επιτελέσουν μια πολύπλοκη δραστηριότητα. Για παράδειγμα, ο
προγραμματισμός υπολογιστών βασίζεται συνήθως στη χρήση ενός απλού
επεξεργαστή κειμένου και μιας σειράς από βοηθητικά προγράμματα
υπολογιστή, τα οποία υποστηρίζουν την προετοιμασία, τον έλεγχο και την
τελική εκτέλεση του νέου προγράμματος. Ο προγραμματιστής του υπολογιστή
αντιλαμβάνεται και χρησιμοποιεί τα παραπάνω ως εργαλεία για να κάνει τη
δουλειά του, που είναι η κατασκευή ή η επισκευή ενός προγράμματος στον
υπολογιστή. \footnote{(\textbf{Εικόνα?})~7 Επεξεργαστής κειμένου VI
  (Wikimedia)} \footnote{(\textbf{Εικόνα?})~8 UNREAL Blueprints (Epic
  Games)}

Η πρωταρχική και για πολλά χρόνια κυρίαρχη αντίληψη του υπολογιστή ως
εργαλείου μπορεί να εντοπιστεί στα πρώτα στάδια της δημιουργίας και
εξέλιξης της γραφικής επιτραπέζιας επιφάνειας εργασίας από την
ερευνητική ομάδα του Xerox PARC. Η ανθρωποκεντρική μελέτη των διεργασιών
του χρήστη βασίστηκε στην παρατήρηση, στις συνεντεύξεις, και στην
ανάλυση της εργασίας που πραγματοποιείται στους εκδοτικούς οργανισμούς,
οι οποίοι ήταν οι βασικοί πελάτες της εταιρείας Xerox κατά τη δεκαετία
του 1970. Οι διεργασίες που εκτελεί αυτή η πολύ καλά ορισμένη ομάδα
χρηστών π.χ. γλωσσική επιμέλεια, γραφικά, σελιδοποίηση, διαχείριση της
παραγωγής του εντύπου, αποτέλεσαν το σημείο αναφοράς για την κατασκευή
της διάδρασης, η οποία οργανώθηκε πάνω στη γραφική επιφάνεια εργασίας,
π.χ. φάκελοι, εργαλεία, καλάθι αχρήστων, κτλ. ως βασική μεταφορά της
εργασίας που κάνουν οι χρήστες. \footnote{(\textbf{Εικόνα?})~9
  Επεξεργαστής κειμένου Gypsy (Xerox PARC)} \footnote{(\textbf{Εικόνα?})~10
  Επιτραπέζιος Υπολογισμός (Xerox)} Αν και η επιφάνεια εργασίας
αποδείχθηκε στην πορεία εξαιρετικά ευέλικτη και προσαρμόσιμη στις
τεχνολογικές εξελίξεις, ταυτόχρονα, αυτό το αρχικό πλαίσιο δημιουργίας
της, ως εργαλείο για εκδοτικές εργασίες, παρέμεινε ο βασικός περιορισμός
που δημιουργεί την ανάγκη για μια διαφορετική αντίληψη της κατασκευής
της διάδρασης, τουλάχιστον σε διαφορετικά πλαίσια ανθρώπινης
δραστηριότητας, όπως η ψυχαγωγία.

Εκτός από τις διεργασίες της προετοιμασίας και της σελιδοποίησης μιας
γραπτής αναφοράς, οι οποίες γίνονται τόσο από ερασιτέχνες όσο και από
επαγγελματίες χρήστες ΗΥ, μια ακόμη δημοφιλής εφαρμογή της διάδρασης
είναι η διαδικασία της σχεδίασης που εφαρμόζουν οι μηχανικοί. Από τις
αρχές της δεκαετίας του 1980 όλες οι κατηγορίες μηχανικών άρχισαν να
χρησιμοποιούν τον υπολογιστή αντί του σχεδιαστηρίου για να ετοιμάσουν
κτίρια, δρόμους, γέφυρες, αεροσκάφη, μηχανές, αντικείμενα, και ηλεκτρικά
κυκλώματα. Στην πράξη, η σχεδίαση με υπολογιστή είναι πολύ πιο
αποδοτική, ειδικά για τη σχεδίαση αντικειμένων που είναι παρόμοια με
άλλα υπάρχοντα ή μοιράζονται κάποια κοινά μοτίβα. Επιπλέον, το βασικό
πλεονέκτημα είναι ότι η χρήση ΗΥ επιτρέπει τη γρήγορη αναίρεση μιας
αλλαγής και την ασφαλή εξερεύνηση εναλλακτικών κατευθύνσεων. Επίσης, ένα
ακόμη πλεονέκτημα της διαδραστικής σχεδίασης είναι ότι επιτρέπει την
εύκολη αποθήκευση και τον διαμοιρασμό, στοιχεία απαραίτητα στις
περισσότερες συνεργατικές δραστηριότητες σχεδίασης. Από την άλλη πλευρά,
από τη στιγμή που η διαδραστική σχεδίαση λειτουργεί μόνο στο στενό και
καλά ορισμένο πλαίσιο ενός προγράμματος διάδρασης στον υπολογιστή,
δυσκολεύει τη σχεδίαση αντικειμένων που δεν έχουν υπάρξει ακόμη.\\
Ως προς τον αυτοσχεδιασμό και τη φαντασία, η διαδραστική σχεδίαση είναι
ακόμα ένα περιοριστικό εργαλείο, αν τη συγκρίνει κανείς με το μολύβι και
το χαρτί στα χέρια ενός έμπειρου χρήστη.

\leavevmode\vadjust pre{\hypertarget{fig:xerox-gypsy}{}}%
\begin{figure}
\hypertarget{fig:xerox-gypsy}{%
\centering
\includegraphics{images/xerox-gypsy.png}
\caption{Εικόνα 9: Η εφαρμογή Gypsy ήταν ο πρώτος επεξεργαστής κειμένου
με μη-τροπικό χειρισμό και λειτουργούσε στον υπολογιστή Xerox Alto. Για
τον σκοπό αυτο το πληκτρολόγιο ήταν μόνο για την είσοδο χαρακτήρων, ενώ
το ποντίκι και τα πλήκτρα του έκαναν τις επιλογές των εντολών, όπως
αντιγραφή και επικόλληση.}\label{fig:xerox-gypsy}
}
\end{figure}

\leavevmode\vadjust pre{\hypertarget{fig:desktop}{}}%
\begin{figure}
\hypertarget{fig:desktop}{%
\centering
\includegraphics{images/desktop.jpg}
\caption{Εικόνα 10: Ο επιτραπέζιος υπολογιστής καθορίζει έμμεσα με τη
μορφή του ένα συγκεκριμένο πλαίσιο χρήσης και αντίστοιχα τις εφαρμογές
και τις διαδικασίες του χρήστη, οι οποίες, συνήθως, σχετίζονται με το
περιβάλλον του γραφείου και την αξία της
παραγωγικότητας.}\label{fig:desktop}
}
\end{figure}

Η εμπορική σημασία του υπολογιστή ως εργαλείο φαίνεται στην περίπτωση
του ηλεκτρονικού εμπορίου, το οποίο σε πολύ σύντομο χρονικό διάστημα
άλλαξε τον τρόπο πώλησης και διανομής πολλών κατηγοριών τυποποιημένων
προϊόντων και, κυρίως, υπηρεσιών. Η πανάρχαια συνήθεια της μετάβασης
στην αγορά για την προμήθεια προϊόντων και υπηρεσιών μετασχηματίζεται σε
μια διαδραστική εμπειρία στον υπολογιστή. Ο υπολογιστής ως εργαλείο
επιλογής προϊόντων και ως εργαλείο πληρωμής μειώνει τη σημασία του
τυπωμένου χρήματος, καθώς η συναλλαγή πραγματοποιείται λογιστικά ανάμεσα
στους λογαριασμούς του προμηθευτή και του πελάτη. Οι επιπτώσεις αυτές
είναι πολύ σημαντικές ειδικά στην περίπτωση που τα ίδια τα αγαθά είναι
ψηφιακά, όπως είναι η μουσική και οι ταινίες.

Η προτίμηση που δείχνουν οι χρήστες διαδραστικών συστημάτων στην
ηλεκτρονική μορφή αγοράς και χρηματικής συναλλαγής δημιουργεί μια σειρά
από πρωτοφανείς επιπτώσεις. Ειδικά στην περίπτωση της παροχής υπηρεσιών
χωρίς τη μεταβίβαση κάποιου απτού προϊόντος, π.χ. αγορά αεροπορικών
εισιτηρίων, κράτηση σε ξενοδοχείο, ψηφιακό ψυχαγωγικό περιεχόμενο, όπως
μουσική, ταινίες, παιχνίδια, το ηλεκτρονικό εμπόριο αποδείχτηκε πιο
αποτελεσματικό από το φυσικό, τόσο για τους χρήστες όσο και για τους
παρόχους της υπηρεσίας. Στις περιπτώσεις όπου η μετάβαση στο ηλεκτρονικό
εμπόριο υπήρξε καθολική, το αποτέλεσμα ήταν η καταστροφή πολλών
επιμέρους ενδιάμεσων, π.χ. δισκοπωλεία, τουριστικοί πράκτορες κ.ά.
Επομένως, βλέπουμε ότι η χρήση του υπολογιστή ως εργαλείου μπορεί να
είναι μια απλή βελτιστοποίηση μιας διεργασίας που ήδη κάνουμε, π.χ.
συγγραφή κειμένου, αλλά μπορεί και να έχει σημαντικές επιπτώσεις σε
επαγγέλματα και ανθρώπινες δραστηριότητες που παύουν να υπάρχουν,
επειδή, δεν χρειάζονται πλέον, αφού γίνονται πολύ πιο αποτελεσματικά με
τον προγραμματισμό της διάδρασης ανθρώπου και υπολογιστή, π.χ. αγορά
αεροπορικού εισιτηρίου.

Η αντίληψη του υπολογιστή ως εργαλείο παραμένει κυρίαρχη και στην εποχή
της μετάβασης από τον επιτραπέζιο στον κινητό και διάχυτο υπολογισμό.
Για παράδειγμα, ένα έξυπνο κινητό τηλέφωνο συνοδεύεται, συνήθως, από μια
ψηφιακή φωτογραφική μηχανή, δηλαδή από ένα εργαλείο αποτύπωσης εικόνων.
Οι ψηφιακές φωτογραφικές μηχανές επικράτησαν, επειδή, εκτός από την
οικειότητα που έχουμε αναπτύξει στη χρήση των βασικών λειτουργιών των
έξυπνων κινητών και την ευχέρεια του να τα έχουμε πάντα μαζί μας,
παρέχουν μια σειρά από επιπλέον λειτουργίες, χάρη στις δυνατότητες του
έξυπνου κινητού τηλεφώνου, π.χ. αισθητήρας θέσης, κίνησης, φίλτρα,
δικτύωση, κτλ. Αυτές οι δυνατότητες κάνουν την ενσωματωμένη φωτογραφική
μηχανή ακόμη πιο χρήσιμη και εύχρηστη, π.χ. για τη δημιουργία
πανοραμικής φωτογραφίας. Όπως με τα κινητά τηλέφωνα που έχουν και
φωτογραφική μηχανή, έτσι και οι φορετοί υπολογιστές και τα έξυπνα
ρολόγια που περιέχουν αισθητήρες κίνησης και βιομετρικών στοιχείων,
μετατρέπουν ένα εργαλείο που έδινε ρυθμό με βάση τον κοινώς αποδεκτό
χρόνο της Γης σε εργαλείο που μετράει περισσότερες παραμέτρους, π.χ.
ώρες ύπνου, βήματα, και μπορεί να δώσει ρυθμό με βάση το βιολογικό
σύστημα του κάθε ανθρώπου χωριστά. Συμπερασματικά, η αντίληψη του
υπολογιστή ως εργαλείου παραμένει δημοφιλής, ίσως περισσότερο από ποτέ,
καθώς η χρήση του διαχέεται σε όλο και πιο πολλές ανθρώπινες
δραστηριότητες. Ταυτόχρονα, ο υπολογιστής αποκτά νέους ρόλους. Πέρα από
εργαλείο, γίνεται και μέσο επικοινωνίας, και αυτός ο νέος ρόλος του ίσως
να είναι και ο πιο σημαντικός από την πλευρά του ατόμου ως μέρους μια
κοινότητας ανθρώπων.

\hypertarget{ux3bcux3adux3c3ux3bf-ux3b5ux3c0ux3b9ux3baux3bfux3b9ux3bdux3c9ux3bdux3afux3b1ux3c2}{%
\subsection{Μέσο
επικοινωνίας}\label{ux3bcux3adux3c3ux3bf-ux3b5ux3c0ux3b9ux3baux3bfux3b9ux3bdux3c9ux3bdux3afux3b1ux3c2}}

\leavevmode\vadjust pre{\hypertarget{fig:bbs}{}}%
\begin{figure}
\hypertarget{fig:bbs}{%
\centering
\includegraphics{images/bbs.png}
\caption{Εικόνα 11: Τα δημοφιλή συστήματα συνεργασίας των χρηστών μέσω
υπολογιστή είναι τόσο παλιά όσο και οι πρώτοι οικιακοί μίκρο-υπολογιστές
με δυνατότητα δικτύωσης μέσω τηλεφώνου, όπου η συνεργασία γινόταν με τα
BBS.}\label{fig:bbs}
}
\end{figure}

\leavevmode\vadjust pre{\hypertarget{fig:xerox-portholes}{}}%
\begin{figure}
\hypertarget{fig:xerox-portholes}{%
\centering
\includegraphics{images/xerox-portholes.png}
\caption{Εικόνα 12: Σε έναν οργανισμό που λειτουργεί κατανεμημένα σε
διαφορετικές γεωγραφικές περιοχές υπάρχει η ανάγκη για μια ήπια επαφή
μεταξύ των συνεργατών, αντίστοιχη με αυτή που υπάρχει σε έναν μεγάλο
ενιαίο χώρο εργασίας. Η ανάγκη αυτή μελετήθηκε με το σύστημα επίγνωσης
πλαισίου απομακρυσμένου συνεργάτη Portholes του Xerox PARC, το οποίο
μοιράζει ασύγχρονα στιγμιότυπα από κάμερες που βρίσκονται στα γραφεία,
έτσι ώστε να υπάρχει κατανόηση της δραστηριότητας σε διαφορετικές
τοποθεσίες.}\label{fig:xerox-portholes}
}
\end{figure}

Οι εφαρμογές επικοινωνίας και συνεργασίας μέσω υπολογιστή είναι μια
μεγάλη περιοχή, την οποία ερευνά το πεδίο των κοινωνικών και
συνεργατικών συστημάτων. Ειδικά οι εφαρμογές επικοινωνίας μέσω
υπολογιστή απέκτησαν μεγάλη σημασία τόσο στην εργασία, π.χ. email όσο
και στην καθημερινή ζωή και τη διασκέδαση, π.χ. τόποι δημόσιων
συζητήσεων, δικτυακά παιχνίδια ρόλων. Αυτό το βήμα συνοδεύτηκε από την
ανάγκη για μια νέα θεώρηση της φιλοσοφίας της διάδρασης ως μέσου
επικοινωνίας. \footnote{Rheingold (2000)} Επιπλέον, η θεώρηση της
διάδρασης ως μέσου επικοινωνίας δίνει έμφαση στη συμμετοχή των χρηστών
στην παραγωγή περιεχομένου και υπηρεσιών, τα οποία κατά τη δεκαετία του
2000 έγιναν ο κυρίαρχος τρόπος διάδρασης με τους υπολογιστές, ειδικά στο
διαδίκτυο. Η σημασία των υπολογιστών ως μέσου επικοινωνίας γίνεται ακόμη
πιο εμφανής κατά τη δεκαετία του 2010, όταν το κοινωνικό δίκτυο
Facebook, καθώς και τα έξυπνα κινητά τηλέφωνα γίνονται το πρώτο, και
πολλές φορές το μόνο, σημείο επαφής των περισσότερων χρηστών με τους
υπολογιστές. \footnote{(\textbf{Εικόνα?})~11 Bulletin Board System
  (blogtoexpress)} \footnote{(\textbf{Εικόνα?})~12 Επίγνωση κοινωνικού
  πλαισίου με το Portholes (Xerox PARC)}

Η ανθρωπιστική θεώρηση της διάδρασης δεν έχει καθιερωθεί, αλλά έχει
υποδεχτεί στην ομάδα της πολλές συμπληρωματικές συνεισφορές, οι οποίες
την έχουν επεκτείνει προς την κοινωνική διάδραση. Μια από τις σημαντικές
θεωρήσεις της διάδρασης εμπνέεται από το θέατρο, όπου οι άνθρωποι
παίζουν ρόλους και συνεργάζονται. Με αυτόν τον τρόπο, ένα σύστημα
διάδρασης μπορεί να θεωρηθεί ως μια συνεργατική αναπαράσταση.
\footnote{Laurel (2013)} Και σε αυτήν τη περίπτωση, όπως στην απλή
διάδραση, η έμφαση βρίσκεται περισσότερο στην επαύξηση της συλλογικής
νοημοσύνης, παρά στην προσομοίωση της φυσικής συνεργασίας. Στην πράξη,
όμως, φαίνεται ότι έχει επικρατήσει η αντίληψη πως οι άνθρωποι πρέπει να
επικοινωνούν και να συνεργάζονται μέσω των υπολογιστών, όπως το κάνουν
και στην πραγματικότητα. Το αποτέλεσμα είναι πως οι αντίστοιχες
εφαρμογές συνεργασίας εστιάζουν μόνο στη μεταφορά του προσώπου και της
φωνής, όπως ακριβώς η επεξεργασία ενός εγγράφου εστιάζει μόνο στο πώς θα
φαίνεται όταν θα τυπωθεί στο φυσικό χαρτί. Μια περισσότερο ανθρωπιστική
σχεδίαση της επικοινωνίας θα μπορούσε να προσθέτει νέες δυνατότητες
αντίληψης, αντί για την ψηφιακή προσομοίωση των φυσικών καναλιών
επικοινωνίας.

\leavevmode\vadjust pre{\hypertarget{fig:videoplace}{}}%
\begin{figure}
\hypertarget{fig:videoplace}{%
\centering
\includegraphics{images/videoplace.png}
\caption{Εικόνα 13: Το ερευνητικό σύστημα Videoplace του Myron Krueger
τοποθετεί τη σιλουέτα του χρήστη μέσα σε ένα συμμετοχικό διαδραστικό
περιβάλλον, όπου δεν υπάρχει διάκριση ανάμεσα σε χρήστες και γραφικά, με
τη χρήση κάμερας σε πραγματικό χρόνο και χωρίς να μεσολαβεί κάποια
συσκευή έμμεσης διάδρασης. Σε αντιδιαστολή με τα πρώτα συστήματα
εικονικής πραγματικότητας εκείνης της περιόδου, αυτό το σύστημα
χαρακτηρίζεται από τον δημιουργό του ως τεχνητή πραγματικότητα, η οποία
αρχικά υλοποιήθηκε με αναλογικά φίλτρα και κινηματογραφικές τεχνικές
πριν περάσει σε τεχνολογίες υπολογιστικής όρασης.}\label{fig:videoplace}
}
\end{figure}

\leavevmode\vadjust pre{\hypertarget{fig:media-space}{}}%
\begin{figure}
\hypertarget{fig:media-space}{%
\centering
\includegraphics{images/media-space.jpg}
\caption{Εικόνα 14: Το ερευνητικό πρόγραμμα Media Space ήταν μια από τις
πρώτες προσπάθειες ανθρώπινης συνεργασίας από απόσταση μέσω υπολογιστή,
το οποίο χρησιμοποιούσε ζωντανή εικόνα βίντεο.}\label{fig:media-space}
}
\end{figure}

Μια από τις πιο δημοφιλείς εφαρμογές των πρώτων συστημάτων
χρονοδιαμοιρασμού ήταν η ηλεκτρονική αλληλογραφία. Οι περισσότεροι
χρήστες ήταν συνδεδεμένοι ταυτόχρονα ή με μικρή χρονική διαφορά σε
τηλέτυπους, με τους οποίους μπορούσαν να επικοινωνούν ασύγχρονα, αλλά
πυκνά. Αν και αυτοί οι χρήστες ήταν, συνήθως, σε σχετικά κοντινή
γεωγραφική απόσταση, επέλεγαν να επικοινωνήσουν με αυτόν τον τρόπο,
γιατί αυτός ο τρόπος είχε το πλεονέκτημα της ασύγχρονης επικοινωνίας και
όχι γιατί ήταν μια ανάγκη που δεν είχε εναλλακτική. Η ηλεκτρονική
αλληλογραφία έγινε ακόμη πιο δημοφιλής στα τέλη της δεκαετίας του 1960,
όταν το ARPANET άρχισε να συνδέει τους λίγους διαθέσιμους υπολογιστές
μεταξύ τους. Η διάδραση με τον υπολογιστή αποκτάει έναν νέο χαρακτήρα, ο
οποίος εκτός από εργαλείο γίνεται και μέσο επικοινωνίας. \footnote{(\textbf{Εικόνα?})~14
  Xerox PARC Media Spaces (Xerox PARC)} Ο τεχνολογικός μετασχηματισμός
προς τη δικτύωση και τη διάδραση σε μεγαλύτερη κλίμακα δεν ήταν
προτεραιότητα για όλους. Για παράδειγμα, η ομάδα της τεχνητής νοημοσύνης
στο ΜΙΤ δεν έβλεπε τη σημασία της ευρύτερης δικτύωσης, αφού ο σκοπός της
δεν ήταν να μιλάνε οι άνθρωποι μεταξύ τους, αλλά με έξυπνες μηχανές. Από
την άλλη πλευρά, η ίδια ομάδα δημιούργησε την ιδέα του ανοιχτού
διαμοιρασμού, κυρίως ως αντίδραση στο σχετικά πολύπλοκο, λόγω ασφάλειας
MULTICS. Στα τέλη της δεκαετίας του 1960, δημιούργησαν ένα απλό σύστημα
χρονοδιαμοιρασμού, το ITS, το οποίο αρχικά δεν έκανε έλεγχο πρόσβασης
χρήστη και αποτέλεσε τη φιλοσοφική πλατφόρμα για τις μελλοντικές
κοινότητες ανοιχτού λογισμικού. \footnote{Markoff (2005)}

Οι καλλιτέχνες ήταν από τους πρώτους που χρησιμοποίησαν τον υπολογιστή
ως εργαλείο για να δημιουργήσουν. Όμως η πιο σημαντική συνεισφορά τους
στην εξέλιξη της χρήσης του υπολογιστή ήταν ότι τον χρησιμοποίησαν ως
μέσο για το έργο τους. Για παράδειγμα, η Brenda Laurel, με υπόβαθρο στις
θεατρικές σπουδές, αναγνωρίζει τη δυνατότητα των υπολογιστών να
λειτουργούν ως σκηνή, στην οποία οι χρήστες μπορούν να παίζουν ρόλους
και να αυτοσχεδιάζουν. Αυτή η αντίληψη έρχεται να συμπληρώσει την
αντίληψη του υπολογιστή ως εργαλείου που σχεδιάζεται με στόχο να
ικανοποιήσει καλά ορισμένες ανθρώπινες ανάγκες και δραστηριότητες.
Επίσης, ο Myron Krueger κατασκεύασε για πρώτη φορά, στα μέσα της
δεκαετίας του 1970, τη διάδραση ανθρώπου και υπολογιστή με έμφαση στην
παρουσία του ανθρώπινου σώματος. Με αυτόν τον τρόπο, η κατανόηση της
διάδρασης με τον υπολογιστή απελευθερώνεται από τη μέχρι τότε αντίληψή
της ως εργαλείο και αρχίζει να γίνεται αντιληπτή και ως μέσο
επικοινωνίας, το οποίο συνδέει τους ανθρώπους είτε άμεσα μεταξύ τους
είτε έμμεσα από την πληροφορία που μπορούμε συλλογικά να αναπαραστήσουμε
στον υπολογιστή. Για παράδειγμα, το σύστημα Videoplace \footnote{(\textbf{Εικόνα?})~13
  Τεχνητή Πραγματικότητα Videoplace (Myron Krueger)} χρησιμοποιεί
κάμερες και προβολείς, έτσι ώστε να μεταφέρει ολόκληρο το ανθρώπινο σώμα
μέσα σε ένα περιβάλλον τεχνητής πραγματικότητας. \footnote{Krueger
  (1991)}

\leavevmode\vadjust pre{\hypertarget{fig:skype-video-call}{}}%
\begin{figure}
\hypertarget{fig:skype-video-call}{%
\centering
\includegraphics{images/skype-video-call.png}
\caption{Εικόνα 15: Τα συστήματα επικοινωνίας και τηλεδιάσκεψης μέσω
υπολογιστή (π.χ. Skype, Zoom) συνιστούν μια διαφορετική θεώρηση του
υπολογιστή σε σχέση με την κλασική θεώρηση του εργαλείου. Οι χρήστες
επικοινωνούν μέσω κειμένου, ήχου, βίντεο και οι εφαρμογές αφορούν την
εργασία, την καθημερινότητα, τη διασκέδαση και την
εκπαίδευση.}\label{fig:skype-video-call}
}
\end{figure}

\leavevmode\vadjust pre{\hypertarget{fig:second-life}{}}%
\begin{figure}
\hypertarget{fig:second-life}{%
\centering
\includegraphics{images/second-life.jpg}
\caption{Εικόνα 16: Οι φωτορεαλιστικοί εικονικοί κόσμοι έγιναν
δημοφιλείς αρχικά σε εφαρμογές ψυχαγωγίας και πολιτισμού και στη
συνέχεια επεκτάθηκαν σε γενικές εφαρμογές επικοινωνίας, όπως το Second
Life.}\label{fig:second-life}
}
\end{figure}

Η αρχική αντίληψη που έχουν οι περισσότεροι άνθρωποι για τη διάδραση και
η οποία στη συνέχεια καθορίζει έμμεσα μια σειρά από άλλες αντιλήψεις και
σχεδιαστικές επιλογές είναι αυτή της \emph{ένας προς έναν} επικοινωνίας
ανάμεσα σε άνθρωπο και υπολογιστή. Αν και η ένας προς έναν διάδραση
ανθρώπου και υπολογιστή αποτελεί βασικό συστατικό κάθε σύνθετης μορφής
διάδρασης, δεν βρίσκεται πάντα στο επίκεντρο του ενδιαφέροντος. Στον
επιτραπέζιο υπολογιστή, όπου έχουμε ένα πληκτρολόγιο και ένα ποντίκι, η
διάδραση ανάμεσα στον άνθρωπο και τον υπολογιστή είναι ένας προς έναν.
Σε αυτό το πλαίσιο έχει δημιουργηθεί και έχει ωριμάσει η περιοχή της
διάδρασης ανθρώπου και υπολογιστή. Καθώς, όμως, οι υπολογιστές άρχισαν
να δικτυώνονται, είτε στον ίδιο χώρο είτε σε μεγάλες γεωγραφικές
αποστάσεις και διαφορετικές χρονικές ζώνες, δημιουργήθηκε η ανάγκη να
μελετήσουμε και να κατασκευάσουμε τη διάδραση που συμβαίνει ανάμεσα σε
ανθρώπους οι οποίοι συνεργάζονται από απόσταση και σε διάφορους τομείς,
όπως στην εργασία, τη διασκέδαση, την εκπαίδευση και την καθημερινή ζωή.
\footnote{(\textbf{Εικόνα?})~15 Βιντεοδιάσκεψη με το Skype (Microsoft)}
\footnote{(\textbf{Εικόνα?})~16 Second Life (Linden Research)}

Τα Συνεργατικά Συστήματα έγιναν γνωστά ως επιστημονική περιοχή στα τέλη
της δεκαετίας του 1980, αλλά η συνεργασία και η επικοινωνία μέσω
υπολογιστή είχε ήδη ξεκινήσει. Στα τέλη της δεκαετίας του 1970, η μεγάλη
διάδοση των πρώτων οικονομικών προσωπικών υπολογιστών και των μόντεμ που
επέτρεπαν την ψηφιακή μετάδοση δεδομένων μέσα από τις αναλογικές
τηλεφωνικές γραμμές, δημιούργησε τα πρώτα Bulletin Board Systems (BBS).
Τα BBS ήταν ένα είδος δημόσιας συζήτησης, όπου οι χρήστες μπορούσαν να
έχουν μια σύγχρονη ή ασύγχρονη συζήτηση, να αναρτήσουν ανακοινώσεις, και
να μοιραστούν αρχεία. Την ίδια περίοδο, αντίστοιχα συστήματα
συνεργατικού λογισμικού, βασισμένα σε μικρο-υπολογιστές και κεντρικούς
υπολογιστές αναπτύχθηκαν για εταιρείες με σκοπό τη διευκόλυνση της
επαγγελματικής συνεργασίας.

Η ερευνητική περιοχή της Συνεργασίας Ανθρώπων μέσω Υπολογιστών θεωρείται
η πρώτη και ίσως η πιο σημαντική συγγενής περιοχή της Διάδρασης ανθρώπου
και υπολογιστή. Επομένως, ένα μεγάλο μέρος της θεωρίας, των μεθόδων και
των τεχνικών που χρησιμοποιούν είναι κοινές ή έστω παρόμοιες. Υπάρχει,
όμως, τουλάχιστον μία σημαντική διαφορά που επιβάλλει την αντιμετώπιση
των Κοινωνικών και Συνεργατικών Συστημάτων ως ξεχωριστή περιοχή. Ενώ στη
διάδραση ανθρώπου και υπολογιστή εστιάζουμε, συνήθως, την προσοχή μας
κατά τον προγραμματισμό και την αξιολόγηση του συστήματος στη διάδραση
ανάμεσα σε έναν άνθρωπο και έναν υπολογιστή, στα Κοινωνικά και
Συνεργατικά Συστήματα η βασική μονάδα ανάλυσης είναι η διάδραση ανάμεσα
σε μια ομάδα τουλάχιστον δύο ανθρώπων, που συμβαίνει μέσω ενός
τουλάχιστον υπολογιστή. Το πιο απλό και δημοφιλές παράδειγμα από την
περιοχή των Συνεργατικών Συστημάτων είναι το Facebook, ένα σύστημα που
επιτρέπει την επικοινωνία και τη συνεργασία ομάδων ανθρώπων.

Η σημασία των Κοινωνικών και Συνεργατικών συστημάτων πέρασε από τις
εταιρείες στην ευρύτερη κοινωνία, και από την εργασία στην
καθημερινότητα, με την ανάπτυξη των κοινωνικών δικτύων μετά τα μισά της
δεκαετίας του 2000. Εκείνη την περίοδο, δημιουργήθηκαν πολλά κοινωνικά
δίκτυα, όπου ο κάθε χρήστης περιέγραφε την προσωπικότητα και τις
προτιμήσεις του και έκανε εικονικούς δεσμούς φιλίας με άλλους χρήστες.
Ανάμεσα στα πολλά κοινωνικά δίκτυα, το Facebook ήταν εκείνο που γνώρισε
τη μεγαλύτερη αποδοχή, με περισσότερο από ένα δισεκατομμύριο
εγγεγραμμένους χρήστες στις αρχές της δεκαετίας του 2010. Οι βασικές
δραστηριότητες των χρηστών του Facebook είναι η ενημέρωση της τρέχουσας
κατάστασής τους, η δημόσια ή ιδιωτική συζήτηση, και η ανάρτηση
φωτογραφιών. Καθώς ο κύριος όγκος της δραστηριότητας στο διαδίκτυο
μετατοπίστηκε από την υπηρεσία του ιστού και την ανάρτηση ιστοσελίδων,
στα δίκτυα κοινωνικής δικτύωσης, αντίστοιχα παρατηρούμε και μια μεταβολή
στο είδος της διάδρασης των νέων εφαρμογών, οι οποίες επιβάλλεται να
έχουν πλέον και μια κοινωνική διάσταση.

\hypertarget{ux3b7-ux3c0ux3b5ux3c1ux3afux3c0ux3c4ux3c9ux3c3ux3b7-ux3c4ux3bfux3c5-facebook}{%
\subsection{Η περίπτωση του
Facebook}\label{ux3b7-ux3c0ux3b5ux3c1ux3afux3c0ux3c4ux3c9ux3c3ux3b7-ux3c4ux3bfux3c5-facebook}}

\leavevmode\vadjust pre{\hypertarget{fig:facebook1}{}}%
\begin{figure}
\hypertarget{fig:facebook1}{%
\centering
\includegraphics{images/facebook1.jpg}
\caption{Εικόνα 17: Μία από τις αρχικές εκδοχές του κοινωνικού δικτύου
Facebook δεν διέφερε πολύ από ένα απλό σύστημα βάσης δεδομένων με προφίλ
χρηστών.}\label{fig:facebook1}
}
\end{figure}

\leavevmode\vadjust pre{\hypertarget{fig:lifestreams}{}}%
\begin{figure}
\hypertarget{fig:lifestreams}{%
\centering
\includegraphics{images/lifestreams.jpg}
\caption{Εικόνα 18: Η οργάνωση και η αλληλεπίδραση με την πληροφορία σε
χρονολογική σειρά παρουσιάστηκε σε μια διεπαφή με τίτλο Lifestreams
περισσότερο από μια δεκαετία πριν γίνει ο βασικός τρόπος διάδρασης στα
ψηφιακά κοινωνικά μέσα, όπως τα Twitter,
Facebook.}\label{fig:lifestreams}
}
\end{figure}

Η ιστορία του κοινωνικού δικτύου Facebook έχει πολλές ενδιαφέρουσες
πλευρές, κάποιες από αυτές αξίζει να ανακτήσει κανείς συνοπτικά και
ευχάριστα στη σχετική ταινία \emph{The Social Network} (Fincher, 2010).
Σε αυτήν την ενότητα θα εστιάσουμε στα στοιχεία του κοινωνικού δικτύου
που άπτονται του προγραμματισμού της διάδρασης. Δεν υπάρχει καλύτερο
σημείο για να ξεκινήσουμε αυτήν την περιγραφή από τη διαδικασία και τα
στάδια ανάπτυξης του κοινωνικού δικτύου Facebook. Τόσο οι λειτουργίες
όσο και η εμφάνιση του συστήματος δεν προέκυψαν από ένα λεπτομερές
συμβόλαιο ή ένα ξεκάθαρο όραμα, αλλά ήταν, κυρίως, το αποτέλεσμα
συνεχούς επανάληψης και δοκιμών. Μάλιστα, δεν ήταν λίγες οι φορές στην
ιστορία του Facebook που πολλοί χρήστες ενοχλήθηκαν ή αντέδρασαν, ακόμα
και κλείνοντας τον λογαριασμό τους, όταν έγιναν αλλαγές που δεν τους
άρεσαν. Ανεξάρτητα από τη μελλοντική αποδοχή του φαινομένου του
Facebook, σίγουρα η επίδρασή του στον τρόπο που αναπτύσσονται και
λειτουργούν οι διαδραστικές εφαρμογές θα είναι διαχρονική, καθώς μια
σειρά από εφαρμογές σε διαφορετικά πεδία (π.χ. Twitter, LinkedIn,
ResearchGate, GitHub, Uber, AirBnB κτλ.) στηρίζονται στις βασικές
λειτουργίες του κοινωνικού δικτύου.

Στα πρώιμα στάδια του Facebook, η ομάδα ανάπτυξης ήταν δύο συγκάτοικοι
στη φοιτητική εστία του πανεπιστήμιου, οι οποίοι ξεκίνησαν να φτιάξουν
μια σελίδα με τις φωτογραφίες των φοιτητών από όλες τις επιμέρους εστίες
του πανεπιστημίου. Οι φωτογραφίες και τα βασικά προφίλ ήταν διαθέσιμα
στις επιμέρους εστίες, αλλά το πανεπιστήμιο δεν πρόσφερε ένα
ολοκληρωμένο σύστημα παρουσίασής τους. Η ομάδα ανάπτυξης ανέκτησε τις
φωτογραφίες από τις επιμέρους εστίες και τις πρόβαλε συγκριτικά, με έναν
πολύ δημοφιλή τρόπο, ο οποίος, όμως, θεωρήθηκε παράνομος από τις
πανεπιστημιακές αρχές. Αν και η αρχική προσπάθεια έπαψε να λειτουργεί
μέσα σε λίγες μέρες, υπήρξε τόσο δημοφιλής που η επόμενη έκδοση -που
βασίστηκε στην εθελοντική συμμετοχή- ετοιμάστηκε και συγκέντρωσε χρήστες
από όλο το πανεπιστήμιο πολύ γρήγορα.\footnote{(\textbf{Εικόνα?})~17
  Αρχική έκδοση του thefacebook (Facebook)} Η έκδοση εκείνη δεν πρόσφερε
τίποτα διαφορετικό, μάλλον είχε λιγότερα στοιχεία, από τα αντίστοιχα
κοινωνικά δίκτυα της εποχής (π.χ. το Friendster). Είχε, όμως, ένα βασικό
πλεονέκτημα που θα έπαιζε σπουδαίο ρόλο στη συνέχεια: η υπηρεσία του
Facebook είχε μεγάλη συμμετοχή από μια συμπαγή και ομοιόμορφη ομάδα
χρηστών, της οποίας η ομάδα ανάπτυξης παρακολουθούσε κάθε διάδραση, και
έτσι μπόρεσε να προσφέρει νέες λειτουργίες που να καλύπτουν τις ανάγκες
των χρηστών της. Η πιο σημαντική λειτουργία είναι η διεπαφή χρονολογικής
εμφάνισης των αναρτήσεων, \footnote{(\textbf{Εικόνα?})~18 Lifestreams
  (Yale)} η οποία αποτελεί και τη μεγαλύτερη μέχρι τότε απομάκρυνση από
το ιεραρχικό σύστημα αρχείων.

Βλέπουμε, λοιπόν, ότι η εμπορική επιτυχία του κοινωνικού δικτύου μπορεί
να εξηγηθεί περισσότερο από την ανθρωποκεντρική του διάσταση, κυρίως
λόγω της συμμετοχής των τελικών χρηστών στην παραγωγή του περιεχομένου,
παρά από την προσφορά νέων καινοτόμων λειτουργιών, τις οποίες δεν έχουν
οι ανταγωνιστές. Το κοινωνικό δίκτυο Facebook πολύ γρήγορα πέρασε από το
πανεπιστημιακό ίδρυμα των δημιουργών του σε όλα τα πανεπιστημιακά
ιδρύματα των ΗΠΑ και από εκεί έγινε διαθέσιμο στον υπόλοιπο κόσμο με την
προσθήκη λειτουργιών και αλλαγές αντίστοιχων της κάθε περιόδου
λειτουργίας του. Η σταδιακή αυτή βελτίωση επέτρεψε στους κατασκευαστές
να παράσχουν στους χρήστες τα στοιχεία της διάδρασης που είχαν ανάγκη
κάθε φορά, και όχι κάτι που είχαν από μόνοι τους προαποφασίσει στο
σχεδιαστικό τραπέζι. Για παράδειγμα, η γρήγορη και μεγάλη αποδοχή του
συστήματος οδήγησε στη βελτιστοποίηση της απόκρισης των σελίδων, ενώ η
μετάβαση των χρηστών στα κινητά μέσα επέβαλε την εξαγορά εταιρειών με
σχετική εξειδίκευση (π.χ. Ιnstagram) και την κατασκευή της κινητής
εφαρμογής του Facebook, ώστε να έχουν πρόσβαση σε αυτό όσο γίνεται
περισσότεροι χρήστες και από όσο γίνεται περισσότερες συσκευές.

Από την πλευρά της σχεδίασης της διάδρασης, το κοινωνικό δίκτυο Facebook
καθιέρωσε μια σειρά από νέα μοτίβα, τα οποία έχουν ανάλογη αποδοχή με τα
μοτίβα της γραφικής επιφάνειας εργασίας. Ανεξάρτητα από το αν μια
εφαρμογή είναι κοινωνικό δίκτυο ή όχι, πολλές φορές δίνει τη δυνατότητα
στους χρήστες της να μοιραστούν απευθείας το περιεχόμενό τους (share) ή
τη δραστηριότητά τους στα κοινωνικά δίκτυα. Επίσης, δίνει τη δυνατότητα
να σχολιάσουν (comment) και να δηλώσουν την αποδοχή τους (like) για το
περιεχόμενο που εμφανίζεται. Επιπλέον, η μαζική χρήση του κοινωνικού
δικτύου και για πολλές διαφορετικές δραστηριότητες, έχει επιτρέψει και
την εκτέλεση μαζικών πειραμάτων στο πεδίο, ακόμα και ερήμην των χρηστών
του, στα οποία μελετήθηκε αν η εμφάνιση διαφορετικού περιεχομένου (π.χ.
θετικό, αρνητικό περιεχόμενο) μπορεί να επηρεάσει αντίστοιχα τη
συμπεριφορά των χρηστών.

Από την πλευρά της κατασκευής και της επιλογής των τεχνολογικών
εργαλείων της διάδρασης, το Facebook έχει υλοποιηθεί αρχικά και κατά ένα
μεγάλο μέρος του στην PHP, παρότι υπάρχουν πολλές άλλες επιλογές που
θεωρούνται περισσότερο κατάλληλες. Στην πράξη, οι κατασκευαστές του
Facebook διάλεξαν ένα οικείο εργαλείο και όταν βρέθηκαν μπροστά σε
θέματα απόδοσης, αντί να ξαναγράψουν το σύστημα, επέλεξαν να φτιάξουν
έναν μεταφραστή (compiler) που μετατρέπει την PHP σε C++, η οποία μπορεί
να εκτελεστεί γρηγορότερα. Τέλος, εκτός από την επίδραση που είχε το
Facebook στον τρόπο που αναπτύσσονται όλες οι διαδραστικές εφαρμογές,
είχε επιπλέον σημαντική επίδραση στον τρόπο που ορίζουμε το διαδίκτυο,
καθώς μειώθηκε η σημασία του παγκόσμιου ιστού.

\hypertarget{ux3b7-ux3c0ux3b5ux3c1ux3afux3c0ux3c4ux3c9ux3c3ux3b7-ux3c4ux3b7ux3c2-ux3c3ux3c5ux3bdux3b1ux3b9ux3c3ux3b8ux3b7ux3bcux3b1ux3c4ux3b9ux3baux3aeux3c2-ux3b5ux3c5ux3c7ux3c1ux3b7ux3c3ux3c4ux3afux3b1ux3c2}{%
\subsection{Η περίπτωση της συναισθηματικής
ευχρηστίας}\label{ux3b7-ux3c0ux3b5ux3c1ux3afux3c0ux3c4ux3c9ux3c3ux3b7-ux3c4ux3b7ux3c2-ux3c3ux3c5ux3bdux3b1ux3b9ux3c3ux3b8ux3b7ux3bcux3b1ux3c4ux3b9ux3baux3aeux3c2-ux3b5ux3c5ux3c7ux3c1ux3b7ux3c3ux3c4ux3afux3b1ux3c2}}

\leavevmode\vadjust pre{\hypertarget{fig:atm-affective}{}}%
\begin{figure}
\hypertarget{fig:atm-affective}{%
\centering
\includegraphics{images/atm-affective.png}
\caption{Εικόνα 19: Τα πειράματα με εναλλακτικές διαρρυθμίσεις των
πλήκτρων σε ένα απλό τραπεζικό ATM έδειξαν ότι ακόμη και αν δεν υπάρχει
ουσιαστική διαφορά στην πραγματική απόδοση της διάδρασης, οι χρήστες
αντιλαμβάνονται κάποια διαφορά, η οποία μπορεί να ερμηνευθεί από τη
σκοπιά της αισθητικής.}\label{fig:atm-affective}
}
\end{figure}

\leavevmode\vadjust pre{\hypertarget{fig:winamp}{}}%
\begin{figure}
\hypertarget{fig:winamp}{%
\centering
\includegraphics{images/winamp.jpg}
\caption{Εικόνα 20: Το Winamp επιτρέπει στους χρήστες του να
χρησιμοποιήσουν εναλλακτικές εμφανίσεις (skins) και επιπλέον, τους
παρέχει ένα καλά τεκμηριωμένο τρόπο για να φτιάξουν τις δικές τους, με
αποτέλεσμα να έχει δημιουργηθεί μια πολύ μεγάλη συλλογή από εμφανίσεις,
με τις περισσότερες από αυτές να δίνουν έμφαση στη μορφή και να
υιοθετούν ένα σκευομορφικό στυλ διάδρασης.}\label{fig:winamp}
}
\end{figure}

Η αρχική θεωρητική θεμελίωση της περιοχής της διάδρασης ανθρώπου και
υπολογιστή βασίστηκε στην τρέχουσα για την εποχή αντίληψη της διάδρασης,
η οποία με τη σειρά της βασιζόταν στον επιτραπέζιο υπολογιστή, στη
γραφική επιφάνεια εργασίας και στο πλαίσιο των εφαρμογών γραφείου και
παραγωγικότητας. Αντίστοιχα, η αρχική θεώρηση της διάδρασης έδωσε έμφαση
στην επεξεργασία της πληροφορίας από τον άνθρωπο, καθώς και στον έλεγχο
και στην απεικόνιση της πληροφορίας στο τερματικό του χρήστη. Οι
θεμελιωτές της αρχικής θεωρίας προσπάθησαν να δημιουργήσουν ένα
αναλυτικό πλαίσιο και τα αντίστοιχα μοντέλα, που να εκφράζουν τον τρόπο
που ο άνθρωπος αντιλαμβάνεται, επεξεργάζεται και ελέγχει την πληροφορία.
Σε ορισμένες περιπτώσεις, όπως στην πρόβλεψη της απόδοσης της συσκευής
εισόδου ποντίκι, η αναλυτική αυτή προσέγγιση είχε άριστα και διαχρονικά
αποτελέσματα.

Από την άλλη πλευρά όμως, καθώς περάσαμε στον κινητό και διάχυτο
υπολογιστή, και καθώς οι υπολογιστές έγιναν μέρος δραστηριοτήτων με
αξίες πέρα από την παραγωγικότητα, η ποιότητα της διάδρασης άρχισε να
αποκτά και άλλες διαστάσεις πέρα από εκείνη της βασικής ευχρηστίας. Στα
τέλη της δεκαετίας του 1990, μια σειρά από πειράματα διάδρασης με
διεπαφές που είχαν μεταξύ τους διαφορές μόνο στην αισθητική (και όχι
στην ευχρηστία τους) οδήγησε στο συμπέρασμα ότι η ευχρηστία δεν είναι ο
μόνος παράγοντας που καθορίζει την αποδοχή μιας διεπαφής.\footnote{Norman
  (2004)} Τα πειράματα έγιναν με πολλές διαφορετικές διεπαφές, με
τραπεζικά ATM,\footnote{(\textbf{Εικόνα?})~19 Αισθητική και ευχρηστία
  (ACM)} μουσικές συσκευές,\footnote{(\textbf{Εικόνα?})~20 Εξατομίκευση
  και προσαρμογή της αισθητικής της διάδρασης (Piotr Gontarczyk)}
συστήματα προσομοίωσης βιομηχανικών διαδικασιών και με ηλεκτρονικά
καταστήματα, έτσι ώστε να εξεταστεί η έκταση του φαινομένου.
Διαπιστώθηκε, μάλιστα, ότι οι χρήστες θεωρούσαν περισσότερο εύχρηστες
τις διεπαφές που είχαν καλύτερη αισθητική.

Για να μελετήσουν την επίδραση της αισθητικής μιας διεπαφής στην
αντίληψη του χρήστη για την ευχρηστία ενός συστήματος, οι ερευνητές
προχώρησαν στην εκτέλεση ενός πειράματος. Ο σχεδιασμός του πειράματος
προέβλεπε ότι διαφορετικές ομάδες χρηστών θα αξιολογούσαν, με βάση τόσο
την εμφάνιση και την ευχρηστία όσο και με βάση τη συμπεριφορά,
διαφορετικά επίπεδα αισθητικής και ευχρηστίας μιας βασικής διεπαφής
μηχανήματος από Αυτόματη Ταμειακή Μηχανή (ΑΤΜ) τράπεζας.

Για τον σκοπό αυτό οι ερευνητές κατασκεύασαν μια προσομοίωση του ΑΤΜ για
διαφορετικές εκδοχές διαρρύθμισης των κουμπιών και ζήτησαν από τους
χρήστες να τα βαθμολογήσουν με βάση την αισθητική και την ευχρηστία
τους. Αμέσως μετά τους ζήτησαν να εκτελέσουν μια σειρά από τυπικές
διεργασίες (π.χ. ανάληψη χρημάτων, ερώτηση υπολοίπου λογαριασμού), με τα
οποία βαθμολόγησαν τις εναλλακτικές διεπαφές. Με αυτόν τον τρόπο, η
ανάλυση των αποτελεσμάτων έδωσε τη συσχέτιση ανάμεσα στην αισθητική και
την ευχρηστία που υπήρχε στα ΑΤΜ. Εκτός από την παραπάνω μεθοδολογία του
πειράματος, ιδιαίτερο ενδιαφέρον παρουσιάζει, επίσης, η χρήση του
λογισμικού διάδρασης το οποίο προσομοίωνε τα ΑΤΜ, αφού αντίστοιχο
λογισμικό χρησιμοποιήθηκε προηγούμενα και για την κατασκευή εναλλακτικών
συσκευών εισόδου ποντίκι.

Τα αποτελέσματα σε αυτό το επιστημονικό άρθρο ήρθαν να συμπληρώσουν τα
κομμάτια που έλειπαν σε ένα παζλ από ευρήματα προηγούμενων εργασιών. Από
τις αρχές της δεκαετίας του 1990 είχε αρχίσει να γίνεται φανερό ότι σε
πολλές περιπτώσεις διάδρασης διαφέρει η αντικειμενική απόδοση του χρήστη
από την υποκειμενική αντίληψη που έχει ο ίδιος ο χρήστης για την απόδοσή
του. Για παράδειγμα, υπάρχουν περιπτώσεις στις οποίες ο χρήστης θα κάνει
περισσότερο χρόνο να ολοκληρώσει μια διεργασία που απαιτεί διάδραση με
τον υπολογιστή και παρόλα αυτά θα τη θεωρεί πιο εύχρηστη από κάποια άλλη
που απαιτεί λιγότερο χρόνο. Αυτές οι περιπτώσεις έρχονται σε αντίθεση με
τον αρχικό ορισμό της ευχρηστίας, η οποία έχει ως βασική διάσταση την
απόδοση ή αλλιώς την ταχύτητα με την οποία ο χρήστης ολοκληρώνει μια
διεργασία. Βλέπουμε, λοιπόν, για μια ακόμη φορά, την ιδιαιτερότητα που
έχει η περιοχή της κατασκευής της διάδρασης, όπου η αντικειμενική
απόδοση ενός συστήματος ανθρώπου και υπολογιστή μπορεί να μην είναι
αρκετή για να χαρακτηρίσει την ποιότητά του. Απαιτείται και η σύμφωνη
γνώμη του χρήστη, η οποία, όμως, είναι εξ΄ ορισμού υποκειμενική και μέσα
σε ένα πλήθος ερευνητικών δεδομένων, η γνώμη των διαφόρων χρηστών μπορεί
να έχει μεγάλη διακύμανση.

\hypertarget{ux3c3ux3cdux3bdux3c4ux3bfux3bcux3b7-ux3b2ux3b9ux3bfux3b3ux3c1ux3b1ux3c6ux3afux3b1-ux3c4ux3bfux3c5-jaron-lanier}{%
\subsection{Σύντομη βιογραφία του Jaron
Lanier}\label{ux3c3ux3cdux3bdux3c4ux3bfux3bcux3b7-ux3b2ux3b9ux3bfux3b3ux3c1ux3b1ux3c6ux3afux3b1-ux3c4ux3bfux3c5-jaron-lanier}}

Ο Jaron Lanier έγινε ευρύτερα γνωστός για τον προβληματισμό που εξέφρασε
σχετικά με τα κοινωνικά μέσα και τους μηχανισμούς εξατομίκευσης που
χρησιμοποιούν. Ήταν, ήδη, όμως, πολύ γνωστός σε έναν μικρότερο κύκλο
ανθρώπων από τη δεκαετιά του 1980 για τα πρώτα εμπορικά συστήματα
εικονικής πραγματικότητας, τα οποία απευθύνονταν κυρίως στην βιομηχανία.
Τα προϊόντα της εταιρείας του VPL βασίζονταν σε μια μάσκα και ένα γάντι
εικονικής πραγματικότητας\footnote{(\textbf{Εικόνα?})~21 Jaron Lanier
  (Jaron Lanier)} με τα οποία μπορούσε να γίνει ο σχεδιασμός της
καμπίνας ενός αυτοκινήτου ή η προετοιμασία για ένα δύσκολο
χειρουργείο.\footnote{(\textbf{Εικόνα?})~22 VPL Virtual Reality (VPL)}

\leavevmode\vadjust pre{\hypertarget{fig:lanier-profile}{}}%
\begin{figure}
\hypertarget{fig:lanier-profile}{%
\centering
\includegraphics{images/lanier-profile.jpg}
\caption{Εικόνα 21: Ο Jaron Lanier κατασκεύασε τα πρώτα εμπορικά
συστήματα εικονικής πραγματικότητας με στόχο να εξερευνήσει την
ανθρώπινη συνείδηση σε φαντασιακά περιβάλλοντα και σε διάδραση με άλλους
χρήστες. Εκτός από τη θεμελίωση μιας σημαντικής περιοχής, έχει συμβάλει
στη διάδοση μιας ανθρωποκεντρικής τεχνολογικής φιλοσοφίας, η οποία
έρχεται σε αντίθεση με το κυρίαρχο αφήγημα της Τεχνητής
Νοημοσύνης.}\label{fig:lanier-profile}
}
\end{figure}

\leavevmode\vadjust pre{\hypertarget{fig:vpl-virtual-reality}{}}%
\begin{figure}
\hypertarget{fig:vpl-virtual-reality}{%
\centering
\includegraphics{images/vpl-virtual-reality.jpg}
\caption{Εικόνα 22: Το γάντι δεδομένων περιέχει αισθητήρες, οι οποίοι
καταγράφουν τη θέση του χεριού και τις κινήσεις των δακτύλων, έτσι ώστε
να υπάρχει λεπτομερής χειρισμός στο περιβάλλον εικονικής
πραγματικότητας, το οποίο απεικονίζεται στη μάσκα. Η δυνατότητα του
χρήστη όχι μόνο να πλοηγηθεί, αλλά κυρίως να εκφραστεί σωματικά και να
χειριστεί με μεγάλη ακρίβεια τον εικονικό εαυτό και το περιβάλλον του,
στο οποίο συνεργάζεται και με άλλους χρήστες, ήταν τα θεμέλια στο πρώτο
εμπορικό σύστημα εικονικής
πραγματικότητας.}\label{fig:vpl-virtual-reality}
}
\end{figure}

Η εικονική πραγματικότητα έγινε περισσότερο δημοφιλής τις επόμενες
δεκαετίας με σημαντικές βελτιώσεις στην ποιότητα των γραφικών και στο
κόστος του κράνους, αλλά χωρίς την αντίστοιχη βελτίωση στη διάδραση με
τον χρήστη. Πράγματι, ο Jaron Lanier θεωρεί τη διάδραση του χρήστη με
τον εικονικό κόσμο ή ακόμη καλύτερα με άλλες αναπαραστάσεις ανθρώπων,
περισσότερο σημαντική από την ανάλυση των γραφικών. Για τον σκοπό αυτό,
το γάντι εικονικής πραγματικότητας προσφέρει μεγάλη λεπτομέρεια στον
χειρισμό εικονικών αντικειμένων ή του χαρακτήρα, ο οποίος δεν είναι
αναγκαίο να αναπαριστά κάποιον άνθρωπο.

Αυτή η έμφαση στην διάδραση με εξειδικευμένες διεπαφές, όπως το γάντι,
καθώς και η επέκταση της ανθρώπινης εμπειρίας με εναλλακτικές εικονικές
αναπαραστάσεις βασίζονται στη μουσική παιδεία του. O Jaron Lanier
ασχολείται από πολύ μικρός με τη μουσική και είναι συλλέκτης μουσικών
οργάνων. Όπως οι Alan Kay και Ted Nelson θεωρούν ότι ο υπολογιστής
μπορεί να γίνει ένα προηγμένο μέσο επικοινωνίας για μια νέα λογοτεχνία,
αντίστοιχα και ο Jaron Lanier θεωρεί πως ο υπολογιστής μπορεί να γίνει
ένα νέο είδος μουσικού οργάνου με απεριόριστες οπτικο-ακουστικές
δυνατότητες.

Περισσότερο σημαντική από την πολύ σπουδαία τεχνολογική συνεισφορά του
παραμένει η ανθρωποκεντρική έμφαση στη δουλειά του. Για παράδειγμα, η
διαμεσολαβούμενη κοινωνική διάδραση είναι θεμελιώδης στους εικονικούς
κόσμους που δημιούργησε, αλλά βασίζεται στην ενεργή πρωτοβουλία και
ελευθερία του κάθε ανθρώπου και όχι σε εξωτερικά συμφέροντα, όπως αυτά
μιας πλατφόρμας ή ενός διαφημιζόμενου. Με αυτόν τον τρόπο, οι
υπολογιστές μπορούν να γίνουν τόσο ένα νέο μέσο έκφρασης όσο και ένα
μέσο οικονομικής ανεξαρτησίας, αφού η αξία κατανέμεται στους ανθρώπους
και στις δράσεις τους, αντί να αναλύεται και να συγκεντρώνεται κεντρικά.

\hypertarget{ux3b2ux3b9ux3b2ux3bbux3b9ux3bfux3b3ux3c1ux3b1ux3c6ux3afux3b1}{%
\subsection*{Βιβλιογραφία}\label{ux3b2ux3b9ux3b2ux3bbux3b9ux3bfux3b3ux3c1ux3b1ux3c6ux3afux3b1}}
\addcontentsline{toc}{subsection}{Βιβλιογραφία}

\hypertarget{refs}{}
\begin{CSLReferences}{0}{0}
\end{CSLReferences}

Fogg, BJ. 2003. \emph{Persuasive Technology: Using Computers to Change
What We Think and Do}. Morgan Kaufmann.

Krueger, M. W. 1991. \emph{Artificial Reality II}. Addison-Wesley.

Laurel, Brenda. 2013. \emph{Computers as Theatre}. Addison-Wesley.

Markoff, John. 2005. \emph{What the Dormouse Said: How the Sixties
Counterculture Shaped the Personal Computer Industry}. Penguin.

McCullough, Malcolm. 1998. \emph{Abstracting Craft: The Practiced
Digital Hand}. MIT press.

Norman, Donald A. 2004. \emph{Emotional Design: Why We Love (or Hate)
Everyday Things}. Basic Civitas Books.

Reeves, Byron, and Clifford Ivar Nass. 1996. \emph{The Media Equation:
How People Treat Computers, Television, and New Media Like Real People
and Places.} Cambridge university press.

Rheingold, Howard. 2000. \emph{The Virtual Community: Homesteading on
the Electronic Frontier}. MIT press.

\hypertarget{ux3c3ux3cdux3bdux3b8ux3b5ux3c3ux3b7}{%
\section{Σύνθεση}\label{ux3c3ux3cdux3bdux3b8ux3b5ux3c3ux3b7}}

\begin{quote}
Ελπίζουμε ότι, σε όχι πάρα πολλά χρόνια, τα ανθρώπινα μυαλά και οι
υπολογιστικές μηχανές θα συνδεθούν πολύ στενά μεταξύ τους, και ότι η
συνεργασία που θα προκύψει θα σκέφτεται όπως δεν έχει σκεφτεί ποτέ ο
ανθρώπινος εγκέφαλος και θα επεξεργάζεται τα δεδομένα με τρόπο που δεν
έχουν πλησιάσει οι μηχανές διαχείρισης της πληροφορίας που γνωρίζουμε
σήμερα. J. C. R. Licklider
\end{quote}

\hypertarget{ux3c0ux3b5ux3c1ux3afux3bbux3b7ux3c8ux3b7}{%
\subsubsection{Περίληψη}\label{ux3c0ux3b5ux3c1ux3afux3bbux3b7ux3c8ux3b7}}

Η μελέτη της κατασκευής της διάδρασης για την περίπτωση των συνεργατικών
συστημάτων, καθώς και για τα συστήματα, τα οποία είναι μία σύνθεση
επιμέρους συστημάτων, απαιτεί επιπλέον μεθόδους και τεχνικές από εκείνες
που είδαμε στο δεύτερο κεφάλαιο. Επίσης, ο βαθμός συμμετοχής των χρηστών
στη διαδικασία της σχεδίασης, της ανάπτυξης, της εισαγωγής και της
ενημέρωσης ενός συστήματος απαιτεί μια διαφορετική αντιμετώπιση.
Συμπληρωματικά, στην περιοχή της διάδρασης ανθρώπου και υπολογιστή,
έχουμε περιοχές όπως τα κοινωνικά και συνεργατικά συστήματα, καθώς και
τα πληροφοριακά συστήματα διοίκησης που αναλύουν φαινόμενα διάδρασης, τα
οποία συμβαίνουν σε σύνθετα κοινωνικά και τεχνολογικά συστήματα, και
μπορεί να λειτουργούν σε πολύ μεγαλύτερη κλίμακα από αυτήν της διάδρασης
ενός ανθρώπου με έναν υπολογιστή. Τόσο η σύνθεση, όσο και η κλίμακα ενός
συστήματος διάδρασης απαιτούν μια διαφορετική προσέγγιση στη μεθοδολογία
της σχεδίασης από εκείνη που καλύπτει τη βασική περίπτωση ανθρώπου και
υπολογιστή.

\hypertarget{ux3c0ux3bbux3bfux3aeux3b3ux3b7ux3c3ux3b7-ux3c3ux3c4ux3b7ux3bd-ux3c0ux3bbux3b7ux3c1ux3bfux3c6ux3bfux3c1ux3afux3b1}{%
\subsection{Πλοήγηση στην
πληροφορία}\label{ux3c0ux3bbux3bfux3aeux3b3ux3b7ux3c3ux3b7-ux3c3ux3c4ux3b7ux3bd-ux3c0ux3bbux3b7ux3c1ux3bfux3c6ux3bfux3c1ux3afux3b1}}

Σε αυτήν την ενότητα θα μελετήσουμε τα είδη της διάδρασης, τα στοιχεία
που τη συνθέτουν, καθώς και το φυσικό, το κοινωνικό και το οργανωσιακό
πλαίσιο μέσα στο οποίο μπορεί να συντελεστεί. Ακόμη, θα εξετάσουμε πόσο
καλά μπορούν να υποστηρίξουν τις ανθρώπινες διαδικασίες τα διαφορετικά
είδη της διάδρασης. Με αυτές τις γνώσεις μπορούμε να μελετήσουμε πώς ο
άνθρωπος χρησιμοποιεί τις συσκευές κυρίως ως εργαλεία για την πλοήγηση
και την ανάκτηση της πληροφορίας \footnote{(\textbf{Εικόνα?})~1
  Αρχιτεκτονική της πληροφορίας στο Hypercard (Apple)} \footnote{(\textbf{Εικόνα?})~2
  Ανάκτηση πληροφορίας στον παγκόσμιο ιστό (Craigslist)} αλλά και ως
μέσα επικοινωνίας, ψυχαγωγίας και συνεργασίας.

Τα συνεργατικά συστήματα είναι από τις σημαντικότερες και τις πιο
γρήγορα αναπτυσσόμενες υποπεριοχές της διάδρασης ανθρώπου και
υπολογιστή, πράγμα αναμενόμενο, αφού ασχολείται με τα πολύ σημαντικά
ζητούμενα της συνεργασίας και της επικοινωνίας μεταξύ ανθρώπων, όταν
αυτές γίνονται μέσω υπολογιστή. Στην πιο κλασική ταξινόμησή τους, οι
συνεργατικές εφαρμογές διακρίνονται βάσει των διαστάσεων της απόστασης
και του χρόνου. Οι πιο δημοφιλείς είναι οι εφαρμογές επικοινωνίας και
συνεργασίας είτε σύγχρονης, είτε ασύγχρονης από απόσταση, όπου το
ζητούμενο είναι ο συντονισμός ομάδων χρηστών. Αναφορικά με την
τροπικότητά τους, οι εφαρμογές συνεργασίας βασίζονται, συνήθως, σε
κείμενο, εικόνα, ήχο, ενώ τα συστήματα υπερμέσων, όπως ο παγκόσμιος
ιστός, \footnote{Bush et al. (1945), Berners-Lee (1996)} διευκολύνουν τη
σύνθεση συστημάτων σε μεγάλη κλίμακα για την εξυπηρέτηση πολλών χρηστών.

\leavevmode\vadjust pre{\hypertarget{fig:hypercard-layout}{}}%
\begin{figure}
\hypertarget{fig:hypercard-layout}{%
\centering
\includegraphics{images/hypercard-layout.png}
\caption{Εικόνα 1: Το πρόγραμμα Hypercard της Apple χρησιμοποιούσε την
απλή μεταφορά της στοίβας με κάρτες, και των αντικειμένων πάνω σε αυτές,
και περιλάμβανε πολλές δυνατότητες σε απλή μορφή (όπως πολυμέσα,
υπερμέσα και μια γλώσσα προγραμματισμού χρήστη), οι οποίες επέτρεψαν την
ευέλικτη προσαρμογή του από τους τελικούς χρήστες σε πολλά πεδία, όπως
σε παρουσιάσεις, στην κατασκευή διαδραστικών πρωτοτύπων και σε βίντεο
παιχνίδια.}\label{fig:hypercard-layout}
}
\end{figure}

\leavevmode\vadjust pre{\hypertarget{fig:web-search}{}}%
\begin{figure}
\hypertarget{fig:web-search}{%
\centering
\includegraphics{images/web-search.png}
\caption{Εικόνα 2: Ανεξάρτητα από τη μορφή του υπολογιστή (επιτραπέζιος,
κινητός), για περισσότερο από δύο δεκαετίες, η ανάκτηση της πληροφορίας
και η διάδραση με το σύστημα των ιστοσελίδων είναι τόσο τεχνολογικά όσο
και συμπεριφορικά ο πιο διαχρονικός και δημοφιλής τρόπος διάδρασης
ανθρώπου και υπολογιστή.}\label{fig:web-search}
}
\end{figure}

Ενώ προχωράμε σε όλο και πιο σύνθετες μορφές διάδρασης, δεν σημαίνει ότι
οι προηγούμενες βασικές μορφές χάνονται. Αντίθετα, οι βασικές μορφές
διάδρασης συνεχίζουν να παίζουν σημαντικό ρόλο ως συστατικά στοιχεία των
πιο σύνθετων συστημάτων. Για παράδειγμα, ένα ηλεκτρονικό κατάστημα
περιέχει πολλές χρήσιμες λειτουργίες, όπως το εικονικό καλάθι αγορών,
στο οποίο ο καταναλωτής συγκεντρώνει τα προϊόντα που θέλει να αγοράσει.
Το ηλεκτρονικό καλάθι αγορών, εκτός από οικεία μεταφορά του καλαθιού από
τον πραγματικό κόσμο, είναι ένα εργαλείο που διευκολύνει τη διάδραση με
το εικονικό κατάστημα. Επιπλέον, ένα σύγχρονο ηλεκτρονικό κατάστημα
περιλαμβάνει και κάποια συνεργατικά στοιχεία, όπως την επικοινωνία με
ηλεκτρονικούς πωλητές ή την ανάγνωση και συγγραφή σχολίων για τα
προϊόντα. Επομένως, τα πιο χρήσιμα συστήματα αποτελούν μια σύνθεση από
επιμέρους ιδέες, τεχνολογίες και πρακτικές, που θα πρέπει να
ολοκληρωθούν σε μεγάλη κλίμακα, ούτως ώστε να εξυπηρετήσουν πολλούς και
διαφορετικούς χρήστες.

Η περιοχή της επικοινωνίας ανθρώπου και υπολογιστή έδωσε αρχικά έμφαση
στη διάδραση με εφαρμογές γραφείου, αναφερόμενη στην αξία της απόδοσης
και της παραγωγικότητας. Είναι χαρακτηριστικό ότι ο Licklider, στη
δεκαετία του 1950, δημιούργησε τις προδιαγραφές που πρέπει να ικανοποιεί
η διάδραση του ανθρώπου με τον υπολογιστή, μελετώντας τις εργασίες που
έκανε ο ίδιος κατά τη διάρκεια μιας τυπικής ημέρας του. Διαπίστωσε ότι
τον περισσότερο χρόνο του τον αφιέρωνε στην ανάκτηση της πληροφορίας,
καθώς και στην επεξεργασία και στην οπτικοποίηση της πληροφορίας, ενώ
αφιέρωνε ελάχιστο χρόνο στην κατανόηση της πληροφορίας και στη λήψη
αποφάσεων, που ήταν και πιο σημαντικά για τον άνθρωπο. Λίγο καιρό
αργότερα, από τη θέση του υπεύθυνου χρηματοδότησης, στήριξε την έρευνα
του Engelbart, η οποία οδήγησε στη δημιουργία της συσκευής εισόδου
ποντίκι, καθώς και σε μια σειρά από τεχνολογίες που χρησιμοποιούνται
σήμερα σε όλα τα γραφεία. Από τη μια πλευρά βλέπουμε ότι το αρχικό όραμα
πήρε πάρα πολύ καιρό μέχρι να γίνει μέρος της καθημερινότητας, αλλά από
την άλλη πλευρά το μεγαλύτερο μέρος της προσπάθειας ξεκίνησε με κίνητρο
να βελτιώσει ένα πολύ συγκεκριμένο αν και σίγουρα όχι το σημαντικότερο
πεδίο της ανθρώπινης δραστηριότητας, αυτό της εργασίας.

Ταυτόχρονα με την ανάπτυξη των υπολογιστών που διευκολύνουν την εργασία,
αυξάνουν την παραγωγικότητα και κερδίζουν χρόνο για τον άνθρωπο,
δημιουργούνται νέες εφαρμογές σε πεδία, όπως στην εκπαίδευση και στην
ψυχαγωγία. Οι εκπαιδευτικές και ψυχαγωγικές εφαρμογές, αν και σε πολλές
περιπτώσεις, εκτελούνται σε υλικό και λογισμικό παρόμοιο με εκείνο των
εφαρμογών γραφείου, έχουν πολύ διαφορετικές απαιτήσεις στην κατασκευή
της διάδρασης με τον χρήστη. \footnote{Shiffman (2009)} Ειδικά οι
ψυχαγωγικές εφαρμογές, αλλά και πολλές εκπαιδευτικές εφαρμογές, που
στόχο έχουν να κεντρίσουν το ενδιαφέρον του χρήστη, δίνουν χαμηλή
προτεραιότητα στην απόδοση της διεργασίας και στον χρόνο που αυτή
παίρνει, ενώ δίνουν έμφαση στην ευχάριστη διάδραση. Για να το πετύχουν
αυτό, χρησιμοποιούν τεχνικές όπως τα πολυμέσα, η αφήγηση, η συμμετοχή
πολλών χρηστών, και, επιπλέον, κάνουν χρήση συσκευών διάδρασης, οι
οποίες διευκολύνουν την εμβύθιση του χρήστη σε ένα εικονικό ή επαυξημένο
περιβάλλον. \footnote{Packer and Jordan (2002)} Η εμβύθιση του χρήστη
μεγιστοποιείται με τη χρήση συσκευών διάδρασης εικονικής
πραγματικότητας, ενώ η μεγιστοποίηση της προσβασιμότητας στην πληροφορία
από διαφορετικούς χρήστες και σε διαφορετικά πλαίσια χρήσης,
επιτυγχάνεται με τις τεχνικές της πολυτροπικής διάδρασης, η οποία δίνει
έμφαση σε συσκευές διάδρασης πέρα από το πληκτρολόγιο και το ποντίκι.

Η χρησιμότητα των υπερμέσων μπορεί να θεωρείται δεδομένη, αφού έγινε
δημοφιλής με την ανάπτυξη του παγκόσμιου ιστού, αλλά η αρχική ιδέα ήταν
του Ted Nelson, το 1965, και περιλαμβάνει επιπλέον λειτουργίες, όπως η
σημασιολογία και το αρχείο αλλαγών, με σκοπό να ενθαρρύνει τη
μη-γραμμική ανάγνωση, καθώς και την ενεργή συμμετοχή των χρηστών ως
συγγραφέων και όχι μόνο ως αναγνωστών. Η χρησιμότητα των υπερμέσων,
αρχικά έγινε αισθητή στους χρήστες με το Hypercard και λίγο αργότερα με
τον παγκόσμιο ιστό, ο οποίος επέτρεψε τη διασύνδεση αντικειμένων που
βρίσκονταν σε απομακρυσμένους δικτυωμένους υπολογιστές. \footnote{Barnet
  (2013)}

\leavevmode\vadjust pre{\hypertarget{fig:we-feel-fine}{}}%
\begin{figure}
\hypertarget{fig:we-feel-fine}{%
\centering
\includegraphics{images/we-feel-fine.jpg}
\caption{Εικόνα 3: Η ανάκτηση της πληροφορίας μπορεί να ξεκίνησε από την
αναζήτηση σε κείμενο στις μεγάλες ψηφιακές βιβλιοθήκες, αλλά γρήγορα
βρήκε εφαρμογή στις νέες δραστηριότητες των ανθρώπων, όπως είναι η
ανάκτηση πληροφορίας σχετικής με το συναίσθημα (π.χ. We Feel Fine) από
ιστολόγια που συνήθως χρησιμοποιούνται από τους χρήστες τους ως
προσωπικά ημερολόγια.}\label{fig:we-feel-fine}
}
\end{figure}

\leavevmode\vadjust pre{\hypertarget{fig:tenbyten}{}}%
\begin{figure}
\hypertarget{fig:tenbyten}{%
\centering
\includegraphics{images/tenbyten.png}
\caption{Εικόνα 4: Η πληθώρα και η πολυπλοκότητα της διαθέσιμης
πληροφορίας δημιούργησε την ανάγκη για μια νέα μορφή ανάκτησης και
απεικόνισης της πληροφορίας, όπου ο ρόλος του δημοσιογράφου είναι
περισσότερο ως προγραμματιστής της διάδρασης.}\label{fig:tenbyten}
}
\end{figure}

Η κατασκευή της διάδρασης για τις εφαρμογές υπερμέσων είναι μια
απαιτητική δραστηριότητα, αλλά αυτό δεν σημαίνει ότι θα πρέπει να
γίνεται μόνο από τους έμπειρους και εκπαιδευμένους χρήστες. Για
παράδειγμα, το πολύ πετυχημένο λογισμικό Hypercard της Apple έδωσε τη
δυνατότητα σε όλους τους χρήστες να δημιουργήσουν τα δικά τους
υπερμεσικά και πολυμεσικά προγράμματα στον υπολογιστή, χωρίς να έχουν
γνώσεις προγραμματισμού, με αποτέλεσμα να δημιουργηθούν πολλά νέα
προϊόντα από ανθρώπους που διαφορετικά δεν θα είχαν πρόσβαση σε αυτήν
την τεχνολογία. Όπως η επιφάνεια εργασίας, έτσι ακριβώς και το WWW
εξελίχθηκε πολύ γρήγορα από μια απλή εφαρμογή στον υπολογιστή του χρήστη
σε μια πλατφόρμα πάνω στην οποία εκτελούνται όλες οι εφαρμογές του
χρήστη, τόσο οι παραδοσιακές, π.χ. εφαρμογές γραφείου όσο και οι νέες
εφαρμογές, όπως η κοινωνική δικτύωση, οι εμπορικές συναλλαγές και η
ανάκτηση της πληροφορίας.

Το πιο συνηθισμένο λάθος στην κατασκευή της διάδρασης για σύνθετα
συστήματα, τα οποία θα χρησιμοποιήσουν πολλοί και διαφορετικοί χρήστες,
είναι να θεωρήσουμε ότι η τεχνολογία μπορεί να προσφέρει μια συνολική
λύση ή, ακόμη χειρότερα, ότι η διαδικασία σχεδίασης μπορεί να προσφέρει
μια λύση από μόνη της, χωρίς τη συμμετοχή του ανθρώπινου παράγοντα.
Ακόμη και στα συστήματα αυτοματισμού γραφείου, όπου μπορούμε να
υποθέσουμε ότι οι χρήστες είναι έμπειροι, η έρευνα έχει δείξει πως
υπάρχουν αναγκαίες διεργασίες συντονισμού ή ακόμη και αντικείμενα, όπως
το χαρτί, \footnote{Sellen and Harper (2003)} που είναι προτιμότερο να
μην γίνουν μέρος του υπολογιστικού συστήματος, αλλά να παραμείνουν μέρος
ενός συνολικού πληροφοριακού συστήματος, το οποίο περιλαμβάνει
υπολογιστές, αντικείμενα, ανθρώπους και πρακτικές. Επιπλέον, επειδή οι
προδιαγραφές που αντικατοπτρίζουν τις ανθρώπινες δραστηριότητες είναι
φευγαλέες κατά την εισαγωγή μιας τεχνολογικής παρέμβασης, ένας τρόπος να
κρατήσουμε το σύστημα χρήσιμο είναι να επιτρέπουμε, ή ακόμη καλύτερα να
ενθαρρύνουμε τη συμμετοχή του τελικού χρήστη στη σχεδίαση αλλά και στην
κατασκευή του.

Μια από τις πιο σημαντικές δυνατότητες των ηλεκτρονικών υπολογιστών
είναι ότι διευκολύνουν την εκτέλεση διεργασιών που βασίζονται στην
επεξεργασία πληροφορίας. Για παράδειγμα, ένας υπολογιστής μπορεί να μας
διευκολύνει να βρούμε άμεσα όλα τα άρθρα ενός συγγραφέα που περιέχουν
κάποιες λέξεις κλειδιά, κάτι που διαφορετικά θα απαιτούσε, πέρα από την
επίσκεψη σε έναν ή περισσότερους χώρους, πολλές ώρες αναζήτησης στα
ράφια της βιβλιοθήκης. Παράλληλα, με τη μετάβαση από το απλό κείμενο στο
εμπλουτισμένο με πολυμέσα κείμενο, η ανάκτηση της πληροφορίας επεκτάθηκε
σε νέες μορφές περιεχομένου, όπως είναι το περιεχόμενο που προσφέρει η
κοινωνική δικτύωση, καθώς και το περιεχόμενο που παράγεται σε πραγματικό
χρόνο από πολλούς χρήστες. Για παράδειγμα, η βραβευμένη εφαρμογή
\emph{We Feel Fine} καταγράφει συνεχώς τα συναισθήματα που εκφράζονται
από τους χρήστες ιστολογίων και προσφέρει εναλλακτικούς τρόπους
πλοήγησης, π.χ. με αφαιρετική οπτικοποίηση, στα συναισθήματα που
εκφράζονται ατομικά ή συλλογικά από τους χρήστες του διαδικτύου.
Βλέπουμε ότι η παραδοσιακή ανάκτηση της πληροφορίας έχει διαχρονική
αξία, αλλά ταυτόχρονα γίνεται μέσα στα χρόνια πολύ διαφορετική, καθώς οι
δραστηριότητες και τα ενδιαφέροντα των χρηστών της μετασχηματίζονται.

Η μεγάλη αποδοχή των υπερμέσων και των πολυμέσων ως δικτυακών εφαρμογών
που βασίζονται σε ψηφιακά διασυνδεδεμένο υλικό, π.χ. σε blogs, εικόνες,
μουσική, ειδήσεις κτλ., δημιούργησε την ανάγκη για εύχρηστους και
διασκεδαστικούς τρόπους ανάκτησης της πληροφορίας. Η χρήση κειμένου σε
μια μηχανή αναζήτησης είναι ένας πολύ αποτελεσματικός τρόπος ανάκτησης
της πληροφορίας, όταν γνωρίζουμε με σχετική ακρίβεια τι ψάχνουμε, ειδικά
αν αυτό που ψάχνουμε περιγράφεται με κείμενο όπως αυτό που
χρησιμοποιούμε για την αναζήτηση. Υπάρχουν όμως πολλές περιπτώσεις που
μια αναζήτηση μπορεί να έχει περισσότερο τη μορφή της ανοιχτής
εξερεύνησης της πληροφορίας, όπως για παράδειγμα η επίσκεψη σε μια
βιβλιοθήκη και η ελεύθερη πλοήγηση στα εξώφυλλα και στο περιεχόμενο των
βιβλίων.

Αντίστοιχα, η πληθώρα της διαθέσιμης πληροφορίας και πιο συγκεκριμένα, η
πολυμεσική της φύση, δημιούργησε την ανάγκη για νέες μορφές
οπτικοποίησης και αναζήτησης της πληροφορίας, που να βασίζονται
περισσότερο στα πολυμέσα και στην αφαίρεση και λιγότερο στο κείμενο και
στη στοχευμένη αναζήτηση. Για παράδειγμα, η εφαρμογή \emph{We Feel Fine}
οπτικοποιεί τις προτάσεις που περιέχουν την λέξη \emph{feel} όταν τις
βρίσκει στα ιστολόγια των χρηστών, τα οποία συνήθως χρησιμοποιούν ως
προσωπικά ημερολόγια. Το αποτέλεσμα αυτής της τεχνικής του
προγραμματισμού της διάδρασης με πολυμέσα και υπερμέσα είναι μια
αφαιρετική απεικόνιση των συναισθημάτων της μπλογκόσφαιρας. Αντίστοιχα,
η εφαρμογή \emph{tenbyten} δημιουργεί κάθε μία ώρα ένα μωσαϊκό της
τρέχουσας ειδησεογραφίας, όπως την ανακτά από εκατό δημοφιλή πρακτορεία
ειδήσεων. Συνολικά, βλέπουμε πως η κατασκευή της διάδρασης μπορεί να
δημιουργήσει ένα νέο επίπεδο ανάγνωσης και αντίληψης του κόσμου, το
οποίο βασίζεται στη σύνθεση των επιμέρους στοιχείων του. \footnote{(\textbf{Εικόνα?})~3
  Οπτικοποίηση δεδομένων συναισθήματων από δικτυακά ημερολόγια χρηστών
  (Jonathan Harris)} \footnote{(\textbf{Εικόνα?})~4 Οπτικοποίηση των
  ειδήσεων (Jonathan Harris)}

\hypertarget{ux3c0ux3bfux3bbux3c5ux3bcux3b5ux3c3ux3b9ux3baux3ae-ux3b4ux3b9ux3acux3b4ux3c1ux3b1ux3c3ux3b7}{%
\subsection{Πολυμεσική
διάδραση}\label{ux3c0ux3bfux3bbux3c5ux3bcux3b5ux3c3ux3b9ux3baux3ae-ux3b4ux3b9ux3acux3b4ux3c1ux3b1ux3c3ux3b7}}

\leavevmode\vadjust pre{\hypertarget{fig:pong}{}}%
\begin{figure}
\hypertarget{fig:pong}{%
\centering
\includegraphics{images/pong.jpg}
\caption{Εικόνα 5: Το Pong αποτελεί μια βελτιωμένη εκδοχή του πρόσφατου
τότε Table Tennis for Two, στο οποίο προσθέτει καλύτερη κίνηση και
γίνεται γρήγορα δημοφιλές αρχικά στις δημόσιες κονσόλες βιντεοπαιχνιδιών
και λίγο αργότερα στους πρώτους οικιακούς μικρο-υπολογιστές. Tο Pong με
τη σειρά του έδωσε την έμπνευση για την δημιουργία του Breakout και τη
δημιουργία της βιομηχανίας των βιντεοπαιχνιδιών, η οποία ήταν μια νέα
κατηγορία τέχνης και εμπορικής δραστηριότητας.}\label{fig:pong}
}
\end{figure}

\leavevmode\vadjust pre{\hypertarget{fig:breakout}{}}%
\begin{figure}
\hypertarget{fig:breakout}{%
\centering
\includegraphics{images/breakout.png}
\caption{Εικόνα 6: Το βιντεοπαιχνίδι Breakout πήρε έμπνευση από το Pong,
το οποίο παίζεται με δύο παίκτες και στη θέση του δεύτερου παίκτη έβαλε
σειρές από τουβλάκια που σπάνε, καθώς ο παίκτης τα χτυπάει με την μπάλα.
Εκτός από τη διασκέδαση και την έμπνευση που έδωσε για τη δημιουργία
νέων συστημάτων διάδρασης, αποτελεί σημείο αναφοράς για τη μετάβαση από
τα βιντεοπαιχνίδια προσομοίωσης της πραγματικότητας σε ένα περισσότερο
αφαιρετικό νέο είδος τέχνης.}\label{fig:breakout}
}
\end{figure}

Η μετάβαση από τα τερματικά κειμένου σε τερματικά γραφικών και νέες
συσκευές εισόδου διευκολύνθηκε και εμπνεύστηκε από την παράλληλη πορεία
της κατασκευής βιντεοπαιχνιδιών. Η κατασκευή ενός καινοτόμου
βιντεοπαιχνιδιού βασίζεται στον προγραμματισμό διαδράσεων που
μετατρέπουν μια μορφή δεδομένων εισόδου σε κάποια διαφορετική μορφή
δεδομένων εξόδου. Για παράδειγμα, στο κλασικό βιντεοπαιχνίδι
επιτραπέζιας αντισφαίρισης οι χρήστες μετακινούν την ψηφιακή ρακέτα με
έναν μοχλό κατακόρυφης εισόδου. Ταυτόχρονα, τα πρώτα δημοφιλή
βιντεοπαιχνίδια που μοιράζονταν σε μορφή πηγαίου κώδικα έπαιξαν τον ρόλο
προδιαγραφών για την κατασκευή των επόμενων διαδραστικών συστημάτων.Για
παράδειγμα, η αντικατάσταση ενός παίκτη στην επιτραπέζια αντισφαίριση με
έναν τοίχο από τουβλάκια δημιούργησε μια νέα κατηγορία βιντεοπαιχνιδιών
και, κυρίως, δημιούργησε τις προδιαγραφές για τον σχεδιασμό του Apple
II, \footnote{(\textbf{Εικόνα?})~5 Pong (Wikipedia)} \footnote{(\textbf{Εικόνα?})~6
  Breakout (Wikimedia)} έτσι ώστε να μπορεί κάποιος να προγραμματίσει
μια εκδοχή του βιντεοπαιχνιδιού στην BASIC, το ίδιο δηλαδή σκεπτικό που
είχε και ο Alan Kay για το Dynabook και το Spacewar.

Το πλαίσιο της τυπολογίας των διαδράσεων που έχουν δημιουργηθεί μέχρι
τώρα μπορεί να εξηγηθεί από τις αντίστοιχες ανάγκες στα πρώτα στάδια
αυτής της περιοχής. Τα πρώτα βήματα του προγραμματισμού της διάδρασης
ανθρώπου και υπολογιστή έγιναν σε τερματικά κειμένου, ενώ ακόμη και τα
πρώτα γραφικά περιβάλλοντα, όπως η επιφάνεια εργασίας, έδιναν έμφαση
στις εφαρμογές γραφείου, ειδικά στην επεξεργασία κειμένου και αριθμών.
Το αποτέλεσμα ήταν να δημιουργηθούν, να βελτιωθούν και να γίνουν
δημοφιλείς, εκείνες οι συσκευές εισόδου και εξόδου, καθώς και εκείνα τα
στυλ διάδρασης που είχαν σχέση με κείμενο. Αντίθετα, τα γραφικά είχαν
αρχικά περισσότερο διακοσμητικό ρόλο ως εικονίδια και παράθυρα, τα οποία
σίγουρα διευκολύνουν τον αρχάριο χρήστη και προσελκύουν ειδικά τον νέο
χρήστη. Καθώς όμως έγιναν περισσότερο δημοφιλή τα πολυμεσικά συστήματα
που δίνουν πρόσβαση σε δικτυακά υπερμέσα, καθώς και σε εικόνα, βίντεο,
κείμενο, ήχο και γραφικά, η κατασκευή της διάδρασης επεκτάθηκε για να
καλύψει και αυτές τις περιοχές. \footnote{(\textbf{Εικόνα?})~7 RAND
  tablet (RAND)} \footnote{(\textbf{Εικόνα?})~8 Genesys (MIT)} Με αυτόν
τον τρόπο, η κατασκευή της διάδρασης μετατρέπεται πλέον στο αναγκαίο
μέσο που συνθέτει όλα τα επιμέρους στοιχεία για τη δημιουργία
πληροφοριακών συστημάτων μεγάλης κλίμακας, είτε λόγω του πλήθους των
χρηστών είτε λόγω του εύρους του πληροφοριακού περιεχομένου.

Ταυτόχρονα, τα συστήματα που επέτρεπαν τη διασύνδεση μεταξύ περιεχομένου
ανεξάρτητα από το είδος του, π.χ. κείμενο, εικόνες κτλ., καθώς και
εκείνα τα συστήματα που βασίζονταν σε πολλαπλά μέσα, στα οποία το
κείμενο είχε έναν ισότιμο ρόλο ανάμεσα σε βίντεο, φωτογραφίες, γραφικά
και ήχο, έγιναν διαθέσιμα στους προσωπικούς υπολογιστές τη δεκαετία του
1990 και σε δικτυακή μορφή από τη δεκαετία του 2000. Το αποτέλεσμα ήταν
να δημιουργηθεί μια νέα σειρά από συστήματα εισόδου και εξόδου, καθώς
και νέα στυλ διάδρασης, ώστε να εξυπηρετηθούν οι νέες ανάγκες των
χρηστών, οι οποίες δημιουργήθηκαν από τη χρήση περιεχομένου που
βασίζεται στα υπερμέσα και στα πολυμέσα. \footnote{Garrett (2010)} Για
παράδειγμα, δημιουργήθηκαν οπτικές γλώσσες προγραμματισμού, οι οποίες
δεν ήταν απλά μια οπτικοποίηση των αντίστοιχων γραπτών γλωσσών
προγραμματισμού. Οι πολυμεσικές οπτικές γλώσσες προγραμματισμού
επιτρέπουν την επεξεργασία πολυμεσικών δεδομένων με χρήση διαγραμμάτων
ροής και λειτουργούν είτε ετεροχρονισμένα είτε σε πραγματικό χρόνο.
Ειδικά τα συστήματα πραγματικού χρόνου επιτρέπουν τον ζωντανό
προγραμματισμό πολυμεσικών έργων, έτσι που μοιάζουν περισσότερο με μια
καλλιτεχνική παράσταση, παρά με τη μηχανική του λογισμικού. Με αυτόν τον
τρόπο ο προγραμματισμός της διάδρασης δεν είναι απλά η είσοδος κειμένου,
αλλά η είσοδος με οποιοδήποτε μέσο επικοινωνίας ταιριάζει στον χρήστη
και στο πεδίο χρήσης.

\leavevmode\vadjust pre{\hypertarget{fig:rand-tablet}{}}%
\begin{figure}
\hypertarget{fig:rand-tablet}{%
\centering
\includegraphics{images/rand-tablet.jpg}
\caption{Εικόνα 7: Η πένα έχει το πλεονέκτημα σε σχέση με το ποντίκι ότι
μπορεί να χρησιμοποιηθεί και ως συσκευή έμμεσης διάδρασης με τη βοήθεια
μιας επιφάνειας ευαίσθητης στη μύτη της πένας και χωρίς να υπάρχει η
ανάγκη αυτή η επιφάνεια να είναι οθόνη.}\label{fig:rand-tablet}
}
\end{figure}

\leavevmode\vadjust pre{\hypertarget{fig:genesys}{}}%
\begin{figure}
\hypertarget{fig:genesys}{%
\centering
\includegraphics{images/genesys.jpg}
\caption{Εικόνα 8: To σύστημα οπτικού προγραμματισμού Genesys
απευθύνεται σε σχεδιαστές γραφικών κίνησης, οπότε βασίζεται στην οικεία
για αυτούς πένα και σε λογισμικό σχεδίασης αντικειμένων. Παρουσιάζει
καινοτόμες λειτουργίες διάδρασης, όπως είναι η χειρονομία για τη
σχεδίαση μιας τροχιάς κίνησης, όπου εκτός από την ίδια την τροχιά
καταγράφεται και ο ρυθμός.}\label{fig:genesys}
}
\end{figure}

Στο τεχνολογικό πεδίο η αναφορά στα πολυμέσα είναι συνήθως συνώνυμη με
την τεχνολογική ολοκλήρωση διαφορετικών μορφών επικοινωνίας, όπως
κείμενο, εικόνα, ήχος, βίντεο κτλ.. Όμως, τα πολυμέσα περιέχουν ακόμη
περισσότερα στοιχεία, όπως είναι τα υπερμέσα, η διάδραση, η συμμετοχή
των χρηστών στην παραγωγή του περιεχομένου, η αφηγηματικότητα και η
εμβύθιση. Στην προσπάθεια για μεγιστοποίηση της εμβύθισης, οι
κατασκευαστές της διάδρασης χρησιμοποιούν φωτορεαλιστικά γραφικά, καθώς
και νέες συσκευές εισόδου και εξόδου, όπως είναι τα συστήματα εικονικής
πραγματικότητας. Σύμφωνα με τον Ted Nelson, ο όρος υπερμέσα περιγράφει
με συνοπτικό τρόπο τον συνδυασμό των πολυμέσων με τη διάδραση, ο οποίος
δίνει τη δυνατότητα για συμμετοχή, μη γραμμική αφήγηση, και μεγαλύτερη
εμβύθιση. Ειδικά για την ενίσχυση της εμβύθισης του χρήστη σε ένα
ψηφιακό περιβάλλον, το οποίο δημιουργεί δυναμικά ο υπολογιστής, έχουν
κατασκευαστεί μια σειρά από νέες συσκευές εισόδου, π.χ. γάντια με
αισθητήρες δακτύλων, και συσκευές εξόδου, όπως μάσκες εικονικής
πραγματικότητας. Η μεγάλη υπόσχεση που δίνουν τα συστήματα εικονικής
πραγματικότητας είναι ότι στο μέλλον δεν θα χρειάζεται να έχουμε
διαφορετικές τεχνητές διεπαφές για τη διάδραση μέσω του υπολογιστή, αφού
αυτές θα μοιάζουν με τις διεπαφές που ήδη έχουμε για τη διάδραση με τους
ανθρώπους και το φυσικό μας περιβάλλον. Αυτή όμως η υπόσχεση πέφτει στην
παγίδα της προσομοίωσης και της ευχρηστίας, οι οποίες μπορεί να έχουν
κάποια πλεονεκτήματαα, αλλά δεν οραματίζονται τις νέες δυνατότητες μέσω
της επαύξησης της ανθρώπινης νοημοσύνης. \footnote{Bolt (1978)}
\footnote{(\textbf{Εικόνα?})~9 Σύστημα ψηφιακής επεξεργασίας εικόνας
  Superpaint (Web Archive)} \footnote{(\textbf{Εικόνα?})~10 Χωρική
  διεπαφή Dataland (MIT Media Lab)}

Η κατασκευή της διάδρασης ήταν αρχικά μια δουλειά μόνο για εξειδικευμένο
προσωπικό σε ερευνητικά εργαστήρια και εταιρείες υψηλής τεχνολογίας.
Σύμφωνα με αυτήν την αυστηρά ιεραρχική και γραμμική αντίληψη της
σχεδίασης, το αποτέλεσμα της εργασίας μιας μικρής, αλλά εξειδικευμένης
ομάδας σχεδιαστών της τεχνολογίας γινόταν προϊόν για τους πολλούς
χρήστες. Αυτή η προσέγγιση χρησιμοποιείται με επιτυχία για πολλά χρόνια
από εταιρείες όπως η Apple, η οποία δοκιμάζει εσωτερικά πολλές εκδοχές
για ένα προϊόν και μετά την αρχική παραγωγή του φροντίζει να το
αναβαθμίζει σταδιακά. Άλλες εταιρείες, όπως η Microsoft, χρησιμοποιούν
πολλούς χρήστες κατά τη διαδικασία σχεδίασης και ανάπτυξης, είτε για να
κάνουν αξιολόγηση είτε απλά για να ακούσουν τη γνώμη τους. Από την άλλη
πλευρά, υπάρχουν εφαρμογές με μια εντελώς μη γραμμική και μη ιεραρχική
αντίληψη της σχεδίασης, οι οποίες αφήνουν ακόμη μεγαλύτερο περιθώριο
στους χρήστες για να επεξεργαστούν απευθείας την εμφάνιση και τη
λειτουργία τους, όπως για παράδειγμα το Winamp, ένα δημοφιλές λογισμικό
αναπαραγωγής μουσικών αρχείων κατά τα τέλη της δεκαετίας του 1990. Το
Winamp έγινε γνωστό όχι τόσο επειδή είχε κάποιο λειτουργικό πλεονέκτημα
έναντι του ανταγωνισμού που ήταν πολύ έντονος, καθώς οι κατασκευαστές
λειτουργικών συστημάτων έβαζαν δωρεάν το δικό τους λογισμικό σε κάθε νέα
εγκατάσταση, όπως π.χ. το Windows Media Player όσο επειδή είχε μια
μεγάλη συλλογή από μορφές και οπτικοποιήσεις, τις οποίες έφτιαχναν και
διαμοίραζαν μεταξύ τους πολλοί από τους τελικούς χρήστες, χωρίς κάποιον
κεντρικό έλεγχο από τον αρχικό κατασκευαστή.

\leavevmode\vadjust pre{\hypertarget{fig:superpaint-setup}{}}%
\begin{figure}
\hypertarget{fig:superpaint-setup}{%
\centering
\includegraphics{images/superpaint-setup.jpg}
\caption{Εικόνα 9: Η αρχική υλοποίηση του προγράμματος ψηφιακής
επεξεργασίας εικόνας Superpaint έγινε σε γλώσσα μηχανής στον μίνι
υπολογιστή Data General Nova, ο οποίος ήταν συνδεμένος με επιπλέον υλικό
για την ψηφιοποίηση αναλογικού βίντεο καθώς και για την είσοδο και έξοδο
του προγράμματος ψηφιακής ζωγραφικής. Η προβολή των εικόνων γινόταν σε
μια αναλογική οθόνη τηλεόρασης, ενώ για την επεξεργασία της εικόνας η
βασική συσκευή εισόδου ήταν μια πένα έμμεσης διάδρασης. Η τεχνολογία
αυτή αναπτύχθηκε παράλληλα με τα γραφικά περιβάλλοντα και οδήγησε σε
εφαρμογές ψηφιακής τέχνης και ψηφιακής
κινηματογραφίας.}\label{fig:superpaint-setup}
}
\end{figure}

\leavevmode\vadjust pre{\hypertarget{fig:dataland}{}}%
\begin{figure}
\hypertarget{fig:dataland}{%
\centering
\includegraphics{images/dataland.jpg}
\caption{Εικόνα 10: Στο ερευνητικό έργο Spatial Data Management System η
πλοήγηση σε πολυμεσική πληροφορία μπορεί να γίνει με μια μεταφορά που
βασίζεται στην πλοήγηση στο φυσικό περιβάλλον με την προσθήκη της
δυνατότητας μεγέθυνσης και μιας ιεαραρχικής οργάνωσης των δεδομένων. Ο
χρήστης πλοηγείται με τα ενσωματωμένα σε μια καρέκλα χειριστήρια και
χρησιμοποιεί συμπληρωματικές οθόνες, οι οποίες μπορούν να μεγενθύνουν
και να εφαρμόσουν εργαλεία πάνω στα πολυμεσικά
δεδομένα.}\label{fig:dataland}
}
\end{figure}

Από την άλλη πλευρά, ο οπτικός προγραμματισμός μπορεί να λειτουργεί και
ως μια μεταφορά για τις σχετικά λιγότερο ελκυστικές, βασικές έννοιες,
όπως είναι ο έλεγχος ροής και η επανάληψη. Όπως ακριβώς στο παρελθόν η
γλώσσα Assembly επέτρεψε αρχικά σε περισσότερους να προγραμματίσουν σε
μια γλώσσα που έμοιαζε έστω και λίγο με τη φυσική γλώσσα και, έπειτα, οι
γλώσσες υψηλού επιπέδου, π.χ. Cobol, C, Pascal κτλ. έφυγαν από τις
λεπτομέρειες της αρχιτεκτονικής του υλικού του κάθε υπολογιστή που
επέβαλε η Assembly, έτσι και ο οπτικός προγραμματισμός έδωσε τη
δυνατότητα σε ακόμη περισσότερους να μιλήσουν μια γλώσσα κατανοητή μεν
από τον υπολογιστή, αλλά και πλησιέστερη στην ανθρώπινη λογική. Ο
οπτικός προγραμματισμός έδωσε τη δυνατότητα ακόμη και στις μικρές
ηλικίες να δημιουργήσουν παιχνίδια με εργαλεία όπως το MIT Scratch.

Ο οπτικός προγραμματισμός είναι μια αναγκαία προϋπόθεση για την γρήγορη
και εύκολη κατασκευή της διάδρασης, δεν αποτελεί, όμως και μια ικανή
συνθήκη της κατασκευής ενός πετυχημένου συστήματος διάδρασης. Υπάρχει η
ανάγκη να βλέπουμε ταυτόχρονα με την κατασκευή και τη συμπεριφορά του
προγράμματος και όχι μόνο τη στατική του κατάσταση όπως μας την
παρουσιάζει ο πηγαίος κώδικας. Σε αναλογία με τον μαθηματικό συμβολισμό
για την κίνηση του απλού εκκρεμούς, ο πηγαίος κώδικας είναι μεν πολύ
ευέλικτος, αλλά δεν επιτρέπει την άμεση κατανόηση κατά τις διάφορες
φάσεις της εκτέλεσης του προγράμματος. Η γρήγορη δοκιμή και η
επαναληπτική βελτίωση του προγράμματος διάδρασης διευκολύνεται από
εκείνα τα περιβάλλοντα ανάπτυξης που ενθαρρύνουν την προσομοίωση της
εκτέλεσης του προγράμματος και τον διαδραστικό έλεγχο της συμπεριφοράς
του. Σε αυτήν την κατεύθυνση, υπάρχουν ερευνητικά και πειραματικά
περιβάλλοντα, τα οποία βασίζονται στον πολυτροπικό και στον ζωντανό
προγραμματισμό, έτσι ώστε ο σχεδιαστής να μπορεί να εξερευνήσει σε
πραγματικό χρόνο διαφορετικές συμπεριφορές με πολλούς τρόπους.

\hypertarget{ux3bfux3bcux3acux3b4ux3b5ux3c2-ux3baux3b1ux3b9-ux3bfux3c1ux3b3ux3b1ux3bdux3b9ux3c3ux3bcux3bfux3af}{%
\subsection{Ομάδες και
οργανισμοί}\label{ux3bfux3bcux3acux3b4ux3b5ux3c2-ux3baux3b1ux3b9-ux3bfux3c1ux3b3ux3b1ux3bdux3b9ux3c3ux3bcux3bfux3af}}

Η ευρύτερη περιοχή της διάδρασης ανθρώπου και υπολογιστή ξεκίνησε και
εξακολουθεί να αναπτύσσεται δίνοντας έμφαση στον διάλογο ανάμεσα σε έναν
άνθρωπο και έναν υπολογιστή. Στην πορεία έχουν δημιουργηθεί νέες
υποπεριοχές, οι οποίες αναπτύσσονται τουλάχιστον το ίδιο γρήγορα, οι
οποίες μελετούν ζητήματα όπως η επικοινωνία, η συνεργασία και η οργάνωση
μικρότερων ή μεγαλύτερων ομάδων ανθρώπων. \footnote{Malone and Crowston
  (1994)} Ειδικότερα, η ανάπτυξη μίας από αυτές, της περιοχής των
κοινωνικών και συνεργατικών συστημάτων, \footnote{Baecker (1993)}
σηματοδοτεί τη μετατόπιση του ενδιαφέροντος από την θεώρηση του
υπολογιστή ως απλό εργαλείο επίλυσης προβλημάτων, στον υπολογιστή ως
μέσο επικοινωνίας που διευκολύνει τη διάδραση ανάμεσα στους ανθρώπους.

Εκ των πραγμάτων, αυτές οι πολύ δημοφιλείς υποπεριοχές της διάδρασης,
λόγω του ανθρωποκεντρικού τους χαρακτήρα, αντλούν ερευνητικά δεδομένα
και από τις ανθρωπιστικές επιστήμες. Εκτός από την επιστήμη της
ψυχολογίας και τη γνωστική επιστήμη, βασίζονται επίσης στην
κοινωνιολογία, την επικοινωνία και την οργάνωση επιχειρήσεων. Επιπλέον,
οι νέες αυτές περιοχές προσαρμόζουν και χρησιμοποιούν μεθόδους και
τεχνικές έρευνας που έχουν αναπτυχθεί στις ανθρωπιστικές επιστήμες, για
να μελετήσουν και να σχεδιάσουν νέα διαδραστικά φαινόμενα, όπως είναι η
συνεργασία και η επικοινωνία ομάδων ανθρώπων μέσω ΗΥ, τόσο στο γραφείο,
όσο και στις νέες μορφές ΗΥ, στον κινητό και στον διάχυτο υπολογισμό.

Αν έπρεπε να διαλέξουμε μία μόνο συνεισφορά των συνεργατικών συστημάτων
στην κατανόηση της κατασκευής της διάδρασης, τότε αυτή θα ήταν η
ταξινόμηση των εφαρμογών σε δύο διαστάσεις: στον χώρο και στον χρόνο.
\footnote{(\textbf{Εικόνα?})~11 Χωροχρονική ταξινόμηση συνεργατικών
  συστημάτων (Public domain)} \footnote{(\textbf{Εικόνα?})~12 Ομότιμη
  αξιολόγηση προϊόντων (Amazon)} Στη διάσταση του χώρου, τα δύο άκρα της
κλίμακας ορίζονται από τη διά ζώσης και την εξ αποστάσεως επικοινωνία,
ενώ στη διάσταση του χρόνου, τα δύο άκρα της κλίμακας ορίζονται από τη
σύγχρονη και την ασύγχρονη επικοινωνία. Για παράδειγμα, το email είναι
μια μορφή ασύγχρονης, εξ αποστάσεως, επικοινωνίας, ενώ το chat είναι μια
σύγχρονη εξ αποστάσεως επικοινωνία. Στις παραπάνω βασικές διαστάσεις,
μπορούμε να προσθέσουμε, επίσης, τη διάσταση της λεκτικής και μη
λεκτικής επικοινωνίας, η οποία έχει γίνει πολύ δημοφιλής με τα εικονίδια
emoticons. Το κείμενο, ο ήχος και το βίντεο ανήκουν τόσο στα παραδοσιακά
μέσα επικοινωνίας, όσο και στη διαμεσολαβούμενη από υπολογιστή
επικοινωνία, στα οποία προστέθηκαν νέες εκφράσεις, όπως π.χ. η μη
λεκτική επικοινωνία με τα emoticons, τα οποία αρχικά σχηματίζονταν μόνο
με τη χρήση συμβόλων κειμένου.

Όλες οι γνώσεις και οι τεχνικές για την κατασκευή της διάδρασης μεταξύ
ανθρώπου και υπολογιστή ισχύουν για τον συνεργατικό παράγοντα, καθώς και
για το τεχνολογικό δίκτυο επικοινωνίας. Επιπλέον, πρέπει να σχεδιάσουμε
και να αναλύσουμε αυτά τα συστήματα λαμβάνοντας υπόψη τα παραπάνω.
Επομένως, η κατασκευή αυτών των συστημάτων είναι περισσότερο δύσκολη από
τη βασική περίπτωση όπου έχουμε έναν άνθρωπο και έναν υπολογιστή, αλλά
αποτελεί και μια δημιουργική-επιχειρηματική πρόκληση, όπως δείχνει το
φαινόμενο της μαζικής αποδοχής των κοινωνικών δικτύων. Το εύρος των
συνεργατικών συστημάτων καλύπτει τεχνολογίες όπως οι εικονικοί κόσμοι,
τα δικτυακά βίντεοπαιχνίδια, η τηλεδιάσκεψη, η ανταλλαγή αρχείων, οι
οποίες έχουν πολλές εφαρμογές τόσο στην εργασία, όσο και στην
εκπαίδευση, στην ψυχαγωγία και στην καθημερινότητα. Συνολικά, με τη
μεσολάβηση του υπολογιστή, όχι μόνο ως προσωπικού εργαλείου,\\
αλλά και ως μέσου επικοινωνίας και συνεργασίας με άλλους χρήστες,
ενθαρρύνεται ο διαμοιρασμός της γνώσης, των ικανοτήτων και των ιδεών. Τα
μέλη μιας ομάδας μπορούν να συζητήσουν και να συνεισφέρουν με μοναδικές
απόψεις σε ένα πρόβλημα, συνθέτοντας από κοινού μια λύση, την οποία δεν
θα μπορούσαν να δώσουν ατομικά ούτε τα ικανότερα μέλη της ομάδας, αφού
ακόμη και οι πιο μικρές προσθήκες μπορεί να δώσουν αξία στις αρχικές
προτάσεις.

Μια πολύ σημαντική εφαρμογή των υπολογιστών έχει να κάνει με τη
διευκόλυνση της επικοινωνίας και της συνεργασίας μικρών ομάδων ανθρώπων.
Για να κατασκευάσουμε ένα σύστημα που θα υποστηρίζει την εργασία σε
ομάδες, θα πρέπει να κατανοήσουμε τον ρόλο του κάθε μέλους της ομάδας
στις κοινές διεργασίες. Αν και η σημασία της κοινωνικής διάστασης της
συνεργασίας ήταν ήδη γνωστή σε συναφείς ερευνητικές περιοχές, όπως τα
πληροφοριακά συστήματα διοίκησης και η οργανωσιακή συμπεριφορά, η
εξειδικευμένη περιοχή των κοινωνικών και συνεργατικών συστημάτων
δημιουργήθηκε στα τέλη της δεκαετίας του 1980. Αρχικά, οι ερευνητές
ασχολήθηκαν με τις ανάγκες που προκύπτουν κατά τη συνεργασία στον χώρο
της εργασίας με επιτραπέζιους υπολογιστές και ενσύρματα δίκτυα. Στη
συνέχεια, το ενδιαφέρον τους στράφηκε προς τον κινητό υπολογισμό, τα
κοινωνικά δίκτυα, και τα δικτυακά βιντεο-παιχνίδια ρόλων. Τέλος, πρέπει
να τονιστεί ότι οι εφαρμογές αυτής της περιοχής δεν περιορίζονται πλέον
στο πεδίο της εργασίας, αλλά έχουν επεκταθεί σε πολλά ακόμη σημαντικά
πεδία κοινωνικής δραστηριότητας, όπως αυτό της εκπαίδευσης.

Η θεωρία για τη διάδραση ανθρώπου και υπολογιστή εμφανίζεται με
διαφορετικές μορφές σε πολλές διαφορετικές περιοχές, οι οποίες έχουν
σχετικά διαφορετικούς στόχους. Για παράδειγμα, η Εργονομία έχει εστιάσει
κυρίως στις σωματικές εργασίες των ανθρώπων που δουλεύουν με
βιομηχανικές μηχανές ή ρομπότ, στην είσοδο δεδομένων και στον χειρισμό
εξοπλισμού ασφαλείας, π.χ., σε αεροσκάφη και πλοία, σε συστήματα
ενέργειας, κτλ.. Από την άλλη πλευρά, η περιοχή των Πληροφοριακών
Συστημάτων Διοίκησης λειτουργεί σε μεγαλύτερη κλίμακα, εκεί όπου πολλοί
άνθρωποι συνεργάζονται ως μέλη επιμέρους ομάδων για να πάρουν αποφάσεις
και να πετύχουν κοινούς στόχους, στα πλαίσια ενός ή περισσότερων
διασυνδεδεμένων οργανισμών και ιεραρχικών δομών αποφάσεων. Ανάμεσα στα
δύο άκρα της κλίμακας της διάδρασης, δλδ. άνθρωπος-υπολογιστής και
οργανισμός που συντονίζει ομάδες ανθρώπων, βρίσκεται η σχετικά νεότερη
περιοχή των Κοινωνικών και Συνεργατικών Συστημάτων. Εκεί η διάδραση
συμβαίνει ανάμεσα στα μέλη μιας μικρής ομάδας ανθρώπων, η οποία
συντονίζεται με τη βοήθεια υπολογιστών. Φυσικά, ο διαχωρισμός ανάμεσα
στις παραπάνω περιοχές δεν είναι στεγανός και παρατηρούνται αρκετές
επικαλύψεις, όπως για παράδειγμα σε θέματα ιδιωτικότητας.

Αν και υπάρχει μια μικρή επικάλυψη της περιοχής των Συνεργατικών
Συστημάτων με το αντικείμενο μελέτης των Πληροφοριακών Συστημάτων
Διοίκησης, τα Συνεργατικά Συστήματα συνήθως δεν έχουν να κάνουν με τα
φαινόμενα μεγάλης κλίμακας που συμβαίνουν σε οργανισμούς και μεγάλες
ομάδες. Η ανάπτυξη και η μελέτη των συνεργατικών συστημάτων σε μεγάλη
κλίμακα αφορά, κυρίως, τον συντονισμό των χρηστών που συνεργάζονται εξ
αποστάσεως. Η ανάπτυξη του λογισμικού ανοικτού κώδικα ήταν μια από τις
πρώτες περιπτώσεις για τις οποίες μελετήθηκε η συνεργασία ομάδων μέσω
της τεχνολογίας, αλλά σίγουρα δεν είναι η μοναδική περίπτωση πλέον, αφού
υπάρχουν πολλά σύνθετα κοινωνικά και τεχνολογικά συστήματα που
λειτουργούν με παρόμοιο τρόπο, όπως για παράδειγμα η ανάπτυξη της
εγκυκλοπαίδειας Wikipedia, των χαρτών του OpenStreetMap, της
ψηφιοποίησης βιβλίων κ.ά. Επομένως, στην ανάπτυξη των πληροφοριακών
συστημάτων εστιάζουμε στον προγραμματισμό της διάδρασης σε μεγάλη
κλίμακα με τη συμμετοχή πολλών χρηστών. Για παράδειγμα, ακόμη και ένας
απλός επεξεργαστής κειμένου που απευθύνεται κυρίως σε έναν χρήστη μπορεί
να προσθέσει λειτουργικότητα μεγάλης κλίμακας, αν παρέχει συντονισμό με
άλλους χρήστες, οι οποίοι θα κάνουν έλεγχο του κειμένου ή μετάφραση σε
άλλες γλώσσες.

\leavevmode\vadjust pre{\hypertarget{fig:time-space-cscw}{}}%
\begin{figure}
\hypertarget{fig:time-space-cscw}{%
\centering
\includegraphics{images/time-space-cscw.jpg}
\caption{Εικόνα 11: Η ταξινόμηση των εφαρμογών στον χώρο (διά ζώσης ή εξ
αποστάσεως) και στον χρόνο (σύγχρονη ή ασύγχρονη) επιτρέπει την εύκολη
ταξινόμηση των κοινωνικών και συνεργατικών εφαρμογών και των λειτουργιών
τους. Επίσης, επιτρέπει τον εντοπισμό ευκαιριών για τη δημιουργία νέων
τύπων συστήματος.}\label{fig:time-space-cscw}
}
\end{figure}

\leavevmode\vadjust pre{\hypertarget{fig:social-reviews}{}}%
\begin{figure}
\hypertarget{fig:social-reviews}{%
\centering
\includegraphics{images/social-reviews.jpg}
\caption{Εικόνα 12: Τα δημοφιλή δικτυακά καταστήματα βασίζονται στις
αξιολογήσεις των προϊόντων και των προμηθευτών τους, τις οποίες κάνουν
οι ίδιοι οι χρήστες και λιγότερο στην προσπάθεια να διαλέξουν μετά από
κόπο μια σειρά προϊόντων, τα οποία οι πελάτες τους θα βρουν σίγουρα
αξιόπιστα.}\label{fig:social-reviews}
}
\end{figure}

Τα πληροφοριακά συστήματα διοίκησης ξεκίνησαν τη δεκαετία του 1970, ως
μια εκδοχή εφαρμοσμένης επιστήμης των υπολογιστών, αλλά στην πορεία
εξελίχθηκαν σε μια εκδοχή κοινωνικής επιστήμης με έμφαση στην εργασία
και στην οικονομική δραστηριότητα μέσω υπολογιστή. Ο βασικός πυλώνας
διαφοροποίησης αυτής της περιοχής από άλλες είναι η έμφαση στις
επιχειρήσεις και στη διοίκηση πόρων. Από τη στιγμή που η απευθείας
διάδραση με υπολογιστές εξαπλώθηκε σε ανθρώπινες δραστηριότητες πέρα από
την εργασία, τα πληροφοριακά συστήματα διοίκησης άρχισαν να
ενδιαφέρονται περισσότερο για τη διάδραση ανθρώπου και υπολογιστή και,
κυρίως, για τα κοινωνικά και συνεργατικά συστήματα. Για παράδειγμα, η
άνοδος του ηλεκτρονικού εμπορίου και γενικότερα των ηλεκτρονικών
συναλλαγών επιβάλλει μια καλύτερη κατανόηση των επιμέρους κανόνων που
καθορίζουν τη διάδραση ενός ανθρώπου με έναν υπολογιστή και, ειδικά για
την περίπτωση του εμπορίου με το ηλεκτρονικό καλάθι, επιβάλλεται η
καταγραφή αξιολογήσεων για προϊόντα και η μέτρηση της αξιοπιστίας των
πωλητών όταν αυτοί είναι απλοί χρήστες. Αντίστοιχα, η καθιέρωση της
ηλεκτρονικής συνεργασίας των ανθρώπων, ακόμα και όταν αυτοί βρίσκονται
στο ίδιο δωμάτιο, πόσο μάλλον όταν βρίσκονται σε μεγάλες αποστάσεις,
επέβαλε την ενασχόληση με τα Κοινωνικά και Συνεργατικά Συστήματα, τα
οποία έχουν μελετήσει το φαινόμενο της επικοινωνίας των ανθρώπων μέσω
υπολογιστή στην κλίμακα της μικρής ομάδας.

Τα Πληροφοριακά Συστήματα Διοίκησης είναι μια συγγενής περιοχή των
Συνεργατικών Συστημάτων, αφού και στις δύο περιπτώσεις η μελέτη εστιάζει
σε ομάδες ανθρώπων και στην εργασία με τον υπολογιστή. Από την άλλη
πλευρά, οι δύο αυτές περιοχές έχουν περισσότερες διαφορές παρά
ομοιότητες, τουλάχιστον αναφορικά με την κλίμακα και το είδος των
φαινομένων που μελετούν. Τα πληροφοριακά συστήματα διοίκησης έχουν ως
αντικείμενο μελέτης μεγάλες ομάδες εργασίας και οργανισμούς, ενώ ο ρόλος
του υπολογιστή και του λογισμικού είναι συμπληρωματικός και σε καμία
περίπτωση δεν είναι ισάξιος με τον ρόλο των ανθρώπων και των ομάδων
εργασίας. Αντίθετα, τα συνεργατικά συστήματα εστιάζουν στην ομάδα
εργασίας ανθρώπων και υπολογιστών σε μικρή κλίμακα και έχουν μεγαλύτερο
ενδιαφέρον για την κατασκευή και τη χρήση του λογισμικού διάδρασης. Για
παράδειγμα, είναι χαρακτηριστικό ότι ο όρος \emph{υλοποίηση} στην
περίπτωση των πληροφοριακών συστημάτων διοίκησης χρησιμοποιείται με
αναφορά στην εισαγωγή κάποιου λογισμικού στον οργανισμό και όχι με
αναφορά στην κατασκευή του. Ακόμη, τα μεν πρώτα εστιάζουν περισσότερο
στις επιπτώσεις σε οικονομικές μετρικές του οργανισμού, ενώ τα δεύτερα
σε μετρικές που επηρεάζουν την σχεδίαση του λογισμικού.

Αν και γίνεται μεγάλη προσπάθεια για πάρα πολλές δεκαετίες να
δημιουργηθεί ένας υπολογιστής με τεχνητή νοημοσύνη παρόμοια με του
ανθρώπου, στην πράξη τα καλύτερα αποτελέσματα και σε πολύ μικρότερο
χρόνο τα έχουμε πετύχει όταν μεγάλες ομάδες ανθρώπων συνεργάζονται με
έμμεσο τρόπο για να λύσουν ένα δύσκολο πρόβλημα. \footnote{Licklider
  (1960)} Σε ένα πρόσφατο παράδειγμα, το σύστημα Captcha χρησιμοποιείται
σε πρώτο επίπεδο για να διασφαλίσει ότι ο χρήστης είναι άνθρωπος και όχι
υπολογιστής, αλλά στην πράξη η αναγνώριση του κειμένου χρησιμοποιείται
σε ένα δεύτερο επίπεδο για την ψηφιοποίηση βιβλίων. Ομοίως, η αναγνώριση
αντικειμένων μέσα σε μια φωτογραφία μπορεί να γίνει αν έχουμε δύο
ανθρώπους που ανταγωνίζονται ποιος θα αναγνωρίσει τα πιο πολλά
αντικείμενα μέσα σε μια φωτογραφία, με αποτέλεσμα την καταγραφή αυτών
των αντικειμένων στα οποία δύο ή περισσότερα ζευγάρια χρηστών συμφωνούν.
Βλέπουμε, λοιπόν, ότι μια χρήσιμη κατεύθυνση συμπληρωματική της
προσπάθειας για αυτόνομη τεχνητή νοημοσύνη είναι η προσπάθεια να
οργανώσουμε και να καταμερίσουμε σε πολλούς ανθρώπους δύσκολα προβλήματα
με τρόπο ευχάριστο ή τουλάχιστον έμμεσο. \footnote{(\textbf{Εικόνα?})~13
  Πληθοπορισμός (Carnegie Mellon University)} Η κατασκευή της διάδρασης
αυτών των συστημάτων είναι ένα σημαντικό κεφάλαιο στα συστήματα
συνεργασίας μεγάλης κλίμακας.

Τα ψηφιακά προϊόντα αρχικά αναπτύσσονταν μόνο από μεγάλες εταιρείες,
γιατί μόνο αυτές είχαν τους αντίστοιχους πόρους (π.χ. οικονομικούς,
ανθρώπινους, τεχνογνωσία), αλλά σταδιακά η καινοτομία πέρασε και στις
μικρότερες εταιρείες, ενώ, πλέον, ακόμη και η χρηματοδότηση μπορεί να
γίνει από το πλήθος (πληθοπορισμός), όπως στην περίπτωση του έξυπνου
ρολογιού pebble που κατασκευάστηκε μόνο από έναν σχεδιαστή, ο οποίος,
όμως, φρόντισε να κάνει τον κατάλληλο καταμερισμό της εργασίας και,
κυρίως, τον καταμερισμό της ανάγκης χρηματοδότησης. Το μοντέλο του
πληθοπορισμού της χρηματοδότησης για νέα προϊόντα έχει εφαρμοστεί επίσης
με επιτυχία στην περίπτωση της παραγωγής ψυχαγωγικού περιεχομένου. Το
βασικό πλεονέκτημα που έχει το μοντέλο της χρηματοδότησης μέσω του
πληθοπορισμού είναι ότι παρέχει μια πολύ γρήγορη ανάδραση για το
πραγματικό ενδιαφέρον των χρηστών να αγοράσουν, οπότε αν αυτό είναι
μικρό μεταφράζεται σε μη χρηματοδότηση, άρα ο δημιουργός μπορεί να
περάσει γρήγορα στην επόμενη ιδέα. \footnote{(\textbf{Εικόνα?})~14
  Χρηματοδότηση από το πλήθος (Kickstarter)}

\leavevmode\vadjust pre{\hypertarget{fig:crowdsourcing}{}}%
\begin{figure}
\hypertarget{fig:crowdsourcing}{%
\centering
\includegraphics{images/crowdsourcing.jpg}
\caption{Εικόνα 13: Η διάδραση σε μεγάλη κλίμακα μπορεί να
χρησιμοποιηθεί για να λύσει δύσκολα προβλήματα, τα οποία οι υπολογιστές
δεν μπορούν να τα αντιμετωπίσουν από μόνοι τους, όπως είναι η αναγνώριση
αντικειμένων σε μια φωτογραφία.}\label{fig:crowdsourcing}
}
\end{figure}

\leavevmode\vadjust pre{\hypertarget{fig:kickstarter-pebble}{}}%
\begin{figure}
\hypertarget{fig:kickstarter-pebble}{%
\centering
\includegraphics{images/kickstarter-pebble.png}
\caption{Εικόνα 14: Το έξυπνο ρολόι pebble κατάφερε να πουλήσει το πρώτο
εκατομμύριο κομμάτια το ίδιο γρήγορα με το πρώτο Apple iPod, με τη
διαφορά ότι το pebble σχεδιάστηκε από μια μικρή ομάδα και
χρηματοδοτήθηκε από χιλιάδες δυνητικούς χρήστες και όχι από επενδυτές
που έχουν ως κίνητρο μόνο το κέρδος.}\label{fig:kickstarter-pebble}
}
\end{figure}

\hypertarget{ux3b7-ux3c0ux3b5ux3c1ux3afux3c0ux3c4ux3c9ux3c3ux3b7-ux3c4ux3bfux3c5-ux3c0ux3b1ux3b3ux3baux3ccux3c3ux3bcux3b9ux3bfux3c5-ux3b9ux3c3ux3c4ux3bfux3cd}{%
\subsection{Η περίπτωση του παγκόσμιου
ιστού}\label{ux3b7-ux3c0ux3b5ux3c1ux3afux3c0ux3c4ux3c9ux3c3ux3b7-ux3c4ux3bfux3c5-ux3c0ux3b1ux3b3ux3baux3ccux3c3ux3bcux3b9ux3bfux3c5-ux3b9ux3c3ux3c4ux3bfux3cd}}

Η κατασκευή του συστήματος World Wide Web (WWW) και, κυρίως, η πολύ
γρήγορη αποδοχή του από ένα μεγάλο εύρος χρηστών ήταν μια εξέλιξη
κομβικής σημασίας για την ανάπτυξη και την ολοκλήρωση της διάδρασης με
τον επιτραπέζιο υπολογιστή. Σε πλήρη αναλογία με την αρχική κατασκευή
της διάδρασης με την επιφάνεια εργασίας του επιτραπέζιου υπολογιστή (με
σκοπό τη διευκόλυνση της εκδοτικής εργασίας), έτσι και το σύστημα WWW
σχεδιάστηκε για να διευκολύνει τον διαμοιρασμό επιστημονικών
δημοσιεύσεων. Η αναλογία του WWW με το Desktop συνεχίζεται και στη
μετεξέλιξή τους, αφού και τα δύο μετασχηματίστηκαν και προσαρμόστηκαν
για να εξυπηρετήσουν τις ανάγκες των χρηστών και σε άλλες εφαρμογές,
όπως η επικοινωνία, η διασκέδαση, η εκπαίδευση κτλ.

Η ιστορία του WWW ξεκίνησε με μια ασύμμετρα χαμηλή αποδοχή όταν η
περιγραφή του συστήματος πέρασε σχεδόν απαρατήρητη από την επιστημονική
κοινότητα, αφού υπήρχαν ήδη αντίστοιχα συστήματα που ήταν περισσότερο
πλήρη, όπως η Standard Generalized Markup Language (SGML). Αν και η
γλώσσα Hyper-Text Markup Language (HTML) δεν θεωρήθηκε ως επιστημονική
πρόοδος, έγινε διαθέσιμη τη σωστή στιγμή, ήταν εύχρηστη και είχε όσες
λειτουργίες ήταν χρήσιμες τότε, με αποτέλεσμα να πετύχει μεγάλη αποδοχή
από μια ευρύτατη ομάδα χρηστών σε πολύ μικρό χρονικό διάστημα. Αυτό το
γεγονός από μόνο του αποτελεί ένα σημαντικό παράδειγμα της περιοχής του
προγραμματισμού της διάδρασης, όπου δεν κερδίζει ούτε η καλύτερη
κατασκευή ούτε η καλύτερη σχεδίαση, αλλά η πιο κατάλληλη και η πιο
προσαρμόσιμη στις συνεχώς μεταλλασσόμενες τεχνολογικές ανάγκες των
χρηστών. Η ανάπτυξη του WWW τόσο από την πλευρά των χρηστών που έβαζαν
περιεχόμενο, όσο και από την πλευρά των κατασκευαστών που ανέπτυσσαν τις
τεχνολογίες έγινε με τρόπο περισσότερο οργανικό, παρά με τρόπο ιεραρχικό
και συντονισμένο βάσει κάποιων κοινά συμφωνημένων προδιαγραφών, οι
οποίες προέρχονταν από συστηματική ανάλυση των αναγκών των χρηστών.

Πράγματι, το σύστημα WWW αποδείχτηκε εξαιρετικά προσαρμόσιμο σε νέα
πεδία εφαρμογών, όπως το ηλεκτρονικό εμπόριο. Αν και αρχικά το σύστημα
WWW σχεδιάστηκε για τον διαμοιρασμό επιστημονικών εργασιών, χάρη στην
πολύ απλή δηλωτική γλώσσα οργάνωσης και μορφοποίησης της πληροφορίας, σε
πολύ σύντομο χρονικό διάστημα γνώρισε την αποδοχή χρηστών που ήθελαν να
μοιραστούν με τον υπόλοιπο κόσμο κάθε λογής πληροφορία. Πριν συμπληρώσει
την πρώτη δεκαετία της ζωής του, το σύστημα WWW με τη βοήθεια
τεχνολογικών επεκτάσεων έδωσε τη δυνατότητα, εκτός από την ανάκτηση
πληροφορίας, και για τη διενέργεια ηλεκτρονικών συναλλαγών. Φυσικά, η
γλώσσα SGML, αν και φαινομενικά χάθηκε, στην πορεία αποτέλεσε τη βάση
για την XML που έγινε μέρος της XHTML και καθόρισε την HTML5, με την
οποία το WWW απέκτησε πλέον τη μορφή πλατφόρμας εκτέλεσης γενικών
υπολογιστικών εφαρμογών και όχι απλών εφαρμογών του Internet όπως ήταν ο
αρχικός ρόλος του.

\leavevmode\vadjust pre{\hypertarget{fig:www}{}}%
\begin{figure}
\hypertarget{fig:www}{%
\centering
\includegraphics{images/www.png}
\caption{Εικόνα 15: Οι πρώτες σελίδες του WWW είχαν στόχο να βελτιώσουν
την ταχύτητα της δημοσίευσης επιστημονικών άρθρων μέσω του Internet, το
οποίο είχε ήδη παρόμοιες υπηρεσίες όπως το FTP, ενώ η ανάπτυξη τόσο του
εξυπηρετητή όσο και του φυλλομετρητή για το WWW αναπτύχθηκαν από έναν
άνθρωπο τον Tim Berners Lee, ο οποίος εργαζόταν στην τεχνολογική
υποστήριξη του ερευνητικού κέντρου CERN.}\label{fig:www}
}
\end{figure}

\leavevmode\vadjust pre{\hypertarget{fig:amaya-web-editor}{}}%
\begin{figure}
\hypertarget{fig:amaya-web-editor}{%
\centering
\includegraphics{images/amaya-web-editor.jpg}
\caption{Εικόνα 16: Το σύστημα Amaya παρέχει ολοκληρωμένη πρόβαση στον
παγκόσμιο ιστό, όχι μόνο ως φυλλομετρητής, αλλά, κυρίως, ως ένα
περιβάλλον επεξεργασίας εγγράφων για
δημοσίευση.}\label{fig:amaya-web-editor}
}
\end{figure}

Η σημαντικότερη εξέλιξη του συστήματος WWW πραγματοποιήθηκε κατά την
πρώτη δεκαετία του 2000, όταν ο συνδυασμός του δυναμικού προγραμματισμού
στις τεχνολογίες του εξυπηρετητή και του φυλλομετρητή επέτρεψαν την
ανάπτυξη πλήρως λειτουργικών εφαρμογών χρήστη. Αν και τα πρώτα δημοφιλή
παραδείγματα δυναμικών εφαρμογών ήταν αυτά των ηλεκτρονικών συναλλαγών
(π.χ. ebay, paypal) και του ηλεκτρονικού καλαθιού (π.χ. Amazon), η
δικτυακή εφαρμογή που επαναπροσδιόρισε την αντίληψη που έχουμε για το
WWW ήταν το Google Mail (Gmail). Το Gmail προσφέρει όλες τις λειτουργίες
του ηλεκτρονικού ταχυδρομείου χωρίς να υπάρχει ανάγκη για την αντίστοιχη
εφαρμογή που τρέχει πάνω από το λειτουργικό σύστημα. Με αυτόν τον τρόπο
το Gmail αποτέλεσε το πρώτο μεγάλο βήμα για την αντιμετώπιση του απλού
μέχρι τότε φυλλομετρητή ως λειτουργικού συστήματος και του WWW ως
πλατφόρμας δικτυακού υπολογισμού (cloud computing).

Η καθιέρωση του συστήματος WWW συνεχίστηκε με ακόμη πιο έντονους ρυθμούς
και μετά το 2005, με την ολοκλήρωση όλο και περισσότερων λειτουργιών, με
περισσότερο χαρακτηριστικές περιπτώσεις το YouTube, το Facebook και το
Twitter, που σε μεγάλο βαθμό επαναπροσδιόρισαν το σύστημα WWW
περισσότερο ως ένα ευέλικτο μέσο επικοινωνίας παρά ως ένα απλό εργαλείο
για τον διαμοιρασμό επιστημονικών δημοσιεύσεων, όπως ήταν ο αρχικός
σκοπός του. Ειδικά η ραγδαία ανάπτυξη και αποδοχή του Facebook αποτελεί
ένα παράδειγμα ανάλογο του ίδιου του WWW, όπου ένα τεχνολογικό σύστημα,
χωρίς να έχει κάποια ιδιαίτερη τεχνολογική ή σχεδιαστική υπεροχή έναντι
του ανταγωνισμού, προσαρμόζεται και εξυπηρετεί τις ανάγκες των χρηστών
του, ενώ διαχρονικά και σταδιακά εξελίσσεται το ίδιο σε πλατφόρμα
εφαρμογών.

Οι τεχνολογικές εξελίξεις μετά το 2010 έδωσαν στο αρχικό σύστημα WWW τη
δυνατότητα ανάπτυξης ενός πολύ μεγάλου εύρους εφαρμογών χρήστη (π.χ.
ανταλλαγή μηνυμάτων, παιχνίδια), σε βαθμό τέτοιο που να μπορούμε πλέον
να μιλάμε για μια πλατφόρμα ανάπτυξης αντίστοιχη με τα μέχρι τότε
διαδεδομένα επιτραπέζια λειτουργικά συστήματα. Πράγματι, ο
μετασχηματισμός της αντίληψης που έχουμε για το WWW, αφού έκανε το πρώτο
βήμα με τις εφαρμογές (π.χ. Wikipedia, Gmail), σταδιακά μετατράπηκε σε
πλατφόρμα για την εκτέλεση εφαρμογών, όπου ο φυλλομετρητής αντικαθιστά
το λειτουργικό σύστημα (π.χ. ChromeOS) και ο χρήστης έχει στην διάθεσή
του δημοφιλείς εφαρμογές επικοινωνίας και γραφείου.

Συνολικά, η ερευνητική μελέτη περίπτωσης του συστήματος World Wide Web
(WWW) αποτελεί μια σύνθεση των διαφορετικών θεωρήσεων της διάδρασης,
εξίσου ενδιαφέρουσα με εκείνη του επιτραπέζιου υπολογιστή. Όπως και ο
επιτραπέζιος υπολογιστής έτσι και το σύστημα WWW, μέσα στο χρονικό
διάστημα μίας δεκαετίας, μετατράπηκε από ένα απλό σύστημα διαμοιρασμού
ερευνητικών δημοσιεύσεων σε μια υπολογιστική πλατφόρμα πάνω στην οποία
μπορούν να εκτελεστούν πολύ διαφορετικές εφαρμογές, από ψυχαγωγικά
παιχνίδια, μέχρι εφαρμογές γραφείου, επικοινωνίας και εμπορικών
συναλλαγών. Τελικά, ο φυλλομετρητής του παγκόσμιου ιστού μετατράπηκε από
μια απλή εφαρμογή σε ένα λειτουργικό σύστημα με τις δικές του εφαρμογές,
όπως φαίνεται στην περίπτωση του ChromeOS.

\hypertarget{ux3b7-ux3c0ux3b5ux3c1ux3afux3c0ux3c4ux3c9ux3c3ux3b7-ux3c4ux3b7ux3c2-wikipedia}{%
\subsection{Η περίπτωση της
Wikipedia}\label{ux3b7-ux3c0ux3b5ux3c1ux3afux3c0ux3c4ux3c9ux3c3ux3b7-ux3c4ux3b7ux3c2-wikipedia}}

Η συμμετοχική ανάπτυξη και η ευέλικτη διανομή και αλλαγή ενός προϊόντος
για υπολογιστές ξεκίνησε με την περίπτωση του λογισμικού ανοικτού
κώδικα. Όμως, η διαδικασία αυτή έγινε δημοφιλής και γνωστή με την
περίπτωση της δικτυακής εγκυκλοπαίδειας Wikipedia. Προσπάθειες παρόμοιες
με της Wikipedia, όπως εκείνη της Nupedia, η οποία είχε ένα πιο
ιεραρχικό και δομημένο μοντέλο δημοσίευσης και ενημέρωσης άρθρων,
παρόμοιο με αυτό μια έντυπης εγκυκλοπαίδειας, δεν πέτυχαν στην πράξη.
Βλέπουμε, λοιπόν, ότι η κατασκευή της διάδρασης είναι κάτι περισσότερο
από τη μετατροπή μιας υπάρχουσας διαδικασίας σε ψηφιακή, αφού
τουλάχιστον σε κάποιες περιπτώσεις, όπως της Wikipedia, απαιτεί και την
υιοθέτηση ενός νέου παραγωγικού μοντέλου, το οποίο έχει και αντίστοιχα
διαφορετικές προδιαγραφές λειτουργίας και χρήσης. Η αποτελεσματικότητα
της Wikipedia ήταν τόσο μεγάλη που μέσα σε λιγότερο από δέκα χρόνια
οδήγησε την Encyclopaedia Britannica στην κατάργηση της έντυπης έκδοσης.
Φυσικά, όπως και οι παραδοσιακές εγκυκλοπαίδειες έτσι και η Wikipedia
έχει δεχτεί κριτική για κάποια άρθρα της, με τη διαφορά όμως ότι στη
Wikipedia η διαδικασία αποσαφήνισης είναι ανοικτή και τεκμηριωμένη.

Το προϊόν της Wikipedia βασίζεται σε δύο πυλώνες που είναι καινοτομικοί
σε σχέση με την κυρίαρχη πρακτική εκείνης της εποχής (2000) για την
παραγωγή και διανομή ψηφιακού περιεχομένου. Πρώτον, οι τελικοί χρήστες
δημιουργούν και συντηρούν το περιεχόμενο της εγκυκλοπαίδειας, και
δεύτερον, η επεξεργασία ενός άρθρου γίνεται με το πρόγραμμα
περιήγησης.\footnote{(\textbf{Εικόνα?})~17 Συνεργατική επεξεργασία
  εγγράφων στην Wikipedia (Wikipedia)} Αν και αυτές οι δύο πρακτικές
μπορεί να φαίνονται σήμερα κοινή λογική, εκείνη την εποχή η λογική αυτή
είχε αποδοχή μόνο σε έναν πολύ μικρό κύκλο προγραμματιστών ανοικτού
κώδικα, όπως ο Richard Stallman. Μάλιστα, στην πρώτη προσπάθεια
δημιουργίας μιας ελεύθερης δικτυακής εγκυκλοπαίδειας, οι ίδιοι οι
δημιουργοί της Wikipedia προσπάθησαν χωρίς επιτυχία (π.χ. Nupedia) να
χρησιμοποιήσουν ειδικούς ανά θεματική κατηγορία και μια αυστηρή
διαδικασία ελέγχου των άρθρων, ακριβώς όπως έκαναν οι έντυπες
εγκυκλοπαίδειες. Ταυτόχρονα, οι περισσότεροι χρήστες και παραγωγοί
περιεχομένου για τον ιστό είχαν την αντίληψη ότι άλλο είναι το πρόγραμμα
περιήγησης που απλά διαβάζει μια σελίδα από τον εξυπηρετητή και άλλο
είναι το πρόγραμμα επεξεργασίας που δημιουργεί και αλλάζει μια σελίδα
για να τη βάλει προς διανομή στον εξυπηρετητή.

H Wikipedia είναι ένα σπουδαίο παράδειγμα προγραμματισμού της διάδρασης
για τους παρακάτω λόγους: Πρώτον, βλέπουμε για μια ακόμη φορά (π.χ.
Facebook) ότι ένα σύστημα διάδρασης ανθρώπου και υπολογιστή μπορεί να
γίνει πολύ επιτυχημένο, όχι επειδή είναι το πιο προηγμένο τεχνολογικά,
αλλά επειδή είναι αυτό που εξυπηρετεί με απλό τρόπο τις πραγματικές
ανάγκες των χρηστών (π.χ. επεξεργασία περιεχομένου χωρίς την ανάγκη
πρόσθετου λογισμικού) και μάλιστα την κατάλληλη χρονική στιγμή.
Δεύτερον, το μοντέλο λογισμικού τύπου wiki, μετά την επιτυχία που είχε
με τη Wikipedia, χρησιμοποιήθηκε σε πάρα πολλές άλλες περιπτώσεις, όπως
στην κατασκευή προσωπικών ιστοσελίδων από τους τελικούς χρήστες (π.χ.
Wordpress). Τέλος, αν και μια εγκυκλοπαίδεια που δίνει εύκολη πρόσβαση
στην ανθρώπινη γνώση σε πολλές γλώσσες είναι από μόνη της ένα σπουδαίο
προϊόν, αυτό που κάνει πραγματικά ξεχωριστή τη Wikipedia είναι ότι
αποτελεί την καλύτερη απόδειξη ότι ένα απλό λογισμικό και μερικοί κοινώς
αποδεκτοί κανόνες συνεργασίας (π.χ. αρχείο αλλαγών και περιοχή
συζήτησης) μπορούν να αντικαταστήσουν τη μέχρι τότε ιεραρχική και
κερδοσκοπική διαδικασία παραγωγής προϊόντων περιεχομένου.

\leavevmode\vadjust pre{\hypertarget{fig:wikipedia-edit}{}}%
\begin{figure}
\hypertarget{fig:wikipedia-edit}{%
\centering
\includegraphics{images/wikipedia-edit.png}
\caption{Εικόνα 17: Η εύκολη και καθολική συμμετοχή των χρηστών στη
δημιουργία και ενημέρωση του περιεχομένου γίνεται με τη βοήθεια
λογισμικού τύπου wiki, το οποίο επιτρέπει την επιτόπου επεξεργασία ενός
άρθρου μέσα στο λογισμικό περιηγητή του
χρήστη.}\label{fig:wikipedia-edit}
}
\end{figure}

\leavevmode\vadjust pre{\hypertarget{fig:open-street-map}{}}%
\begin{figure}
\hypertarget{fig:open-street-map}{%
\centering
\includegraphics{images/open-street-map.jpg}
\caption{Εικόνα 18: Η λογική της συμμετοχικής δημιουργίας και συντήρησης
σημαντικού περιεχομένου, όπως στην εγκυκλοπαίδεια Wikipedia, εφαρμόστηκε
με επιτυχία και σε άλλους τομείς, όπως οι χάρτες, οι οποίοι σε μια
περίπτωση φυσικής καταστροφής ενημερώθηκαν πιο γρήγορα και πιο
αποτελεσματικά από τους αντίστοιχους
ιδιωτικούς.}\label{fig:open-street-map}
}
\end{figure}

Το παραπάνω παράδειγμα λειτουργίας (καθολική συμμετοχή και συνεργατικό
λογισμικό) της Wikipedia εφαρμόστηκε σε πολλές ακόμη περιπτώσεις
συνεργατικής παραγωγής περιεχομένου, με ιδιαίτερη έμφαση στις
περιπτώσεις εκείνες που οι κερδοσκοπικοί οργανισμοί δεν είχαν κίνητρα
συμμετοχής (π.χ. Ushahidi). Ένα ακόμη πολύ επιτυχημένο παράδειγμα είναι
η παραγωγή χαρτογραφικής πληροφόρησης και ειδικά η προσπάθεια του
OpenStreetMap,\footnote{(\textbf{Εικόνα?})~18 Ψηφιακοί γεωγραφικοί
  χάρτες ανοιχτού κώδικα (OpenStreetMap)} το οποίο επιτρέπει σε όλους
τους χρήστες να εισάγουν δρόμους και χαρτογραφική πληροφορία σε μια
κοινή περιοχή, με σκοπό να δημιουργηθεί μια εναλλακτική στα κυρίαρχα
προϊόντα, όπως το Google Maps, τα οποία ελέγχονται από κερδοσκοπικές
εταιρείες. Για τους δημιουργούς των αντίστοιχων συστημάτων ανοικτής
πρόσβασης, τα κίνητρα συμμετοχής τους είναι, συνήθως, κάτι περισσότερο
από μια φιλοσοφική άρνηση του μοντέλου λειτουργίας των κερδοσκοπικών
επιχειρήσεων, αφού ο στόχος τους είναι περισσότερο η πρόσβαση σε μια
διαφορετική αντίληψη στην παραγωγή του περιεχομένου. Άλλωστε, ακόμη και
οι κυρίαρχες πλατφόρμες χαρτών αποδέχτηκαν την αδυναμία τους να καλύψουν
τις επιμέρους ανάγκες που συνέχεια προκύπτουν, και έδωσαν (από το 2013)
τη δυνατότητα στους χρήστες να κάνουν προτάσεις για αλλαγές στους χάρτες
τους (π.χ. Google Map Maker, Here Map Creator).

\hypertarget{ux3c3ux3cdux3bdux3c4ux3bfux3bcux3b7-ux3b2ux3b9ux3bfux3b3ux3c1ux3b1ux3c6ux3afux3b1-ux3c4ux3bfux3c5-ted-nelson}{%
\subsection{Σύντομη βιογραφία του Ted
Nelson}\label{ux3c3ux3cdux3bdux3c4ux3bfux3bcux3b7-ux3b2ux3b9ux3bfux3b3ux3c1ux3b1ux3c6ux3afux3b1-ux3c4ux3bfux3c5-ted-nelson}}

O Ted Nelson αναφέρεται συχνά ως ένας από τους θεωρητικούς του
παγκόσμιου ιστού, αλλά αυτό είναι μόνο εν μέρει σωστό. Πράγματι
ασχολήθηκε με τα ψηφιακά έγγραφα και, κυρίως, με τη φύση των
υπερσυνδέσμων ανάμεσα τους, έτσι ώστε να είναι πάντα σαφές το νόημα ενός
κειμένου που είναι παράθεση από άλλο κείμενο. \footnote{(\textbf{Εικόνα?})~19
  Ted Nelson (The Transcopyright License)} Για αυτόν τον σκοπό η
σχεδίαση του επιβάλλει την ύπαρξη αμφίδρομων υπερσυνδέσμων, οι οποίοι
εμπλουτίζουν και το νόημα της αρχικής πηγής, σε αντίθεση με τον σύγχρονο
παγκόσμιο ιστό που βασίζεται σε μονόδρομους υπερσυνδέσμους.

Οι μονόδρομοι υπερσύνδεσμοι είναι σίγουρα πολύ απλοί στην υλοποίηση,
αλλά τελικά έχουν ένα πολύ μεγάλο κόστος μακροπρόθεσμα, το οποίο δεν
είναι μόνο οικονομικό. Το οικονομικό κόστος αφορά την ανάγκη να
δημιουργηθούν εμπορικές υπηρεσίες οι οποίες θα παρέχουν την
λειτουργικότητα των αμφίδρομων συνδέσμων όπως είναι οι Google και
Faceobok, για τις περιπτώσεις των ψηφιακών εγγράφων και κοινωνικών
δικτύων αντίστοιχα. Η συγκέντρωση της οικονομικής ισχύος σε μοναδικούς
πυλώνες ενισχύει την οικονομική ανισότητα και ταυτόχρονα μειώνει την
πολυφωνία και τελικά απειλεί τη δημοκρατία, ειδικά σε μια κοινωνία που
λειτουργεί πάνω σε ατελή ψηφιακά μέσα.

Όπως και ο καλός του φίλος ο Douglas Engelbart, ο οποίος έμεινε γνωστός
περισσότερο για το ποντίκι, έτσι ακριβώς και ο Ted Nelson είναι γνωστός
ως ο δημιουργός του πλήκτρου της επιστροφής από ένα ψηφιακό έγγραφο στο
προηγούμενο, το οποίο, όμως, δεν είναι παρά μια μικρή τεχνολογική
συνεισφορά σε ένα πολύ ευρύτερο όραμα. \footnote{Nelson (1974)} Το όραμά
του βασίζεται στη θεώρηση των ψηφιακών εγγράφων ως ένα νέο μέσο και όχι
ως μια προσομοίωση των φυσικών εγγράφων που τυπώνονται σε χαρτί. Σε αυτό
το όραμα ο υπολογιστής δεν είναι απλά ένα εργαλείο, αλλά ένα νέο μέσο,
το οποίο διατηρεί το ιστορικό για κάθε έγγραφο που είναι μοναδικό, έτσι
ώστε να μην υπάρχει η ανάγκη για αντιγραφή και επικόλληση, αλλά μόνο για
δημιουργία υπερσυνδέσμων που αυξάνουν το νόημα αλλά και την αξία κάθε
εγγράφου, είτε αυτό είναι κείμενο, είτε πολυμεσικό υλικό.

\leavevmode\vadjust pre{\hypertarget{fig:nelson-profile}{}}%
\begin{figure}
\hypertarget{fig:nelson-profile}{%
\centering
\includegraphics{images/nelson-profile.jpg}
\caption{Εικόνα 19: Το έργο του Ted Nelson αντανακλά την κατασκευή της
διάδρασης, όπου υπάρχουν πολλές εναλλακτικές ιδέες για την οργάνωση και
την οπτικοποίηση της πληροφορίας και τελικά επικρατούν κάποιες που ήταν
ευκολότερο να υλοποιηθούν και να διαδοθούν γρήγορα ώστε να γίνουν
οικείες. Ταυτόχρονα, η σύνθεση γνώσεων και δεξιοτήτων του Ted Nelson
δείχνει τη δυσκολία του αντικειμένου, όπου απαιτούνται τόσο
ανθρωπιστικές όσο και τεχνολογικές ιδιότητες.}\label{fig:nelson-profile}
}
\end{figure}

\leavevmode\vadjust pre{\hypertarget{fig:xanadu-viewer3d}{}}%
\begin{figure}
\hypertarget{fig:xanadu-viewer3d}{%
\centering
\includegraphics{images/xanadu-viewer3d.png}
\caption{Εικόνα 20: Σε μια από τις πολλές προσπάθειες υλοποίησης μιας
διεπαφής για το Xanadu δοκιμάστηκαν τα γραφικά τριών διαστάσεων και η
πλοήγηση με διεπαφή μεγέθυνσης. Για αυτήν την περίπτωση, το κείμενο ήταν
η Παλαιά Διαθήκη, η οποία συνοδεύεται από πολλές εναλλακτικές σημειώσεις
και ερμηνείες, έτσι ώστε ο αναγνώστης να μπορεί να τις μελετήσει
συγκριτικά.}\label{fig:xanadu-viewer3d}
}
\end{figure}

Το έργο του Ted Nelson θέτει θεμελιώδη ερωτήματα στην τεχνολογία
λογισμικού και στα διαδραστικά συστήματα, \footnote{Nelson (2010)} γιατί
είναι από τους λίγους που συμμετείχε στις συζητήσεις δημιουργίας των
πρώτων προσωπικών υπολογιστών και είχε άποψη ορμώμενος από τις
ανθρωπιστικές επιστήμες και τα ανθρωπιστικά ιδεώδη. Αν και ο ίδιος δεν
κατάφερε να υλοποιήσει σε μεγάλη κλίμακα το όραμα του, οι ιδέες του για
τη φύση των υπολογιστών ως μέσων επικοινωνίας και έκφρασης αποτελούν
προδιαγραφές για μελλοντικά διαδραστικά συστήματα,\footnote{(\textbf{Εικόνα?})~20
  Xanadu μεγέθυνσης σε τρεις διαστάσεις (The Transcopyright License)} τα
οποία δεν θα είναι απλά μια προσομοίωση της πραγματικότητας.

\hypertarget{ux3b2ux3b9ux3b2ux3bbux3b9ux3bfux3b3ux3c1ux3b1ux3c6ux3afux3b1}{%
\subsection*{Βιβλιογραφία}\label{ux3b2ux3b9ux3b2ux3bbux3b9ux3bfux3b3ux3c1ux3b1ux3c6ux3afux3b1}}
\addcontentsline{toc}{subsection}{Βιβλιογραφία}

\hypertarget{refs}{}
\begin{CSLReferences}{0}{0}
\end{CSLReferences}

Baecker, Ronald M. 1993. \emph{Readings in Groupware and
Computer-Supported Cooperative Work: Assisting Human-Human
Collaboration}. Elsevier.

Barnet, Belinda. 2013. \emph{Memory Machines: The Evolution of
Hypertext}. Anthem Press.

Berners-Lee, Tim. 1996. {``WWW: Past, Present, and Future.''}
\emph{Computer} 29 (10): 69--77.

Bolt, Richard A. 1978. {``Spatial Data Management System.''}
MASSACHUSETTS INST OF TECH CAMBRIDGE ARCHITECTURE MACHINE GROUP.

Bush, Vannevar et al. 1945. {``As We May Think.''} \emph{The Atlantic
Monthly} 176 (1): 101--8.

Garrett, Jesse James. 2010. \emph{Elements of User Experience, the:
User-Centered Design for the Web and Beyond}. Pearson Education.

Licklider, Joseph Carl Robnett. 1960. {``Man-Computer Symbiosis.''}
\emph{IRE Transactions on Human Factors in Electronics}, no. 1: 4--11.

Malone, Thomas W, and Kevin Crowston. 1994. {``The Interdisciplinary
Study of Coordination.''} \emph{ACM Computing Surveys (CSUR)} 26 (1):
87--119.

Nelson, Theodor H. 1974. {``Computer Lib/Dream Machines.''}

---------. 2010. \emph{POSSIPLEX: Movies, Intellect, Creative Control,
My Computer Life and the Fight for Civilization: An Autobiography of Ted
Nelson}. Mindful Press.

Packer, Randall, and Ken Jordan. 2002. \emph{Multimedia: From Wagner to
Virtual Reality}. WW Norton \& Company.

Sellen, Abigail J, and Richard HR Harper. 2003. \emph{The Myth of the
Paperless Office}. MIT press.

Shiffman, Daniel. 2009. \emph{Learning Processing: A Beginner's Guide to
Programming Images, Animation, and Interaction}. Morgan Kaufmann.

\hypertarget{ux3bcux3bfux3c1ux3c6ux3adux3c2}{%
\section{Μορφές}\label{ux3bcux3bfux3c1ux3c6ux3adux3c2}}

\begin{quote}
Αν υπάρχει ένα προηγούμενο στην ανθρώπινη εμπειρία που να δείχνει πώς θα
πρέπει να μοιάζει ένας υπολογιστής, αυτό είναι το μουσικό όργανο, μια
επινόηση με την οποία μπορείς να διερευνήσεις ένα τεράστιο εύρος
δυνατοτήτων μέσω μιας διεπαφής που συνδέει το μυαλό με το σώμα σου.
Jaron Lanier
\end{quote}

\hypertarget{ux3c0ux3b5ux3c1ux3afux3bbux3b7ux3c8ux3b7}{%
\subsubsection{Περίληψη}\label{ux3c0ux3b5ux3c1ux3afux3bbux3b7ux3c8ux3b7}}

Οι μορφές των συστημάτων διάδρασης βρίσκονται σε μια συνεχή εξέλιξη, η
οποία εξαρτάται τόσο από την τεχνολογία και τις ανθρώπινες δεξιότητες
όσο και από το κοινωνικό και πολιτιστικό πλαίσιο δημιουργίας και χρήσης
τους. Οι διάτρητες κάρτες και το πληκτρολόγιο ήταν διαθέσιμα ως
συστήματα εισόδου για την βιομηχανία υφασμάτων και την τυπογραφία,
πολλές δεκαετίες πριν συνδεθούν σε έναν υπολογιστή. Αν και η σύνδεση
τους με τον υπολογιστή έδωσε μια οικειότητα σε ένα μηχάνημα, στο οποίο,
μέχρι τότε, είχε πρόσβαση μόνο κάποιος ηλεκτρονικός και ταυτόχρονα όρισε
και ένα σχετικά στενό πλαίσιο διάδρασης, όπως είναι η εισαγωγή δεδομένων
και η πληκτρολόγηση κειμένου, τα οποία είναι συμβατά περισσότερο με το
περιβάλλον εργασίας στο γραφείο. Από την άλλη πλευρά, ο υπολογιστής δεν
έχει κάποια προτίμηση όσον αφορά τις συσκευές εισόδου και εξόδου. Θα
χρειαστεί, ωστόσο, να περάσουν πολλές ακόμη δεκαετίες μέχρι να αρχίσουμε
να συνδέουμε ψηφιακές οθόνες, κάμερες και αισθητήρες. Παράλληλα με τις
συσκευές διάδρασης, αναπτύσσονται διαφορετικές μορφές λογισμικού
διάδρασης, όπως ένας επεξεργαστής κειμένου που βασίζεται σε
πληκτρολόγιο, αλλά μπορεί να έχει πολλά διαφορετικά στυλ διάδρασης, όπως
παράθυρα, μενού, γραφικά ή να βασίζεται στη γραμμή εντολών.

\hypertarget{ux3bcux3bfux3c1ux3c6ux3bfux3bbux3bfux3b3ux3afux3b1}{%
\subsection{Μορφολογία}\label{ux3bcux3bfux3c1ux3c6ux3bfux3bbux3bfux3b3ux3afux3b1}}

Υπάρχουν πολλοί τρόποι διάδρασης του χρήστη με τη συσκευή. Αρχικά,
έχουμε τη γραμμή εντολών και τις εντολές δέσμης. Αυτοί παραμένουν πολύ
αποτελεσματικοί τρόποι διάδρασης, ειδικά όταν ο χρήστης δίνει
επαναλαμβανόμενες και σταθερές οδηγίες προς τη συσκευή. Όταν οι οδηγίες
προς το σύστημα πρέπει να αλλάζουν συχνά και δυναμικά, τότε έχουμε τον
απευθείας χειρισμό και την εικονική πραγματικότητα. Όταν η διάδραση
γίνεται δυναμικά, οι χρήστες λαμβάνουν συνέχεια ανάδραση. Στο ενδιάμεσο
αυτών των ακραίων τύπων διάδρασης (γραμμή εντολής και απευθείας
χειρισμός) υπάρχει ένα ολόκληρο φάσμα τύπων, μεταξύ των οποίων και τύποι
διάδρασης που εμφανίζονται με τις κινητές εφαρμογές και τις συσκευές
διάχυτου υπολογισμού.

Πριν την εφικτή και οικονομική ανάπτυξη των μικρο-υπολογιστών, που
βασίζονται σε ολοκληρωμένα κυκλώματα ημιαγωγών μεγάλης κλίμακας, η
διάδραση σε πραγματικό χρόνο ήταν διαθέσιμη μόνο σε συστήματα που είχαν
λογισμικό χρονοδιαμοιρασμού, αφού στους περισσότερους υπολογιστές την
δεκαετία του 1960 η εκτέλεση των προγραμμάτων γινόταν με εργασίες
δέσμης. Πέρα από τα καινοτόμα ερευνητικά συστήματα που εξετάζουμε στην
επόμενη ενότητα, τα εμπορικά διαθέσιμα διαδραστικά συστήματα βασίζονταν
σε τερματικά κειμένου, στα οποία η διάδραση γινόταν με γραμμή εντολών,
μενού και φόρμες. Τα τερματικά κειμένου αρχικά ήταν προσαρμοσμένες και
εξελιγμένες εκδοχές του παραδοσιακού τηλέτυπου, τα οποία σταδιακά
αντικαταστάθηκαν από τερματικά με ηλεκτρονικές οθόνες καθοδικού σωλήνα.
Σε όλες αυτές τις περιπτώσεις ο χρήστης πληκτρολογούσε και έβλεπε το
αποτέλεσμα στο χαρτί ή στην οθόνη. Η πληκτρολόγηση γινόταν συνήθως για
την εισαγωγή ενός προγράμματος FORTRAN, BASIC, COBOL ή για τη χρήση του
προγράμματος μέσω της γραμμής εντολών, μενού και φόρμας. Εκτός από τους
κεντρικούς υπολογιστές της IBM, που ήταν ήδη μια μεγάλη εταιρεία στις
λύσεις για την ηλεκτρονική επεξεργασία δεδομένων, ήρθαν να προστεθούν
και οι μίνι-υπολογιστές της DEC. Η DEC, με τους μινι-υπολογιστές της
σειράς PDP, δημιούργησε τη γέφυρα από τους πολύ ακριβούς κεντρικούς
υπολογιστές της δεκαετίας του 1960 προς τους πολύ οικονομικούς
μικρο-υπολογιστές της δεκαετίας του 1970. Οι μίνι-υπολογιστές ήταν μεν
πολύ ακριβοί για να είναι προσωπικοί, αλλά αρκετά οικονομικοί για να
εγκατασταθούν σε πολλούς οργανισμούς και να εκπαιδεύσουν μια νέα γενιά
χρηστών, η οποία θα απαιτήσει μια πιο προσωπική εκδοχή τους.

\leavevmode\vadjust pre{\hypertarget{fig:macintosh-desktop}{}}%
\begin{figure}
\hypertarget{fig:macintosh-desktop}{%
\centering
\includegraphics{images/macintosh-desktop.png}
\caption{Εικόνα 1: Το λειτουργικό σύστημα του επιτραπέζιου υπολογιστή
Macintosh ήταν το πρώτο επιτυχημένο εμπορικό γραφικό περιβάλλον εργασίας
και το πρώτο που έδωσε πρόσβαση σε πολλές εφαρμογές λογισμικού, ακόμη
και σε χρήστες που δεν είχαν γνώσεις υπολογιστών, χάρη στην ευχρηστία
του.}\label{fig:macintosh-desktop}
}
\end{figure}

\leavevmode\vadjust pre{\hypertarget{fig:tmux-desktop}{}}%
\begin{figure}
\hypertarget{fig:tmux-desktop}{%
\centering
\includegraphics{images/tmux-desktop.png}
\caption{Εικόνα 2: Η επιφάνεια εργασίας έχει συνδεθεί με τα γραφικά και
τα πολυμέσα αλλά μπορεί να πραγματοποιηθεί σε μεγάλο βαθμό μόνο με τη
χρήση της γραμμής εντολών, μερικών συνοδευτικών εργαλείων και αρκετών
ρυθμίσεων με αποτέλεσμα μια πολύ γρήγορη, απλή και πυκνή σε πληροφορία
διεπαφή.}\label{fig:tmux-desktop}
}
\end{figure}

Ο μικρο-υπολογιστής Apple II ήταν μια πολύ μεγάλη εμπορική επιτυχία με
εκατομμύρια πωλήσεις, αν και απευθυνόταν κυρίως σε χομπίστες των
ηλεκτρονικών και του προγραμματισμού. Η προσιτή τιμή, η αξιοπιστία του
και η απλότητα στη χρήση διεύρυναν πολύ το κοινό του, πέρα από τους
χομπίστες, προς τους σπουδαστές και τους επαγγελματίες. Σε αντίθεση με
τους πρώτους μικρο-υπολογιστές που ήταν διαθέσιμοι ως συναρμολογούμενοι,
ο Apple II είχε έναν προσεγμένο βιομηχανικό σχεδιασμό που έμοιαζε με τα
οικιακά ηλεκτρονικά συστήματα, όπως είναι ένα οικειακό ηχοσύστημα. Το
βασικό σύστημα είχε πληκτρολόγιο, αλλά δεν είχε ούτε οθόνη ούτε
αποθηκευτικό μέσο, για τα οποία ο χρήστης μπορούσε να χρησιμοποιήσει μια
τηλεόραση και ένα κασετόφωνο. Το βασικό λογισμικό ήταν μια εκδοχή της
γλώσσας προγραμματισμού BASIC, με την οποία ο χρήστης μπορούσε να
αναπτύξει τα δικά του προγράμματα, ή να αντιγράψει αυτά που έβρισκε σε
περιοδικά και βιβλία της εποχής. Η μεγάλη διάδοση αυτού του
μικρο-υπολογιστή έδωσε το κίνητρο στην Apple να αναπτύξει κάποιες
βελτιώσεις, όπως την οθόνη και το εξωτερικό μέσο αποθήκευσης, καθώς και
νέα μοντέλα με περισσότερες δυνατότητες. Επίσης, το εύρος των
διαφορετικών χρηστών και οι πολυδιάστατες ανάγκες τους οδήγησαν στη
δημιουργία πολλών εμπορικών εφαρμογών, όπως το πρώτο φύλλο εργασίας
VisiCalc, καθώς και στην ανάπτυξη πολλών βιντεοπαιχνιδιών.

Παράλληλα με την ανάπτυξη των προσωπικών διαδραστικών μικρο-υπολογιστών,
μερικές εταιρείες κατασκεύασαν τις πρώτες κονσόλες βιντεοπαιχνιδιών. Οι
πρώτες κονσόλες δεν είχαν εξωτερικό αποθηκευτικό μέσο, οπότε τα
βιντεοπαιχνίδια ήταν διαθέσιμα μόνο εσωτερικά σε τσιπάκια μνήμης
ανάγνωσης, αλλά αυτό άλλαξε με το Atari 2600, το οποίο δεχόταν
εξωτερικές κασέτες, όπου βρίσκονταν τα τσιπ με τη μνήμη ανάγνωσης. Οι
κονσόλες συνοδεύονταν από συσκευές εισόδου με τη μορφή του μοχλού
ελέγχου και δεν είχαν ούτε πληκτρολόγιο ούτε ποντίκι, ενώ η έξοδος
γινόταν προς την τηλεόραση. Όπως και οι μορφές των προσωπικών
διαδραστικών συστημάτων έχουν μόνο μικρές αλλαγές από το 1984 με το
Apple Macintosh, \footnote{(\textbf{Εικόνα?})~1 Γραφική επιφάνεια
  εργασίας Apple Macintosh (Apple)} \footnote{(\textbf{Εικόνα?})~2
  Επιφάνεια εργασίας σε περιβάλλον τερματικού (Fair use)} έτσι και οι
κονσόλες δεν έχουν αλλάξει σημαντικά. To ίδιο ισχύει και με την
αρχιτεκτονική του λογισμικού τους, το οποίο συνήθως δεν περιλαμβάνει το
ενδιάμεσο επίπεδο ενός λειτουργικού συστήματος, αφού οι κατασκευαστές
βιντεοπαιχνιδιών προτιμάνε να έχουν πλήρη έλεγχο πάνω στο υλικό, γιατί
αυτό τους δίνει μεγαλύτερη δημιουργικότητα και έλεγχο στο τελικό
αποτέλεσμα, που είναι το ζητούμενο σε αυτήν την βιομηχανία. Αν και τα
πρώτα βιντεοπαιχνίδια ξεκίνησαν κυρίως ως προσομοιώσεις επίκαιρων
θεμάτων, όπως το τένις ή οι διαστημικές μάχες, σταδιακά η
δημιουργικότητα των σχεδιαστών βιντεοπαιχνιδιών δημιούργησε νέα είδη
και, κυρίως, νέες συσκευές εισόδου. Σε αντίθεση με τους προσωπικούς
υπολογιστές με γραφική διεπαφή που έχουν μείνει σταθεροί, η βιομηχανία
των βιντεοπαιχνιδιών φαίνεται πιο δημιουργική, γιατί δεν περιορίζεται
ούτε από συσκευές εισόδου ούτε από ένα λειτουργικό σύστημα με γραφική
διεπαφή.

\leavevmode\vadjust pre{\hypertarget{fig:apple2}{}}%
\begin{figure}
\hypertarget{fig:apple2}{%
\centering
\includegraphics{images/apple2.jpg}
\caption{Εικόνα 3: Ο δεύτερος μικρο-υπολογιστής της Apple ήταν άμεσα
διαθέσιμος σε μια καλαίσθητη συσκευασία βιομηχανικού σχεδιασμού με
πληκτρολόγιο και είχε εισόδους και εξόδους για πολλά διαφορετικά
περιφερειακά, όπως δίσκοι και οθόνη. Μπορούσε να συνδεθεί σε τηλεόραση
και να αποθηκεύσει προγράμματα σε ένα μαγνητόφωνο κασέτας ήχου, οπότε το
κόστος ήταν χαμηλό και έγινε αμέσως δημοφιλής, ενώ έμεινε σε παραγωγή
για περισσότερα από δεκαπέντε χρόνια.}\label{fig:apple2}
}
\end{figure}

\leavevmode\vadjust pre{\hypertarget{fig:visicalc}{}}%
\begin{figure}
\hypertarget{fig:visicalc}{%
\centering
\includegraphics{images/visicalc.png}
\caption{Εικόνα 4: Ανάμεσα στις πολλές διαφορετικές χρήσεις του
υπολογιστή ως εργαλείο (π.χ. επεξεργασία κειμένου, σχεδίαση κτλ.),
ξεχωρίζει η περίπτωση του VisiCalc που είναι το πρώτο δημοφιλές εργαλείο
επεξεργασίας υπολογιστικών φύλλων, το οποίο μεταφέρει από το χαρτί στην
οθόνη του υπολογιστή μια πολύ βασική διεργασία που κάνουν οι εταιρείες
και τα νοικοκυριά για τη διαχείριση των οικονομικών τους. Η διάθεση μιας
εφαρμογής γραφείου σε μικρο-υπολογιστές θα στρέψει την προσοχή της IBM
στους προσωπικούς υπολογιστές και θα δημιουργήσει πολλές νέες μεγάλες
εταιρείες όπως η Microsoft και η Apple.}\label{fig:visicalc}
}
\end{figure}

Η εμπορική επιτυχία των μικρο-υπολογιστών και ειδικά του Apple II, καθώς
και η εμφάνιση λειτουργικών συστημάτων, όπως το CPΜ, και εφαρμογών
γραφείου, όπως ο επεξεργαστής κειμένου WordStar και η λογιστική εφαρμογή
VisiCalc, έστρεψαν την προσοχή της IBM από τους κεντρικούς υπολογιστές
που ήταν η διαχρονική της αγορά προς την κατεύθυνση ενός προσωπικού
υπολογιστή για τις εργασίες του γραφείου. Αν και δεν υπήρχε ακόμη κάποιο
πετυχημένο εμπορικά εύχρηστο γραφικό περιβάλλον εργασίας, υπήρχαν ήδη
πάρα πολλές και πολύ οικονομικές λύσεις υλικού και λογισμικού για το
περιβάλλον γραφείου, μια αγορά δηλαδή που η IBM θεωρούσε ότι της ανήκε
και όπου είχε διαχρονικά τον έλεγχο των τιμών. Αυτή η νέα αγορά
προσωπικών υπολογιστών είχε πολύ διαφορετικές ιδιότητες από την
παραδοσιακή αγορά της IBM με τους κεντρικούς ή μίνι-υπολογιστές. Τόσο η
κεντρική μονάδα όσο και τα συστήματα εισόδου και εξόδου είχαν πλέον
προσβάσιμη τιμή ώστε να είναι προσιτά από πολλές μικρές επιχειρήσεις,
καθώς και νοικοκυριά. Ταυτόχρονα, η κατασκευή του λογισμικού αλλά και η
χρήση του ήταν πλέον αρκετά διαδεδομένη, ώστε νέες μικρές εταιρείες,
ακόμη και ανεξάρτητοι κατασκευαστές, να δημιουργούν πολλές εφαρμογές, ή
λειτουργικά συστήματα, όπως το CPM, τα οποία μπορεί να μην ήταν τόσο
καλά όσο αυτά της IBM, αλλά ήταν αρκετά καλά και πολύ οικονομικά.
\footnote{Freiberger and Swaine (1984)} Η είσοδος της IBM με τον
προσωπικό της υπολογιστή ουσιαστικά έδωσε μια σφραγίδα ποιότητας και
σοβαρότητας σε έναν ιδιαίτερα πολυφωνικό χώρο, ταυτόχρονα όμως με την
επικράτησή του έδωσε και ένα τέλος στη δημιουργικότητα αυτού του κλάδου,
η οποία θα περάσει σε έναν μικρότερο βαθμό προς το Apple Macintosh.
\footnote{(\textbf{Εικόνα?})~3 Apple II (Apple)} \footnote{(\textbf{Εικόνα?})~4
  Φύλλο εργασίας Visicalc (Visicorp)}

\leavevmode\vadjust pre{\hypertarget{fig:xerox-parc-tab}{}}%
\begin{figure}
\hypertarget{fig:xerox-parc-tab}{%
\centering
\includegraphics{images/xerox-parc-tab.png}
\caption{Εικόνα 5: Η συσκευή τύπου Tab ήταν η μικρότερη στην οικογένεια
συσκευών διάχυτου υπολογισμού Tab-Pad-Board και ήταν σχεδιασμένη έτσι
ώστε να χωράει στην παλάμη και να μπορεί να λειτουργήσει τα κουμπιά μόνο
με το ένα χέρι, ενώ επέτρεπε και την αφή με το δεύτερο χέρι, με τη χρήση
μιας πένας.}\label{fig:xerox-parc-tab}
}
\end{figure}

\leavevmode\vadjust pre{\hypertarget{fig:tabs-pads-boards}{}}%
\begin{figure}
\hypertarget{fig:tabs-pads-boards}{%
\centering
\includegraphics{images/tabs-pads-boards.png}
\caption{Εικόνα 6: Το όραμα για τον διάχυτο υπολογισμό από το εργαστήριο
Xerox PARC βασίζεται σε τρεις διακριτές φόρμες συσκευών διάδρασης με
τους χρήστες, οι οποίες επικοινωνούν διαφανώς μεταξύ τους έτσι ώστε οι
χρήστες να μπορούν να πραγματοποιήσουν τους στόχους τους είτε ατομικά
είτε συνεργατικά.}\label{fig:tabs-pads-boards}
}
\end{figure}

Η παρουσία ενός διακριτού επιπέδου λογισμικού ανάμεσα στο υλικό του
υπολογιστή και στις εφαρμογές του ήταν για πολλές δεκαετίες κάτι
περιττό, αφού δεν υπήρχαν πολλές διαφορετικές αρχιτεκτονικές, ενώ και οι
εφαρμογές δεν ήταν πολλές. Σε μερικές περιπτώσεις, κάποιος κατασκευαστής
έφτιαχνε μερικές βιβλιοθήκες προγραμματισμού, έτσι ώστε να μην
χρειάζεται να υλοποιεί συνέχεια κάποιες βασικές λειτουργίες, αλλά, στην
πράξη, οι περισσότερες εφαρμογές φτιάχνονταν από την αρχή είτε σε
γλώσσες υψιλού επιπέδου όπως οι FORTRAN, COBOL, BASIC είτε σε συμβολική
γλώσσα μηχανής. Η διάθεση οικονομικών επεξεργαστών από την Intel, οι
οποίοι τοποθετήθηκαν σε πολλές διαφορετικές αρχιτεκτονικές
μικρο-υπολογιστών, οδήγησε στη δημιουργία των πρώτων λειτουργικών
συστημάτων, τα οποία επέτρεπαν στους προγραμματιστές να εστιάσουν στις
λειτουργίες της εφαρμογής τους, χωρίς να νοιάζονται για την πρόσβαση
στον δίσκο και στις βασικές συσκευές εισόδου και εξόδου. Η επιλογή του
λειτουργικού συστήματος MSDOS από την IBM για τον πρώτο της προσωπικό
υπολογιστή και η ανάπτυξη πολλών εφαρμογών για αυτήν την πλατφόρμα
δημιούργησαν μια αγορά και έναν τρόπο διάδρασης που θα παραμείνει
επίκαιρος ακόμη και μετά την εμφάνιση της γραφικής διεπαφής. Πράγματι,
οι πρώτες εκδόσεις του γραφικού περιβάλλοντος της Microsoft μέχρι και
την έκδοση Windows Me του 2000 βασίζονται στο MS-DOS, ενώ ακόμη και
επόμενες εκδόσεις που έχουν κατασκευαστεί από την αρχή, υλοποιούν για
λόγους συμβατότητας ένα περιβάλλον εξομοίωσης για τις διαχρονικά
δημοφιλείς εφαρμογές που τρέχουν μόνο πάνω σε MS-DOS.

Τα επιτραπέζια συστήματα με οθόνη γραφικών, πληκτρολόγιο και ποντίκι,
ήταν, από τη δεκαετία του 1980 και μέχρι τις αρχές της δεκαετίας του
2010, η βασική μορφή υλικού διάδρασης. Για παράδειγμα, ο επιτραπέζιος
υπολογιστής Apple Lisa απέτυχε εμπορικά, αλλά ήταν καθοριστικής σημασίας
για τη μετάβαση από τον Xerox Star, ο οποίος απευθυνόταν μόνο στο
περιβάλλον του γραφείου, προς την κατεύθυνση του Apple Macintosh που
έφερε τη γραφική επιφάνεια εργασίας σε ένα ευρύτερο κοινό. \footnote{Hertzfeld
  (2004)} Για τον σκοπό αυτό, η Apple υιοθετεί από το Xerox Star το
ποντίκι και τη γραφική επιφάνεια εργασίας με τα έγγραφα ως αρχεία.
Ταυτόχρονα, προσθέτει στο Macintosh την ιδέα της διάκρισης ανάμεσα στις
εφαρμογές και στο λειτουργικό σύστημα, έτσι ώστε να μπορεί να
προσαρμοστεί σε διαφορετικές ανάγκες. Το Apple Macintosh δημιουργεί ένα
σημείο αναφοράς για τη διάδραση με επιτραπέζια συστήματα, το οποίο στην
συνέχεια θα αντιγραφεί από την Microsoft με τα Windows 95, καθώς και από
τα γραφικά περιβάλλοντα των συστημάτων Linux με τα GNOME, KDE. Ο χρήστης
μπορεί με το ποντίκι να εξερευνήσει τις εφαρμογές και τα έγγραφα του
συστήματος, ενώ δεν χρειάζεται να θυμάται εντολές, αφού μπορεί να τις
ανακαλύψει σταδιακά, μέσα από μενού, φόρμες και παλέτες εργαλείων. Η
δημιουργία εφαρμογών που βασίζονται στις ίδιες βιβλιοθήκες και σε
κάποιους βασικούς κανόνες ενισχύει ακόμη περισσότερο την ευχρηστία του
συστήματος, αφού ακόμη και μια νέα εφαρμογή έχει πολλές διαδράσεις
παρόμοιες με προηγούμενες, όπως το άνοιγμα, η αποθήκευση, και η εκτύπωση
εγγράφων. Η διάκριση ανάμεσα στις εφαρμογές και στο λειτουργικό σύστημα
δημιουργεί τα θεμέλια για ένα σύστημα διάδρασης που θα διατηρηθεί και θα
επεκταθεί ακόμη περισσότερο με τα κινητά και φορετά συστήματα διάδρασης
των επόμενων δεκαετιών. \footnote{(\textbf{Εικόνα?})~5 Xerox Tab (Bill
  Buxton collection)} \footnote{(\textbf{Εικόνα?})~6 Διάχυτος
  Υπολογισμός (Mark Stefik)}

\leavevmode\vadjust pre{\hypertarget{fig:apple-newton}{}}%
\begin{figure}
\hypertarget{fig:apple-newton}{%
\centering
\includegraphics{images/apple-newton.jpg}
\caption{Εικόνα 7: Πολύ πριν την επιτυχία του iPad, το Apple Newton
προσπάθησε να προσφέρει εύχρηστη διάδραση σε κινητή μορφή για μαθητές
και εργαζόμενους, βασιζόμενο στη φυσική διάδραση της γραφής, η οποία,
όμως, εκείνη την εποχή δεν είχε την αναγκαία
ακρίβεια.}\label{fig:apple-newton}
}
\end{figure}

\leavevmode\vadjust pre{\hypertarget{fig:iphone-jobs}{}}%
\begin{figure}
\hypertarget{fig:iphone-jobs}{%
\centering
\includegraphics{images/iphone-jobs.jpg}
\caption{Εικόνα 8: Η εισαγωγή του πρώτου iphone αποτελεί σημείο
αναφοράς, καθώς, μέσα σε λιγότερο από δέκα χρόνια, περισσότεροι άνθρωποι
θα είχαν μια παρόμοια κινητή συσκευή με οθόνη αφής ως βασικό σύστημα
καθημερινής διάδρασης, παρά τον παραδοσιακό επιτραπέζιο υπολογιστή, ο
οποίος θα κρατήσει τη θέση του περισσότερο ως υπολογιστής
ανάπτυξης.}\label{fig:iphone-jobs}
}
\end{figure}

Η διαδραστική τηλεόραση είναι μια προσπάθεια για τη βελτίωση ενός
παραδοσιακού μέσου που έχει κατηγορηθεί για τη δημιουργία παθητικότητας
από την πλευρά του τηλεθεατή. Από τις αρχές της δεκαετίας του 1990,
πολλές εταιρείες και ερευνητικές ομάδες προσθέτουν τεχνολογίες διάδρασης
σε τηλεοπτικούς δέκτες, έτσι ώστε τα οφέλη της διάδρασης να μην
βρίσκονται μόνο στους επιτραπέζιους υπολογιστές. Αρχικά, οι περισσότερες
προσπάθειες μετέφεραν στον τηλεοπτικό δέκτη τη λειτουργικότητα των
επιτραπέζιων εφαρμογών, όπως την ηλεκτρονική αλληλογραφία και την
περιήγηση στον παγκόσμιο ιστό. Σταδιακά, οι σχεδιαστές κατανόησαν πως το
πλαίσιο και οι στόχοι χρήσης της διαδραστικής τηλεόρασης δεν είναι
καθόλου ίδιοι με αυτούς του επιτραπέζιου υπολογιστή. Ο τηλεοπτικός
δέκτης βρίσκεται συνήθως σε ένα σαλόνι και οι χρήστες παρακολουθούν από
απόσταση και με παρέα προγράμματα που έχουν μια σημαντική ψυχαγωγική
διάσταση. Με αυτόν τον τρόπο, η διάδραση πήρε μια λιγότερο κυριαρχική
θέση ως συμπλήρωμα της ροής οπτικοακουστικού περιεχομένου, με εφαρμογές
ψηφοφορίας, μηνυμάτων, και πρόσθετης πληροφορίας.

Την ίδια περίοδο που η τηλεόραση γίνεται περισσότερο διαδραστική,
αναδύεται μια νέα μορφή προσωπικού συστήματος διάδρασης με την ονομασία
\emph{έξυπνο τηλέφωνο}, το οποίο θα γίνει σύντομα η πιο δημοφιλής
διαδραστική συσκευή. Οι πρώτες προσπάθειες στην κατασκευή φορητών
συστημάτων μεγέθους παλάμης είχαν ασύμβατα στοιχεία διάδρασης, τα οποία
ήταν συνήθως απευθείας δανεισμένα από τα επιτραπέζια συστήματα, όπως
ακριβώς και στην περίπτωση της διαδραστικής τηλεόρασης. Στην πορεία,
όμως, το έξυπνο τηλέφωνο με οθόνη αφής θα υιοθετήσει την ιδέα των
εφαρμογών από τα επιτραπέζια συστήματα σε μια περισσότερο απλή μορφή και
με έμφαση στο οπτικοακουστικό περιεχόμενο. Με αυτόν τον τρόπο, πολλοί
χρήστες θα αποκτήσουν πρόσβαση σε διαδραστική πληροφορία και επικοινωνία
με μια φορητή συσκευή με κύρια στοιχεία, όχι τόσο το τηλέφωνο, αλλά
κυρίως την ασύρματη πρόσβαση σε δίκτυα δεδομένων, καθώς και την κάμερα
και τη γεωγραφική θέση, τα οποία θα αποτελέσουν δομικά στοιχεία
διάδρασης με τις κινητές εφαρμογές. Σε πολύ σύντομο χρονικό διάστημα από
την εμφάνιση του, το έξυπνο τηλέφωνο θα αρχίσει να χρησιμοποιείται
λιγότερο ως τηλέφωνο και περισσότερο ως τερματικό κατανάλωσης
περιεχομένου, κάτι δηλαδή που ήταν ένα από τα κακά χαρακτηριστικά της
παραδοσιακής τηλεόρασης. \footnote{(\textbf{Εικόνα?})~7 Αpple Newton
  (DAVIDSDIEGO)} \footnote{(\textbf{Εικόνα?})~8 Apple iPhone (Wikimedia)}

\hypertarget{ux3c0ux3adux3c1ux3b1-ux3b1ux3c0ux3cc-ux3c4ux3bfux3bd-ux3c5ux3c0ux3bfux3bbux3bfux3b3ux3b9ux3c3ux3bcux3cc}{%
\subsection{Πέρα από τον
υπολογισμό}\label{ux3c0ux3adux3c1ux3b1-ux3b1ux3c0ux3cc-ux3c4ux3bfux3bd-ux3c5ux3c0ux3bfux3bbux3bfux3b3ux3b9ux3c3ux3bcux3cc}}

Οι πρώτοι κεντρικοί υπολογιστές προγραμματίζονταν με διάτρητες κάρτες,
γιατί αυτή ήταν μια έμπιστη τεχνολογία που είχε χρησιμοποιηθεί ήδη για
πολλές δεκαετίες σε άλλες εφαρμογές, όπως η κλωστοϋφαντουργία και η
απογραφή του πληθυσμού. \footnote{(\textbf{Εικόνα?})~9 Μηχάνημα
  διάτρησης κάρτας (wikimedia)} Παράλληλα, η συγγενική τεχνολογία της
διάτρητης ταινίας χρησιμοποιήθηκε για την αποθήκευση μεγαλύτερων
προγραμμάτων και δεδομένων, καθώς και για την ανάλυση εργαστηριακών
πειραμάτων. \footnote{(\textbf{Εικόνα?})~10 LINC PC (MIT)} Αν και ο
υπολογιστής αριστεύει στην αποδοτική εκτέλεση υπολογισμών, αυτό δεν
είναι η μόνη εφαρμογή του, αφού μπορεί να προσομοιώσει και, κυρίως, να
εξομοιώσει υπάρχοντα και νέα συστήματα, τα οποία βασίζονται μεν στον
υπολογισμό, αλλά παρουσιάζουν μια διαφορετική εικόνα στον χρήστη, έτσι
ώστε να ταιριάζει στις ανάγκες και τις δυνατότητες του. \footnote{denning1998beyond}

\leavevmode\vadjust pre{\hypertarget{fig:card-puncher}{}}%
\begin{figure}
\hypertarget{fig:card-puncher}{%
\centering
\includegraphics{images/card-puncher.jpg}
\caption{Εικόνα 9: Τα συστήματα διάτρησης κάρτας υπάρχουν από τον 19ο
αιώνα, με βασικές εφαρμογές τον προγραμματισμό και την αποθήκευση
δεδομένων για μηχανήματα πλεξίματος και αναπαραγωγής μουσικής, οπότε
υπήρχε η τεχνογνωσία για την προσαρμογή τους ως συσκευές εισόδου στους
πρώτους κεντρικούς ηλεκτρονικούς υπολογιστές.}\label{fig:card-puncher}
}
\end{figure}

\leavevmode\vadjust pre{\hypertarget{fig:linc-pc}{}}%
\begin{figure}
\hypertarget{fig:linc-pc}{%
\centering
\includegraphics{images/linc-pc.jpg}
\caption{Εικόνα 10: O LINC (Laboratory INstrument Computer) θεωρείται ο
πρώτος προσωπικός υπολογιστής, γιατί σε έναν σχετικά μικρό όγκο
περιλαμβάνει μια συσκευή εξωτερικής αποθήκευσης και κυρίως μία οθόνη και
ένα πληκτρολόγιο ως σύστημα εξόδου και εισόδου. Σε αντίθεση με τους
μεγάλους κεντρικούς υπολογιστές εκείνης της εποχής, ήταν σχεδιασμένος
από την αρχή σε συναρμολογούμενη μορφή, οπότε ήταν σχετικά προσιτός σε
πολλά ερευνητικά εργαστήρια που είχαν τις δεξιότητες να τον
κατασκευάσουν και να τον συντηρήσουν.}\label{fig:linc-pc}
}
\end{figure}

Ένας από τους πιο σημαντικούς μορφολογικούς μετασχηματισμούς της έννοιας
του υπολογιστή ήταν η δημιουργία του Sketchpad στο MIT από τον Ivan
Sutherland, το 1963. Ο κεντρικός υπολογιστής που είχε στη διάθεσή του
μπορούσε να προγραμματιστεί με διάτρητες κάρτες και η βασική
αλληλεπίδραση γινόταν σε εργασίες δέσμης, όπου υπήρχε ένας σημαντικός
ετεροχρονισμός, ανάμεσα στην ανάγνωση του προγράμματος και στην τελική
εκτύπωση του αποτελέσματος. Αντίθετα, το Sketchpad εστιάζει στη διάδραση
σε πραγματικό χρόνο με χρήση πένας και οθόνης, όπου τα γραφικά
ελέγχονται τόσο από την πένα όσο και από τον υπολογιστή σε πραγματικό
χρόνο, δίνοντας έτσι μια από τις πρώτες εμπειρίες συμβίωσης ανάμεσα στον
άνθρωπο και τη μηχανή. Αν και αυτή η υπέρβαση φαίνεται πολύ μεγάλη,
υπήρχαν ήδη σε χρήση οι επιμέρους τεχνολογίες, για διαφορετικούς
σκοπούς, όπως η είσοδος με πένα αντί για διάτρητες κάρτες και η έξοδος
σε οθόνη αντί για εκτύπωση σε χαρτί.\\
Πράγματι, την ίδια περίοδο, οι μεταπτυχιακοί ερευνητές στο MIT
χρησιμοποιούν τους μεγάλους υπολογιστές της εποχής για να φτιάξουν
διαδραστικά προγράμματα, όπως τα ψυχαγωγικά προγράμματα τρίλιζα,
λαβύρινθος και Spacewar ή την \emph{πολύ ακριβή γραφομηχανή}. Το
συμπέρασμα είναι ότι ο μορφολογικός μετασχηματισμός συμβαίνει με
σταδιακές μετατροπές και περισσότερο ως δημιουργική σύνθεση τεχνολογιών
που ήδη υπάρχουν, παρά ως καινοτομία χωρίς προηγούμενο. Παρόμοια
μετάβαση διαπιστώνεται και κατά την δημιουργία της πρώτης μάσκας
επαυξημένης πραγματικότητας, η οποία εμπνέεται από την τεχνολογία
προβολής ροής βίντεο πάνω σε μικρό διαφανή φακό κοντά στο μάτι, και
επαυξάνεται με την προβολή γραφικών στην θέση του βίντεο, τα οποία
ελέγχονται από την κίνηση του κεφαλιού. \footnote{(\textbf{Εικόνα?})~11
  Electrocular (ARPA)} \footnote{(\textbf{Εικόνα?})~12 Γραφικά που
  ακολοθούν την κίνηση του βλέματος (Ivan Shutherland)} Αυτό που είναι
κοινό έδαφος σε όλες τις περιπτώσεις καινοτομικού μορφολογικού
μετασχηματισμού είναι ότι συμβαίνει με την κατασκευή νέου λογισμικού
διάδρασης, που δεν βασίζεται στα υπάρχοντα εργαλεία κατασκευής.
Επόμενως, η δημιουργία νέων εργαλείων κατασκευής είναι περισσότερο
σημαντική από την απλή χρήση τους, όπως ακριβώς ο ψηφιακός αλφαβητισμός
έχει να κάνει περισσότερο με την κατασκευή παρά με την χρήση εφαρμογών.

Το αρχικό όραμα για το Dynabook ήταν η δημιουργία ενός φορητού
συστήματος διάδρασης για παιδιά, το οποίο μορφολογικά έμοιαζε με τους
σύγχρονους υπολογιστές ταμπλέτας. Το λογισμικό διάδρασής του, όμως, δεν
είχε τίποτα κοινό με αυτό που έχουν οι σύγχρονες ταμπλέτες iOS και
Android. Ο στόχος του Dynabook ήταν να δημιουργήσει μια νέα μορφή
ψηφιακού αλφαβητισμού, η οποία βασίζεται στην ανάγνωση του πηγαίου
κώδικα που έχουν γράψει άλλοι, στη μετατροπή και στη κατανόησή του και,
τελικά, στην ανάπτυξη πρωτότυπων διαδραστικών έργων λογισμικού, ως μια
νέα μορφή αλφαβητισμού. Για αυτόν τον σκοπό, ο Άλαν Κέη και η ομάδα του
δημιούργησαν το λογισμικό Smalltalk, το οποίο είχε ως βασική προδιαγραφή
τη δυνατότητα υλοποίησης του δημοφιλούς βιντεοπαιχνιδιού Spacewar με
σχετικά μικρή προσπάθεια και λίγο κώδικα. Ο προγραμματισμός του
Spacewar, εκτός από τη διασκέδαση του ίδιου του παιχνιδιού με έναν
συμπαίκτη, έδινε πρόσβαση σε ένα περιβάλλον προσομοίωσης της βαρύτητας.
Με άλλα λόγια, ο χρήστης του Dynabook δεν ήταν απλά ένας καταναλωτής
βιντεοπαιχνιδιών που αγόραζε, αλλά ήταν κάποιος που θα μελετούσε την
κατασκευή τους, θα έκανε μετατροπές και, τελικά, θα άρχιζε να φτιάχνει
τα δικά του παιχνίδια και τις δικές του προσομοιώσεις για άλλα φυσικά ή
τεχνητά φαινόμενα.

Το λογισμικό και το υλικό διάδρασης των δημοφιλών επιτραπέζιων
συστημάτων δημιουργήθηκε ως ένα ενδιάμεσο πρωτότυπο, αλλά τελικά
εδραιώθηκε, ειδικά στον χώρο του γραφείου και της εργασίας. \footnote{Waldrop
  (2001)} Το επιτραπέζιο σύστημα Xerox Alto δημιουργήθηκε ως ένα
ενδιάμεσο πρωτότυπο για το Dynabook, γιατί ήταν ένα εφικτό ενδιάμεσο
βήμα από τους μινι-υπολογιστές του 1970 προς τη μελλοντική κατεύθυνση
φορητών μορφών. Ο βασικός στόχος ήταν η ανάπτυξη λογισμικού στο
περιβάλλον Smalltalk και οι δοκιμές με παιδιά, έτσι ώστε να βελτιωθεί η
κατανόηση των ερευνητών για αυτόν τον σχεδιαστικό χώρο. Τα εδραιωμένα
εμπορικά συμφέροντα στον χώρο των εκδόσεων της μητρικής εταιρείας Xerox
οδήγησαν τελικά στη δημιουργία του Star, το οποίο απευθύνεται στον χώρο
του γραφείου. Με τη σειρά της, η Apple παρουσίασε το σύστημα Macintosh
για ένα ευρύτερο κοινό, το οποίο, εκτός από τις εφαρμογές γραφείου,
ενδιαφέρεται, επίσης, για τη δημιουργικότητα και την ψυχαγωγία. Με αυτόν
τον τρόπο, ένα πειραματικό σύστημα για την εκπαίδευση τελικά βρίσκεται
στο γραφείο με πολύ διαφορετικό λογισμικό, γιατί μια σειρά από
οργανισμοί είχαν διαφορετικά κίνητρα και συμφέροντα και όχι γιατί η
επιφάνεια εργασίας είναι ο αντικειμενικά καλύτερος τρόπος διάδρασης για
τις διεργασίες του γραφείου.

Η διαθεσιμότητα οικονομικών μικροεπεξεργαστών στις αρχές της δεκαετίας
του 1970 επέτρεψε σε πολλές νέες μικρές εταιρείες να κατασκευάσουν
προσιτούς μικρο-υπολογιστές, δημιουργώντας έτσι μια νέα αγορά, την οποία
καμία από τις μεγάλες εταιρείες εκείνης της εποχής, όπως οι IBM, DEC,
Xerox, δεν μπορούσαν να φανταστούν. Εκείνη την εποχή, η πιο γρήγορα
αναπτυσσόμενη εταιρεία υπολογιστών ήταν η DEC, η οποία έφτιαχνε
οικονομικούς μινι-υπολογιστές για ένα πολύπλοκο λογισμικό
χρονοκαταμερισμού, το οποίο ήταν πολύ δημοφιλές στα πανεπιστήμια και
στις μικρές εταιρείες, ενώ οι μεγαλύτεροι οργανισμοί συνήθως προτιμούσαν
έναν κεντρικό υπολογιστή της IBM. Όπως ακριβώς ο διευθυντής της IBM
έκανε τη λανθασμένη πρόβλεψη το 1958 ότι ο κόσμος δεν χρειάζεται
περισσότερους από δέκα υπολογιστές, έτσι ακριβώς και ο διευθυντής της
DEC, ο οποίος διέψευσε την προφητεία της IBM με εκατοντάδες
μίνι-υπολογιστές, ήρθε με τη σειρά του να κάνει μια λάθος πρόβλεψη το
1977, δηλώνοντας ότι κανένας άνθρωπος δεν χρειάζεται έναν προσωπικό
υπολογιστή στο σπίτι του. Ήδη είχε εμφανιστεί ο πρώτος συναρμολογούμενος
μικρο-υπολογιστής, ο οποίος μπορούσε να προγραμματιστεί για την εκτέλεση
του βιντεοπαιχνιδιού Lunar Lander, με τη βοήθεια της BASIC, που ήταν το
πρώτο προϊόν της νέας εταιρείας Microsoft. Αρχικά ο Altair απευθυνόταν,
κυρίως, σε χομπίστες ηλεκτρονικών κατασκευών, οι οποίοι απλά ήθελαν έναν
πιο ευέλικτο σταθμό εργασίας. Η δυνατότητά του, όμως, να εκτελεί
διαφορετικά προγράμματα λογισμικού τον μετέτρεψε σε ένα κομβικό σημείο
για τη δημιουργία μιας νέας κατηγορίας προσιτών προσωπικών υπολογιστών,
αφού πολλές νέες μικρές εταιρείες επρόκειτο να δημιουργηθούν για να
εκμεταλλευτούν τις ευκαιρίες σε υλικό και λογισμικό μικρής κλίμακας,
όπως οι Apple, Microsoft, Commodore και Digital Research. \footnote{Nelson
  (2008)}

Τα περισότερα συστήματα με γραφική διεπαφή χρησιμοποιούν την επιφάνεια
εργασίας, τις εφαρμογές, και τα αρχεία εγγράφων ως τη βασική μορφή
λογισμικού διάδρασης, αλλά όλα αυτά δεν είναι παρά μόνο μια εκδοχή της
διάδρασης που μπορούμε να έχουμε για τις δουλειές του γραφείου.
Παράλληλα με την κατασκευή του Lisa, μια ομάδα της Apple κατασκευάζει
ένα αρχικό προσχέδιο του Macintosh, υπό την καθοδήγηση του Jef Raskin.
Σε αντίθεση με το Lisa, που απευθύνεται στην εταιρική αγορά του
γραφείου, το Macintosh αρχικά στοχεύει στην επεξεργασία κειμένου, η
οποία θεωρείται η πιο χρήσιμη λειτουργία των υπολογιστών και απευθύνεται
σε όλους, δηλαδή σε σχολεία, σπίτια και γραφεία. Για τον σκοπό αυτό, ο
Jef Raskin κατασκευάζει ένα μηχάνημα που είναι πολύ οικονομικό και απλό
στη χρήση του και δεν περιλαμβάνει ούτε λειτουργικό σύστημα ούτε
εφαρμογές ούτε αρχεία. Το λειτουργικό υπόδειγμα βασίζεται, όπως ακριβώς
και το Lisa, σε μια κάρτα επέκτασης του δημοφιλούς Apple II, αλλά δεν
περιλαμβάνει ούτε γραφικό περιβάλλον με παράθυρα ούτε είσοδο με ποντίκι.
Η διάδραση βασίζεται στο πληκτρολόγιο, με το οποίο ο χρήστης
επεξεργάζεται έγγραφα, τα οποία αποθηκεύονται αυτόματα σε μια δισκέτα, η
οποία αποτελεί τη φυσική αναπαράσταση του εγγράφου χωρίς να μεσολαβεί η
έννοια του αρχείου. Η πιο σημαντική διαφορά από τα συστήματα διάδρασης
εκείνης της εποχής, αλλά και όσα ακολούθησαν, είναι ότι, αντί για
εφαρμογές και λειτουργικό σύστημα, χρησιμοποιεί μια συλλογή από εντολές,
οι οποίες μπορούν να εφαρμοστούν πάνω στο έγγραφο κειμένου, παρόμοια με
τα συστήματα UNIX. Αν και αυτή η φιλοσοφία διάδρασης δεν εφαρμόστηκε
τελικά στην εμπορική εκδοχή του Macintosh, o Jef Raskin εφάρμοσε αυτές
τις ιδέες λίγο αργότερα στα εμπορικά προϊόντα SwyftCard, Canon Cat και
Archy.

Με την εξαίρεση του Jef Raskin, για τους περισσότερους κατασκευαστές
διάδρασης και σίγουρα για όλους τους χρήστες θεωρείται δεδομένο ότι τα
αρχεία, οι εφαρμογές και το λειτουργικό σύστημα είναι θεμελιώδη
συστατικά. Παρόμοια και για τον Alan Kay και την ομάδα του, αρχικά στο
Xerox PARC και αργότερα στην Disney, η αρχιτεκτονική ενός διαδραστικού
συστήματος παραμένει ανοιχτή σε ερμηνείες και σε κατευθύνσεις. Όπως στο
περιβάλλον Smalltalk δεν υπάρχει η διάκριση ανάμεσα σε εφαρμογές και
λειτουργικό σύστημα, έτσι και στο νέο περιβάλλον Squeak δεν υπάρχουν
αυτές οι έννοιες, ούτε τα αρχεία. \footnote{Kay and Goldberg (1977)} Η
βασική θεμελίωση του συστήματος Squeak γίνεται με την έννοια του
αντικειμένου, που στέλνει μηνύματα σε άλλα αντικείμενα. Με αυτόν τον
τρόπο, όλα τα γραφικά στοιχεία του συστήματος είναι αντικείμενα, τα
οποία μπορούν να προγραμματιστούν σε πραγματικό χρόνο. Όπως ακριβώς στα
συστήματα UNIX όλα είναι αρχεία, έτσι ακριβώς και στο σύστημα Squeak όλα
είναι αντικείμενα, που μπορούν να αναγνωρίσουν κάποια μηνύματα από άλλα
αντικείμενα. Ο απλός χρήστης οργανώνει το γραφικό περιβάλλον σε
επιμέρους περιοχές έργων, όπου όλα τα διαθέσιμα εργαλεία μπορούν να
εφαρμοστούν σε όλο το διαθέσιμο οπτικοακουστικό περιεχόμενο, χωρίς να
γίνεται κάποια διάκριση ανάμεσα σε διαφορετικά είδη περιεχομένου ή
διαφορετικά είδη εφαρμογής, αφού ο γραφικός χώρος εργασίας είναι
ενιαίος. Αυτό το περιβάλλον διάδρασης επιτρέπει στον χρήστη να κάνει
πολύ περισσότερα λάθη, ενώ για την καλύτερη χρήση του θα πρέπει να
υπάρχει γνώση της γλώσσας προγραμματισμού του συστήματος, η οποία είναι
μια εξέλιξη της Smalltalk. Το τίμημα της αρχικά αυξημένης δυσκολίας
χρήσης είναι η μεγαλύτερη δημιουργικότητα, πέρα από τα όρια που θέτουν
τα συμφέροντα των κατασκευαστών εφαρμογών. Τόσο οι δημιουργοί του UNIX,
όσο και αυτοί της Smalltalk θεωρούν ότι οι αντίστοιχες αρχιτεκτονικές
είναι ενδεικτικές και αντιπροσωπευτικές ενός σχεδιαστικού χώρου.
Επομένως, οι κατασκευαστές μελλοντικών συστημάτων διάδρασης θα πρέπει να
προτείνουν νέες αρχιτεκτονικές που ταιριάζουν στο θεματικό τους
ενδιαφέρον, αντί να κτίζουν απλά πάνω σε προηγούμενες ιδέες.

\leavevmode\vadjust pre{\hypertarget{fig:electrocular}{}}%
\begin{figure}
\hypertarget{fig:electrocular}{%
\centering
\includegraphics{images/electrocular.jpg}
\caption{Εικόνα 11: Η προβολή βίντεο πάνω σε έναν διάφανο φακό που
βρίσκεται στερεωμένος σε ένα κράνος πιλότου ελικοπτέρου χρησιμοποιήθηκε
από τον στρατό για να διευκολύνει την προσγείωση ελικοπτέρων με τη
βοήθεια κάμερας, σε δύσκολες συνθήκες. Αυτό το σύστημα επεκτάθηκε τα
επόμενα χρόνια ώστε, αντί για βίντεο, να προβάλει γραφικά, τα οποία
μετασχηματίζονταν ανάλογα με την κίνηση του
κεφαλιού.}\label{fig:electrocular}
}
\end{figure}

\leavevmode\vadjust pre{\hypertarget{fig:damocles-sword}{}}%
\begin{figure}
\hypertarget{fig:damocles-sword}{%
\centering
\includegraphics{images/damocles-sword.jpg}
\caption{Εικόνα 12: Tο Σπαθί του Δαμοκλή θεωρείται μια θεμελιώδης
τεχνολογία για τη δημιουργία της εικονικής πραγματικότητας, όπου τα
γραφικά στην οθόνη ακολουθούν την κίνηση του κεφαλιού. Το αρχικό σύστημα
βασιζόταν στην προβολή βίντεο από μια βιντεοκάμερα που βρισκόταν στη
βάση ενός ελικοπτέρου έτσι ώστε να διευκολύνει την προσγείωση σε
δύσκολες συνθήκες.}\label{fig:damocles-sword}
}
\end{figure}

Ένα από τα πιο σημαντικά ερευνητικά παραδείγματα διάδρασης στον χώρο της
εργασίας χωρίς λειτουργικό σύστημα, εφαρμογές και αρχεία δημιουργήθηκε
από το Αγγλικό παράρτημα του Xerox PARC στις αρχές της δεκαετίας του
ενενήντα. Την ίδια περίοδο που τα κεντρικά του PARC στην Καλιφόρνια
εξερευνούσαν τις τεχνολογίες διάδρασης με τον διάχυτο υπολογισμό, η
ερευνητική ομάδα στο Κέμπριτζ, δοκίμαζε μια εναλλακτική κατεύθυνση για
το ψηφιακό γραφείο του μέλλοντος. Αν και το Xerox PARC με τον
επιτραπέζιο Star και τον επεξεργαστή κειμένου Bravo είχε ήδη καθορίσει
τη μορφή του σύγχρονου ψηφιακού γραφείου, οι ερευνητές γνώριζαν καλύτερα
από τους χρήστες ότι αυτά τα συστήματα διάδρασης δεν ήταν μονοσήμαντα.
Μια εναλλακτική κατεύθυνση για τη διάδραση στην πραγματική επιφάνεια
εργασίας είναι η επαύξηση των αντικειμένων του γραφείου και όχι η
αντικατάστασή τους με προσομοιωμένες μορφές στον υπολογιστή. Πράγματι,
με τη χρήση της υπολογιστικής όρασης και ενός προβολέα που βρίσκονται
πάνω από το γραφείο είναι εφικτή η επαύξηση του φυσικού χαρτιού, πάνω
στο οποίο μπορούν να προβάλονται δυναμικά γραφικά. Το φυσικό χαρτί και η
γραφή με το χέρι παραμένουν στο γραφείο, το οποίο επαυξάνεται με τις
δυνατότητες του υπολογιστή και της κάμερας που είναι η βασική συσκευή
εισόδου. Τα συστήματα επαυξημένης πραγματικότητας αποτελούν ένα
παράδειγμα διάδρασης που βασίζεται περισσότερο στη σύζευξη με τον
πραγματικό κόσμο, παρά με τη προσομοίωση ή την αντικατάστασή του από μια
εικονική πραγματικότητα.

Τα συστήματα εικονικής πραγματικότητας αποτελούν μια ιδιαίτερη μορφή
διάδρασης, αφού δεν παρουσιάζουν συγγένειες με τα αντίστοιχα συστήματα
εισόδου και εξόδου και τις γραφικές διεπαφές. Η κατασκευή των πρώτων
συστημάτων εικονικής πραγματικότητας από την ομάδα του Jaron Lanier στα
τέλη της δεκαετίας του 1980 ξεκίνησε με βασικό κίνητρο ένα φιλοσοφικό
αντίβαρο στην δημοφιλία των συστημάτων τεχνητής νοημοσύνης. \footnote{Lanier
  (2017)} Τα συστήματα τεχνητής νοημοσύνης, μετά την αρχική τους
σχεδίαση και την περιστασιακή ενημέρωσή τους, δεν έχουν ανάγκη
ανθρώπινης διάδρασης για να λειτουργήσουν. Αντίθετα, το αρχικό όραμα για
τα συστήματα εικονικής πραγματικότητας βασίζεται στη συνεχή ανθρώπινη
παρουσία και διάδραση, μέσω του υπολογιστή με έναν εικονικό κόσμο, καθώς
και με τις αναπαραστάσεις άλλων χρηστών. Τόσο οι αναπαραστάσεις των
χρηστών, όσο και τα εικονικά περιβάλλοντα αρχικά σχεδιάζονται με
δημιουργικούς τρόπους, πέρα από την προσομοίωση της πραγματικότητας. Για
παράδειγμα, ένας χρήστης μπορεί να εμφανιστεί στην εικονική
πραγματικότητα με τη μορφή κάποιου ζώου, ή κάποιου κυττάρου και να
προσπαθήσει να αλληλεπιδράσει κάνοντας μια χαρτογράφηση ανάμεσα στα
διαθέσιμα συστήματα εισόδου και στις δυνατότητες κίνησης της αντίστοιχης
αναπαράστασης. Τελικά, η εικονική πραγματικότητα απομακρύνθηκε από το
αρχικό όραμα, όπου η έμφαση ήταν στον άνθρωπο και στη δημιουργικότητα
προς την εμπορική κατεύθυνση της προσομοίωσης και, κυρίως, της πιστής
απεικόνισης περισσότερο και όχι της διάδρασης.

\hypertarget{ux3c3ux3c5ux3bdux3b5ux3c1ux3b3ux3b1ux3c4ux3b9ux3baux3cc-ux3bfux3b9ux3baux3bfux3c3ux3cdux3c3ux3c4ux3b7ux3bcux3b1-ux3b4ux3b9ux3b5ux3c0ux3b1ux3c6ux3ceux3bd}{%
\subsection{Συνεργατικό οικοσύστημα
διεπαφών}\label{ux3c3ux3c5ux3bdux3b5ux3c1ux3b3ux3b1ux3c4ux3b9ux3baux3cc-ux3bfux3b9ux3baux3bfux3c3ux3cdux3c3ux3c4ux3b7ux3bcux3b1-ux3b4ux3b9ux3b5ux3c0ux3b1ux3c6ux3ceux3bd}}

\leavevmode\vadjust pre{\hypertarget{fig:xerox-pad-board}{}}%
\begin{figure}
\hypertarget{fig:xerox-pad-board}{%
\centering
\includegraphics{images/xerox-pad-board.jpg}
\caption{Εικόνα 13: Η επικοινωνία ανάμεσα στη συσκευή Tab και στη
συσκευή Board έχει πολλές συνεργατικές εφαρμογές, όπως, σε αίθουσες
συναντήσεων, καθώς και σε αίθουσες
διδασκαλίας.}\label{fig:xerox-pad-board}
}
\end{figure}

\leavevmode\vadjust pre{\hypertarget{fig:xerox-liveboard}{}}%
\begin{figure}
\hypertarget{fig:xerox-liveboard}{%
\centering
\includegraphics{images/xerox-liveboard.jpg}
\caption{Εικόνα 14: Το τρίτο και μεγαλύτερο σε μέγεθος μέλος της
οικογένειας του διάχυτου υπολογισμού του Xerox PARC είναι το Liveboard.
Βασίζεται στην είσοδο με πένα και προσομοιώνει τις λειτουργίες των
σημειώσεων και της παρουσίασης σε μια μεγάλη οθόνη τοίχου, έτσι ώστε οι
ομάδες συνεργασίας να μπορούν να μεταφέρουν χρήσιμη πληροφορία από τα
προσωπικά τερματικά τους.}\label{fig:xerox-liveboard}
}
\end{figure}

Η συνεργασία μεταξύ των ανθρώπων και μέσω των υπολογιστών είναι ένα
διαχρονικό θέμα στα συστήματα διάδρασης. \footnote{(\textbf{Εικόνα?})~13
  Σύγχρονη συνεργασία στον ίδιο χώρο (Xerox PARC)} \footnote{(\textbf{Εικόνα?})~14
  Διαδραστικός πίνακας τοίχου (Xerox PARC)} Ακόμη και στα πρώτα
πολυχρηστικά συστήματα χρονοδιαμοιρασμού υπάρχουν κοινά αποθετηρία και
ανταλλαγή ηλεκτρονικής αλληλογραφίας, έστω και με ασύγχρονο τρόπο. Το
πρώτο διαδραστικό σύστημα σύγχρονης συνεργασίας είναι το NLS του Douglas
Engelbart, όπου οι χρήστες βλέπουν το ίδιο έγγραφο στις προσωπικές
οθόνες και μπορούν να συνεργαστούν για την επεξεργασία του με ενέργειες
και χειρονομίες, οι οποίες γίνονται μέσω πολλαπλών δεικτών που
οδηγούνται από το ποντίκι του κάθε χρήστη. \footnote{Engelbart (1988)}
Χρειάστηκε να περάσουν τουλάχιστον δύο δεκαετίες από το NLS για να
φτάσουμε στο σύστημα διάχυτου υπολογισμού του PARC, το οποίο προσθέτει
πολλές διαφορετικές συσκευές συνεργασίας. Σε όλες αυτές τις περιπτώσεις
συνεργασίας ανάμεσα σε ομάδες ανθρώπων και μηχανών, η πιο σημαντική
παράμετρος σχεδίασης δεν είναι η ποιότητα της διεπαφής και η εργονομία
της συσκευής, αλλά κυρίως ένας ισορροπημένος σχεδιασμός του μίγματος
επικοινωνίας, που γίνεται μέσω των μηχανών και απευθειάς μεταξύ των
ανθρώπων.

Μετά το περιβάλλον εργασίας, οι κοινότητες μάθησης όρισαν διαφορετικά
μίγματα στην συνεργασία ανθρώπων και μηχανών. Η κατασκευή και η εκτέλεση
των δημοφιλών βιντεοπαιχνιδιών, αρχικά δεν γινόταν μόνο για ψυχαγωγικούς
σκοπούς, αλλά και για εμπορικούς, καθώς και για εκπαιδευτικούς σκοπούς.
Το βιντεοπαιχνίδι Spacewar δημιουργήθηκε αρχικά για να εξερευνηθούν τα
όρια των δυνατοτήτων διάδρασης σε πραγματικό χρόνο με τον
πρωτοεμφανιζόμενο μινι-υπολογιστή DEC PDP-1. Το Spacewar, στη συνέχεια,
ήταν σημείο αναφοράς για την δημιουργία του Dynabook και της Smalltalk
από τον Alan Kay, ο οποίος ήθελε τα παιδιά να μπορούν όχι μόνο να
διασκεδάσουν, αλλά να μπορούν και να υλοποιήσουν το Spacewar σε μια
προσβάσιμη για αυτά γλώσσα προγραμματισμού. Τόσο η Smalltalk, όσο και το
Dynabook δεν σχεδιάστηκαν ανεξάρτητα από το πλαίσιο λειτουργίας τους,
αλλά έγιναν με την παραδοχή ότι τα παιδιά υποστηρίζονται από έναν
δάσκαλο και το μαθησιακό υλικό περιλαμβάνει ένα χαρτοφυλάκιο έτοιμων
έργων.

Μερικά χρόνια αργότερα, με παρόμοιο σκεπτικό, ο Paul Allen υλοποιεί το
πρώτο προϊόν της νεοσύστατης Microsoft, την πρώτη Basic για τον
συναρμολογούμενο μικρο-υπολογιστή Altair, έτσι ώστε να μπορεί κάποιος να
κατασκευάσει το δημοφιλές βιντεοπαιχνίδι Lunar Lander. Σχεδόν παράλληλα,
το ίδιο ακριβώς πνεύμα διατρέχει και την δουλειά του Steve Wozniak κατά
την ανάπτυξη του μικρο-υπολογιστή Apple II και της Apple Basic, τα οποία
υλοποιήθηκαν έτσι ώστε να μπορεί κάποιος να αναπτύξει σχετικά εύκολα το
δημοφιλές βιντεοπαιχνίδι Breakout. Σε όλες αυτές τις περιπτώσεις, ο
στόχος είναι εξερεύνηση και η επίδειξη των δυνατοτήτων ενός υπολογιστή
αλλά και η μάθηση μέσα από την αντιγραφή κώδικα διαθέσιμου σε βιβλία και
περιοδικά, καθώς και η μετατροπή και η προσαρμογή του, σύμφωνα με τις
προτιμήσεις του κάθε χρήστη.

\leavevmode\vadjust pre{\hypertarget{fig:printed-code}{}}%
\begin{figure}
\hypertarget{fig:printed-code}{%
\centering
\includegraphics{images/printed-code.jpg}
\caption{Εικόνα 15: Πριν την εξάπλωση του δικτύου και των μέσων
αποθήκευσης, ο βασικός τρόπος διαμοιρασμού του λογισμικού ήταν σε πηγαίο
κώδικα, τυπωμένο σε χαρτί περιοδικού ή βιβλίου, τα οποία ο χρήστης θα
μελετούσε και μετά θα πληκτρολογούσε στον μικρο-υπολογιστή του. Αν και η
διαδικασία αυτή φαίνεται κάπως ξεπερασμένη, φαίνεται ότι είχε το
πλεονέκτημα μια αίσθησης ιδιοκτησίας, καθώς και μιας έμμεσης ενθάρρυνσης
για σκόπιμες ή ακούσιες μετατροπές, οι οποίες τελικά λειτουργούσαν
θετικά για τις γνώσεις και τις δεξιότητες του
χρήστη.}\label{fig:printed-code}
}
\end{figure}

\leavevmode\vadjust pre{\hypertarget{fig:c64-demoscene}{}}%
\begin{figure}
\hypertarget{fig:c64-demoscene}{%
\centering
\includegraphics{images/c64-demoscene.jpg}
\caption{Εικόνα 16: Η προσιτή τιμή του μικρο-υπολογιστή Commodore 64 τον
έκανε δημοφιλή σε περισσότερο από δέκα εκατομύρια χρήστες, οι οποίοι,
εκτός από την αγορά λογισμικού και περιφερειακών, δημιούργησαν μια
κοινότητα ερασιτεχνικής δημιουργίας και ανταλλαγής προγραμμάτων και
παιχνιδιών, που διατηρήθηκε ως μέρος της ψηφιακής κουλτούρας και μετά
την απόσυρση αυτού του μοντέλου.}\label{fig:c64-demoscene}
}
\end{figure}

Τα πρώτα χρόνια διάθεσης των μικρο-υπολογιστών δεν υπήρχε αρκετό
διαθέσιμο λογισμικό και αυτό οδήγησε τους περισσότερους χρήστες στην
αναζήτηση πηγαίου κώδικα από βιβλία, περιοδικά, καθώς και από άλλους
χρήστες. Με αυτόν τον τρόπο, δημιουργείται μια νέα κατηγορία περιοδικού
τύπου, όπου δημοσιεύονται ολοκληρωμένα προγράμματα ή δίνονται λύσεις σε
επιμέρους προβλήματα προγραμματισμού. \footnote{(\textbf{Εικόνα?})~15
  Διαμοιρασμός κώδικα με βιβλία και περιοδικά (David H. Ahl)}
Ταυτόχρονα, οι χρήστες δημιουργούν ομάδες ενδιαφέροντος και οργανώνουν
φυσικές συναντήσεις, \footnote{(\textbf{Εικόνα?})~16 Commodore64
  demoscene (Wikimedia)} με στόχο τη συνεργασία ή και τον ανταγωνισμό
στην κατασκευή πειραματικών προγραμμάτων και βιντεοπαιχνιδιών.

Παράλληλα με την επιτυχία και τη διάδοση των πρώτων μικρο-υπολογιστών
από τις Apple και Commodore, η Atari ακολουθεί μια διαφορετική πορεία,
όπου ο χρήστης αποκτάει πρόσβαση σε μια συλλογή από εμπορικά
βιντεοπαιχνίδια. Οι χρήστες δεν έχουν τη δυνατότητα να χρησιμοποιήσουν
την κονσόλα για ανάπτυξη, παρά μόνο για να διαδράσουν με το έτοιμο
λογισμικό, το οποίο δημιουργεί τον νέο μεγάλο κλάδο της βιομηχανίας των
βιντεοπαιχνιδιών, που θα γνωρίσει συνεχή ανάπτυξη τις επόμενες
δεκαετίες. Αν και η Atari δεν μπόρεσε να εκμεταλλευτεί την αγορά που
δημιούργησε με την κονσόλα 2600 και με κλασικά βιντεοπαιχνίδια, όπως το
Space Invaders, η επίδρασή της θα είναι καταλυτική τόσο στη βιομηχανία
των βιντεοπαιχνιδιών αλλά και σε σχετικούς κλάδους, όπως είναι τα
καταναλωτικά ηλεκτρονικά, αλλά και στην ευρύτερη ψηφιακή κουλτούρα.
Πράγματι, νέες εταιρείες θα δημιουργηθούν για να εκμεταλλευτούν την
ευκαιρία που δημιούργησε η Atari, όπως οι Nintento και η Sega, ενώ και
οι υπάρχουσες εταιρείες θα προσπαθήσουν να αποκτήσουν ένα μερίδιο από
αυτήν τη νέα και αναπτυσσόμενη αγορά, όπως οι Sony και η Microsoft.

Ο ψηφιακός αλφαβητισμός είναι καθολικά αποδεκτός ως μια πολύ βασική
δεξιότητα, ανεξάρτητα από τις προσωπικές και επαγγελματικές επιδιώξεις
του κάθε ανθρώπου. Στις πιο πρόσφατες φάσεις διάδοσης της διάδρασης με
επιτραπέζιους υπολογιστές γραφείου, ο ψηφιακός αλφαβητισμός εξαντλήθηκε
στην κατανόηση της χρήσης του υπολογιστή, αλλά τελικά έγινε σαφές ότι ο
αλφαβητισμός, εκτός από την ανάγνωση, έχει ως αναγκαία προϋπόθεση και
ένα βαθμό δεξιότητας στη συγγραφή, η οποία βασίζεται στην γραφή που
είναι από τις σημαντικότερες τεχνολογικές εφευρέσεις της ανθρωπότητας.
Φυσικά, όπως δεν έχουμε την απαίτηση από τον μέσο άνθρωπο να γράφει
κείμενο όπως ένας κορυφαίος συγγραφέας, έτσι ακριβώς δεν έχουμε την
απαίτηση να μπορεί να δημιουργήσει καινοτόμα προγράμματα διάδρασης. Από
την άλλη πλευρά, η δυνατότητα να παρέμβει στη δημιουργία και προσαρμογή
προγραμμάτων διάδρασης ανοιχτού κώδικα που του ταιριάζουν είναι μια
δεξιότητα που αυξάνει τις δυνατότητές του για έκφραση και δημιουργία,
τόσο στην προσωπική όσο και στην επαγγελματική του ζωή.

Η εφαρμογή των υπολογιστών στην εκπαίδευση ξεκίνησε με τα ρομπότ-χελώνες
του Seymour Papert στο ΜΙΤ, τα οποία συνέχισε ο Alan Kay με τη Smalltalk
στο Xerox PARC. Αυτές οι αρχικές φιλόδοξες προσπάθειες παραμένουν για
πολλές δεκαετίες μετέωρες, όχι τόσο γιατί είναι ανέφικτες, αλλά, κυρίως,
γιατί δεν έχει δημιουργηθεί το κατάλληλο οργανωσιακό και γνωστικό
πλαίσιο για την ευρύτερη ορθή εφαρμογή τους. Αντίθετα, η εφαρμογή των
υπολογιστών στην εκπαίδευση έχει βρει πολύ γόνιμο έδαφος εκεί που οι
υπολογιστές χρησιμοποιούνται ως εργαλεία για την επιφανειακή μετάδοση
γνώσεων σε άλλες γνωστικές περιοχές ή στην καλύτερη περίπτωση για την
ανάπτυξη δεξιοτήτων για τον προγραμματισμό τους. Για παράδειγμα, η σειρά
υπολογιστών Plato περιλαμβάνει διαδραστικές ασκήσεις στις φυσικές
επιστήμες, ενώ ο BBC Micro αντιγράφει τη δημοφιλή φόρμα των
μικρο-υπολογιστών δίνοντας έμφαση, εκτός από την διανομή εκπαιδευτικού
λογισμικού, και στον προγραμματισμό.

\hypertarget{ux3b7-ux3c0ux3b5ux3c1ux3afux3c0ux3c4ux3c9ux3c3ux3b7-ux3c4ux3bfux3c5-unix}{%
\subsection{Η περίπτωση του
UNIX}\label{ux3b7-ux3c0ux3b5ux3c1ux3afux3c0ux3c4ux3c9ux3c3ux3b7-ux3c4ux3bfux3c5-unix}}

\leavevmode\vadjust pre{\hypertarget{fig:pdp11-tty-unix}{}}%
\begin{figure}
\hypertarget{fig:pdp11-tty-unix}{%
\centering
\includegraphics{images/pdp11-tty-unix.jpg}
\caption{Εικόνα 17: Ο μίνι-υπολογιστής PDP11 της DEC θεωρείται κομβικό
σημείο, γιατί ήταν πολύ δημοφιλής σε πανεπιστήμια και ερευνητικά κέντρα,
όπου χρησιμοποιήθηκε για την διάδοση του λειτουργικού συστήματος UNIX,
της γλώσσας προγραμματισμού C και, κυρίως, των εξομοιωτών που επέτρεψαν
την ανάπτυξη νέων συστημάτων για τους μικρο-υπολογιστές που αναπτύχθηκαν
στο τέλος της δεκαετίας του 1970 και έφεραν τον υπολογισμό στα σπίτια
και στους απλούς χρήστες.}\label{fig:pdp11-tty-unix}
}
\end{figure}

\leavevmode\vadjust pre{\hypertarget{fig:unix-shell}{}}%
\begin{figure}
\hypertarget{fig:unix-shell}{%
\centering
\includegraphics{images/unix-shell.png}
\caption{Εικόνα 18: Η διεπαφή του πυρήνα του λειτουργικού συστήματος
UNIX ονομάστηκε κέλυφος, καθώς κρύβει τις λεπτομέρειες της υλοποίησης
και παρέχει μια προσβάσιμη διάδραση για τον χρήστη. Εκτός από τις
βασικές λειτουργίες του συστήματος, παρέχει και μια απλή γλώσσα
προγραμματισμού μεταγλώτισης, η οποία επιτρέπει τη φορητότα των
προγραμμάτων του χρήστη σε παρόμοια συστήματα.}\label{fig:unix-shell}
}
\end{figure}

Το Unix είναι περισσότερο γνωστό σε όσους ασχολούνται με τα λειτουργικά
συστήματα και σχεδόν καθόλου γνωστό στην κοινότητα της διάδρασης και των
γραφικών διεπαφών χρήστη. Πράγματι, οι γραφικές διεπαφές σε όλα τα Unix
είναι συνήθως αντίγραφα από τις αντίστοιχες εμπορικές, με βασικό σημείο
διαφοροποίησης τη διάθεση ανοικτού πηγαίου κώδικα. Εκτός από τη γραφική
διεπαφή όμως, το Unix παρέχει και, κυρίως, βασίζεται στη διεπαφή της
γραμμής εντολών, η οποία είναι ένα από τα σημαντικότερα κεφάλαια στην
διάδραση.

Η διάδραση με τη γραμμή εντολών είναι ένας από τους πρώτους τρόπους
χρήσης των υπολογιστών, αλλά η αξία της παραμένει διαχρονική. Η γραμμή
εντολών έγινε αρχικά δημοφιλής με το σύστημα Unix, ενώ διατηρεί τη
χρησιμότητα της σε σύγχρονα συστήματα, όπως στις διεπαφές γραπτών
μηνυμάτων και στις φωνητικές πύλες. Αν και θεωρείται δύσκολη στη χρήση,
τουλάχιστον για τους αρχάριους, η διαχρονικότητά της, καθώς και η
σταθερή προτίμηση από τους ειδικούς τεκμηριώνουν τη σημασία της.

Εκτός από τη γραμμή εντολών, το Unix έκανε δημοφιλή την οργάνωση ενός
λειτουργικού συστήματος σε αρχεία και φακέλους, τα οποία
χρησιμοποιήθηκαν και από τα περισσότερα γραφικά περιβάλλοντα που
ακολούθησαν. Η πιο σημαντική συνεισφορά αυτού του συστήματος είναι στο
σημείο συνάντησης των αρχείων με τη γραμμή εντολής, όπου δημιουργήθηκαν
οι γλώσσες κελύφους, \footnote{(\textbf{Εικόνα?})~18 UNIX shell
  (Wikimedia)} καθώς και η διασωλήνωση των προγραμμάτων.\footnote{Kernighan
  (2019)} Σε αντίθεση με την ιδέα της εμπορικής διάθεσης εφαρμογών με
πολλές δυνατότητες, το Unix βασίζεται στην ιδέα των πολλών μικρών
προγραμμάτων, τα οποία παραμετροποιούνται, συνδέονται μεταξύ τους και
τελικά συνθέτουν νέα προγράμματα, σύμφωνα με τις ανάγκες του χρήστη.

Ακόμη και η γραφική διεπαφή έγινε με έμφαση στο δίκτυο, έτσι ώστε να
υπάρχει ένας διαχωρισμός ανάμεσα στο μηχάνημα που εκτελεί μια εφαρμογή
και στο τερματικό του χρήστη που κάνει τη διάδραση. Τέλος, το Unix, από
τη δημιουργία του τη δεκαετία του 1960, δίνει μεγάλη έμφαση σε μια
κοινότητα χρηστών, οι οποίοι δουλεύουν μαζί για την ανάπτυξη του βασικού
συστήματος και, κυρίως, για την ανταλλαγή προγραμμάτων.\footnote{(\textbf{Εικόνα?})~17
  Μίνι υπολογιστής PDP11 (Courtey of Wikimedia)} Η ιδέα της κοινότητας,
σε αντίθεση με την ιδέα του προϊόντος, ήταν θεμελιώδης για τη δημιουργία
αντίστοιχων κοινοτήτων κατά τις επόμενες δεκαετίες, όπως αυτές των
πρώτων δικτυακών συζητήσεων, της σκηνής των δοκιμαστικών προγραμμάτων
και, κυρίως, για τη δημιουργία του ανοιχτού λογισμικού, τη δεκαετία του
1990.

\hypertarget{ux3b7-ux3c0ux3b5ux3c1ux3afux3c0ux3c4ux3c9ux3c3ux3b7-ux3c4ux3b7ux3c2-ux3baux3bfux3bdux3c3ux3ccux3bbux3b1ux3c2-ux3b2ux3b9ux3bdux3c4ux3b5ux3bfux3c0ux3b1ux3b9ux3c7ux3bdux3b9ux3b4ux3b9ux3ceux3bd}{%
\subsection{Η περίπτωση της κονσόλας
βιντεοπαιχνιδιών}\label{ux3b7-ux3c0ux3b5ux3c1ux3afux3c0ux3c4ux3c9ux3c3ux3b7-ux3c4ux3b7ux3c2-ux3baux3bfux3bdux3c3ux3ccux3bbux3b1ux3c2-ux3b2ux3b9ux3bdux3c4ux3b5ux3bfux3c0ux3b1ux3b9ux3c7ux3bdux3b9ux3b4ux3b9ux3ceux3bd}}

Οι κονσόλες βιντεοπαιχνιδιών και τα βιντεοπαιχνίδια που τρέχουν σε αυτές
αναπτύχθηκαν παράλληλα με τους οικιακούς μικρο-υπολογιστές και έχουν
ιδιαίτερη σημασία γιατί σε πολλές περιπτώσεις δεν βασίζονται στα
κυρίαρχα συστήματα διάδρασης. Πράγματι, ένα βιντεοπαιχνίδι αξιολογείται
με κριτήριο τη δημιουργικότητα, οπότε οι κατασκευαστές του συνήθως
αποφεύγουν τα κυρίαρχα αρχέτυπα διάδρασης και σχεδιάζουν νέες
διεπαφές.\footnote{(\textbf{Εικόνα?})~20 Mattel PowerGlove (Mattel)} Για
την περίπτωση των δημοφιλών παιχνιδιών πρώτου προσώπου ή για τα
παιχνίδια ρόλων δημιουργούνται εύχρηστα εργαλεία κατασκευής για τις
αντίστοιχες κατηγορίες.

\leavevmode\vadjust pre{\hypertarget{fig:atari-2600}{}}%
\begin{figure}
\hypertarget{fig:atari-2600}{%
\centering
\includegraphics{images/atari-2600.jpg}
\caption{Εικόνα 19: Σε αντίθεση με τους πρώτους μικρο-υπολογιστές
εκείνης της χρονιάς, η κονσόλα της Atari εστιάζει μόνο στη διασκέδαση με
βιντεοπαιχνίδια, τα οποία αγοράζονται όπως τα βιβλία και οι δίσκοι και
φορτώνονται με εξωτερικές κάρτες μνήμης ανάγνωσης. Αντί για
πληκτρολόγιο, έχει μοχλό, για τον έλεγχο των χαρακτήρων στα
βιντεοπαιχνίδια. Ο βιομηχανικός σχεδιασμός ταιριάζει με αυτόν της
τηλεόρασης, με την οποία συνδέεται ως συσκευή εξόδου και, με τη βοήθεια
του βιντεοπαιχνιδιού Space Invaders, δημιουργεί μια νέα κατηγορία
διαδραστικού υπολογιστή.}\label{fig:atari-2600}
}
\end{figure}

\leavevmode\vadjust pre{\hypertarget{fig:power-glove}{}}%
\begin{figure}
\hypertarget{fig:power-glove}{%
\centering
\includegraphics{images/power-glove.jpg}
\caption{Εικόνα 20: Το πρώτο προσιτά οικονομικό και ευρέως διαθέσιμο
γάντι διάδρασης είχε πολλά προβλήματα και λίγο διαθέσιμο λογισμικό, αλλά
αυτό δεν εμπόδισε πολλούς προχωρημένους χρήστες να το προσαρμόσουν σε
διαφορετικές εφαρμογές πέρα από τα βίντεοπαιχνίδια για τα οποία
κατασκευάστηκε.}\label{fig:power-glove}
}
\end{figure}

Η οικιακή ανάπτυξη και ο διαμοιρασμός του κώδικα είναι μια δημοφιλής
πρακτική, από τα πρώτα βήματα διάδοσης των βιντεοπαιχνιδιών στους
μικρο-υπολογιστές. Ο άμεσος προγραμματισμός των πρώτων μικρο-υπολογιστών
προσφέρει σε μια γενιά χρηστών μια περισσότερο παραγωγική οπτική στο
φαινόμενο της διάδρασης με τους υπολογιστές. Επίσης, οι κατασκευαστές
βιντεοπαιχνιδιών συχνά επιλέγουν να παρακάμψουν το λειτουργικό σύστημα,
έτσι ώστε να πετύχουν βελτιστοποιημένες επιδόσεις για τις δημιουργίες
τους.

Οι εμπορικές κονσόλες βιντεοπαιχνιδιών συνεισφέρουν στο φαινόμενο της
διάδρασης με νέες συσκευές εισόδου, όπως είναι τα χειριστήρια με πολλά
κουμπιά και αισθητήρες, καθώς και η αναγνώριση εικόνας. Σε αντίθεση με
το παραδοσιακό πληκτρολόγιο και το ποντίκι, οι κονσόλες γίνονται
πλατφόρμες πειραματισμού για νέες συσκευές διάδρασης, οι οποίες
δημιουργούν νέες κατηγορίες βιντεοπαιχνιδιών, σε πεδία όπως είναι η
μουσική και η γυμναστική. Νέες συσκευές διάδρασης, όπως είναι η μάσκα
εικονικής πραγματικότητας, χρησιμοποιούν την δημοτικότητα των
βιντεοπαιχνιδιών για να διατεθούν εμπορικά σε περισσότερο προσιτές
τιμές.

Η ευρεία διαθεσιμότητα πολύ ισχυρών πολυμεσικών συστημάτων επιτρέπει την
εξομοίωση παλιότερων συστημάτων που δεν είναι πλέον διαθέσιμα, όπως ήταν
οι κονσόλες βιντεοπαιχνιδιών με κέρματα και οι κονσόλες με
κασέτες.\footnote{(\textbf{Εικόνα?})~19 Atari 2600 (Wikipedia)} Η
πρακτική αυτή επιτρέπει τη μουσειακή διατήρηση παλιότερων συστημάτων,
αλλά και τη μελέτη τους από τους νεότερους που δεν έχουν δει αυτά τα
συστήματα. Επιπλέον, η ίδια η πρακτική της εξομοίωσης επιτρέπει την
ανάπτυξη νέων εικονικών συστημάτων, για τα οποία δεν υπάρχει αντίστοιχο
υλικό, με στόχο τον πειραματισμό και τη δημιουργικότητα.

\hypertarget{ux3c3ux3cdux3bdux3c4ux3bfux3bcux3b7-ux3b2ux3b9ux3bfux3b3ux3c1ux3b1ux3c6ux3afux3b1-ux3c4ux3bfux3c5-jef-raskin}{%
\subsection{Σύντομη βιογραφία του Jef
Raskin}\label{ux3c3ux3cdux3bdux3c4ux3bfux3bcux3b7-ux3b2ux3b9ux3bfux3b3ux3c1ux3b1ux3c6ux3afux3b1-ux3c4ux3bfux3c5-jef-raskin}}

Ο Jef Raskin, όπως και οι περισσότεροι πρωτοπόροι των υπολογιστών, δεν
σπούδασε κάτι σχετικό με την πληροφορική, ενώ παράλληλα είχε πολλά
ενδιαφέροντα με κύριο τη μουσική εκτέλεση. Ξεκίνησε την καριέρα του ως
καθηγητής πανεπιστημίου, αλλά η εισαγωγή των πρώτων μικρο-υπολογιστών
τον βρήκε να δημιουργεί μια εταιρεία που έφτιαχνε εγχειρίδια χρήσης για
απλούς χρήστες και να αρθρογραφεί στον περιοδικό τύπο του κλάδου. Με
αυτόν τον τρόπο συνάντησε τους ιδρυτές της Apple και άρχισε να γράφει
ένα καλύτερο εγχειρίδιο για τον Apple II.

\leavevmode\vadjust pre{\hypertarget{fig:swyftware}{}}%
\begin{figure}
\hypertarget{fig:swyftware}{%
\centering
\includegraphics{images/swyftware.jpg}
\caption{Εικόνα 21: Το σύστημα SwyftWare βασιζόταν σε μια κάρτα
επέκτασης για τον Apple II και σε λογισμικό διάδρασης για την
επεξεργασία εγγράφων κειμένου, η οποία γινόταν μόνο με το πληκτρολόγιο
και μερικές εντολές. Η κεντρική ιδέα σε αυτό το σύστημα είναι ότι η
διάδραση με τα έγγραφα γίνεται πιο απλή αν αποφύγουμε ιδέες που
σχετίζονται με μηχανήματα γενικής χρήσης, όπως είναι τα αρχεία, οι
εφαρμογές, και το λειτουργικό σύστημα.}\label{fig:swyftware}
}
\end{figure}

\leavevmode\vadjust pre{\hypertarget{fig:raskin-profile}{}}%
\begin{figure}
\hypertarget{fig:raskin-profile}{%
\centering
\includegraphics{images/raskin-profile.jpg}
\caption{Εικόνα 22: Ο Jef Raskin είναι από τους λίγους κατασκευαστές της
διάδρασης που συνδύασε μια ανθρωποκεντρική φιλοσοφία με την πρακτική
εφαρμογή σε ένα προϊόν επεξεργασίας κειμένου. Το αποτέλεσμα αυτής της
προσπάθειας αποτελεί παράδειγμα προς μίμηση και ταυτόχρονα μας θυμίζει
ότι τα περισσότερα πετυχημένα εμπορικά προϊόντα δεν εξυπηρετούν την
αντικειμενική ευχρηστία, αλλά μόνο μια βολική συνήθεια των χρηστών και
τα συμφέροντα των κατασκευαστών.}\label{fig:raskin-profile}
}
\end{figure}

Τόσο η τεχνολογική κατανόηση που είχε για τους μικρο-υπολογιστές, όσο
και η έμφαση που έδινε στην απλότητα της χρήσης, του έδωσαν τη θέση του
υπεύθυνου ανάπτυξης για το έργο Macintosh. Από αυτήν την θέση προσπάθησε
να φτιάξει ένα απλό και οικονομικό μηχάνημα, το οποίο έδινε έμφαση στην
επεξεργασία κειμένου. Τελικά, ο Macintosh πέρασε στη διοίκηση του Steve
Jobs, ο οποίος πρόσθεσε το ποντίκι και το παραθυρικό περιβάλλον, που
ήταν μια πολύ διαφορετική κατεύθυνση. Για τον Jef Raskin η
αντιλαμβανόμενη ευχρηστία ενός συστήματος διάδρασης επηρεάζεται κυρίως
από την υποκειμενική οικειότητα με προηγούμενα παρόμοια συστήματα ή από
μεταφορές από τον φυσικό κόσμο, τα οποία, όμως, δεν είναι αντικειμενικά
βέλτιστα, ειδικά για τους συχνούς χρήστες.

Η στοχοπροσήλωση που είχε για ένα πολύ απλό και εστιασμένο στην
επεξεργασία κειμένου μηχάνημα διάδρασης τον οδήγησαν στη δημιουργία μιας
ακόμη εταιρείας. Το πρώτο προϊόν είναι το SwyftWare,\footnote{(\textbf{Εικόνα?})~21
  SwyftWare (hackaday)} το οποίο είναι ένας επεξεργαστής κειμένου χωρίς
αρχεία, χωρίς εφαρμογές, και χωρίς λειτουργικό σύστημα, τα οποία θεωρεί
πως μπερδεύουν τον χρήστη με περιττές επιλογές, που αυξάνουν το κόστος
χωρίς να προσφέρουν σημαντικές λειτουργίες. Για τη λειτουργικότητα του
επεξεργαστή κειμένου, το SwyftWare και το Canon Cat\footnote{(\textbf{Εικόνα?})~22
  Jef Raskin (Jef Raskin)} βασίζονται στο πληκτρολόγιο και σε
συντομεύσεις για την εκτέλεση εντολών.

Την περίοδο που ήταν καθηγητής στο τμήμα τεχνών, ανέπτυξε τη γλώσσα
προγραμματισμού FLOW, η οποία απευθύνεται σε σπουδαστές των
ανθρωπιστικών επιστημών. Όπως η BASIC έδωσε εύκολη πρόσβαση στον
προγραμματισμό για τις φυσικές επιστήμες και τα μαθηματικά ως σκαλοπάτι
πριν τη FORTRAN, έτσι και η FLOW άνοιξε τον προγραμματισμό για ένα νέο
κοινό. Η FLOW είχε έμφαση στην επεξεργασία δεδομένων κειμένου και όχι
αριθμών και καινοτόμησε με τη χρήση της αυτόματης συμπλήρωσης των
εντολών προγραμματισμού. Η FDLOW χρησιμοποιήθηκε για να διδάξει
προγραμματισμό σε σπουδαστές τεχνών τριάντα χρόνια πριν η ιδέα αυτή
υλοποιηθεί πάλι στο δημοφιλές Processing.

\hypertarget{ux3b2ux3b9ux3b2ux3bbux3b9ux3bfux3b3ux3c1ux3b1ux3c6ux3afux3b1}{%
\subsection*{Βιβλιογραφία}\label{ux3b2ux3b9ux3b2ux3bbux3b9ux3bfux3b3ux3c1ux3b1ux3c6ux3afux3b1}}
\addcontentsline{toc}{subsection}{Βιβλιογραφία}

\hypertarget{refs}{}
\begin{CSLReferences}{0}{0}
\end{CSLReferences}

Engelbart, Douglas. 1988. {``The Augmented Knowledge Workshop.''} In
\emph{A History of Personal Workstations}, 185--248.

Freiberger, Paul, and Michael Swaine. 1984. \emph{Fire in the Valley:
The Making of the Personal Computer}. McGraw-Hill, Inc.

Hertzfeld, Andy. 2004. \emph{Revolution in the Valley
{{[}}Paperback{{]}}: The Insanely Great Story of How the Mac Was Made}.
" O'Reilly Media, Inc.".

Kay, Alan, and Adele Goldberg. 1977. {``Personal Dynamic Media.''}
\emph{Computer} 10 (3): 31--41.

Kernighan, Brian W. 2019. \emph{UNIX: A History and a Memoir}. Kindle
Direct Publishing.

Lanier, Jaron. 2017. \emph{Dawn of the New Everything: Encounters with
Reality and Virtual Reality}. Henry Holt; Company.

Nelson, Ted. 2008. \emph{Geeks Bearing Gifts}. Mindful Pr.

Waldrop, M Mitchell. 2001. \emph{The Dream Machine: JCR Licklider and
the Revolution That Made Computing Personal}. Viking Penguin.

\hypertarget{ux3c4ux3b5ux3c7ux3bdux3bfux3bbux3bfux3b3ux3afux3b1}{%
\section{Τεχνολογία}\label{ux3c4ux3b5ux3c7ux3bdux3bfux3bbux3bfux3b3ux3afux3b1}}

\begin{quote}
Οι κατασκευαστές που σχεδιάζουν προσεκτικά το λογισμικό τους φτιάχνουν
και το δικό τους υλικό. Alan Kay
\end{quote}

\hypertarget{ux3c0ux3b5ux3c1ux3afux3bbux3b7ux3c8ux3b7}{%
\subsubsection{Περίληψη}\label{ux3c0ux3b5ux3c1ux3afux3bbux3b7ux3c8ux3b7}}

Η τεχνολογία διάδρασης είναι πλέον ευρέως διαθέσιμη σε κάποιες λίγες
μορφές, που καθιστούν δύσκολη τη διάκριση ανάμεσα στη βασική τεχνολογία
και στα στιγμιότυπα οργάνωσής της. Ο δημοφιλής επιτραπέζιος υπολογιστής
για αρκετές δεκαετίες συνοδεύεται από ένα γραφικό περιβάλλον εργασίας,
το οποίο από τους περισσότερους θεωρείται τεχνολογία, αλλά, όπως είδαμε
στα προηγούμενα, είναι περισσότερο μια τεχνολογική μορφή, που
κατασκευάστηκε σε ένα δεδομένο τεχνο-οικονομικό πλαίσιο. Παρόμοια, η
τεχνολογία εμβύθισης των συστημάτων εικονικής πραγματικότητας μπορεί να
χρησιμοποιηθεί για την επαύξηση της ανθρώπινης εμπειρίας προς
κατευθύνσεις που δεν είναι απαραίτητα συμβατές με το φυσικό περιβάλλον.
Στην πράξη όμως, τα περισσότερα συστήματα εικονικής πραγματικότητας
χρησιμοποιούνται για να δημιουργήσουν προσομοιώσεις της φυσικής
πραγματικότητας. Ιδιαίτερο ενδιαφέρον παρουσιάζει η τεχνολογία ανοιχτού
κώδικα, ο οποίος επιτρέπει τη μελέτη και, ανάλογα με την άδεια χρήσης,
την προσαρμογή του αρχικού κώδικα για διαφορετικούς σκοπούς. Ο ανοιχτός
κώδικας είναι μια σημαντική ιδιότητα στην τεχνολογία διάδρασης, αλλά δεν
μπορεί να είναι ο σκοπός της, γιατί τότε τα νέα συστήματα διάδρασης
μπορεί να είναι απλά αντίγραφα των μορφών διάδρασης από τα ήδη υπάρχοντα
κλειστού κώδικα.

\hypertarget{ux3c4ux3b5ux3c7ux3bdux3bfux3bbux3bfux3b3ux3b9ux3baux3ac-ux3c0ux3c1ux3bfux3caux3ccux3bdux3c4ux3b1}{%
\subsection{Τεχνολογικά
προϊόντα}\label{ux3c4ux3b5ux3c7ux3bdux3bfux3bbux3bfux3b3ux3b9ux3baux3ac-ux3c0ux3c1ux3bfux3caux3ccux3bdux3c4ux3b1}}

Στην τεχνολογία λογισμικού υπάρχουν συνήθως πολλά επίπεδα αφαίρεσης, τα
οποία κάνουν δύσκολη τη διάκριση ανάμεσα στη βασική τεχνολογία και στο
τεχνολογικό προϊόν. Επίσης, συμβαίνει συχνά κάποια δημοφιλή προϊόντα να
δίνουν το όνομά τους σε όλη την κατηγορία, όπως ακριβώς συμβαίνει σε
πολλά καταναλωτικά προϊόντα. Για παράδειγμα, υπήρχε μια περίοδος που η
οδοντόκρεμα λεγόταν kolynos και η χλωρίνη καθαρισμού λεγόταν kleenex.
Αντίστοιχα, ο παγκόσμιος ιστός δεν είναι τεχνολογία, αλλά ένα δημοφιλές
προϊόν της τεχνολογικής κατηγορίας των υπερμέσων. Φυσικά, ούτε τα
Facebook, iPhone, Android, Windows, κτλ. είναι τεχνολογίες. Το ίδιο
ισχύει και για τις δημοφιλείς γλώσσες αντικειμενοστραφούς
προγραμματισμού, όπως είναι η Java, η οποία δημιουργήθηκε για να
εξυπηρετήσει τις ανάγκες μιας εποχής και, κυρίως, τις ανάγκες μιας
εταιρείας που λειτουργούσε σε ένα δεδομένο πολιτισμικό πλαίσιο, οπότε
έχει πολλές ιδιότητες διαφορετικές από αυτές της Smalltalk, \footnote{Ingalls
  (2020)} που ήταν η πρώτη γλώσσα του είδους είκοσι χρόνια νωρίτερα.
\footnote{Roszak (1986)} Η διάκριση ανάμεσα σε τεχνολογία και προϊόν
γίνεται ακόμη δυσκολότερη στην περίπτωση που πολλές διαφορετικές
εταιρείες χρησιμοποιούν την ίδια έννοια, όπως για παράδειγμα τα αρχεία
και τις εφαρμογές. Η συμφωνία χρήσης αρχείων και εφαρμογών σε προϊόντα
διαφορετικών εταιρειών δημιουργεί την ψευδαίσθηση ότι αυτά είναι βασικές
τεχνολογίες διάδρασης, αλλά στην πράξη είναι απλά δημοφιλείς και οικείες
συμβάσεις για χρήστες που δεν θέλουν να κατανοήσουν άλλες κατευθύνσεις
της τεχνολογίας διάδρασης. \footnote{(\textbf{Εικόνα?})~1 Norton
  Commander (Wikepdia)} \footnote{(\textbf{Εικόνα?})~2 SugarOS
  neighborhood (wikimedia)}

\leavevmode\vadjust pre{\hypertarget{fig:norton-commander}{}}%
\begin{figure}
\hypertarget{fig:norton-commander}{%
\centering
\includegraphics{images/norton-commander.png}
\caption{Εικόνα 1: Ο κεντρικός ρόλος των αρχείων στα περισσότερα
δημοφιλή λειτουργικά συστήματα φαίνεται και από την αποδοχή που έχουν οι
αντίστοιχες εφαρμογές διαχείρισης αρχείων. Ο διαχειριστής αρχείων αρχικά
βασιζόταν μόνο σε μια διεπαφή κειμένου για να διευκολύνει τις βασικές
εντολές πάνω στα αρχεία ενός συστήματος.}\label{fig:norton-commander}
}
\end{figure}

\leavevmode\vadjust pre{\hypertarget{fig:sugar-neighborhood}{}}%
\begin{figure}
\hypertarget{fig:sugar-neighborhood}{%
\centering
\includegraphics{images/sugar-neighborhood.png}
\caption{Εικόνα 2: Στο λειτουργικό σύστημα SugarOS, το οποίο απευθύνεται
σε μικρά παιδιά, δεν υπάρχει γραφική διεπαφή με αρχεία, φακέλους και
εφαρμογές, όπως στην επιφάνεια εργασίας, αλλά η έμφαση βρίσκεται στις
δραστηριότητες και, κυρίως, σε μια οπτικοποίηση των συνδέσεων με άλλους
χρήστες που βρίσκονται κοντά και είναι απευθείας συνδεδεμένοι με την
ασύρματη σύνδεση.}\label{fig:sugar-neighborhood}
}
\end{figure}

Η κατανόηση της τεχνολογίας απαιτεί αρχικά μια σημειωτική ανάλυση των
βασικών λέξεων που χρησιμοποιούμε, \footnote{Mumford (2010) Ihde (2012)}
γιατί υπάρχουν πολλά προϊόντα τα οποία θεωρούνται τεχνολογίες, αλλά δεν
είναι. \footnote{Nelson (2010)} Αν θέλουμε να έχουμε κατανόηση πέρα από
τα συμφέροντα των κατασκευαστών και τις εμπορικές συγκυρίες, τότε η
ονομασία μιας τεχνολογίας διάδρασης θα πρέπει να σχετίζεται άμεσα με την
λειτουργία που πραγματικά κάνει. Για παράδειγμα, το σύστημα SwyftWare
σχεδιάστηκε από τον Jef Raskin για να κάνει επεξεργασία κειμένου, που
είναι μια από τις πιο δημοφιλείς χρήσεις για τον επιτραπέζιο υπολογισμό.
\footnote{Raskin (2000)} Για αυτόν τον σκοπό, από την πλευρά του χρήστη,
δεν έχει ούτε λειτουργικό σύστημα ούτε σύστημα αρχείων, ούτε εφαρμογές,
αφού όλα αυτά είναι απλά προϊόντα και όχι βασικές τεχνολογίες, που είναι
απαραίτητες για να έχουμε μια ποιοτική διάδραση με την επεξεργασία
μικρών ή μεγάλων κειμένων και με ό,τι αυτή συνδέεται, όπως σημειώσεις
και αλληλογραφία.

Το σύστημα αρχείων είναι η πιο δημοφιλής περίπτωση προϊόντος, το οποίο
οι περισσότεροι θεωρούν πως είναι τεχνολογία. Στην πραγματικότητα, η
θεώρηση ενός συστήματος λογισμικού ως σύνολο αρχείων έγινε δημοφιλής με
το σύστημα UNIX. Η σημασία των αρχείων ενδυναμώθηκε στη συνέχεια από την
γραφική επιφάνεια εργασίας και τα έγγραφα, που αναπαραστάθηκαν ως
αρχεία. Τόσο το υλικό του συστήματος όσο και το βασικό επίπεδο
λογισμικού δεν έχουν αρχεία, τα οποία είναι ένα κατασκεύασμα που
εξυπηρετεί καλά πολλούς σκοπούς, αλλά σίγουρα δεν είναι ο μόνος τρόπος
οργάνωσης του λογισμικού. Για παράδειγμα, την ίδια περίοδο που οι
μηχανικοί στα Bell Labs κατασκεύασαν το σύστημα αρχείων του UNIX, οι
μηχανικοί στο Xerox PARC κατασκεύασαν μια εναλλακτική οργάνωση του
λογισμικού, που βασίζεται στα αντικείμενα, τα οποία ανταλλάσσουν
μηνύματα. Επίσης, η επεξεργασία εγγράφων κειμένου στο SwyftWare δεν είχε
αρχεία, όπως αρχεία δεν είχαν οι αρχικές εκδόσεις του λειτουργικού
συστήματος iOS για τις κινητές συσκευές της Apple.

Αμέσως μετά τα αρχεία, το λειτουργικό σύστημα και οι εφαρμογές του
επιτραπέζιου και κινητού υπολογισμού αποτελούν σημαντικά παραδείγματα
προϊόντων, τα οποία, όμως, δεν είναι βασική τεχνολογία για το λογισμικό
διάδρασης. Το λειτουργικό σύστημα παρέχει στον προγραμματιστή και στον
απλό χρήση ένα ενιαίο περιβάλλον διάδρασης για πολλές διαφορετικές
εφαρμογές. Το λειτουργικό σύστημα είναι μια ιδέα χρήσιμη για πολύ ακριβά
μηχανήματα, για τα οποία δεν γνωρίζουμε τις πιθανές εφαρμογές τους. Από
την πλευρά του χρήστη, για μια δεδομένη λειτουργία, όπως η επεξεργασία
κειμένου, το λογισμικό διάδρασης δεν χρειάζεται να έχει ενδιάμεσα
επίπεδα ή αρχιτεκτονικές της πληροφορίας που εξυπηρετούν και άλλους
σκοπούς. Ειδικά οι εφαρμογές, όπως τις γνωρίζουμε από τις δημοφιλείς
γραφικές διεπαφές σε επιτραπέζια και κινητά συστήματα, δημιουργούν ένα
περιβάλλον χρήστη με έμφαση στην κατανάλωση, παρά στη δημιουργία.
Πράγματι, αν μια εφαρμογή δεν έχει μια λειτουργία, τότε ο χρήστης θα
πρέπει να αγοράσει κάποια άλλη εφαρμογή, που πιθανόν είναι πολύ παρόμοια
με την αρχική. Επίσης, πολλές λειτουργίες παγιδεύονται σε μια εφαρμογή
και δεν μπορούν να χρησιμοποιηθούν σε άλλη ή να γίνει μια σύνθεσή τους
με τρόπο που να βολεύει. Για παράδειγμα, η γραμμή εντολών, όπως έγινε
αρχικά γνωστή με το Unix, δεν περιέχει εφαρμογές, αλλά τη δυνατότητα
διασύνδεσης εντολών για τη δημιουργία σύνθετων προγραμμάτων, που μπορούν
να κάνουν ό,τι και μια εφαρμογή, χωρίς να παγιδεύονται σε ένα κλειστό
κουτί. Από τις αρχές της δεκαετίας του 1980, ο Gary Gildall είχε
διαπιστώσει ότι, ακόμη και όταν υπάρχει ανάγκη για διακριτό λειτουργικό
σύστημα και εφαρμογές, αυτά δεν θα πρέπει να φτιάχνονται από τον ίδιο
κατασκευαστή, γιατί δημιουργείται σύγκρουση συμφερόντων. \footnote{Gildall
  (1993)}

Η γραφική διεπαφή των δημοφιλών επιτραπέζιων συστημάτων, όπως είναι τα
Windows, MacOS και GNOME, συνήθως αποτελείται από παράθυρα που
αντιπροσωπεύουν εφαρμογές ή έγγραφα, καθώς, επίσης, και από μενού
εργαλείων που εμφανίζονται ως εικονίδια. Όπως ακριβώς είδαμε και στις
προηγούμενες περιπτώσεις προϊόντων παραπάνω, αυτή η τόσο δημοφιλής
οργάνωση και η σημασιολογία των γραφικών στοιχείων είναι ένα ακόμη
δημοφιλές προϊόν, το οποίο οι περισσότεροι θεωρούν ως τεχνολογία χωρίς
εναλλακτικές. Πράγματι, είναι πολύ δύσκολο να εντοπίσουμε εμπορικές
εναλλακτικές, καθώς η ευχρηστία αυτού του μοντέλου έχει επικρατήσει για
πολλές δεκαετίες και οποιαδήποτε αλλαγή είναι τουλάχιστον τόσο δύσκολη
όσο η οδήγηση από την αντίθετη κατεύθυνση. Η ευχρηστία αυτού του
μοντέλου διάδρασης μπορεί να ερμηνευτεί τόσο από την καθολική επικράτηση
του όσο και από την οικειότητα που έχει δημιουργηθεί, καθώς είναι το πιο
απλό στην εκμάθηση, ειδικά για περιστασιακούς χρήστες. Τα εναλλακτικά
συστήματα γραφικής διεπαφής που υπάρχουν βασίζονται περισσότερο στις
προσαρμογές, τις οποίες θα κάνει ένας προγραμματιστής, ή είναι λιγότερο
εύχρηστα γιατί απευθύνονται σε συχνούς χρήστες μεγαλύτερης δεξιότητας.
Για παράδειγμα, τα συστήματα Xerox Cedar και Oberon δημιουργήθηκαν με
έμπνευση το Alto, αλλά με χρήση δομημένης γλώσσας προγραμματισμού, όπως
είναι οι Mesa και η Pascal. Αυτά τα ερευνητικά συστήματα δίνουν κεντρικό
ρόλο στα έγγραφα, τα οποία εμφανίζονται στα παράθυρα, ενώ η διάδραση
γίνεται με σύνθεση εντολών όπως στο UNIX, οι οποίες επιτρέπουν τους
υπερσυνδέσμους ανάμεσα στα έγγραφα, καθώς και τον διαμοιρασμό τους με
άλλους χρήστες. Ο παγκόσμιος ιστός, τα κοινωνικά δίκτυα και τα έξυπνα
κινητά είναι τα πιο πρόσφατα παραδείγματα προϊόντων, τα οποία, όμως, δεν
αποτελούν βασικές τεχνολογίες.

Η τεχνολογία του αντικειμενοστραφούς προγραμματισμού έχει γίνει
δημοφιλείς με γλώσσες προγραμματισμού όπως η Java και η C++, αλλά αυτές
όχι μόνο είναι απλά προϊόντα, αλλά και έχουν συγκεκριμένες ιδιότητες που
δεν ταιριάζουν με την αρχική σχεδίαση. Η αρχική σχεδίαση και υλοποίηση
του αντικειμενοστραφούς προγραμματισμού από τον Άλαν Κέη, ήταν
εμπνευσμένη από την περιοχή της βιολογίας, τα κύτταρα και την πολύπλοκη
κλίμακα των ζωντανών οργανισμών. Αντί να βασίζεται σε πολύπλοκες δομές
δεδομένων, η Smalltalk βασίζεται σε πολύ απλές δομές, που ανταλλάσσουν
μηνύματα μεταξύ τους, έτσι ώστε να είναι εφικτή η δημιουργία κλίμακας
από απλά δομικά στοιχεία. Για τη δημιουργία πολύπλοκων συστημάτων δεν
υπάρχει λόγος να έχουμε πολύπλοκες γλώσσες προγραμματισμού, αφού αρκεί
να έχουμε ένα συμβολικό σύστημα, το οποίο ταιριάζει στο πεδίο εφαρμογής.
Για τον σκοπό αυτό, τα σύγχρονα συστήματα που βασίζονται στη φιλοσοφία
της Smalltalk κατασκευάζονται με τη σταδιακή υλοποίηση ενός μεταφραστή
που είναι γραμμένος στην ίδια γλώσσα πρόγραμματισμού με αυτήν του κώδικα
που μετατρέπει σε εκτελέσιμο. Με αυτόν τον τρόπο, οι προδιαγραφές του
συστήματος είναι καθολικές ανάμεσα σε συστήματα με διαφορετικό υλικό. Τα
παραπάνω δεν σημαίνουν ότι υπάρχει κάποια σωστή ή λάθος σχεδίαση, αλλά
σίγουρα σημαίνει ότι πολλές τεχνολογικές ετικέτες θα πρέπει να
χρησιμοποιούνται με περισσότερες επεξηγήσεις σχετικά με το πεδίο
ερφαρμογής και τους τυπικούς χρήστες, έτσι ώστε να είναι διακριτό το
πραγματικό τους νόημα. Ταυτόχρονα, μπορούμε να εντοπίσουμε καινοτόμα
συστήματα, όπως το Superpaint, τα οποία δεν βασίζονται ούτε σε κάποιο
λειτουργικό σύστημα ούτε σε κάποια γλώσσα προγραμματισμού. Πράγματι, ένα
σύστημα διάδρασης μπορεί να φτιαχτεί για έναν σημαντικό σκοπό, όπως
είναι η ψηφιακή επεξεργασία εικόνας, χωρίς τη φιλοδοξία να γίνει
πλατφόρμα για κάτι άλλο.

\hypertarget{ux3b5ux3beux3bfux3bcux3bfux3afux3c9ux3c3ux3b7-ux3baux3b1ux3b9-ux3c0ux3c1ux3bfux3c3ux3bfux3bcux3bfux3afux3c9ux3c3ux3b7}{%
\subsection{Εξομοίωση και
προσομοίωση}\label{ux3b5ux3beux3bfux3bcux3bfux3afux3c9ux3c3ux3b7-ux3baux3b1ux3b9-ux3c0ux3c1ux3bfux3c3ux3bfux3bcux3bfux3afux3c9ux3c3ux3b7}}

\leavevmode\vadjust pre{\hypertarget{fig:paper-simulation}{}}%
\begin{figure}
\hypertarget{fig:paper-simulation}{%
\centering
\includegraphics{images/paper-simulation.jpg}
\caption{Εικόνα 3: Καθώς η εταιρεία Xerox ασχολείται διαχρονικά με το
φυσικό χαρτί και τις εκτυπώσεις, οι πελάτες της είναι, κυρίως, εκδοτικοί
οργανισμοί και γενικά άτομα που εργάζονται στο γραφείο με έγγραφα που
αποθηκεύονται σε φυσικό χαρτί. Με αυτόν τον τρόπο, όλες οι καινοτομίες
του Alto μεταφέρθηκαν στο Star, δίνοντας έμφαση στη δυνατότητα
προσομοίωσης του φυσικού χαρτιού στην οθόνη του υπολογιστή και γενικά
στην επιφάνεια εργασίας με τις αντίστοιχες μεταφορές από το φυσικό
γραφείο.}\label{fig:paper-simulation}
}
\end{figure}

\leavevmode\vadjust pre{\hypertarget{fig:magic-cap}{}}%
\begin{figure}
\hypertarget{fig:magic-cap}{%
\centering
\includegraphics{images/magic-cap.png}
\caption{Εικόνα 4: Η χρήση της μεταφοράς αντικειμένων από τον πραγματικό
κόσμο στο λειτουργικό σύστημα Magic Cap του φορητού υπολογιστή
επικοινωνίας της General Magic είναι εμπνευσμένη από την επιτυχία που
είχαν οι ανάλογες μεταφορές στον επιτραπέζιο υπολογιστή, αν και σε αυτήν
την περίπτωση δεν μεταφράστηκαν σε αντίστοιχη εμπορική
επιτυχία.}\label{fig:magic-cap}
}
\end{figure}

Η εξομοίωση και η προσομοίωση παρέχουν δύο σημαντικούς τρόπους θεώρησης
της λειτουργίας του υπολογισμού και των εφαρμογών του. Χρονολογικά, η
εξομοίωση είναι προγενέστερη της προσομοίωσης, καθώς περιγράφεται από
τον Alan Turing ως ένα χαρακτηριστικό της γενικής μηχανής υπολογισμού. Η
δυνατότητα της εξομοίωσης επιτρέπει σε ένα μηχάνημα να εξομοιώνει τη
λειτουργία κάποιου άλλου διαφορετικού μηχανήματος. Υπάρχουν πολλές
δημοφιλείς μηχανές εξομοίωσης για τους μικρο-υπολογιστές και τις
παιχνιδομηχανές της δεκαετίας του 1980, αλλά η εξομοίωση δεν είναι
δημοφιλής τρόπος σχεδίασης νέων συστημάτων διάδρασης. Η προσομοίωση
αναφέρεται στη δυνατότητα ενός μηχανήματος να υπολογίζει και να
οπτικοποιεί τη συμπεριφορά συστήματων, των οποίων τη λειτουργία δεν
μπορεί να γνωρίζει επακριβώς, όπως είναι τα πολύπλοκα φυσικά ή βιολογικά
φαινόμενα. Για παράδειγμα, οι πρώτοι υπολογιστές χρησιμοποιήθηκαν για
τον υπολογισμό της τροχιάς ενός πυραύλου, καθώς και για την πρόβλεψη του
καιρού, ενώ οι πιο πρόσφατες εφαρμογές της προσομοίωσης περιλαμβάνουν τη
λειτουργία των βιολογικών κυττάρων, καθώς και των μικροσκοπικών
σωματιδίων της ύλης.

Οι εφαρμογές της προσομοίωσης είναι από τις πιο πετυχημένες εμπορικά
εφαρμογές των υπολογιστών, αλλά αυτή η επιτυχία τους έχει επισκιάσει τη
συμπληρωματική θεώρηση του υπολογισμού που βρίσκεται στην εξομοίωση.
Πράγματι, οι σύγχρονες προσομοιώσεις είναι τόσο εξελιγμένες σε
συμπεριφορά, που μοιάζουν πολύ με το αντίστοιχο φαινόμενο που
προσομοιώνουν. Για παράδειγμα, ένας προσομοιωτής πτήσης, από τον πιο
απλό οικιακό μέχρι τον πιο εξελιγμένο, που χρησιμοποιείται για την
εκπαίδευση των πιλότων, μοιάζει πάρα πολύ με τον χειρισμό ενός αληθινού
αεροσκάφους. H θεώρηση της εξομοίωσης δεν έχει γίνει ακόμη τόσο
δημοφιλής όσο η προσομοίωση, γιατί είναι πιο δύσκολη και απαιτεί
δημιουργικότητα και φαντασία. Η προσομοίωση ενός φυσικού φαινομένου
απαιτεί από εμάς την παρατήρηση και την κατανόησή του, αλλά η εξομοίωση
ενός νέου μηχανήματος απαιτεί από εμάς να το σχεδιάσουμε και κυρίως να
φανταστούμε κάτι που δεν υπάρχει. Για παράδειγμα, πολλά βιντεοπαιχνίδια,
αν και αρχικά ο κλάδος τους ξεκίνησε επίσης ως προσομοίωση, παρουσιάζουν
γραφικά και συμπεριφορές που δεν βασίζονται σε προσομοίωση της
πραγματικότητας.

Η δημοφιλία της προσομοίωσης και η δυσκολία που παρουσιάζει η εξομοίωση
έχουν οδηγήσει τις περισσότερες εφαρμογές του παραδοσιακού επιτραπέζιου
υπολογισμού στην κατεύθυνση της προσομοίωσης του πραγματικού κόσμου. Για
παράδειγμα, η γραφική διεπαφή στον επιτραπέζιο υπολογιστή είναι μια
προσομοίωση της επιφάνειας εργασίας στον αληθινό χώρο του γραφείου.
\footnote{(\textbf{Εικόνα?})~3 Η επιφάνεια εργασίας ως προσομοίωση του
  φυσικού χαρτιού (Xerox)} \footnote{(\textbf{Εικόνα?})~4 Η χρήση της
  μεταφοράς στο κινητό σύστημα Magic Cap (wikimedia)} Αντίστοιχα, οι
εφαρμογές επεξεργασίας εγγράφων είναι μια προσομοίωση των χάρτινων
εγγράφων του αληθινού κόσμου. Στην πραγματικότητα όμως, ο υπολογιστής,
με γραφικά ή χωρίς, δεν έχει κανέναν περιορισμό για το πώς θα είναι μια
γραφική διεπαφή ή μια εφαρμογή επεξεργασίας εγγράφων. Σίγουρα, η
προσομοίωση του αληθινού κόσμου στον κόσμο του υπολογιστή δημιουργεί μια
αρχική αίσθηση ευχρηστίας μέσα από την οικειότητα. Επομένως, για τα
μηχανήματα εκείνα που πρέπει να είναι εύχρηστα για ευκαιριακούς χρήστες,
η προσομοίωση είναι μια χρήσιμη τεχνική, αλλά δεν είναι αντιπροσωπευτική
των δυνατοτήτων που θέλουμε να έχουμε, αν ο στόχος μας είναι η επαύξηση
της ανθρώπινης νοημοσύνης. Επιπλέον, για κάποιες εφαρμογές διάδρασης σε
πραγματικό χρόνο, όπως είναι η μουσική και η ζωγραφική, η προσομοίωση
τους στον συμβολικό χώρο του υπολογισμού αφαιρεί δυνατότητες που
βρίσκονται πέρα από τις ρητές γνώσεις μας, αλλά ήταν προσβάσιμες στα
αρχικά αναλογικά εργαλεία ήχου και χρώματος.

Η πιο δημοφιλής μορφή και ταυτόχρονα η πιο ιδιάζουσα περίπτωση
προσομοίωσης συναντάται στον κινητό υπολογισμό της δεκαετίας του 2010,
όπου οι συσκευές διάδρασης μικραίνουν σε μέγεθος και προσομοιώνουν τις
εφαρμογές του επιτραπέζιου υπολογισμού, αντί να ορίσουν νέα
παραδείγματα, συμβατά με το κινητό πλαίσιο χρήσης. Για παράδειγμα, τα
πρώτα κινητά λειτουργικά συστήματα της Microsoft έχουν γραφικό
περιβάλλον με κουμπί εκκίνησης, ενώ όλα τα συστήματα βασίζονται στην
ιδέα των εφαρμογών για την οργάνωση του λογισμικού τους. Όπως έχουμε ήδη
αναλύσει στην ενότητα των Μορφών, οι εφαρμογές λογισμικού είναι μια πολύ
συγκεκριμένη μορφή οργάνωσης του λογισμικού, η οποία ωφελεί, κυρίως, την
κυρίαρχη πλατφόρμα ενός λειτουργικού συστήματος. Με άλλα λόγια, οι
κατασκευαστές λογισμικού επέλεξαν, για άλλη μια φορά, να
βελτιστοποιήσουν τα συμφέροντά τους, τα οποία παρουσιάζουν στους χρήστες
ως οικειότητα και ευχρηστία. Στην πράξη όμως, αυτό που πραγματικά
προσφέρουν στους τελικούς χρήστες είναι μια προσομοίωση ενός
παραδείγματος του επιτραπέζιου υπολογισμού σε ένα νέο πλαίσιο χρήσης, το
οποίο είναι πολύ διαφορετικό. Με αυτόν τον τρόπο, τα κινητά συστήματα
είναι μια ακόμη χαμένη ευκαιρία, μετά τα επιτραπέζια, αφού παγιδεύονται
στη θεώρηση της προσομοίωσης και δεν εξετάζουν καθόλου την εξομοίωση
νέων συστημάτων διάδρασης.

\leavevmode\vadjust pre{\hypertarget{fig:ed-editor}{}}%
\begin{figure}
\hypertarget{fig:ed-editor}{%
\centering
\includegraphics{images/ed-editor.jpg}
\caption{Εικόνα 5: Ο επεξεργαστής κειμένου ED λειτουργεί σε επίπεδο
γραμμής, γιατί αναπτύχθηκε σε ένα περιβάλλον χρήστη, στο οποίο η βασική
συσκευή εισόδου και εξόδου είναι ο τηλέτυπος. Αν και δεν είναι
εύχρηστος, παραμένει, από τη δημιουργία του συστήματος UNIX, ο βασικός
επεξεργαστής κειμένου, καθώς είναι απλός στην υλοποίηση και συμβατός με
τις γλώσσες προγραμματισμού σε γραμμή εντολών.}\label{fig:ed-editor}
}
\end{figure}

\leavevmode\vadjust pre{\hypertarget{fig:unix-tmg}{}}%
\begin{figure}
\hypertarget{fig:unix-tmg}{%
\centering
\includegraphics{images/unix-tmg.png}
\caption{Εικόνα 6: Για τον μετασχηματισμό από την αρχική έκδοση του UNIX
που ήταν γραμμένη σε συμβολική γλώσσα μηχανής σε μια έκδοση που
βασίζεται σε μια γλώσσα υψηλού επιπέδου, όπως η B και η C, χρειάστηκε να
δημιουργηθεί από την αρχή ένας μεταφραστής. Η διαδικασία δημιουργίας του
μεταφραστή είναι σημαντική, γιατί είναι μια δεξιότητα που επιτρέπει τη
δημιουργία μιας νέας γλώσσας, αφού η αρχική εκδοχή του μεταφραστή μπορεί
να είναι πολύ απλή και να γραφεί απευθείας σε συμβολική γλώσσα
μηχανής.}\label{fig:unix-tmg}
}
\end{figure}

Η περίπτωση των συστημάτων διάδρασης ανοιχτού κώδικα είναι εξίσου
ιδιάζουσα με αυτή του κινητού υπολογισμού, γιατί προσομοιώνει το γραφικό
περιβάλλον εργασίας και τις εφαρμογές των κυρίαρχων συστημάτων. Για
παράδειγμα, τα πιο δημοφιλή γραφικά περιβάλλοντα με επιφάνεια εργασίας
όπως τα Gnome και KDE, είναι αντίγραφα των Windows και MacOS.
Ταυτόχρονα, οι αντίστοιχες εφαρμογές γραφείου ανοιχτού κώδικα OpenOffice
και LibreOffice είναι επίσης αντίγραφα των αντίστοιχων εφαρμογών
Microsoft Office και Apple iWork. Ο ανοιχτός κώδικας σε επίπεδο
εφαρμογής και περιβάλλοντος ήταν δεδομένος στο σύστημα Smalltalk του
Alto, αλλά οι κατασκευαστές του είχαν κάνει πολλές ακόμη αρχιτεκτονικές
επιλογές. Ακόμη και πέρα από τον χώρο της διάδρασης που
διαπραγματευόμαστε εδώ, οι κατασκευαστές ανοιχτού λογισμικού φαίνονται
παγιδευμένοι περισσότερο στην αναπαραγωγή της υπάρχουσας κατάστασης παρά
στην καινοτομία. Πράγματι, ο κατασκευαστής του πυρήνα του Linux επέλεξε
στις αρχές τις δεκαετίας του 1990 να φτιάξει μια εκδοχή ενός
παραδοσιακού πυρήνα που υπάρχει από τις αρχές του 1970 και ταυτόχρονα να
βρει καθολική ανταπόκριση από πολλές διαφορετικές ομάδες χρηστών.
Σίγουρα η διάθεση του λογισμικού με ανοιχτό κώδικα έχει περισσότερα
πλεονεκτήματα από το ίδιο λογισμικό με κλειστό κώδικα, αλλά αν η
αρχιτεκτονική του και η διάδρασή του είναι ίδια ακριβώς με αυτή του
κλειστού κώδικα, τελικά το κίνημα του ανοιχτού κώδικα καταλήγει να
εξυπηρετεί όχι μόνον τον εαυτό του, αλλά, κυρίως, τα συμφέροντα των
κατασκευαστών κλειστού λογισμικού, οι οποίοι μπορούν πλέον να
ετεροπροσδιορίζονται ως τεχνολογία. Τελικά αυτό δεν εξυπηρετεί την
πραγματική βελτίωση της ποιότητας του λογισμικού διάδρασης, που μπορεί
να γίνει μόνο με τους κατάλληλους εξομοιωτές για νέα συστήματα που δεν
υπάρχουν, καθώς και με εξομοίωση συστημάτων εισόδου και εξόδου
δεδομένων. \footnote{(\textbf{Εικόνα?})~5 Επεξεργαστής κειμένου γραμμής
  ED (Wikimedia)} \footnote{(\textbf{Εικόνα?})~6 UNIX TMG (Wikipedia)}

Τόσο τα πρώτα πειραματικά βιντεοπαιχνίδια όσο και τα δημοφιλή προϊόντα
των επόμενων δεκαετιών συνήθως βασίζονται στη θεώρηση της προσομοίωσης.
Για παράδειγμα, το βιντεοπαιχνίδι Tennis for two \footnote{(\textbf{Εικόνα?})~7
  Tennis for Two (wikimedia)} δημιουργήθηκε για έναν αναλογικό
υπολογιστή με οπτικοποίηση στην οθόνη ενός φασματοσκόπιου, στον οποίο οι
δύο παίκτες έπαιζαν τένις σε πρόσοψη. Μερικά χρόνια αργότερα, το
δημοφιλές βιντεοπαιχνίδι Pong αντιγράφει το Table Tennis for Two
\footnote{(\textbf{Εικόνα?})~8 Magnavox Odyssey (Philips)} και
χρησιμοποιεί την οπτικοποίηση της κάτοψης. Επίσης, ένα από τα πρώτα
δημοφιλή βιντεοπαιχνίδια τη δεκαετία του 1960 σε μινι-υπολογιστές είναι
το Spacewar, στο οποίο οι παίκτες κάνουν μια αερομαχία στο διάστημα σε
συνθήκες βαρύτητας γύρω από ένα πλανητικό κέντρο. Αντίστοιχα, τη
δεκαετία του 1970, πολλοί προγραμματιστές αποκτούν την πρώτη τους επαφή
με τον υπολογιστή με τον ευρέως διαθέσιμο κώδικα του βιντεοπαιχνιδιού
Lunar Lander, όπου ο παίκτης προσπαθεί να προσγειώσει ένα διαστημόπλοιο
στη σελήνη. Παρατηρούμε ότι και στις δύο περιπτώσεις, εκτός από την
προσομοίωση της φυσικής πραγματικότητας, η θεματολογία των
βιντεοπαιχνιδιών είναι έντονα επηρεασμένη από την τότε πολιτική και
πολιτιστική πραγματικότητα ή από μια αθλητική και παιγνιώδη
δραστηριότητα. Μετά τη δεκαετία του 1970, εμφανίζονται, πολύ δειλά και
ως αφαιρετικοί μετασχηματισμοί των προηγούμενων, τα πρώτα καινοτόμα
βιντεοπαιχνίδια, όπως ήταν τα Breakout και Space Invaders, αλλά το
παράδειγμα της προσομοίωσης παραμένει κυρίαρχο.

\leavevmode\vadjust pre{\hypertarget{fig:tennis-for-two}{}}%
\begin{figure}
\hypertarget{fig:tennis-for-two}{%
\centering
\includegraphics{images/tennis-for-two.jpg}
\caption{Εικόνα 7: Το βιντεοπαιχνίδι Tennis for Two είναι μια σχετικά
πιστή προσομοίωση του τένις για δύο παίκτες, όπως αυτό φαίνεται από το
πλάϊ, σε πρόσοψη. Κατασκευάστηκε σε αναλογικό ηλεκτρονικό υπολογιστή και
για την οπτικοποίησή του χρησιμοποιήθηκε ένα φασματοσκόπιο. Η επίδρασή
του ήταν καταλυτική, αφού έδωσε την έμπνευση για το δημοφιλές Pong, ενώ
για πολλές δεκαετίες μετά τα βιντεοπαιχνίδια προσομοίωσης αθλημάτων
παραμένουν πολύ δημοφιλή.}\label{fig:tennis-for-two}
}
\end{figure}

\leavevmode\vadjust pre{\hypertarget{fig:magnavox-odyssey}{}}%
\begin{figure}
\hypertarget{fig:magnavox-odyssey}{%
\centering
\includegraphics{images/magnavox-odyssey.jpg}
\caption{Εικόνα 8: Η πρώτη εμπορικά πετυχημένη οικιακή κονσόλα
βιντεοπαιχνιδιών είχε μόνο τη δυνατότητα να μετακινεί τρία σημεία στην
τηλεόραση του χρήστη, με τα δύο χειριστήρια με τροχαλία, ενώ δεν είχε τη
δυνατότητα να εκτελέσει νέα παιχνίδια, εκτός από αυτά που ήταν ήδη
αποθηκευμένα στην εσωτερική μνήμη ανάγνωσης. Με τη βοήθεια ενός σετ
περιφερειακών, όπως χρωματισμένες ημιδιαφανείς ζελατίνες για την
τηλεόραση και κάρτες επιτραπέζιων παιχνιδιών, δημιούργησε τη βιομηχανία
των βιντεοπαιχνιδιών. To πιο γνωστό παιχνίδι ήταν το Table Tennis for
Two, το οποίο πολύ γρήγορα βελτιώθηκε από την Atari με το δημοφιλές
Pong.}\label{fig:magnavox-odyssey}
}
\end{figure}

Τα βιντεοπαιχνίδια σε κονσόλες και σε μικρο-υπολογιστές των δεκαετιών
του 1970 και 1980 απέκτησαν μια δεύτερη ζωή μακριά από το αρχικό υλικό
τους με την τεχνολογία της εξομοίωσης, μετά το 2000. Η βελτιωμένη
επεξεργαστική ισχύς των επιτραπέζιων συστημάτων επέτρεψε την ανάπτυξη
εξομοιωτών που μπορούσαν να εκτελέσουν το παλαιότερο λογισμικό σε μια
αντίστροφη κίνηση, όπου τα διαθέσιμα συστήματα και η θεώρηση της
εξομοίωσης δεν χρησιμοποιούνται για να μας πάνε παρακάτω, αλλά πίσω στον
χρόνο. Με αυτόν τον τρόπο, μια νέα γενιά χρηστών χρησιμοποιεί συστήματα
που δεν υπάρχουν, ενώ και πολλοί παλιότεροι ζουν στην πράξη τις
αναμνήσεις τους. Σίγουρα, η εξομοίωση παλιών συστήματων, τα οποία δεν
είναι, πλέον, ευρέως διαθέσιμα, είναι μια άριστη πρακτική για να έχουμε
πρόσβαση σε παλαιότερο λογισμικό, ακόμη και για να αναπτύξουμε νέο
λογισμικό για αυτό το δυσεύρετο, πλέον, υλικό. Ταυτόχρονα, όμως, αυτή η
κάπως οπισθοδρομική εφαρμογή της θεώρησης της εξομοίωσης επιβεβαιώνει
ότι τόσο οι κατασκευαστές όσο και η χρήστες έχουν μια έμφυτη τάση στην
αναπαραγωγή της υπάρχουσας κατάστασης, παρά στη δημιουργία καινοτομίας,
ακόμη και όταν έχουν στη διάθεση τους τα κατάλληλα εργαλεία, όπως είναι
οι ισχυροί και δικτυωμένοι προσωπικοί υπολογιστές και η τεχνική της
εξομοίωσης.

Πέρα από το γραφικό περιβάλλον και τα έγγραφα στον επιτραπέζιο
υπολογισμό, η μονοθεματική θεώρηση της προσομοίωσης είναι, επίσης, το
κυρίαρχο παράδειγμα, ακόμη και σε καινοτόμα συστήματα διάδρασης, όπως
είναι η εικονική πραγματικότητα. Η εικονική πραγματικότητα όπως
περιγράφηκε αρχικά από τον Jaron Lanier ήταν μια προσπάθεια εξομοίωσης
ενός κόσμου που δεν υπάρχει. Δεν είναι τυχαίο ότι ο δημιουργός και νονός
της θεωρεί ατυχές το όνομα \emph{εικονική πραγματικότητα}, αφού δηλώνει
μια προσομοίωση. Πράγματι, στις δικές του πειραματικές εφαρμογές
εικονικής πραγματικότητας, ο χαρακτήρας που ελέγχεται μπορεί να είναι
ένα χταπόδι ή κάποιο άλλο δημιούργημα που δεν υπάρχει. Ταυτόχρονα, η
οπτικοποίηση και η συμπεριφορά του εικονικού κόσμου δεν προσομοιώνει τον
πραγματικό κόσμο της ανθρώπινης εμπειρίας, αλλά έναν κόσμο που δεν
υπάρχει. Ο στόχος της εξομοίωσης μιας δυνητικής πραγματικότητας είναι να
εμπλουτίσει την ανθρώπινη εμπειρία και να επαυξήσει την ανθρώπινη
νοημοσύνη σε ευρύτερα πεδία, στα οποία δεν υπάρχει απευθείας πρόσβαση
μέσα από το φυσικό περιβάλλον. \footnote{Engelbart (1962) Lanier (2010)}
Αντίθετα, οι περισσότερες εφαρμογές εικονικής πραγματικότητας προσπαθούν
να προσομοιώσουν την πραγματικότητα όσο πιο πιστά γίνεται, τόσο στην
εμφάνιση της με γραφικά όσο και στη συμπεριφορά του κόσμου και των
χαρακτήρων. Αν και τα συστήματα εικονικής πραγματικότητας είχαν μικρή
εμβέλεια για πολλές δεκαετίες, αποτελούν άλλη μια σημαντική χαμένη
ευκαιρία για την εξερεύνηση της εξομοίωσης ως βασικής θεώρησης
κατασκευής νέων διαδραστικών συστημάτων.

\leavevmode\vadjust pre{\hypertarget{fig:lisa-bootstrapping}{}}%
\begin{figure}
\hypertarget{fig:lisa-bootstrapping}{%
\centering
\includegraphics{images/lisa-bootstrapping.jpg}
\caption{Εικόνα 9: Η κατασκευή λογισμικού για ένα νέο υλικό υπολογιστή
που δεν υπάρχει ακόμη μπορεί να ξεκινήσει με έναν διαθέσιμο υπολογιστή,
στον οποίο γίνεται εξομοίωση της αρχιτεκτονικής του νέου συστήματος. Με
αυτόν τον τρόπο, ο διαθέσιμος Apple II χρησιμοποιήθηκε για την κατασκευή
της γραφικής διεπαφής του Apple Lisa σε μια επαναληπτική διαδικασία, η
οποία ξεκίνησε από τα διαθέσιμα εργαλεία που υποστήριζαν μόνο χαρακτήρες
κειμένου σε μια μαύρη οθόνη.}\label{fig:lisa-bootstrapping}
}
\end{figure}

\leavevmode\vadjust pre{\hypertarget{fig:emulators}{}}%
\begin{figure}
\hypertarget{fig:emulators}{%
\centering
\includegraphics{images/emulators.jpg}
\caption{Εικόνα 10: Ο εξομοιωτής σε ένα σύγχρονο γραφικό περιβάλλον
μπορεί να εκτελέσει το λογισμικό για ένα παλιότερο γραφικό περιβάλλον,
το οποίο με τη σειρά του μπορεί να εκτελέσει έναν εξομοιωτή για ένα
ακόμη παλιότερο σύστημα με γραμμή εντολών. Με αυτόν τον τρόπο, μπορούμε
να εξομοιώσουμε ένα παλιότερο σύστημα ακόμη και αν δεν έχουμε εξομοιωτή
για αυτό, αρκεί να έχουμε για μια σειρά από ενδιάμεσα συστήματα. Το
ιδανικό είναι να εκμεταλλευτούμε αυτήν την τεχνική για νέα συστήματα και
όχι μόνο για την αναβίωση των παλιότερων.}\label{fig:emulators}
}
\end{figure}

Μια ερμηνεία για την τάση προσομοίωσης ή την τάση εξομοίωσης του
παρελθόντος μπορούμε να αντλήσουμε από το γνωστικο πεδίο των μέσων
επικοινωνίας. Για παράδειγμα, στο παρελθόν, η εισαγωγή της τεχνολογίας
της τηλεόρασης θεωρήθηκε μια συνέχεια της τεχνολογίας του ραδιοφώνου.
Οπότε, η παραγωγή του περιεχομένου που θα φιλοξενούσε το νέο μέσο θα
ήταν μια γραμμική βελτίωση του περιεχομένου που υπήρχε στο ραδιόφωνο. Με
αυτό το σκεπτικό δεν ήταν καθόλου περίεργο που η τηλεόραση αρχικά
ορίστηκε ως \emph{ραδιόφωνο με εικόνα} και, με δεδομένο αυτόν τον
σχετικά στενό ορισμό, ήταν επόμενο το περιεχόμενο των εκπομπών
τηλεόρασης, τα πρώτα χρόνια, να μην ήταν κάτι παραπάνω από μια στατική
εικόνα με ήχο. \footnote{Bolter and Grusin (2000)} Μια ακόμη ερμηνεία
για την ανάγκη να χρησιμοποιούμε μεταφορές από τον πραγματικό κόσμο, τις
οποίες τις προσαρμόζουμε στον ψηφιακό κόσμο των υπολογιστών, μπορούμε να
αντλήσουμε από το γνωστικό πεδίο της φιλοσοφίας. Πράγματι, είναι
ευκολότερο να κατανοήσουμε κάτι νέο, αν το χρησιμοποιήσουμε ως κάτι
παλιότερο. {{[}}lakoff2008metaphors{{]}} Αυτές οι ιδέες αποτελούν την
θεωρητική θεμελίωση δημοφιλών προϊόντων διάδρασης, όπως είναι η γραφική
επιφάνεια εργασίας και οι εφαρμογές τηλεδιάσκεψης με βίντεο. Αν και
παρέχουν ευχρηστία και οικειότητα, αυτές οι ιδέες δεν προσφέρουν
υποστήριξη για την κατασκευή μελλοντικών καινοτόμων συστημάτων
διάδρασης.

Όσο εύκολο είναι να περιγράψουμε τα δημοφιλή συστήματα διάδρασης που
βασίζονται στη θεώρηση της προσομοίωσης, άλλο τόσο δύσκολο είναι να
περιγράψουμε εκείνα που βασίζονται στην εξομοίωση, \footnote{(\textbf{Εικόνα?})~10
  Εξομοιωτές παλαιότερων συστημάτων (wikimedia)} αφού δεν έχουμε κρίσιμη
μάζα γνώσης για αυτά. \footnote{Bardini (2000)} Ένα από τα λίγα γνωστά
συστήματα διάδρασης που βασίζονται στην εξομοίωση περιγράφεται από τον
Alan Kay στην κατασκευή του Alto. Το λογισμικό για το λειτουργικό
υπόδειγμα του Alto, το οποίο βασίζεται στην περιβάλλον προγραμματισμού
Smalltalk, κατασκευάστηκε με την εξομοίωσή του πάνω σε έναν
μινι-υπολογιστή στις αρχές του 1970 και πολύ πριν οι ερευνητές του Xerox
PARC κατασκευάσουν το πραγματικό υλικό του Alto. \footnote{Hiltzik
  (1999)} Εκείνος ο μινι-υπολογιστής δούλευε με τηλέτυπο χωρίς οθόνη
γραφικών και χωρίς ποντίκι, ενώ και η γλώσσα προγραμματισμού του ήταν
πολύ διαφορετική από τη Smalltalk, αλλά αυτά δεν εμπόδισαν τους
κατασκευαστές του Alto να φανταστούν ένα διαδραστικό γραφικό περιβάλλον
και να το εξομοιώσουν αρχικά πάνω σε ξένο υλικό, πριν τελικά το
κατασκευάσουν και στην πραγματικότητα. Παρόμοια τεχνική εξομοίωσης
χρησιμοποίησε και ο Bill Atkinson για την κατασκευή του λογισμικού
διάδρασης για το Apple Lisa. \footnote{(\textbf{Εικόνα?})~9 Επαναληπτική
  ανάπτυξη του Lisa (Bill Atkinson)} Για τον σκοπό αυτό, χρησιμοποίησε
μια επέκταση του δημοφιλούς εκείνη την εποχή Apple II για να
κατασκευάσει σταδιακά με την γλώσσα PASCAL ένα γραφικό περιβάλλον που
λίγο διαφέρει από αυτό των σύγχρονων επιτραπέζιων συστημάτων και όλα
αυτά ενώ είχε μπροστά του μόνο έναν μικρο-υπολογιστή με διάδραση σε
γραμμή εντολών, χωρίς ποντίκι και χωρίς παραθυρικό περιβάλλον. Επομένως,
ένα βασικό συστατικό διάδρασης που εμφανίζουν τα συστήματα, τα οποία
έχουν κατεύθυνση την εξομοίωση είναι ότι επιτρέπουν την κατασκευή νέων
εργαλείων, που δεν περιορίζονται από εσωτερικές παραδοχές και κανόνες,
όπως συνήθως συμβαίνει στα συστήματα προσομοίωσης.

\leavevmode\vadjust pre{\hypertarget{fig:nova}{}}%
\begin{figure}
\hypertarget{fig:nova}{%
\centering
\includegraphics{images/nova.jpg}
\caption{Εικόνα 11: Ακόμη και ο μικρότερος μίνι-υπολογιστής στα τέλη της
δεκαετίας του 1960 δεν μοιάζει καθόλου με ένα επιτραπέζιο σύστημα
διάδρασης, αφού το βασικό μοντέλο έχει μόνο διακόπτες με φωτάκια και με
μια επέκταση μπορεί να συνδεθεί σε έναν τηλέτυπο. Αυτό δεν ήταν εμπόδιο
για τους ερευνητές του Xerox PARC, οι οποίοι με την τεχνική της
εξομοίωσης άρχισαν πάνω σε αυτό το μηχάνημα την κατασκευή του λογισμικού
για το Alto που είναι πολύ διαφορετικό, αφού βασίζεται στο ποντίκι και
στη γραφική διεπαφή.}\label{fig:nova}
}
\end{figure}

\leavevmode\vadjust pre{\hypertarget{fig:altair-teletype}{}}%
\begin{figure}
\hypertarget{fig:altair-teletype}{%
\centering
\includegraphics{images/altair-teletype.jpg}
\caption{Εικόνα 12: O Altair θεωρείται ο πρώτος δημοφιλής
μικρο-υπολογιστής και ήταν διαθέσιμος σε οικονομικό κιτ, το οποίο ο
χρήστης συναρμολογούσε μόνος του, και στη συνέχεια θα έπρεπε να αγοράσει
και τα σχετικά περιφεριακά, εισόδου και εξόδου, όπως ο τηλέτυπος για την
είσοδο και την έξοδο κειμένου, τα οποία κόστιζαν ακόμη περισσότερο. Ήταν
όμως πολύ πιο προσιτός από τους μινι-πολογιστές εκείνης της εποχής και
επιπλέον ήταν διαθέσιμος με την επίσης προσιτή γλώσσα προγραμματισμού
BASIC.}\label{fig:altair-teletype}
}
\end{figure}

Αμέσως μετά τους πρωτοπόρους Douglas Engelbart και Ivan Sutherland του
δημιουργικού μετασχηματισμού του υπολογιστών σε μια νέα μορφή, οι
ερευνητές του Xerox PARC εφάρμοσαν την τεχνική της εξομοίωσης, έτσι ώστε
να μπορούν να εργάζονται με το σύστημα που ήθελαν και όχι απλά με αυτό
που είχαν διαθέσιμο. Πράγματι, η Xerox μόλις είχε αγοράσει την
Scientific Data Systems (SDS) και αγνόησε το αίτημα των ερευνητών του
PARC για την αγορά του ανταγωνιστικού DEC PDP, το οποίο ήταν πολύ
δημοφιλές ανάμεσα στους συναδέλφους ερευνητές σε άλλους οργανισμούς και
στο δίκτυο ερευνητικής συνεργασίας ARPANET. Στην πράξη, όμως, αυτό δεν
ήταν μεγάλο πρόβλημα, αφού με την τεχνική της εξομοίωσης οι ερευνητές
του PARC δημιούργησαν το PDP με μικροκώδικα πάνω στο φυσικό μηχάνημα της
SDS. Με τον ίδιο ακριβώς τρόπο, χρησιμοποίησαν το μηχάνημα Data General
Nova για να αναπτύξουν το λογισμικό για το Alto, πολύ πριν έχουν στη
διάθεση τους το υλικό της αρχικής έκδοσης του Alto. Παρόμοια, ο Niklaus
Wirth θα καταφέρει να εκτελέσει και να ενημερώσει τη βάση για το αρχικό
λογισμικό διάδρασης για το σύστημα Oberon \footnote{Wirth and Gutknecht
  (1992)} είκοσι πέντε χρόνια μετά, με τη δημιουργία ενός νέου
υπολογιστή με την τεχνική των προγραμμάτων για FPGA, δηλαδή υλικό που
επιτρέπει την εξομοίωση άλλου υλικού υπολογιστή. Για τους ερευνητές του
PARC και τους συνεργάτες τους, η διάδραση με τους υπολογιστές σημαίνει,
κυρίως, την κατασκευή ενός νέου υπολογιστή και του περιβάλλοντος
ανάπτυξης, που περιλαμβάνει τον μεταγλωττιστή και τον επεξεργαστή
κειμένου, τα οποία αρχικά θα πρέπει να εξομοιωθούν στα πρώτα στάδια της
ανάπτυξης πάνω σε ένα διαφορετικό από το τελικό φυσικό μηχάνημα. Η
επιλογή της εξομοίωσης μπορεί να γίνει αρχιτεκτονική επιλογή, όπως στην
περίπτωση της JAVA, όπου ο κατασκευαστής φροντίζει για την εκτέλεση μόνο
πάνω σε μια εικονική μηχανή και αφήνει την υλοποίηση του κατώτερου
επιπέδου στους επιμέρους κατασκευαστές υλικού. \footnote{(\textbf{Εικόνα?})~11
  Data General Nova (Wikimedia)} \footnote{(\textbf{Εικόνα?})~12 Altair
  και Τηλέτυπος (wikimedia)}

\hypertarget{ux3bcux3bfux3bdux3c4ux3b5ux3bbux3bfux3c0ux3bfux3afux3b7ux3c3ux3b7-ux3baux3b1ux3b9-ux3b1ux3bdux3b1ux3bbux3bfux3b3ux3afux3b1}{%
\subsection{Μοντελοποίηση και
Αναλογία}\label{ux3bcux3bfux3bdux3c4ux3b5ux3bbux3bfux3c0ux3bfux3afux3b7ux3c3ux3b7-ux3baux3b1ux3b9-ux3b1ux3bdux3b1ux3bbux3bfux3b3ux3afux3b1}}

\leavevmode\vadjust pre{\hypertarget{fig:digital-desk}{}}%
\begin{figure}
\hypertarget{fig:digital-desk}{%
\centering
\includegraphics{images/digital-desk.jpg}
\caption{Εικόνα 13: Το σύστημα Digital Desk δημιουργήθηκε είχε στόχο να
φέρει πάνω στο τραπέζι τις δυνατότητες διάδρασης μεταξύ του ψηφιακού και
φυσικού κόσμου, ακριβώς δηλαδή, το αντίθετο από την κατεύθυνση που
ακολουθεί ο επιτραπέζιος υπολογισμός με την γραφική επιφάνεια
εργασίας.}\label{fig:digital-desk}
}
\end{figure}

\leavevmode\vadjust pre{\hypertarget{fig:mit-clearboard}{}}%
\begin{figure}
\hypertarget{fig:mit-clearboard}{%
\centering
\includegraphics{images/mit-clearboard.jpg}
\caption{Εικόνα 14: Το σύστημα συνεργασίας Clearboard του MIT Media Lab
προβάλλει το βίντεο δύο συνεργατών πάνω στην κοινή εφαρμογή ζωγραφικής,
έτσι ώστε να υπάρχει σύγχρονη επίγνωση του βλέμματος και των χειρονομιών
τους.}\label{fig:mit-clearboard}
}
\end{figure}

Η βασική τεχνολογία για την κατασκευή της διάδρασης παραμένει διαχρονικά
ο προγραμματισμός ενός υπολογιστή. Ο προγραμματισμός με τη σειρά του
είναι μια σημαντική, αλλά ξεχωριστή περίπτωση διάδρασης με τον
υπολογιστή, ο οποίος έχει περάσει από πολλά στάδια και συνεχίζει να
εξελίσσεται. Οι πρώτοι κεντρικοί υπολογιστές όπως και ο πρώτος
μικρο-υπολογιστής Altair προγραμματίζονταν με τους φυσικούς διακόπτες,
αλλά αυτό ήταν δύσκολο, ειδικά για μεγάλα προγράμματα και σετ δεδομένων.
Για αυτόν τον λόγο, ο προγραμματισμός των υπολογιστών ξεκίνησε να
γίνεται με ασύγχρονο τρόπο μέσω της χρήσης διάτριτων καρτών σε σετ ή
ακόμη καλύτερα, με τη χρήση της διάτρητης χαρτοταινίας. Η εγγραφή και η
ανάγνωση από τη διάτρητη χαρτοταινία ήταν πολύ οικονομική και τόσο
δημοφιλής ώστε να βρίσκεται ενσωματωμένη ακόμη και πάνω στον τηλέτυπο, ο
οποίος ήταν για πολλές δεκαετίες ο βασικός τρόπος συγγραφής προγραμμάτων
υπολογιστή και, επόμενως, και διάδρασης, αφού το πληκτρολόγιο και η
διάτρητη χαρτοταινία ήταν τα συστήματα εισόδου, ενώ η χαρτοταινία και ο
εκτυπωτής γραμμής ήταν τα συστήματα εξόδου. Με αυτόν τον τρόπο,
φτιάχτηκαν και τα πρώτα συστήματα διάδρασης σε πραγματικό χρόνο, όπως
ήταν η LISP και το JOSS, αλλά η πλειοψηφία των συστημάτων λειτουργούσε
ασύγχρονα με εργασίες δέσμης. Όλα αυτά τα ηλεκτρομηχανικά συστήματα
εισόδου και εξόδου από τον υπολογιστή, τα οποία βασίζονταν στο χάρτι, θα
αντικατασταθούν από τις ηλεκτρονικές και, αργότερα, ψηφιακές οθόνες και
το ποντίκι, τη δεκαετία του 1980, αλλά ο προγραμματισμός του υπολογιστή
θα παραμείνει μια διαδικασία που βασίζεται σε γραπτές γλώσσες κειμένου
που εισάγονται με το πληκτρολόγιο, γραμμή γραμμή, όπως ακριβώς
απαιτούσαν συσκευές εισόδου και εξόδου που ήταν πλέον παρωχημένες.

Ανάμεσα στις πολλές τεχνολογίες λογισμικού για την κατασκευή νέων
διαδραστικών συστημάτων, οι γλώσσες προγραμματισμού που επιτρέπουν τόσο
την ανάπτυξη νέων συστημάτων όσο και την εξέλιξη της ίδιας της γλώσσας
έχουν διαχρονική αποτελεσματικότητα, αλλά με μεγάλο κόστος για την
εκμάθησή τους. Για παράδειγμα, η εκφραστική δύναμη της αρχικής γλώσσας
Lisp επιτρέπει την συγγραφή του μεταγλωττιστή της στην ίδια τη γλώσσα,
μια επιλογή που δείχνει προς την κατεύθυνση της συνεχούς βελτίωσης της
γλώσσας προγραμματισμού, η οποία δεν είναι πλέον κάτι στατικό. Αυτή η
τεχνική χρησιμοποιήθηκε και κατά τη δημιουργία του διαδραστικού
περιβάλλοντος Smalltalk, η οποία εκτός από γλώσσα προγραμματισμού για
τον χρήστη επιτρέπει και τη μεταβολή του αρχικού συστήματος, δηλαδή και
της ίδιας της γλώσσας. Παρόμοια τακτική ακολούθησαν και άλλοι
κατασκευαστές για να δημιουργήσουν από την αρχή νέα διαδραστικά
συστήματα, όπως είναι τα Lilith και Oberon, τα οποία υλοποιήθηκαν με τις
Modula και Oberon, οι οποίες με τη σειρά τους δεν ήταν εντελώς νέες,
αλλά επεκτάσεις της δομημένης και στατικής Pascal, ώστε να είναι συμβατή
με το πλαίσιο ανάπτυξης του νέου διαδραστικού συστήματος. Η βασική
διαφορά που έχουν αυτά τα συστήματα από το αρχικό Alto και τη Smalltalk
είναι ότι επιλέγουν μια δομημένη γλώσσα προγραμματισμού με στατικό
ορισμό τύπων, παρότι και στις δύο περιπτώσεις αυτά τα συστήματα στόχευαν
στο ίδιο κοινό, δηλαδή στην εκπαίδευση. Από αυτές τις προσπάθειες
προκύπτει ότι οι ίδιες οι γλώσσες και τα περιβάλλοντα προγραμματισμού
δεν είναι ποτέ τεχνολογίες, οι οποίες αναφέρονται κυρίως σε κάποιες
σημαντικές ιδιότητές τους. Επίσης, προκύπτει η κατανόηση ότι η δεξιότητα
χρήσης ενός ισχυρού και εκφραστικού συστήματος διαδραστικής κατασκευής
και δημιουργίας είναι συνήθως αντιστρόφως ανάλογο με την ευκολία
εκμάθησής του.

\leavevmode\vadjust pre{\hypertarget{fig:terminal-emulator}{}}%
\begin{figure}
\hypertarget{fig:terminal-emulator}{%
\centering
\includegraphics{images/terminal-emulator.png}
\caption{Εικόνα 15: Η απλότητα και η διαχρονικότητα του πληκτρολογίου
και της οθόνης κειμένου για τη διάδραση με τον υπολογιστή οδήγησαν στη
δημιουργία εφαρμογών εξομοίωσης, οι οποίες δίνουν πρόσβαση σε έναν
υπολογιστή ανεξάρτητα από την τοποθεσία του. Η εφαρμογή εξομοίωσης
τερματικού επιτρέπει σε έναν χρήστη να αλληλεπιδράσει τόσο με τον
προσωπικό υπολογιστή όσο και με ένα σύστημα που βρίσκεται κάπου αλλού ή
ακόμη και με ένα εικονικό σύστημα στο υπολογιστικό
νέφος.}\label{fig:terminal-emulator}
}
\end{figure}

\leavevmode\vadjust pre{\hypertarget{fig:vnc}{}}%
\begin{figure}
\hypertarget{fig:vnc}{%
\centering
\includegraphics{images/vnc.png}
\caption{Εικόνα 16: Σε αναλογία με την τεχνολογία εξομοιωτή τερματικού
κειμένου, η τεχνολογία εικονικού υπολογιστή επιτρέπει τη διάδραση με
γραφικές διεπαφές και συσκευές εισόδου πέρα από το πληκτρολόγιο. Η
διαθεσιμότητα των ευρυζωνικών δικτύων έδωσε τη δυνατότητα της μετάδοσης
σε πραγματικό χρόνο μιας γραφικής διεπαφής από έναν απομακρυσμένο
υπολογιστή σε ένα απλό τερματικό, το οποίο έχει μόνο συστήματα εισόδου
και εξόδου με τον χρήστη. Το τερματικό αναλαμβάνει την απεικόνιση της
διεπαφής και την είσοδο από τον χρήστη, αλλά όλη η επεξεργασία, καθώς
και η σύνθεση της εικόνας, γίνεται στον απομακρυσμένο υπολογιστή και
στέλνεται ως συμπιεσμένη εικόνα.}\label{fig:vnc}
}
\end{figure}

Μια θεμελιώδης τεχνική που διατρέχει τη δημιουργία όλων των καινοτόμων
τεχνολογιών διάδρασης είναι ότι βασίζονται σε ένα μικρό σύνολο από
αρχιτεκτονικές επιλογές, περισσότερο ως αναλογία και λιγότερο ως
προσομοίωση του πραγματικού κόσμου. Για παράδειγμα, η δημιουργία της
Smalltalk και της ανταλλαγής μηνυμάτων ανάμεσα σε αντικείμενα είναι
εμπνευσμένη από τη λειτουργία του βιολογικού κυττάρου. Όπως δηλαδή ένα
κύτταρο αλληλεπιδρά δυναμικά και χωρίς κάποιο πλάνο με διπλανά του
κύτταρα, έτσι ακριβώς και τα αντικείμενα στη Smalltalk ανταλλάσσουν
μηνύματα δημιουργώντας συνδέσμους δυναμικά κατά την εκτέλεση, ακόμη και
αν αυτές οι διαδράσεις δεν είχαν αρχικά σχεδιαστεί. Αν και η
αρχιτεκτονική για την κατασκευή της Smalltalk βασίζεται στην εξομοίωση,
ένα από τα βασικά κίνητρα της δημιουργίας, καθώς και ένα από τα πεδία
εφαρμογής, έχει να κάνει με την προσομοίωση του φυσικού κόσμου, έτσι
ώστε οι προγραμματιστές να μπορούν να μελετήσουν τη συμπεριφορά των
πολύπλοκων συστημάτων που θα μοντελοποιήσουν. Βλέπουμε, δηλαδή, ότι οι
κατασκευαστές του Alto και της Smalltalk όχι μόνο δεν είναι αντίθετοι
στην προσομοίωση, αλλά είναι πολύ θετικοί, αρκεί όμως αυτό να μην
γίνεται εμπόδιο για τη διαδικασία κατασκευής, η οποία θα πρέπει να
βασίζεται στην εξομοίωση ενός νέου καινοτόμου συστήματος. \footnote{Kay
  (1993)} Μια αντίστοιχη αλλά πολύ διαφορετική αναλογία μπορούμε να
εντοπίσουμε και στην περίπτωση του UNIX, όπου οι κατασκευαστές
χρειάστηκε μαζί με το λειτουργικό σύστημα να δημιουργήσουν και τη νέα
γλώσσα προγραμματισμού C, η οποία σχεδιάστηκε έτσι ώστε να είναι
κατάλληλη για τη συνεχή βελτίωση αυτού του λειτουργικού συστήματος και
του διαδραστικού κελύφους γραμμής εντολών, το οποίο απευθύνεται αρχικά
σε εργαζόμενους τηλεφωνικών οργανισμών που πρέπει να διαχειριστούν
δεδομένα σε αρχεία και να ετοιμάσουν την τεκμηρίωση για τις
τηλεπικοινωνιακές τεχνολογίες. Στην περίπτωση του UNIX, η αναλογία είναι
ότι όλα τα δομικά στοιχεία του συστήματος είναι αρχεία, μια αναλογία που
έφτασε στην ωρίμανσή της με το επόμενο δημιούργημα των ερευνητών στα
Bell Labs, το Plan9.

Στις προηγούμενες ενότητες είδαμε τον κεντρικό ρόλο του τηλέτυπου, ο
οποίος μετασχηματίστηκε σταδιακά από ένα αυτόνομο τερματικό για την
αποστολή μηνυμάτων στη βασική συσκευή διάδρασης για την κατασκευή πολλών
καινοτόμων συστημάτων, όπως τα Sketchpad, NLS, UNIX, Smalltalk, JOSS,
CPM και MS-BASIC. Ταυτόχρονα, παρατηρούμε ότι ο ρόλος του τηλέτυπου στα
ίδια αυτά συστήματα είχε πολύ διαφορετικό βαθμό επιρροής στο τελικό
τερματικό διάδρασης με τον χρήστη. Για παράδειγμα, η διάδραση στο τελικό
σύστημα Sketchpad δεν περιλαμβάνει καθόλου στοιχεία ούτε από το
περιβάλλον ανάπτυξης, αλλά ούτε και από τις συσκευές διάδρασης κατά την
ανάπτυξή του. Στην πιο σύγχρονη εκδοχή του, αυτό το σενάριο
παρουσιάζεται στην περίπτωση ανάπτυξης κινητών εφαρμογών σε έναν
επιτραπέζιο υπολογιστή, αλλά η διάδραση θα γίνει με διαφορετικά
συστήματα εισόδου και εξόδου στο έξυπνο κινητό. Σε ένα ενδιάμεσο επίπεδο
συναντάμε το σύστημα NLS, καθώς και τη Smalltalk, όπου το κείμενο και η
επεξεργασία του έχουν κεντρικό ρόλο, αλλά αυτό είναι μόνο ένα μικρό
υποσύνολο από τις δυνατότητες του γραφικού περιβάλλοντος, αλλά και της
συσκευής εισόδου ποντίκι που είναι κάτι νέο. Στο αντίθετο άκρο, τα
συστήματα JOSS και UNIX, όχι μόνο έχουν τον τηλέτυπο ως βασικό μοντέλο
για τη διάδραση, αλλά υιοθετούν τον τηλέτυπο και ως συσκευή εισόδου και
εξόδου για τη διάδραση με τον τελικό χρήστη, όπου έχει κεντρικό ρόλο στο
τερματικό γραμμής εντολών. Μια διαφορετική αναλογία, αντί για τον
τηλέτυπο μπορούμε να εντοπίσουμε στην τεχνολογία του εικονικού
τερματικού εικονοστοιχείων Virtual Network Computing (VNC), το οποίο
επιτρέπει τη διάδραση με γραφική διεπαφή με έναν απομακρυσμένο
υπολογιστή. Με αυτήν την τεχνολογία η διάδραση σε ένα τερματικό χρήστη
ενορχηστρώνεται σε έναν απομακρυσμένο υπολογιστή, ο οποίος με την
τεχνική της εξομοίωσης μπορεί να έχει όποια συμπεριφορά μπορούμε να
φανταστούμε και να υλοποιήσουμε. \footnote{(\textbf{Εικόνα?})~15
  Εξομοιωτής Τερματικού (Wikimedia)} \footnote{(\textbf{Εικόνα?})~16
  Virtual Network Computing (Wikipedia)}

\hypertarget{ux3b7-ux3c0ux3b5ux3c1ux3afux3c0ux3c4ux3c9ux3c3ux3b7-ux3c4ux3bfux3c5-xerox-alto}{%
\subsection{Η περίπτωση του Xerox
Alto}\label{ux3b7-ux3c0ux3b5ux3c1ux3afux3c0ux3c4ux3c9ux3c3ux3b7-ux3c4ux3bfux3c5-xerox-alto}}

Ο προσωπικός υπολογιστής θεωρείται δεδομένος για τους χρήστες μετά τη
δεκαετία του 1980, αλλά ήταν μόνο μια ιδέα για πολύ λίγους ερευνητές τη
δεκαετία του 1970. Η αρχή έγινε με το όραμα για το Dynabook, το οποίο
έμοιαζε με ένα σύγχρονο τάμπλετ. Αν και ήταν αδύνατο να κατασκευαστεί
ένα τάμπλετ τότε, οι ερευνητές αποφάσισαν να χρησιμοποιήσουν το υλικό
της εποχής τους, ώστε να φτιάξουν ένα σχετικά μικρό υπολογιστή, πάνω
στον οποίο θα αναπτύξουν το λογισμικό του μελλοντικού προσωπικού
υπολογιστή.

\leavevmode\vadjust pre{\hypertarget{fig:xerox-alto}{}}%
\begin{figure}
\hypertarget{fig:xerox-alto}{%
\centering
\includegraphics{images/xerox-alto.jpg}
\caption{Εικόνα 17: Ο επιτραπέζιος υπολογιστής της Xerox Alto ήταν ένα
λειτουργικό πρωτότυπο πάνω στην ιδέα του Dynabook, το οποίο βελτιωνόταν
συνεχώς από τους ερευνητές του PARC και οδήγησε τελικά στην κατασκευή
του πρώτου σύγχρονου υπολογιστή με γραφική επιφάνεια εργασίας, του Xerox
Star.}\label{fig:xerox-alto}
}
\end{figure}

\leavevmode\vadjust pre{\hypertarget{fig:smalltalk-paint}{}}%
\begin{figure}
\hypertarget{fig:smalltalk-paint}{%
\centering
\includegraphics{images/smalltalk-paint.png}
\caption{Εικόνα 18: Οι γραφικές διεπαφές μετατοπίζουν τη σημασία της
διάδρασης από το κείμενο προς τις εικόνες και τα γραφικά. Για αυτόν τον
σκοπό, μαζί με τη δημιουργία του περιβάλλοντος προγραμματισμού είναι
εξίσου σημαντικό να δημιουργηθεί ένα πρόγραμμα επεξεργασίας εικόνας, με
το οποίο θα διευκολυνθεί η κατασκευή των πρόσθετων γραφικών στοιχείων,
καθώς, και των εικονιδίων και των
γραμματοσειρών.}\label{fig:smalltalk-paint}
}
\end{figure}

Ο προσωπικός υπολογιστής Xerox Alto\footnote{(\textbf{Εικόνα?})~17 Xerox
  Alto (DigiBarn)} δημιουργήθηκε με το υλικό των μινι-υπολογιστών
εκείνης της εποχής, αλλά με έμφαση στη διάδραση με έναν χρήστη που
γινόταν με οθόνη γραφικών και ποντίκι. Αρχικά, οι σχεδιαστές του
χρησιμοποίησαν έναν μίνι-υπολογιστή για να εξομοιώσουν την λειτουργία
του και στη συνέχεια κατασκεύασαν εκατοντάδες αντίγραφα του, έτσι ώστε
πολλοί διαφορετικοί χρήστες να αποκτήσουν πρόσβαση και να δημιουργηθεί
μια κοινότητα γύρω από αυτόν. Με αυτόν τον τρόπο, το Alto ήταν κάτι
περισσότερο από ένα εύθραστο εργαστηριακό πείραμα και επέτρεψε στους
χρήστες του όχι απλά να πάρουν μια γεύση από το μέλλον, αλλά να
μεταφερθούν σε αυτό.

Η επιτραπέζια μορφή του Alto ήταν απλά ένα ενδιάμεσο στάδιο πριν
φτάσουμε στους φορητούς προσωπικούς υπολογιστές, αλλά το λογισμικό του
αναπτύχθηκε με αφετηρία το όραμα του Dynabook για έναν υπολογιστή για
παιδιά, όλων των ηλικιών. Πράγματι, το σύστημα προγραμματισμού
Smalltalk\footnote{(\textbf{Εικόνα?})~18 Eπεξεργασία εικόνας (Squeak)}
διευκόλυνε τη γρήγορη ανάπτυξη εφαρμογών και, κυρίως, τον πειραματισμό,
με στόχο την κατανόηση της επιστήμης και της τεχνολογίας. Τα παιδιά
ανέπτυξαν εφαρμογές ζωγραφικής και κίνησης, ενώ παράλληλα έμαθαν να
σκέφτονται μαζί με το σύστημα Alto, το οποίο, εκτός από εργαλείο, ήταν
και μέσο επικοινωνίας.

Εκτός από τα παιδιά, οι προγραμματιστές του Alto ανέπτυξαν λογισμικό και
για πολλές άλλες κατηγορίες χρηστών, όπως είναι οι υπάλληλοι γραφείου,
οι οποίοι αποτελούν βασικούς πελάτες της Xerox. Ο Larry Tesler
μετασχημάτισε τον επεξεργαστή κειμένου Bravo στον Gypsy, έτσι ώστε η
χρήση του να είναι μη τροπική. Με αυτόν τον τρόπο, το Alto αποτέλεσε
παράδειγμα για την ανάπτυξη των επόμενων επιτραπέζιων προσωπικών
υπολογιστών, όπως είναι το Xerox Star και ο Apple Macintosh, ενώ το
λογισμικό του έθεσε τη βάση για τις αντικειμενοστραφείς γλώσσες
προγραμματισμού, τα παραθυρικά περιβάλλοντα και τις δικτυακές εφαρμογές.

\hypertarget{ux3b7-ux3c0ux3b5ux3c1ux3afux3c0ux3c4ux3c9ux3c3ux3b7-ux3c4ux3bfux3c5-ux3b5ux3c1ux3b5ux3c5ux3bdux3b7ux3c4ux3b9ux3baux3bfux3cd-ux3baux3adux3bdux3c4ux3c1ux3bfux3c5-xerox-parc}{%
\subsection{Η περίπτωση του ερευνητικού κέντρου Xerox
PARC}\label{ux3b7-ux3c0ux3b5ux3c1ux3afux3c0ux3c4ux3c9ux3c3ux3b7-ux3c4ux3bfux3c5-ux3b5ux3c1ux3b5ux3c5ux3bdux3b7ux3c4ux3b9ux3baux3bfux3cd-ux3baux3adux3bdux3c4ux3c1ux3bfux3c5-xerox-parc}}

Το ερευνητικό κέντρο της Xerox με την ονομασία PARC (Palo Alto Research
Center) δημιουργήθηκε το 1970 με στόχο να επενδύσει τα μεγάλα κέρδη που
είχε η μητρική εταιρεία από την αγορά των βιομηχανικών εκτυπώσεων στην
έρευνα και την ανάπτυξη νέων προϊόντων και υπηρεσιών. Ειδικά η πρώτη
δεκαετία λειτουργίας του PARC θα είναι γεμάτη εφευρέσεις που θα
καθορίσουν το μέλλον του προσωπικού επιτραπέζιου υπολογιστή, αφού εκεί
θα ωριμάσουν σημαντικές τεχνολογίες, όπως το ποντίκι, η γραφική
επιφάνεια εργασίας, και οι αντικειμενοστραφείς γλώσσες προγραμματισμού.
Ειδικά η δημιουργική σύνθεση αυτών των τεχνολογιών στον Xerox Alto και
αργότερα στον Xerox Star θα βάλουν τα θεμέλια για τον επιτραπέζιο
υπολογιστή με γραφική επιφάνεια εργασίας.\footnote{(\textbf{Εικόνα?})~19
  Ο επιτραπέζιος υπολογιστής Xerox Star (Xerox PARC)}

\leavevmode\vadjust pre{\hypertarget{fig:xerox-star-pc}{}}%
\begin{figure}
\hypertarget{fig:xerox-star-pc}{%
\centering
\includegraphics{images/xerox-star-pc.jpg}
\caption{Εικόνα 19: Ο επιτραπέζιος υπολογιστής με πληκτρολόγιο, ποντίκι
και γραφική επιφάνεια εργασίας (παράθυρα, εικονίδια, φάκελοι), ο οποίος
δημιουργήθηκε από τη Xerox στα τέλη της δεκαετίας του 1970 λίγο διαφέρει
από τον μοντέρνο επιτραπέζιο υπολογιστή.}\label{fig:xerox-star-pc}
}
\end{figure}

\leavevmode\vadjust pre{\hypertarget{fig:xerox-cedar}{}}%
\begin{figure}
\hypertarget{fig:xerox-cedar}{%
\centering
\includegraphics{images/xerox-cedar.png}
\caption{Εικόνα 20: Το σύστημα Xerox Cedar αποτελεί τη συνέχεια του
αποτυχημένου εμπορικά Star, δανειζόμενο στοιχεία από το πετυχημένο
ερευνητικό πρωτότυπο Alto. Η διάδραση βασίζεται στα έγγραφα, τα οποία
είναι το βασικό αντικείμενο της εταιρείας και το σύστημα παρέχει την
ομώνυμη γλώσσα προγραμματισμού, με την οποία ο χρήστης μπορεί να
προσαρμόσει τη διάδραση μέσα από ένα παραθυρικό περιβάλλον, που
εμφανίζει παράλληλα τα έγγραφα και τον πηγαίο κώδικα των
εντολών.}\label{fig:xerox-cedar}
}
\end{figure}

Ένας από τους βασικούς παράγοντες επιτυχίας του PARC είναι οι άνθρωποι
που δούλεψαν εκεί. Ένας από τους πρώτους ήταν ο Alan Kay, ο οποίος,
εκτός από την επιμέρους συμβολή του, είχε καταλυτικό ρόλο στην
ανθρωποκεντρική φιλοσοφία που διαπνέει τις καινοτομίες του PARC.
Επιπλέον, η δημιουργία του εργαστηρίου συνέπεσε με την περίοδο που το
σπουδαίο εργαστήριο SRI (Stanford Research Institute) είχε προβλήματα
χρηματοδότησης, με αποτέλεσμα σημαντικοί επιστήμονες των υπολογιστών,
όπως ο μηχανικός του ποντικιού Bill English, να αναζητήσουν δουλειά στο
PARC. Εκτός λοιπόν από τη γεωγραφική εγγύτητα με το πανεπιστήμιο του
Stanford, το οποίο παράγει συνέχεια νέους ερευνητές, το εργαστήριο είχε
την τύχη να πάρει τη σκυτάλη από την ομάδα του SRI που δημιούργησε το
σύστημα NLS και το πρώτο ποντίκι.

Τη δεκαετία του 1980, η άνθιση της βιομηχανίας των προσωπικών
υπολογιστών θα δελεάσει με τη σειρά της πολλούς από τους ερευνητές του
PARC, οι οποίοι θα δουλέψουν σε εταιρείες όπως η Apple για τη δημιουργία
επιτυχημένων εμπορικών εκδόσεων των δημιουργιών του PARC. Δεν είναι
τυχαίο ότι ο Steve Jobs το 1980 εμπνεύστηκε τον Macintosh κατά τη
διάρκεια μιας επίσκεψης εκεί. Το γεγονός ότι η ίδια η μητρική εταιρεία
Xerox δεν μπόρεσε να εκμεταλλευτεί τις δικιές της εφευρέσεις είναι
αντιπροσωπευτικό της ανάγκης η έρευνα και η γνώση να είναι ανοιχτές για
όλους, καθώς πολλές φορές οι χρηματοδότες της μπορεί να μην βλέπουν ή να
μην ενδιαφέρονται για ευρήματα που τελικά είναι χρήσιμα για τους
πολλούς.

Το 2020 το PARC συμπληρώνει πενήντα χρόνια συμβολής στην περιοχή της
διάδρασης ανθρώπου-υπολογιστή, περίοδος που αντιστοιχεί και στην ηλικία
αυτής της περιοχής. Αν και έχουν υπάρξει πολλά πανεπιστημιακά ή
βιομηχανικά ερευνητικά κέντρα, καθώς και πολλές εταιρείες με μεγάλη
συμβολή σε αυτόν τον χώρο, το PARC είναι με διαφορά το μόνο που έχει μια
διαχρονική συμβολή.\footnote{(\textbf{Εικόνα?})~20 Xerox Cedar
  (toastytech.com)} Πράγματι, τη δεκαετία του 2010, ο επιτραπέζιος
υπολογισμός δίνει τη θέση του στον διάχυτο υπολογισμό (κινητά, φορετά)
και τα οικοσυστημάτα συνεργασίας ανθρώπων και συσκευών, που ήταν ένα από
τα βασικά ερευνητικά θέματα του Marc Weiser στο PARC στις αρχές της
δεκαετίας του 1990.

\hypertarget{ux3c3ux3cdux3bdux3c4ux3bfux3bcux3b7-ux3b2ux3b9ux3bfux3b3ux3c1ux3b1ux3c6ux3afux3b1-ux3c4ux3bfux3c5-alan-kay}{%
\subsection{Σύντομη βιογραφία του Alan
Kay}\label{ux3c3ux3cdux3bdux3c4ux3bfux3bcux3b7-ux3b2ux3b9ux3bfux3b3ux3c1ux3b1ux3c6ux3afux3b1-ux3c4ux3bfux3c5-alan-kay}}

Από πολύ μικρή ηλικία ο Alan Kay έχει παρουσιάσει σημαντικά ταλέντα,
όπως την ικανότητα ανάγνωσης πριν φοιτήσει στο δημοτικό σχολείο, με
αποτέλεσμα να βαριέται στην τάξη, αφού ήδη γνώριζε την ύλη. Οι βασικές
του σπουδές ήταν στα μαθηματικά και τη βιολογία, ενώ ο ίδιος για μεγάλο
διάστημα αυτοπροσδιορίζεται περισσότερο ως μουσικός παρά ως επιστήμονας
των υπολογιστών,\footnote{(\textbf{Εικόνα?})~21 Alan Kay (Alan Kay)} που
ήταν τελικά η επαγγελματική κατεύθυνση που τον έκανε γνωστό. Μάλιστα,
λέγεται ότι αποφεύγει τα ταξίδια και τις μετακινήσεις μακριά από το
σπίτι του γιατί θέλει να μπορεί να συντηρεί και να παίζει στο μεγάλο
εκλησιαστικό μουσικό όργανο που έχει στο σπίτι του.

\leavevmode\vadjust pre{\hypertarget{fig:kay-profile}{}}%
\begin{figure}
\hypertarget{fig:kay-profile}{%
\centering
\includegraphics{images/kay-profile.jpg}
\caption{Εικόνα 21: Ο Alan Kay μετέφερε τον προγραμματισμό με γραφικά σε
ένα προσωπικό διαδραστικό σύστημα, το οποίο σε μια εκδοχή του
δημιούργησε όλα τα σύγχρονα επιτραπέζια συστήματα. Παράλληλα, ο ίδιος
παραμένει με συνέπεια ένας από τους λίγους συνεχιστές της φιλοσοφίας του
Douglas Engelbart, με σύγχρονα συστήματα, τα οποία δίνουν έμφαση
περισσότερο στις προσωπικές δεξιότητες και στην εκφραστική δύναμη, παρά
στην ευχρηστία ως οικειότητα και ευκολία.}\label{fig:kay-profile}
}
\end{figure}

\leavevmode\vadjust pre{\hypertarget{fig:squeakos}{}}%
\begin{figure}
\hypertarget{fig:squeakos}{%
\centering
\includegraphics{images/squeakos.jpg}
\caption{Εικόνα 22: Το σύστημα Squeak βασίζεται στο περιβάλλον
προγραμματισμού Smalltalk, το οποίο επιτρέπει τις απευθείας αλλαγές σε
όλα τα αντικείμενα του συστήματος και ταυτόχρονα δεν κάνει καμία
διάκριση ανάμεσα σε αρχεία, εφαρμογές και λειτουργικό σύστημα, έτσι ώστε
ο χρήστης να έχει τον πλήρη έλεγχο, με έμφαση στα έργα
του.}\label{fig:squeakos}
}
\end{figure}

Η πρώτη σημαντική συνεισφορά του ήταν το όραμα και ο σχεδιασμός του
Dynabook το 1968, το οποίο μορφολογικά και λειτουργικά μοιάζει πολύ με
ένα σύγχρονο (2010) τάμπλετ. Αυτός ο αρχικός σχεδιασμός, όμως, ήταν τόσο
φιλόδοξος που ακόμη και ο ίδιος θεωρεί πως οι σύγχρονοι υπολογιστές δεν
εφαρμόζουν ακόμη όλες εκείνες τις οδηγίες, τουλάχιστον από την ποιοτική
πλευρά. Για να φτάσει στο όραμα του Dynabook ο Alan Kay εστίασε την
προσοχή του στο λογισμικό, το οποίο αρχικά ήταν το περιβάλλον Smalltalk
που έτρεχε σε έναν από τους πρώτους πειραματικούς επιτραπέζιους
υπολογιστές, τον Xerox Alto.

Μετά την εργασία του στο Xerox Parc συνέχισε να δουλεύει στην Apple και
μετά στο Viewpoints Institute, αλλά, σε κάθε περίπτωση, συνέχισε να
εργάζεται πάνω στο αρχικό όραμα του Dynabook, δίνοντας ιδιαίτερη έμφαση
στον τρόπο που μαθαίνουν τα παιδιά. Δουλεύοντας επίμονα πάνω στο ίδιο
αντικείμενο, έκανε πολλές επαναλήψεις με τα προσωρινά Dynabook (Xerox
Alto, Xerox Notetaker), τα οποία μπορούσαν να εκτελέσουν προγράμματα
Smalltalk. Το περιβάλλον Smalltalk αποτέλεσε τη βάση για τη δημιουργία
της Squeak (eToys),\footnote{(\textbf{Εικόνα?})~22 Squeak (Bauman
  National Library)} η οποία με τη σειρά της επηρέασε περισσότερο ως
κίνητρο και λιγότερο ως σχεδίαση τα σύγχρονα εργαλεία εκμάθησης
προγραμματισμού, όπως το Scratch Blocks. Τρεις δεκαετίες μετά θα
συμβάλει στη δημιουργία της διεπαφής για το πρόγραμμα του ενός φορητού
υπολογιστή για κάθε παιδί (MIT OLPC).

Δεν είναι τυχαίο ότι ο Alan Kay επηρεάστηκε ιδιαίτερα από το έργο του
Ivan Sutherland, ενώ o ίδιος ο Alan Kay αποτέλεσε σημαντική επιρροή του
Bret Victor. Παρατηρούμε ότι η δουλειά όλων βασίζεται στην κατασκευή της
διάδρασης με αντικείμενα στην οθόνη του υπολογιστή με στόχο τη
διευκόλυνση της επεξεργασίας της πληροφορίας και, κυρίως, της προσωπικής
έκφρασης. Η πιο σημαντική, όμως, συνεισφορά του Alan Kay δεν είναι τόσο
το ίδιο το υλικό και το λογισμικό των υπολογιστών που οραματίστηκε και
έφτιαξε, αλλά η προσπάθεια να ενδυναμώσει την ανθρώπινη σκέψη και
αντίληψη μέσα από την κατασκευή της διάδρασης.

\hypertarget{ux3b2ux3b9ux3b2ux3bbux3b9ux3bfux3b3ux3c1ux3b1ux3c6ux3afux3b1}{%
\subsection*{Βιβλιογραφία}\label{ux3b2ux3b9ux3b2ux3bbux3b9ux3bfux3b3ux3c1ux3b1ux3c6ux3afux3b1}}
\addcontentsline{toc}{subsection}{Βιβλιογραφία}

\hypertarget{refs}{}
\begin{CSLReferences}{0}{0}
\end{CSLReferences}

Bardini, Thierry. 2000. \emph{Bootstrapping: Douglas Engelbart,
Coevolution, and the Origins of Personal Computing}. Stanford University
Press.

Bolter, Jay David, and Richard Grusin. 2000. \emph{Remediation:
Understanding New Media}. mit Press.

Engelbart, Douglas C. 1962. \emph{Augmenting Human Intellect: A
Conceptual Framework}. SRI, Menlo Park, CA.

Gildall, Gary. 1993. \emph{Computer Connections: People, Places, and
Events in the Evolution of the Personal Computer Industry}. Unpublished.

Hiltzik, Michael. 1999. {``Dealers of Lightning: Xerox PARC and the
Dawning of the Computer Age.''}

Ihde, Don. 2012. \emph{Technics and Praxis: A Philosophy of Technology}.
Vol. 24. Springer Science \& Business Media.

Ingalls, Daniel. 2020. {``The Evolution of Smalltalk: From Smalltalk-72
Through Squeak.''} \emph{Proceedings of the ACM on Programming
Languages} 4 (HOPL): 1--101.

Kay, Alan C. 1993. {``The Early History of Smalltalk.''} \emph{ACM
SIGPLAN Notices} 28 (3): 69--95.

Lanier, Jaron. 2010. \emph{You Are Not a Gadget: A Manifesto}. Vintage.

Mumford, Lewis. 2010. \emph{Technics and Civilization}. University of
Chicago Press.

Nelson, Theodor H. 2010. \emph{POSSIPLEX: Movies, Intellect, Creative
Control, My Computer Life and the Fight for Civilization: An
Autobiography of Ted Nelson}. Mindful Press.

Raskin, Jef. 2000. \emph{The Humane Interface: New Directions for
Designing Interactive Systems}. Addison-Wesley Professional.

Roszak, Theodore. 1986. \emph{From Satori to Silicon Valley: San
Francisco and the American Counterculture}. Don't Call It Frisco Press.

Wirth, Niklaus, and Jürg Gutknecht. 1992. \emph{Project Oberon}.
Addison-Wesley Reading.

\hypertarget{ux3b5ux3b9ux3c3ux3b1ux3b3ux3c9ux3b3ux3ae}{%
\section{Εισαγωγή}\label{ux3b5ux3b9ux3c3ux3b1ux3b3ux3c9ux3b3ux3ae}}

\begin{quote}
Η μάθηση δεν είναι το αποτέλεσμα της διδασκαλίας, αλλά το αποτέλεσμα της
δραστηριότητας του μαθητή. John Holt
\end{quote}

Η κατασκευή της διάδρασης είναι μια σχετικά νέα γνωστική περιοχή, η
οποία δημιουργήθηκε από τη μεγάλη αποδοχή που γνώρισαν τα συστήματα
διάδρασης ανθρώπου και υπολογιστή σε ένα ευρύτατο φάσμα εφαρμογών της
καθημερινότητας και της εργασίας. Είναι τόσες πολλές οι ψηφιακές ανάγκες
των ανθρώπων σε διαφορετικές πτυχές της ζωής τους (π.χ.η ευζωία, η
ψυχαγωγία, η μάθηση, το εμπόριο, η εργασία κτλ.), ενώ ταυτόχρονα
δημιουργούνται συνέχεια νέες συσκευές αλλά και νέες συνδέσεις μεταξύ
τους, ώστε η κατασκευή της διάδρασης αναδεικνύεται οργανικά σε
πρωταγωνιστή στη σχεδίαση νέων ανθρώπινων και κοινωνικών δραστηριοτήτων.
Το βιβλίο αυτό βασίζεται στην άποψη ότι η κατασκευή της διάδρασης, εκτός
του ότι είναι μια σύνθεση πέρα από το άθροισμα των επιμέρους τμημάτων,
είναι κυρίως ένα νέο τεχνολογικό επίπεδο, το οποίο έχει τη δυνατότητα να
επαναπροσδιορίσει με καλό ή κακό τρόπο όλες τις ανθρώπινες και
κοινωνικές δραστηριότητες.

Συνήθως, όταν έχουμε μια νέα γνωστική περιοχή, οι επιστήμονες θα
προσπαθήσουν να την προσεγγίσουν μεθοδικά, σύμφωνα με τις τεχνικές που
έχουν δουλέψει σε παρόμοιες περιοχές στο παρελθόν. Για παράδειγμα, ο
προγραμματισμός αντιμετωπίζεται ως υποπερίπτωση της ευρύτερης περιοχής
των μηχανικών (π.χ., μηχανολόγοι μηχανικοί), αφού έχει να κάνει με την
κατασκευή και τη λειτουργία μιας μηχανής. Ταυτόχρονα, είναι λογικό η
διάδραση να αντιμετωπίζεται ως υποπερίπτωση της ευρύτερης περιοχής του
βιομηχανικού σχεδιασμού (όπως π.χ. η γραφιστική και η εργονομία). Στην
ειδική περίπτωση της κατασκευής της διάδρασης και με δεδομένο ότι
αναφερόμαστε σε μια σύνθετη περιοχή, διαφορετικού επιπέδου από τις
επιμέρους, δεν έχουμε την ευχέρεια να κάνουμε τις παραπάνω
απλουστεύσεις.

Οι συσκευές διάδρασης με τους υπολογιστές, και αντίστοιχα η κατασκευή
της διάδρασής τους, είναι έννοιες φευγαλέες τουλάχιστον για την περίοδο
από τη δεκαετία του 1970 μέχρι και τη δεκαετία του 2010, αφού η διάδραση
με τους υπολογιστές ξεκινάει από το τραπέζι και περνάει στα κινητά,
φορετά, και διάχυτα συστήματα. Tη δεκαετία του 1970, η τυπική μορφή του
προσωπικού υπολογιστή ήταν ο επιτραπέζιος υπολογιστής χωρίς γραφικό
περιβάλλον εργασίας, το οποίο υπήρξε αντικείμενο έρευνας στα εργαστήρια.
Τη δεκαετία του 1980, η γραφική επιφάνεια εργασίας έγινε εμπορικά
διαθέσιμη, ενώ, παράλληλα, το μεγαλύτερο μέρος του λογισμικού είχε
περάσει από τη γραμμή εντολών στα μενού και στις φόρμες, οπότε το
πληκτρολόγιο παρέμεινε η πιο δημοφιλής συσκευή εισόδου. Τη δεκαετία του
1990, η γραφική επιφάνεια εργασίας και το ποντίκι έγιναν ο κυρίαρχος
τρόπος διάδρασης με τον προσωπικό υπολογιστή, οπότε η συσκευή εισόδου
ποντίκι και η έμμεση διάδραση με αντικείμενα στην οθόνη μέσω του δείκτη
καθόρισε τα πιο δημοφιλή στυλ διάδρασης. Στα τέλη της δεκαετίας του
2000, ο κινητός υπολογιστής με οθόνη αφής έφερε στο προσκήνιο τις
χειρονομίες και την άμεση διάδραση στην οθόνη, ενώ τη δεκαετία του 2010,
ο υπολογιστής διαχέεται πέρα από το γραφείο, τόσο στο περιβάλλον όσο και
στο ανθρώπινο σώμα, δημιουργώντας έτσι ένα οικοσύστημα συσκευών και
εφαρμογών για τον χρήστη. Αντίστοιχα, η κατασκευή της διάδρασης
εξελίσσεται έτσι ώστε τα βασικά αρχέτυπα και εργαλεία να διευκολύνουν
τον χειρισμό των νέων συσκευών του χρήστη, όπως είναι το πληκτρολόγιο, η
οθόνη, το ποντίκι, η οθόνη αφής, κτλ.

Παράλληλα, και πάντα αλληλένδετα με την εξέλιξη του υλικού και της
φυσικής μορφής του υπολογιστή, έχουμε μια εξέλιξη του λογισμικού και του
στυλ διάδρασης με τον υπολογιστή, η οποία σχετίζεται περισσότερο με τις
εφαρμογές και τις διεργασίες του χρήστη. Οι πρώτες δημοφιλείς εφαρμογές
του προσωπικού υπολογιστή ήταν ο επεξεργαστής κειμένου και τα φύλλα
εργασίας, τα οποία αποτελούσαν το βασικό κίνητρο αγοράς κατά τις
δεκαετίες του 1970 και του 1980. Τη δεκαετία του 1990 είχαμε τη μεγάλη
υπόσχεση των εκπαιδευτικών και ψυχαγωγικών πολυμέσων, τα οποία τελικά
δεν έφτασαν στον τελικό χρήστη όπως αρχικά είχε σχεδιαστεί (μέσω της
καλωδιακής τηλεόρασης), αλλά περισσότερο μέσω του οπτικού δίσκου, των
κονσολών για βιντεο-παιχνίδια, και του διαδικτύου. Από το τέλος της
δεκαετίας του 2000, έχουμε την επικράτηση των κοινωνικών μέσων δικτύωσης
ως κυρίαρχο στυλ διάδρασης με τον υπολογιστή. Πλέον, όλες οι εφαρμογές,
ανεξάρτητα από το αν έχουν στόχο την παραγωγικότητα, την εκπαίδευση, την
ψυχαγωγία, τις εμπορικές συναλλαγές ή την πληροφόρηση, βασίζονται ή
τουλάχιστον έχουν μια διάσταση κοινωνικού δικτύου. Αντίστοιχα, η
κατασκευή της διάδρασης εξελίσσεται, έτσι ώστε τα βασικά αρχέτυπα και
εργαλεία να διευκολύνουν τον χειρισμό των οντοτήτων του χρήστη, όπως
είναι τα τοπικά αρχεία, τα πολυμέσα, τα υπερμέσα, το κοινωνικό δίκτυο,
κτλ.

Η κατασκευή της διάδρασης ανθρώπου και υπολογιστή, όπως είδαμε συνοπτικά
παραπάνω, έχει παραμείνει για πολύ καιρό μια φευγαλέα περιοχή, επειδή σε
κάθε χρονική περίοδο έχουμε διαφορετικές τεχνολογικές μορφές υπολογιστών
(π.χ. επιτραπέζιος, κινητός, φορετός, διάχυτος) διεπαφών με τους χρήστες
(π.χ. η γραμμή εντολών, το γραφικό περιβάλλον, οι χειρονομίες, η φυσική
γλώσσα) και εφαρμογών (π.χ. η προσομοίωση, το γραφείο, η πλοήγηση, η
φωτογραφία). Για παράδειγμα, ένας χρήστης υπολογιστών, ο οποίος έλαβε τη
βασική, τη δευτεροβάθμια, και την τριτοβάθμια εκπαίδευση τη δεκαετία του
1970, ή το πολύ μέχρι τα μισά της δεκαετίας του 1980, είναι πολύ πιθανό
να έχει μεγάλη εξοικείωση με τη γραμμή εντολών και τους επιτραπέζιους
υπολογιστές, αφού αυτή ήταν η βασική μορφή στα χρόνια της εκπαίδευσής
του. Αντίθετα, ένας χρήστης που έλαβε την εκπαίδευσή του μετά το 2000
και κατά τη δεκαετία του 2010, είναι πολύ πιθανό να μην έχει καθόλου
προσωπικό επιτραπέζιο υπολογιστή, αφού οι βασικές διεργασίες του χρήστη
αυτήν τη χρονική περίοδο (π.χ. αναζήτηση στον παγκόσμιο ιστό, κοινωνική
δικτύωση, ψηφιακό περιεχόμενο κτλ.) μπορούν να γίνουν εξίσου καλά, αν
όχι καλύτερα, με έναν κινητό υπολογιστή με διεπαφή χειρονομίας, η οποία
δεν απαιτεί σχεδόν καμία ανάπτυξη νέων δεξιοτήτων. Η αποδοχή και η
επικράτηση της έννοιας της ευχρηστίας, περισσότερο ως οικειότητας με τις
πρώτες εμπειρίες μας έχει αυξήσει μεν την προσβασιμότητα στην
πληροφορία, αλλά, ταυτόχρονα, έχει μειώσει τη διαφάνεια των τεχνολογιών
διάδρασης, καθώς και τις δεξιότητες που απαιτούνται για την κατασκευή
της διάδρασης.

Βλέπουμε, λοιπόν, ότι, στην πράξη, τόσο ο υπολογιστικός όσο και ο
ψηφιακός αλφαβητισμός είναι έννοιες περισσότερο σχετικές με τη
δημογραφία και την ημερομηνία γέννησης, παρά με μια διαχρονική αξία. Για
παράδειγμα, ο όρος υπολογιστής για πολλές δεκαετίες πριν τη δημιουργία
των πρώτων ηλεκτρονικών και ψηφιακών υπολογιστών αναφερόταν στον άνθρωπο
που έκανε μαθηματικούς υπολογισμούς για να φτιάξει τριγωνομετρικούς και
λογαριθμικούς πίνακες. Για αυτόν τον λόγο, το περιεχόμενο του βιβλίου
σκόπιμα αποφεύγει τις πιο νέες εξελίξεις και τα νέα προϊόντα, έτσι ώστε
να είναι όσο γίνεται πιο διαχρονικό. Η έμφαση δίνεται σε παλαιότερα
συστήματα, όχι επειδή υπάρχει μια ρετρολαγνεία, αλλά επειδή υπάρχουν
διαχρονικές τάσεις, οι οποίες είναι παρούσες και σε σύγχρονα προϊόντα
και οι οποίες ενδέχεται να επηρεάσουν τα μελλοντικά. Η μελέτη
παλαιότερων συστημάτων δεν έχει απλά ιστορικό χαρακτήρα, αλλά σκοπεύει
να φωτίσει εκείνα τα τεχνολογικά και ανθρωπιστικά μοτίβα που
εμφανίζονται και σε σύγχρονα συστήματα και, πολύ πιθανόν, και σε
μελλοντικά.

Εκτός από την έμφαση στα σύγχρονα και επίκαιρα συστήματα, τα περισσότερα
βιβλία σε θέματα τεχνολογίας προσπαθούν να χωρέσουν όσο γίνεται
περισσότερο περιέχομενο στο τυπικό μέγεθος ενός τυπωμένου ή ηλεκτρονικού
βιβλίου. Σε αυτό το βιβλίο, ο στόχος ήταν να καλύψουμε όσο γίνεται
περισσότερα θέματα σε όσο γίνεται μικρότερο χώρο, άρα και σε λιγότερο
χρόνο για τον αναγνώστη. Επιπλέον, το ύφος της γραφής παραμένει
προφορικό και σκόπιμα αποφεύγει το εγκυκλοπαιδικό, αφού όλες οι
πληροφορίες είναι πλέον διαθέσιμες σε ηλεκτρονικά μέσα, καθώς και στα
κλαδικά βιβλία αναφοράς του τομέα. Ακόμη, το βιβλίο συνοδεύεται από
πολλές εικόνες συσκευών και λογισμικού διάδρασης με τον χρήστη. Οι
εικόνες αυτές σκόπιμα παρουσιάζονται σε ζευγάρια με σχετικά εκτενείς
λεζάντες στην ίδια σελίδα, έτσι ώστε να παρέχουν μια παράλληλη διεπαφή
ανάγνωσης, η οποία είναι σίγουρα πολύ οικεία στην εποχή της εικόνας.
Ακόμη περισσότερες εικόνες και πρόσθετους τρόπους οργάνωσής τους θα βρει
ο αναγνώστης στην ιστοσελίδα του βιβλίου, όπου υπάρχουν εικόνες σε
χρονολόγια και σε διαφάνειες. Με αυτόν τον τρόπο, το βιβλίο γίνεται
περισσότερο προσβάσιμο στον αναγνώστη, ενώ παράλληλα, λειτουργεί
συμπληρωματικά με άλλες προσπάθειες.

Αυτό το βιβλίο απευθύνεται σε όσους εμπλέκονται με οποιονδήποτε ρόλο στη
σχεδιάση και στην κατασκευή συστημάτων διάδρασης ανθρώπου και
υπολογιστή. Επομένως, είναι χρήσιμο τόσο σε επαγγελματίες όσο και σε
φοιτητές μαθημάτων πληροφορικής, μηχανικής και σχεδίασης, οι οποίοι
θέλουν να αποκτήσουν εισαγωγικές γνώσεις στη συγκεκριμένη θεματική
περιοχή ή θέλουν να τακτοποιήσουν σκόρπιες γνώσεις. Επιπλέον, με
δεδομένη την εξάπλωση των εργαλείων της πληροφορικής σε πολλούς
συγγενείς τεχνολογικούς και επιστημονικούς κλάδους, αλλά και σε ακόμη
περισσότερους κλάδους που ωφελούνται ή ακόμη και επηρεάζονται από τις
εφαρμογές της, το βιβλίο αυτό απευθύνεται σε όλους αυτούς που
συμμετέχουν σε μια ομάδα που καλείται να σχεδιάσει ή να βελτιώσει ένα
διαδραστικό σύστημα το οποίο εμπλέκεται σε μια ανθρώπινη δραστηριότητα,
ανεξάρτητα από τον ρόλο τους και ανεξάρτητα από τη βασική τους
δεξιότητα.

Υπάρχουν πολλά βιβλία και ακόμη περισσότερες ελεύθερες πηγές στο δίκτυο,
τα οποία είναι πλούσια σε περιεχόμενο και σε εγκυκλοπαιδικές γνώσεις,
και στα οποία αξίζει να ανατρέξουμε κάθε φορά που θα έχουμε ένα
συγκεκριμένο ερώτημα ή όταν θέλουμε να ενημερωθούμε σε βάθος. Η ανάγνωση
ενός βιβλίου είναι μεν αναγκαία συνθήκη, αλλά όχι και ικανή για να
μεταδώσει πρακτικές γνώσεις, ακόμη και όταν ο αναγνώστης μπορεί να
θυμάται το περιεχόμενο. Για αυτόν τον σκοπό, το βιβλίο συνοδεύεται με
συμπληρωματικό πολυμεσικό περιεχόμενο και κυρίως με τη δυνατότητα για
την προσθήκη περιεχομένου από τους αναγνώστες σε δικό τους αντίγραφο του
πηγαίου κώδικα. Η εποικοδομητική μελέτη του συμπληρωματικού περιεχόμενου
δίνει τη δυνατότητα στον αναγνώστη να μεταβεί σταδιακά στη δραστηριότητα
της σκέψης και της συγγραφής και, μέσα από αυτήν την προσπάθεια, να
κατανόησει καλύτερα όχι απλά το περιεχόμενο, αλλά και την ευρύτερη
γνωστική περιοχή. Η ουσιαστική όμως κατανόηση πρακτικών ζητημάτων, όπως
η κατασκευή της διάδρασης, απαιτεί και την πρακτική ενασχόληση με τα
αντίστοιχα ζητήματα, η οποία υλοποιείται μέσα από τις προτάσεις για
πρόσθετες δραστηριότητες κατασκευής διαδραστικών συστημάτων.

Στα επόμενα κεφάλαια αυτού του βιβλίου μελετάμε εκείνα τα θέματα τα
οποία, ανεξάρτητα από τις τεχνολογικές εξελίξεις των τελευταίων
δεκαετιών, παραμένουν διαχρονικά και επίκαιρα.

\hypertarget{ux3c0ux3c1ux3ccux3bbux3bfux3b3ux3bfux3c2}{%
\section{Πρόλογος}\label{ux3c0ux3c1ux3ccux3bbux3bfux3b3ux3bfux3c2}}

\begin{quote}
Τα πράγματα που πρέπει να κάνεις, τα μαθαίνεις κάνοντάς τα. Αριστοτέλης
\end{quote}

Ο σκοπός αυτού του βιβλίου είναι να δώσει μια σύντομη εισαγωγή στα
συστήματα διάδρασης ανθρώπου και υπολογιστή και, κυρίως, να ενθαρρύνει
έναν κριτικό διάλογο αναφορικά με τις ατομικές και συλλογικές επιλογές,
οι οποίες έχουν διαμορφώσει τα σύγχρονα συστήματα. Η μελέτη των
παλαιότερων συστημάτων έχει ιστορικό χαρακτήρα μόνο σε μια πρώτη
επιφανειακή ανάγνωση, γιατί ο βασικός σκοπός είναι να εντοπιστούν
εκείνες οι συνθήκες, οι δυνάμεις και τα υλικά, τα οποία θα επιτρέψουν
την κατασκευή νέων συστημάτων. Μέσα από την κριτική ανάλυση των
παλαιότερων συστημάτων προκύπτουν ερμηνείες για τη μορφή τους. Επιπλέον,
η μελέτη των παλαιότερων συστημάτων αποκαλύπτει τις διαχρονικές αξίες
και τις βέλτιστες πρακτικές, οι οποίες μπορούν να οδηγήσουν σε καλύτερα
συστήματα διάδρασης, με τρόπο συστηματικό και με τεκμηριωμένες
παραδοχές.

Μια προσεκτική μελέτη των παραδοσιακών και των σύγχρονων συστημάτων
διάδρασης δείχνει ότι δημιουργήθηκαν σε ένα συγκεκριμένο τεχνολογικό και
πολιτισμικό πλαίσιο και ότι εξυπηρετούν συγκεκριμένα κίνητρα και
στόχους. Αυτή η διαπίστωση απελευθερώνει τον αναγνώστη, καθώς του
επιτρέπει να κατανοήσει όλα τα σύγχρονα συστήματα, απλά ως ένα
στιγμιότυπο μιας διαδρομής με πολλές εναλλακτικές, και όχι ως κάτι
αναπόφεκτο, ούτε καν ως αναγκαίο βήμα, γι' αυτά που θα μπορούσαν να
δημιουργηθούν. Πέρα από μια κριτική ανάγνωση της τεχνολογικής εξέλιξης,
το κυρίαρχο αφήγημα που διατρέχει το σώμα του κειμένου, αλλά και το
συμπληρωματικό του περιεχόμενο, είναι η έμφαση σε εκείνες τις
διαχρονικές τεχνολογίες και τεχνικές που επιτρέπουν την κατασκευή νέων
εναλλακτικών συστημάτων διάδρασης για τις ανάγκες του σήμερα αλλά και
του αύριο.

Αυτό που παραμένει διαχρονικό δεν είναι τόσο κάποια δεδομένη γραφική
διεπαφή, όπως αυτή του κινητού ή του επιτραπέζιου συστήματος, αλλά
κυρίως εκείνη η σύνθεση υλικών και δυνάμεων όπως οι διαδραστικές αξίες,
οι μέθοδοι, τα αρχέτυπα, οι τεχνικές, και τα μοντέλα τα οποία
δημιούργησαν εκείνες τις διεπαφές. Με αυτόν τον τρόπο, ο αναγνώστης
μαθαίνει κυρίως να σκέφτεται για τις συνθήκες και για τον τρόπο που
κατασκευάστηκαν τα υπάρχοντα διαδραστικά συστήματα, έτσι ώστε να
μπορέσει στη συνέχεια, αφού κρίνει και ερμηνεύσει το παρελθόν, να
συνθέσει τις δυνάμεις της σύγχρονης εποχής για την κατασκευή μελλοντικών
συστημάτων διάδρασης.

Οι παραπάνω παραδοχές επηρεάζουν το περιεχόμενο και τη μορφή του
βιβλίου. Το περιεχόμενο μοιάζει σαν να γράφτηκε τη δεκαετία του 2000,
έτσι ώστε σε δέκα χρόνια από σήμερα να έχει την ίδια αξία, αφού η
κατανόηση μας για εκείνα τα συστήματα θα έχει αλλάξει λίγο στο μεταξύ.
Επιπλέον, δίνεται έμφαση στα κλασικά συστήματα διάδρασης, γιατί με τη
βοήθεια της χρονικής απόστασης που έχουμε, μπορούμε να ξεχωρίσουμε πιο
εύκολα τις ιδιότητες εκείνες οι οποίες είναι διαχρονικές από εκείνες που
απλά εξυπηρετούσαν παλιά κίνητρα, ή που ταίριαζαν στο οργανωσιακό
πλαίσιο μιας εποχής ή ενός οργανισμού. Αντίστοιχα, και η μορφή του
βιβλίου ακολουθεί το περιεχόμενό του και συνοδεύεται από εικόνες
συστημάτων που θεωρούνται κλασικά, ανεξάρτητα από την αρχική αποδοχή
τους, με την προϋπόθεση ότι περιέχουν αξίες και ιδέες που τελικά
συναντάμε διαχρονικά.

Συνοπτικά, σε αυτό το βιβλίο γίνεται μια σύνθεση γνώσεων με στόχο την
έμπνευση του αναγνώστη, ο οποίος θα αναζητήσει περισσότερα έξω από αυτό,
και, γιατί όχι, θα μπει σε έναν διάλογο με τον συγγραφέα, στο αποθετήριο
ανοιχτού πηγαίου κειμένου του βιβλίου, το οποίο υπάρχει για αυτόν τον
σκοπό. Με αυτόν τον τρόπο, τόσο η συγγραφή όσο και η ανάγνωση αυτού του
βιβλίου, γίνονται με τη μορφή ενός κριτικού διαλόγου, όπως ακριβώς
δηλαδή και το πνεύμα που διατρέχει το βιβλίο απέναντι στις τεχνολογίες
των συστημάτων διάδρασης.

\hypertarget{ux3b9ux3c3ux3c4ux3bfux3c3ux3b5ux3bbux3afux3b4ux3b1-ux3baux3b1ux3b9-ux3c3ux3c5ux3bdux3bfux3b4ux3b5ux3c5ux3c4ux3b9ux3baux3cc-ux3c0ux3b5ux3c1ux3b9ux3b5ux3c7ux3ccux3bcux3b5ux3bdux3bf}{%
\section{Ιστοσελίδα και συνοδευτικό
περιεχόμενο}\label{ux3b9ux3c3ux3c4ux3bfux3c3ux3b5ux3bbux3afux3b4ux3b1-ux3baux3b1ux3b9-ux3c3ux3c5ux3bdux3bfux3b4ux3b5ux3c5ux3c4ux3b9ux3baux3cc-ux3c0ux3b5ux3c1ux3b9ux3b5ux3c7ux3ccux3bcux3b5ux3bdux3bf}}

\begin{quote}
Συνέχεια προσπαθώ να κάνω αυτό που δεν μπορώ, έτσι ώστε κάποια στιγμή να
μπορώ να το κάνω. Πάμπλο Πικάσο
\end{quote}

Αν και φτάσατε στην τελευταία σελίδα, το βιβλίο αυτό δεν τελειώνει εδώ,
αλλά συνεχίζεται στη συνοδευτική ηλεκτρονική του έκδοση, η οποία
περιέχει πολυμεσικό και διαδραστικό περιεχόμενο, όπως είναι θεματικές
διαφάνειες και τα χρονολόγια. Επίσης, σκόπιμα το βιβλίο δεν περιέχει μια
σειρά από ενότητες όπως οι ασκήσεις και οι εργασίες, γιατί αυτές
βρίσκονται στην ιστοσελίδα, έτσι ώστε να μπορούν να ενημερώνονται
συνέχεια: \url{https://pibook.epidro.me}

\end{document}
